\secly{Структура встраиваемого микро\linux а}

\begin{description}
  \item{\prog{syslinux}}\ Загрузчик

em\linux\ поставляется в виде двух файлов:

\begin{enumerate}[nosep]
  \item ядро \verb|$(HW)$(APP).kernel|
  \item сжатый образ корневой файловой системы \verb|$(HW)$(APP).rootfs|
\end{enumerate}

Загрузчик считывает их с одного из носителей данных, который поддерживается
загрузчиком\note{IDE/SATA HDD, CDROM, USB флешка, сетевая загрузка с
BOOTP-сервера по Ethernet}, распаковывает в память, включив защищенный режим
процессора, и передает управление ядру \linux.

\begin{framed}\noindent
Для работы em\linux\ не требуются какие-либо носители данных: вся
(виртуальная) файловая система располагается в ОЗУ. При необходимости к любому
из каталогов корневой ФС может быть \term{подмонтирована}\ любая существующая
дисковая или сетевая файловая система (FAT,NTFS,Samba,NFS,..), причем можно явно
запретить возможность записи на нее, защитив данные от
разрушения.

\emph{Использование rootfs в ОЗУ позволяет гарантировать защиту базовой ОС и
пользовательских исполняемых файлов от внезапных выключений питания и
ошибочной записи на диск. Еще большую защиту даст отключение драйверов
загрузочного носителя в ядре.} Если отключить драйвера SATA/IDE и грузиться
с USB флешки, практически невозможно испортить основную установку ОС и
пользовательские файлы на чужом компьютере.
\end{framed}
   
  \item{\prog{kernel}}\ Ядро \linux\ 3.13.xx
  \item{\prog{ulibc}}\ Базовая библиотека языка Си
  \item{\prog{busybox}}\ Ядро командной среды UNIX, базовые сетевые сервера
  \item{дополнительные библиотеки}
  \begin{description}[nosep]
    \item{\prog{zlib}} сжатие/распаковка gzip
    \item{\prog{???}} библиотека помехозащищенного кодирования
    \item{\prog{png}} библиотека чтения/записи графического формата .png 
    \item{\prog{freetype}} рендер векторных шрифтов (TTF) 
	\item{\prog{SDL}} полноэкранная (игровая) графика, аудио, джойстик
  \end{description}
  \item{кодеки аудио/видео форматов}: ogg vorbis, mp3, mpeg, ffmpeg/gsm
\end{description}

\begin{framed}\noindent
К базовой системе добавляются пользовательские
программы \dir{/usr/bin}\\ и динамические библиотеки \dir{/usr/lib}.
\end{framed}

