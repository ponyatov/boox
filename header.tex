% e-book
\documentclass[oneside,12pt]{book}
%% screen paper layout (A5 landscape)
\usepackage[paperwidth=210mm,paperheight=148mm,margin=5mm]{geometry}
%%% font setup for screen reading
%\renewcommand{\familydefault}{\sfdefault}\normalfont
%% hyperlinks pdf style
\usepackage[unicode,colorlinks=false,pdfborderstyle={/S/U/W 1}]{hyperref}
% [nosep] option in lists/enums
\usepackage{enumitem}
% frame box
\usepackage{framed}

% Cyrillization
\usepackage[T1,T2A]{fontenc}
\usepackage[utf8]{inputenc}
\usepackage[english,russian]{babel}

% relative sectioning
\usepackage{ifthen}
\newcounter{secdepth}\setcounter{secdepth}{0}
\newcommand{\secup}{\addtocounter{secdepth}{1}}
\newcommand{\secdown}{\addtocounter{secdepth}{-1}}
\newcommand{\secrel}[1]{
\ifthenelse{\equal{\value{secdepth}}{0}}{\part{#1}}{}
\ifthenelse{\equal{\value{secdepth}}{-1}}{\chapter{#1}}{}
\ifthenelse{\equal{\value{secdepth}}{-2}}{\section{#1}}{}
\ifthenelse{\equal{\value{secdepth}}{-3}}{\subsection{#1}}{}
\ifthenelse{\equal{\value{secdepth}}{-4}}{\subsubsection{#1}}{}
}
\newcommand{\secly}[1]{\section*{#1}\addcontentsline{toc}{section}{#1}}

% computer-related markup: manuals, listings, CS

%% typical =objects
\newcommand{\file}[1]{\textbf{#1}}
\newcommand{\dir}[1]{\textbf{\textit{#1}}}
\newcommand{\prog}[1]{\textbf{#1}}
\newcommand{\class}[1]{\textbf{#1}}

%% languages
\newcommand{\cpp}{$C^+_+$}
\newcommand{\py}{$Python$}
\newcommand{\bi}{$bI$}
\newcommand{\st}{$SmallTalk$}
\newcommand{\lisp}{$Lisp$}

%% OSes
\newcommand{\linux}{$Linux$}

%% MCU
\newcommand{\cm}{Cortex-M}

%% listings
\usepackage{verbatim}
\usepackage{listings}
\lstset{
basicstyle=\small,
frame=single,
numbers=left,numberstyle=\small,numbersep=2mm,
tabsize=4,
keywordstyle=\textbf,
commentstyle=\textit
}
\newcommand{\lstx}[2]{\lstinputlisting[title=#1]{#2}}
\newcommand{\lst }[3]{\lstinputlisting[title=#1,language=#3]{#2}}


% misc
%% typical macros
\newcommand{\email}[1]{$<$\href{mailto:#1}{#1}$>$}
\newcommand{\note}[1]{\footnote{\ #1}}
\renewcommand{\emph}[1]{\textbf{#1}}
\newcommand{\cp}[1]{\note{\copyright\ #1}}
\newcommand{\term}[1]{\textit{#1}}
\newcommand{\termdef}[2]{\textbf{\textit{#1}}}
