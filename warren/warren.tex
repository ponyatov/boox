\secrel{Warren’s Abstract Machine\\Абстрактная машина
Варрена}\label{warren}\secdown

\cp{http://wambook.sourceforge.net/}

\copyright\ Hassan A\"it-Kaci \email{hak@cs.sfu.ca}

\copyright\ David H. D. Warren

\secly{Предисловие к репринтному изданию}

Этот докуент\ --- репринтное издание книги имеющей то же название, которая была
опубликована MIT Press, в 1991 году с кодом ISBN 0-262-51058-8 (мягкая обложка)
and ISBN 0-262-01123-9 (тканый переплет). Редакция книги MIT Press сейчас
не перездается, и права на издание были переданы автору.
Оригинальная версия\note{английская \url{http://wambook.sourceforge.net/}}\
была бесплатно доступна всем, кто хочет ее использовать в некоммерческих целях,
с веб-сайта автора:

\bigskip
\url{http://www.isg.sfu.ca/˜hak/documents/wam.html}
\bigskip

\textit{Сейчас ссылка недоступна, книга пеерехала на
\url{http://wambook.sourceforge.net/}}

\bigskip
Если вы используете ее, пожалуйста дайте мне знать кто вы и для каких целей
хотите ее использовать.

\bigskip
Thank you very much.

\bigskip
Hassan A\"it-Kaci

Burnaby, BC, Canada

May 1997

\secly{Предисловие}

Язык \prolog\ был задуман в начале 1970х Alain Colmerauer a и его коллегами из
Марсельского университета. Его реализация языка была первым практическим
воплощением концепции \term{логического программирования}, предложенной Robert
Kowalski. Ключевая идея логического программирования\ --- вычисления могут быть
выражены в виде конктролируемого вывода (дедукции) из набора декларативных
утверждений. Несмотря на то что эта область значительно развилась за последнее
время, \prolog\ остается наиболее фундаментальным и широко известным языком
логического программирования.

Первой реализацией \prolog а был интерпретатор, написанный на Фортране членами
группы Colmerauer а. Несмотря на очень ущербную в некотором смысле реализацию,
эта версия считается в некотором смысле первым камнем: она доказала
жизнеспособность \prolog а, помогла распространению языка, и заложила основные
принципы реализаций \prolog а. Следующим шагом возможно была \prolog-система для
PDPD-10, разработанная в Университете Эдинбурга мной и коллегами. Эта система
построена на базе техник Марсельской реализации, с добавлением понятия
компиляции \prolog а в низкоуровневый язык (в случае PDP-10 это машинный код), а
также различные техники экономии памяти. Позже я уточнил и абстрагировал
принципы реализации \prolog\ DEC-10 в то, что я называю \prog{WAM} (Warren
Abstract Machine).

\prog{WAM}\ --- абстрактная (виртуальная) машина с архитектурой памяти и набором
команд, заточенных под язык \prolog. Она может быть эффективно реализована на
широком наборе аппаратных архитектур, и служить целевой платформой для
переносимых компиляторов \prolog а. Сейчас она принимается как стандартный базис
при реализации \prolog а. Это конечно лично приятно, но неудобно в том, что WAM
слишком легко принимается как стандарт. Несмотря на то что WAM явилась
результатом длительной работы и большого опыта в реализации \prolog а, это
отнюдь не единственно возможный подход. Например, в то время как WAM применяет
\term{копирование структуры}\note{structure copying}\ для представления
\term{термов}\ \prolog а, метод \term{общих структур}\note{structure
sharing}, использованный в Марсельской и DEC-10 реализациях, все еще можно
рекомендовать к применению. Как бы то ни было, я считаю WAM хорошей отправной
точкой для изучения технологий реализации \prolog-машины.

К сожалению до сих пор не было хорошей книги для ознакомления с внутренним
устройством WAM. Мой оригинальный технический отчет слишком сложен, и написан
для опытного читателя. Другие работы обсуждают WAM с различных точек зрения, но
все же не могут быть использованы в качестве хорошего вводного руководства.

Поэтому очень приятно видеть появление этого прекрасного учебника, написанного
Hassan A\"it-Kaci. Эту книгу приятно читать. Она объясняет WAM c большой
ясностью и элегантностью. Я думаю что читатели, интересующиеся информатикой,
найдут эту книгу очень стимулирующим введением в увлекательную тему\ ---
реализацию \prolog а. Я очень благодарен Хассану за донесение моей работы до
широкой аудитории.

\bigskip
\copyright\ David H. D. Warren

Бристоль, UK

Февраль 1991

\secrel{1 Введение 3}\secdown

В 1983 году Дэвид Варрэн разработал абстрактную машину для реализации языка
\prolog, содержащую специальную архитектуру памяти и набор инструкций
\cite{War83}. Эта разработка стала известка как Warren Abstract Machine (WAM)
и стала стандартом де-факто для реализаций компиляторов \prolog а. В
\cite{War83} Варрэн описан WAM в минималистичном стиле, который слишком сложен
для понимания неподготовленным читателем, даже заранее знакомым в операциями
\prolog а. Слишком многое было несказанным, и very little is justified in clear
terms\note{David H. D. Warren поделился в частной беседе что он ``чувствовал
что WAM важна, но к деталям ее реализации вряд ли будет широкий интерес, поэтому
он использовал стиль личных заметок''}. Это привело к очень скудному количеству
поклонников WAM, которые могли был похвастаться пониманием деталей ее работы.
Обычно это были реализаторы \prolog а, которые решили уделить необходимое время
для обучения через делание и кропотливого достижения просветления.

\secrel{1.1 Существующая литература 3}

 

\secrel{1.2 Этот учебник 5 }\secdown

\secrel{1.2.1 Disclaimer and motivation . . . . . . . . . . . . . . . . . 5}

\secrel{1.2.2 Organization of presentation . . . . . . . . . . . . . . . . 6}

\secup
\secup
\secrel{2 Unification—Pure and Simple 9}\secdown

\secrel{2.1 Term representation . . . . . . . . . . . . . . . . . . . . . . . .
. 10 }

\secrel{2.2 Compiling L
queries . . . . . . . . . . . . . . . . . . . . . . . . 11}

\secrel{2.3 Compiling L
programs . . . . . . . . . . . . . . . . . . . . . . . 13}

\secrel{2.4 Argument registers . . . . . . . . . . . . . . . . . . . . . . . . .
19}

\secup

\secrel{3 Flat Resolution 25}\secdown

\secrel{3.1 Facts . . . . . . . . . . . . . . . . . . . . . . . . . . . . . . .
. . 26}

\secrel{3.2 Rules and queries . . . . . . . . . . . . . . . . . . . . . . . . .
. 27}

\secup
\secrel{4 Prolog 33}\secdown

\secrel{4.1 Environment protection . . . . . . . . . . . . . . . . . . . . . . .
34}

\secrel{4.2 What’s in a choice point . . . . . . . . . . . . . . . . . . . . . .
36}

\secup

\secrel{5 Optimizing the Design 45}\label{warren5}\secdown

\secrel{5.1 Heap representation . . . . . . . . . . . . . . . . . . . . . . . .
. 46}

\secrel{5.2 Constants, lists, and anonymous variables . . . . . . . . . . . . .
47}

\secrel{5.3 A note on set instructions . . . . . . . . . . . . . . . . . . . . .
52}

\secrel{5.4 Register allocation . . . . . . . . . . . . . . . . . . . . . . . .
. 54}

\secrel{5.5 Last call optimization . . . . . . . . . . . . . . . . . . . . . . .
. 56}

\secrel{5.6 Chain rules . . . . . . . . . . . . . . . . . . . . . . . . . . . .
. 57}

\secrel{5.7 Environment trimming . . . . . . . . . . . . . . . . . . . . . . .
58}

\secrel{5.8 Stack variables . . . . . . . . . . . . . . . . . . . . . . . . . .
. 60}\secdown

\secrel{5.8.1 Variable binding and memory layout . . . . . . . . . . . . 62}

\secrel{5.8.2 Unsafe variables . . . . . . . . . . . . . . . . . . . . . . 64}

\secrel{5.8.3 Nested stack references . . . . . . . . . . . . . . . . . . . 67}

\secup

\secrel{5.9 Variable classification revisited . . . . . . . . . . . . . . . . .
. . 69}

\secrel{5.10 Indexing . . . . . . . . . . . . . . . . . . . . . . . . . . . . .
. . 75}

\secrel{5.11 Cut . . . . . . . . . . . . . . . . . . . . . . . . . . . . . . . .
. . 83}

\secup

\secrel{6 Conclusion 89}
\secrel{A Prolog in a Nutshell 91}
\secrel{B The WAM at a glance 97}\label{warrenB}\secdown

\secrel{B.1 WAM instructions . . . . . . . . . . . . . . . . . . . . . . . . . .
97}

\secrel{B.2 WAM ancillary operations . . . . . . . . . . . . . . . . . . . . .
112}

\secrel{B.3 WAM memory layout and registers . . . . . . . . . . . . . . . . .
117}

\secup


\secup
