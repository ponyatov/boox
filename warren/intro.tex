\secrel{1 Введение 3}\secdown

В 1983 году Дэвид Варрэн разработал абстрактную машину для реализации языка
\prolog, содержащую специальную архитектуру памяти и набор инструкций
\cite{War83}. Эта разработка стала известка как Warren Abstract Machine (WAM)
и стала стандартом де-факто для реализаций компиляторов \prolog а. В
\cite{War83} Варрэн описан WAM в минималистичном стиле, который слишком сложен
для понимания неподготовленным читателем, даже заранее знакомым в операциями
\prolog а. Слишком многое было несказанным, и very little is justified in clear
terms\note{David H. D. Warren поделился в частной беседе что он ``чувствовал
что WAM важна, но к деталям ее реализации вряд ли будет широкий интерес, поэтому
он использовал стиль личных заметок''}. Это привело к очень скудному количеству
поклонников WAM, которые могли был похвастаться пониманием деталей ее работы.
Обычно это были реализаторы \prolog а, которые решили уделить необходимое время
для обучения через делание и кропотливого достижения просветления.

\secrel{1.1 Существующая литература 3}

Свидетельством недостатка понимания может служить тот факт, что за первые шесть
лет было крайне мало публикаций о WAM, не говоря о том чтобы формально доказать
ее корректность. Кроме оригинального герметического доклада Варрэна
\cite{War83}, практически не было никаких официальных публикаций о WAM.
Несколько лет спустя группой Аргонской Национальной Лаборатории был выпущен
единственный черновой стандарт \cite{GLLO85}. Но следует отметить что этот
манускрипт был еще менее понятен, чем оригинальный отчет Варрэна. Его
недостатком была цель описать готовую WAM как есть, а не как пошагово
трансформируемый и оптимизируемый проект.

Стиль пошагового улучшения фактически был использован в публикации David Maier и
David S. Warren\note{Это другой человек, а не разработчик WAM, работа которого
вдохновила S.Warren на исследования. В свою очередь достаточно интересно что
David H. D. Warren позже работал над параллельной архитектурой реализации
\prolog а, поддерживая некоторые идеи, независимо предложенные David S.
Warren.}\ \cite{MW88}. В этой работе можно найти описание техник компиляции
\prolog а похожие на принципы WAM\note{chap.9}. Тем не менее мы считаем что
эта похвальная попытка все еще страдает от нескольких недостатков, если его
рассматривать как окончательный учебник.
Прежде всего эта работа описывает собственный достаточно близкий вариант WAM, но
строго говоря не ее саму. Так что описаны не все особенности WAM.
Более того, объяснения ограничены иллюстративными примерами, и редко четко и
исчерпывающие очерчивают контекст, в котором применяются некоторые оптимизации.
Во-вторых, часть посвященная компиляции \prolog а, идет очень поздно\ --- в
предпоследней главе, полагаясь в деталях реализации на свердетализированные
процедуры на Паскакле, и структуры данных, последовательно улчшаемые в течение
предыдущих разделов. Мы чувствуем что это уводит и запутывает читателя,
интересующегося абстрактной машиной. Наконец, несмотря на то что публикация
содержит серию последовательно улучшаемых вариантов реализации, этот учебник не
отделяет независимые части \prolog а в процессе. Все представленные версии\ ---
полные \prolog-машины. В результате, читатель интересующися выбором и сравнением
отдельных техник, которые он хочет применить, не может различить отдельные
техники в тексте. По всей справедливости, книга Майера и С.Варрэна имеет амбиции
быть первой книгой по логическому программирования. Так что они совершили
подвиг, охватывая так много материала, как теоретического так и практического, и
даже включили техники компиляции \prolog а. Более важно, что их книга была
первой доступной официальной публикацией, содержащей реальный учебник по
техникам WAM.

After the preliminary version of this book had been completed, another recent
publication containing a tutorial on the WAM was brought to this author’s
attention. It is a book due to Patrice Boizumault \cite{Boi88} whose Chapter 9
is devoted to explaining the WAM. There again, its author does not use a gradual
presentation of partial \prolog\ machines. Besides, it is written in French\ ---
a somewhat restrictive trait as far as its readership is concerned. Still,
Boizumault’s book is very well conceived, and contains a detailed discussion
describing an explicit implementation technique for the \var{freeze}
meta-predicate\note{chap.10}.

Even more recently, a formal verification of the correctness of a slight
simplification of the WAM was carried out by David Russinoff \cite{Rus89}. That
work deserves justified praise as it methodically certifies correctness of most
of the WAM with respect to \prolog’s SLD resolution semantics. However, it is
definitely not a tutorial, although Russinoff defines most of the notions he
uses in order to keep his work self-contained. In spite of this effort,
understanding the details is considerably impeded without working familiarity
with the WAM as a prerequisite. At any rate, Russinoff’s contribution is
nevertheless a \emph{premi\`ere} as he is the first to establish rigorously
something that had been taken for granted thus far. Needless to say, that report
is not for the fainthearted.
 
\secrel{1.2 Этот учебник 5}\secdown

\secrel{1.2.1 Disclaimer and motivation 5}

The length of this monography has been kept deliberately short. Indeed, this
author feels that the typical expected reader of a tutorial on the WAM would
wish to get to the heart of the matter quickly and obtain complete but short
answers to questions. Also, for reasons pertaining to the specificity of the
topic covered, it was purposefully decided not to structure it as a real
textbook, with abundant exercises and lengthy comments. Our point is to make the
WAM explicit as it was conceived by David H. D. Warren and to justify its
workings to the reader with convincing, albeit informal, explanations. The few
proposed exercises are meant more as an aid for understanding than as food for
further thoughts.

The reader may find, at points, that some design decisions, clearly correct as
they may be, appear arbitrarily chosen among potentially many other
alternatives, some of which he or she might favor over what is described. Also,
one may feel that this or that detail could be ``simplified'' in some local or
global way. Regarding this, we wish to underscore two points: (1) we chose to
follow Warren’s original design and terminology, describing what he did as
faithfully as possible; and, (2) we warn against the casual thinking up of
alterations that, although that may appear to be “smarter” from a local
standpoint, will generally bear subtle global consequences interfering with
other decisions or optimizations made elsewhere in the design.
This being said, we did depart in some marginal way from a few original WAM
details. However, where our deviations from the original conception are
proposed, an explicit mention will be made and a justification given.

Our motivation to be so conservative is simple: our goal is not to teach the
world how to implement Prolog optimally, nor is it to provide a guide to the
state of the art on the subject. Indeed, having contributed little to the craft
of Prolog implementation, this author claims glaring incompetence for carrying
out such a task. Rather, this work’s intention is to explain in simpler terms,
and justify with informal discussions, David H. D. Warren’s abstract machine
\emph{specifically} and \emph{exclusively}. Our source is what he describes in
\cite{War83, War88}. The expected achievement is merely the long overdue filling
of a gap so far existing for whoever may be curious to acquire \emph{basic}
knowledge of Prolog implementation techniques, as well as to serve as a spring
board for the expert eager to contribute further to this field for which the WAM
is, in fact, just the tip of an iceberg. As such, it is hoped that this
monograph would constitute an interesting and self-contained complement to basic
textbooks for general courses on logic programming, as well as to those on
compiler design for more conventional programming languages. As a stand-alone
work, it could be a quick reference for the computer professional in need of
direct access to WAM concepts.

Our style of teaching the WAM makes a special effort to consider carefully each
feature of the WAM design in isolation by introducing separately and
incrementally distinct aspects of Prolog. This allows us to explain as limpidly
as possible specific principles proper to each. We then stitch and merge the
different patches into larger pieces, introducing independent optimizations one
at a time, converging eventually to the complete WAM design as described in
\cite{War83} or as overviewed in \cite{War88}. Thus, in \ref{warren2}, we
consider unification alone. Then, we look at flat resolution (that is, Prolog
without backtracking) in \ref{warren3}. Following that, we turn to disjunctive
definitions and backtracking in \ref{warren4}. At that point, we will have a
complete, albeit na\"ive, design for pure Prolog. In \ref{warren5}, this
first-cut design will be subjected to a series of transformations aiming at
optimizing its performance, the end product of which is the full WAM. We have
also prepared an index for quick reference to most critical concepts used in the
WAM, something without which no (real) tutorial could possibly be complete.

It is expected that the reader already has a basic understanding of the
operational semantics of \prolog\ --- in particular, of unification and
backtracking. Nevertheless, to make this work also profitable to readers lacking
this background, we have provided a quick summary of the necessary \prolog\
notions in \ref{warrenA}. As for notation, we implicitly use the syntax of
so-called Edinburgh Prolog (see, for instance, \cite{CM84}), which we also
recall in that appendix. Finally, \ref{warrenB} contains a recapitulation of all
explicit definitions implementing the full WAM instruction set and its
architecture so as to serve as a complete and concise summary.

\secrel{1.2.2 Organization of presentation 6}

\secup
\secup