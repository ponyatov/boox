\secrel{Основы Standard ML \copyright\ Michael P. Fourman}\secdown

\url{http://homepages.inf.ed.ac.uk/mfourman/teaching/mlCourse/notes/L01.pdf}

\secrel{Введение}

\ml\ обозначает “MetaLanguage”: МетаЯзык. У Robin Milnerа была идея создания
языка программирования, специально адаптированного для написания приложений для
обработки логических формул и доказательств. Этот язык должен быть метаязыком
для манипуляции объектами, представляющими формулы на логическом объектном
языке.

Первый \ml\ был метаязыком вспомогательного пакета автоматических доказательств
Edinburgh LCF. Оказалось что метаязык Милнера, с некоторыми дополнениями и
уточнениями, стал инновационным и универсальным языком программирования общего
назначения. Today there are many languages that adopt many or all of Milner’s
innovations. Standard ML (SML) is the most direct descendant of the original,
CAML is another, Haskell is a more-distant relative. In this note, we introduce
the SML language, and see how it can be used to compute some interesting results
with very little programming effort.

In the beginning, you were told that a program is a sequence of instructions
to be executed by a computer. This is not true! Giving a sequence of
instructions is just one way of programming a computer. A program is a text
specifying a computation. The degree to which this text can be viewed as a
sequence of instructions varies from programming language to programming
language. In these notes we will write programs in the language ML, which
is not quite so explicit as languages such as C and Pascal about the steps
needed to perform a desired computation. In many ways, ML is simpler than
Pascal and C. However, you may take some time to appreciate this.

ML is primarily a functional language: most ML programs are best viewed
as specifying the values we want to compute, without explicitly describing
the primitive steps used to achieve this. In particular, we will not generally
describe (or be concerned with) the way values stored in particular memory
locations change as the program executes. This will allow us to concentrate
on the organisation of data and computation, without becoming mired in
detailed housekeeping.

ML programs are fundamentally different from those you have become
accustomed to write in a traditional imperative language. It will not be
fruitful to begin by trying to translate between ML and the more-familiar
paradigm — resist this temptation!

We begin this chapter with a quick introduction to a small fragment of
ML. We then use this to investigate some functions that will be useful later
on. Finally, we give an overview of some important aspects of ML.

It is strongly recommended that you try some examples on the computer
as you read this text.\note{User input is terminated with a semicolon, “;”. In most systems, this must be followed
by <return>, to tell the system to send the line to ML. The examples have been run using
Abstract Hardware Limited’s Poly/ML system. The Poly/ML prompt is > — or, if input
is incomplete, \#.}

\secup
