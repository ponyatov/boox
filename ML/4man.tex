\secrel{Основы Standard ML \copyright\ Michael P.
Fourman}\label{fourmanru}\secdown

\url{http://homepages.inf.ed.ac.uk/mfourman/teaching/mlCourse/notes/L01.pdf}

\secrel{Введение}

\ml\ обозначает “MetaLanguage”: МетаЯзык. У Robin Milner была идея создания
языка программирования, специально адаптированного для написания приложений для
обработки логических формул и доказательств. Этот язык должен быть
\emph{метаязыком}\ для манипуляции объектами, представляющими формулы на
логическом \emph{объектном языке}.

Первый \ml\ был метаязыком вспомогательного пакета автоматических доказательств
Edinburgh LCF. Оказалось что метаязык Милнера, с некоторыми дополнениями и
уточнениями, стал инновационным и универсальным языком программирования общего
назначения. Standard ML (SML) является наиболее близким потомком оригинала,
другой\ --- CAML, Haskell является более дальним родственником. В этой статье мы
представляем язык SML, и рассмотрим, как он может быть использован для
вычисления некоторых интересных результатов с очень небольшим усилием по
программированию.

Для начала, вы считаете, что программа представляет собой последовательность
команд, которые будут выполняться компьютером. Это неверно\,! Предоставление
последовательности инструкций является лишь одним из способов программирования
компьютера. Точнее сказать, что \emph{программа\ --- это текст
\term{спецификации вычисления}}. Степень, в которой этот текст можно
рассматривать как последовательность инструкций, изменяется в разных языках
программирования. В этих заметках мы будем писать программы на языке \ml,
который не является столь явно императивным, как такие языки, как Си и Паскаль,
в описании мер, необходимых для выполнения требуемого вычисления. Во многих
отношениях \ml\ \emph{проще}\ чем Паскаль и Си. Тем не менее, вам может
потребоваться некоторое время, чтобы оценить это.

\ml\ в первую очередь функциональный язык: большинство программ на \ml\  лучше
всего рассматривать как спецификацию \emph{значения}, которое мы хотим
вычислить, без явного описания примитивных шагов, необходимых для достижения
этой цели. В частности, мы не будем описывать, и вообще беспокоиться о способе,
каким значения, хранимые где-то в памяти, изменяются по мере выполнения
программы. Это позволит нам сосредоточиться на \emph{организации}\ данных и
вычислений, не втягиваясь в детали внутренней работы самого вычислителя.

В этом программирование на \ml\ коренным образом отличается от тех приемов,
которыми вы привыкли пользоваться в привычном императивном языке.
\emph{Попытки транслировать ваши программисткие привычки на \ml\ неплодотворны\
--- сопротивляйтесь этому искушению\,!}

Мы начнем этот раздел с краткого введения в небольшой фрагмент на \ml. Затем мы
используем этот фрагмент, чтобы исследовать некоторые функции, которые будут
полезны в дальнейшем. Наконец, мы сделаем обзор некоторых важных аспектов \ml.

\emph{Крайне важно}\ пробовать эти примеры на компьютере, когда вы читаете этот
текст.\note{Пользовательский ввод завершается точкой с запятой ``;''.
В большинстве систем, ``;'' должна завершаться нажатием [Enter]/[Return], чтобы
сообщить системе, что надо послать строку в \ml. Эти примеры тестировались на
системе Abstract Hardware Limited’s Poly/ML. В \prog{Poly/ML}\ запрос ввода
символ >\ или, если ввод неполон\ --- \#.}


\bigskip
\paragraph{Примечение переводчика}
Для целей обучения очень удобно использовать онлайн среды, не требующие
установки программ, и достуные в большинстве браузеров на любых мобильных
устройствах. В качестве рекомендуемых online реализаций Standrard ML можно
привести следующие:

\begin{description}
\item{CloudML}\ \url{https://cloudml.blechschmidt.saarland/}\\ описан в
\href{https://blog.blechschmidt.saarland/cloudml-online-standard-ml-editor-and-interpreter/}{блогпосте
B. Blechschmidt}\ как онлайн-интерпретатор диалекта
\href{http://mosml.org/}{Moscow ML}

\item{TutorialsPoint SML/NJ}
\url{http://www.tutorialspoint.com/execute_smlnj_online.php}

\item{Moscow ML}\ (\emph{offline})\ \url{http://mosml.org/}\ реализация
Standart ML
\begin{itemize}[nosep]
  \item Сергей Романенко, Келдышевский институт прикладной математики,
РАН, Москва
\item Claudio Russo, Niels Kokholm, Ken Friis Larsen, Peter Sestoft
\item используется движок и некоторые идеи из Caml Light
\copyright\ Xavier Leroy, Damien Doligez.
\item порт на MacOS \copyright\ Doug Currie.
\end{itemize} 
 
\end{description}

\secup
