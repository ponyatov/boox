\secrel{Синтаксический анализ текстовых данных}\label{syntax}\secdown

\secrel{Универсальный \file{Makefile}}\label{lexmake}

Универсальный Makefile сделан на базе \ref{bimake}, с добавлением переменной
\var{APP}\ указывающий какой пример парсера следуует скомпилировать и выполнить.

Для хранения (и возможной обработки) отпарсенных данных используем ядро языка
\bi\ \ref{bicore}\ --- используем файлы \file{../bi/hpp.hpp}\ и
\file{../bi/cpp.cpp}. Ядро \emph{очень компактно}, но умеет работать со
скалярными, составными и функциональными данными, и содержит минимальную
реализацию \term{ядра динамического языка}.

\lstx{Универсальный \file{Makefile}}{parser/minimal.mk}

\secrel{\cpp\ интерфейс синтаксического анализатора}\label{lexinterface}

\begin{verbatim}
extern int yylex();             // получить код следующиго токена, и yylval.o 
extern int yylineno;            // номер текущей строки файла исходника
extern char* yytext;            // текст распознанного токена, asciiz
#define TOC(C,X) { yylval.o = new C(yytext); return X; }

extern int yyparse();           // отпарсить весь текущий входной поток токенов
extern void yyerror(string);    // callback вызывается при синтаксической ошибке
#include "ypp.tab.hpp"
\end{verbatim}

\secrel{Минимальный парсер}\label{miniparser}

Рассмотрим минимальный парсер, который может анализировать файлы текстовых
данных (например исходники программ), и вычленять из них последовательности
символов, которые можно отнести к \termdef{скалярам}: символ, строка и число.
\note{эти три типа можно назвать атомами computer science}

\lstx{Лексер \file{minimal.lpp}\
/\prog{flex}/}{parser/minimal.lpp}\label{minilexer}

\begin{description}
\item{(../bi/)\file{hpp.hpp}} содержит определения интерфейса лексера
\ref{lexinterface}, и ядра языка \bi\ \ref{bicore}\ для хранения результатов
разбора текстовых данных
\item{\var{noyywrap}} выключает использование функции \var{yywrap()}
\item{\var{yylineno}} включает отслеживание строки исходного файла, используется
при выводе сообщений об ошибках. В минимальном парсере не используется, но
требуется для сборки \bi-ядра.
\item{\verb|%%..%%|} набор правил группировки отдельных символов в
элементы данных\ --- \termdef{токены}{токен}, правила задаются с помощью \term{регулярных
выражений}
\item{\verb|TOC(Sym,SYM)|}\label{minisym} единственное правило, распознающее
любые группы сиволов как класс \class{bi::sym}: латинские буквы, цифры и символы
\_\ и .\ (точка)\note{точка добавлена, так часто используется в именах файлов}
\end{description}

\lstx{Парсер \file{minimal.ypp}\ /\prog{bison}/}{parser/minimal.ypp}

\begin{description}
\item{\file{hpp.hpp}} заголовок аналогичен лексеру \ref{minilexer}
\item{\verb|%defines %union|} указывает какие типы данных могут храниться в
узлах разобранного \termdef{синтаксического дерева}{синтаксическое дерево}.
Поскольку мы используем \bi-ядро, нам будет достаточно пользоваться
только классами языка \bi,
прежде всего универсальным символьным типом AST \ref{ast}\ и его прозводными
классами.
\item{\verb|%token|} описывает токены, которые может возвращать лексер
\ref{minlexer}, причем набор токенов должен быть согласованным между лексером и
парсером\note{определение токенов генерируется в файл \file{ypp.tab.hpp}}
\item{\verb|%type|} описывает типы синтаксических выражений, которые может
распознавать \termdef{грамматика}{грамматика} синтаксического анализатора, 
\item{\verb|REPL|} выражение, описывающее грамматику, аналогичную простейшему
варианту цикла REPL: Read Eval Print
Loop\note{чтение/вычисление/вывод/повторить}. В нашем случае часть вычисления
Eval не выполняется\note{разобранное выражение не вычисляется, хотя
используемое ядро \bi\ и поддерживает такой функционал}, а часть Print
выполняется через метод \verb|Sym.tagval()|, возвращающий котороткую строку
вида \verb|<класс:значение>|\ для найденного токена.
\item{\verb|ex|} (expression) универсальное символьное выражение языка \bi, в
нашем случае оно должно представлять только \verb|scalar|
\item{\verb|scalar|} выражение, представляющиее только распознаваемые скаляры:
\item{\verb|SYM|} символ, 
\item{\verb|STR|} строку \emph{или}
\item{\verb|NUM|} число\note{числа в грамматике языка \bi\ по типам не делятся,
токен соответствует как \class{int}, так и \class{num}}
\end{description}

В качестве тестового исходника возьмем \cpp\ код ядра языка \bi:
\file{../bi/cpp.cpp}:

\lstx{\file{minimal.src}: Тестовый исходник}{parser/minimal.src}

\lstx{\file{minimal.log}: Результат прогона}{parser/minimal.log}

Как видно по логу \file{minimal.log}, все группы сиволов, соответствующих
правилу лексера \verb|SYM|\ref{minisym}, распознались как объекты \bi, остальные
остались символами и попали в лог без изменений.


\secrel{Добавляем обработку комментариев}\label{minicomment}

В тестах программ и файлов конфигурации очень часто используются
\term{комментарии}. В языке \py, \bi\ и UNIX shell комментарием является все
от символа \#\ до конца строки.

Для обработки таких \termdef{строчных комментариев}{строчный комментарий}\
достаточно добавить одно правило лексера, \emph{обязательно первым правилом}:

\lstx{Лексер со строчными комментариями}{parser/comment.lpp}

Группа символов, начинающаяся с символа \#, затем идет ноль или более \verb|[]*|
любых символов не равных \verb|^|\ концу строки \verb|\n|.
Пустое тело правила: \cpp\ код в \verb|{}|\ скобках\ --- выполняется и ничего не
делает.

Тело правила SYM\ --- вызов макроса \verb|TOC(C,X)|, наоборот, при своем
выполнении создает токен, и возвращает код токена \verb|=SYM|.

\lstx{\file{comment.log}: Результат прогона}{parser/comment.log}

Как видно из лога, из вывода исчезли первые 2 строки, начинающиеся на \#, причем
концы этих строк остались (но не были как-либо распознаны).

\secrel{Разбор строк}\label{ministring}

Для разбора строк необходимо использовать лексер с применением
\termdef{состояний}{состояние лексера}. Строки имеют сильно отличающийся от
основного кода синтаксис, и для его обработки нужно \emph{переключать набор
правил лексера}.

\lstx{Лексер с состоянием для строк}{parser/string.lpp}\label{lexstring}

\begin{description}
\item{\verb|string LexString|} строковая буферная переменная, накапливающая
символы строки
\item{\verb|%x lexstring|} создание отдельного состояния лексера
\verb|lexstring|
\item{\verb|INITIAL|} основное состояние лексера
\item{\verb|<lexstring>\n|} правило конца строки позволяет использовать
многострочные строки\note{символ конца строки не распознается
метасимволом . (точка) в регулярном выражении, и требует явного указания}
\item{\verb|<lexstring>.|} любой символ в состоянии \verb|<lexstring>|
\end{description}

\lstx{Лог разбора со строками}{parser/string.log}

Обратите внимание, что ранее попадавшие в лог строки в двойных кавычках, типа
\verb|"]\n\n"|, стали распознаваться как строковые токены \verb|<str:']\n\n'>|.
\note{использованы 'одинарные кавычки' как в \py/\bi}

\secrel{Добавляем операторы}\label{miniops}

Для разбора языков программирования необходима поддержка операторов,
включим общепринятые одиночные операторы, операторы \cpp\ и \bi.
\emph{Скобки различного вида тоже будет рассматривать как операторы.}
Операторы реализованы в ядре \bi\ как отдельный класс \class{op}, зададим
пачку правил разбора операторов, создающих токены \verb|TOC(Op,XXX)|:

\lstx{Лексер с операторами}{parser/ops.lpp}

\lstx{Парсер с операторами}{parser/ops.ypp}

Лог уже стал нечитаем, переключаемся на древовидный вывод через метод
\verb|Sym.dump()|.

\lstx{Разбор с операторами}{parser/ops.log}

\secrel{Обработка вложенных структур (скобок)}

Обработка вложенных структур возможна только парсером, лексер оставляем
без изменений. Хранение вложенных структур в виде дерева\ --- главная фича
типа \bi\ AST\ref{ast}. Заменяем грамматическое выражение \verb|bracket|\ на
отдельные выражения для скобок:

\lstx{Парсер со скобками}{parser/brackets.ypp}

\lstx{Разбор со скобками}{parser/brackets.log}


\secup