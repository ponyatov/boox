\secrel{Разбор строк}\label{ministring}

Для разбора строк необходимо использовать лексер с применением
\termdef{состояний}{состояние лексера}. Строки имеют сильно отличающийся от
основного кода синтаксис, и для его обработки нужно \emph{переключать набор
правил лексера}.

\lstx{Лексер с состоянием для строк}{parser/string.lpp}\label{lexstring}

\begin{description}
\item{\verb|string LexString|} строковая буферная переменная, накапливающая
символы строки
\item{\verb|%x lexstring|} создание отдельного состояния лексера
\verb|lexstring|
\item{\verb|INITIAL|} основное состояние лексера
\item{\verb|<lexstring>\n|} правило конца строки позволяет использовать
многострочные строки\note{символ конца строки не распознается
метасимволом . (точка) в регулярном выражении, и требует явного указания}
\item{\verb|<lexstring>.|} любой символ в состоянии \verb|<lexstring>|
\end{description}

\lstx{Лог разбора со строками}{parser/string.log}

Обратите внимание, что ранее попадавшие в лог строки в двойных кавычках, типа
\verb|"]\n\n"|, стали распознаваться как строковые токены \verb|<str:']\n\n'>|.
\note{использованы 'одинарные кавычки' как в \py/\bi}