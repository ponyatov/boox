\secrel{\prog{os86}: низкоуровневое программирование i386}\secdown\label{os86}

Если вам по каким-то причинам не подходит одна из типовых распространенных ОС,
например требуется сделать систему управления жесткого реального
времени\note{или вы любитель гадить из прикладного ПО в аппаратные порты в обход
всех соглашений и средств защиты ОС}, информация в этом разделе поможет сделать
ОС-поделку для типового Wintel ПК.

\bigskip
Для компиляции кода вам потребуется специально собранный из исходников
кросс-\gnut\ для целевой архитектуры i386\ --- \term{триплет}\
\verb|TARGET=i386-pc-elf|. Процесс сборки подробно описан в отдельном разделе
\ref{cross}.

\bigskip

Для упрощения не будем завязываться на особенности конкретного ПК или
эмулятора \qemu\note{VMWare, VirtualPC}, все они вполне аппаратно совместимы
с любым i386 компьютером в базовой конфигурации, для которого мы и будем
рассматривать примеры кода:

\begin{itemize}[nosep]
  \item \verb|APP=bare| metal программирование, без базовой ОС
  \item \verb|HW=x86| типовой минимальный i386 компьютер 
\end{itemize}

\lst{os86/Makefile}{os86/Makefile}{make}

\secly{\prog{MultiBoot}-загрузчик}

Благодаря усилиям сообщества разработчиков OpenSource была успешно решена одна
из проблем начинающего системного программиста\ --- было создано несколько
универсальных \termdef{загрузчиков}{загрузчик}, берущих на себя заботу о чтении
ядра ОС или bare metal программы, начальную инициализацию оборудования,
включении защищенного режима, и передачу управления вашей ОС.

Чтобы ваша bare metal программа была успешно загружена, она должна удовлетворять
требованиям
\termdef{спецификации MultiBoot}{спецификация MultiBoot}\ \ref{multiboot}
быть слинкована в формат ELF и включать заголовок multiboot.

\secup

\secrel{Cпецификация MultiBoot}\label{multiboot}\secdown

\url{http://www.gnu.org/software/grub/manual/multiboot/multiboot.html}

\bigskip

Этот файл документирует \termdef{Спецификацию Multiboot}{спецификация
MultiBoot}, проект стандарта на последовательность загрузки.
Этот документ имеет редакцию 0.6.96.

Copyright \copyright\ 1995,96 Bryan Ford \email{baford@cs.utah.edu}

Copyright \copyright\ 1995,96 Erich Stefan Boleyn \email{erich@uruk.org}

Copyright \copyright\ 1999,2000,2001,2002,2005,2006,2009 
Free Software Foundation, Inc.

Permission is granted to make and distribute verbatim copies of this manual
provided the copyright notice and this permission notice are preserved on all
copies.

Permission is granted to copy and distribute modified versions of this manual
under the conditions for verbatim copying, provided also that the entire
resulting derived work is distributed under the terms of a permission notice
identical to this one.

Permission is granted to copy and distribute translations of this manual into
another language, under the above conditions for modified versions.

\secrel{Introduction to Multiboot Specification}\secdown

This chapter describes some rough information on the Multiboot Specification.
Note that this is not a part of the specification itself.

\secrel{The background of Multiboot Specification}

Every operating system ever created tends to have its own boot loader.
Installing a new operating system on a machine generally involves installing a
whole new set of boot mechanisms, each with completely different install-time
and boot-time user interfaces. Getting multiple operating systems to coexist
reliably on one machine through typical chaining mechanisms can be a nightmare.
There is little or no choice of boot loaders for a particular operating system\
--- if the one that comes with the operating system doesn't do exactly what you
want, or doesn't work on your machine, you're screwed.

While we may not be able to fix this problem in existing proprietary operating
systems, it shouldn't be too difficult for a few people in the free operating
system communities to put their heads together and solve this problem for the
popular free operating systems. That's what this specification aims for.
Basically, it specifies an interface between a boot loader and a operating
system, such that any complying boot loader should be able to load any complying
operating system. This specification does not specify how boot loaders should
work — only how they must interface with the operating system being loaded.

\secrel{The target architecture}

This specification is primarily targeted at i386 PC, since they are the most
common and have the largest variety of operating systems and boot loaders.
However, to the extent that certain other architectures may need a boot
specification and do not have one already, a variation of this specification,
stripped of the x86-specific details, could be adopted for them as well.

\secrel{The target operating systems}

This specification is targeted toward free 32-bit operating systems that can be
fairly easily modified to support the specification without going through lots
of bureaucratic rigmarole. The particular free operating systems that this
specification is being primarily designed for are Linux, the kernels of FreeBSD
and NetBSD, Mach, and VSTa. It is hoped that other emerging free operating
systems will adopt it from the start, and thus immediately be able to take
advantage of existing boot loaders. It would be nice if proprietary operating
system vendors eventually adopted this specification as well, but that's
probably a pipe dream.

\secrel{Boot sources}

It should be possible to write compliant boot loaders that load the OS image
from a variety of sources, including floppy disk, hard disk, and across a
network.

Disk-based boot loaders may use a variety of techniques to find the relevant OS
image and boot module data on disk, such as by interpretation of specific file
systems\note{e.g. the BSD/Mach boot loader}, using precalculated
\term{blocklists}\note{e.g.
LILO}, loading from a special \term{boot partition}\note{e.g. OS/2}, or even
loading from within another operating system\note{e.g. the 
VSTa boot code, which loads from DOS}.
Similarly, network-based boot loaders could use a variety of network hardware
and protocols.

It is hoped that boot loaders will be created that support multiple loading
mechanisms, increasing their portability, robustness, and user-friendliness.

\secrel{Configure an operating system at boot-time}

It is often necessary for one reason or another for the user to be able to
provide some configuration information to an operating system dynamically at
boot time. While this specification should not dictate how this configuration
information is obtained by the boot loader, it should provide a standard means
for the boot loader to pass such information to the operating system.

\secrel{How to make OS development easier}

OS images should be easy to generate. Ideally, an OS image should simply be an
ordinary 32-bit executable file in whatever file format the operating system
normally uses. It should be possible to \prog{nm} or disassemble OS images just
like normal executables. Specialized tools should not be required to create OS
images in a \emph{special} file format. If this means shifting some work from
the operating system to a boot loader, that is probably appropriate, because all
the memory consumed by the boot loader will typically be made available again
after the boot process is created, whereas every bit of code in the OS image
typically has to remain in memory forever. The operating system should not have
to worry about getting into 32-bit mode initially, because mode switching code
generally needs to be in the boot loader anyway in order to load operating
system data above the 1MB boundary, and forcing the operating system to do this
makes creation of OS images much more difficult.

Unfortunately, there is a horrendous variety of executable file formats even
among free Unix-like pc-based operating systems\ --- generally a different
format for each operating system. Most of the relevant free operating systems
use some variant of a.out format, but some are moving to elf. It is highly
desirable for boot loaders not to have to be able to interpret all the different
types of executable file formats in existence in order to load the OS image\ ---
otherwise the boot loader effectively becomes operating system specific again.

This specification adopts a compromise solution to this problem.
Multiboot-compliant OS images always contain a magic \term{Multiboot header}
(see OS image format\ \ref{osimage}), which allows the boot loader to load the
image without having to understand numerous a.out variants or other executable formats. This magic
header does not need to be at the very beginning of the executable file, so
kernel images can still conform to the local a.out format variant in addition to
being Multiboot-compliant.

\secrel{Boot modules}

Many modern operating system kernels, such as Mach and the microkernel in VSTa,
do not by themselves contain enough mechanism to get the system fully
operational: they require the presence of additional software modules at boot
time in order to access devices, mount file systems, etc. While these additional
modules could be embedded in the main OS image along with the kernel itself, and
the resulting image be split apart manually by the operating system when it
receives control, it is often more flexible, more space-efficient, and more
convenient to the operating system and user if the boot loader can load these
additional modules independently in the first place.

Thus, this specification should provide a standard method for a boot loader to
indicate to the operating system what auxiliary boot modules were loaded, and
where they can be found. Boot loaders don't have to support multiple boot
modules, but they are strongly encouraged to, because some operating systems
will be unable to boot without them.

\secup

\secly{The definitions of terms used through the specification}

\begin{description}

\item[must]
We use the term must, when any boot loader or OS image needs to follow a rule\
--- otherwise, the boot loader or OS image is not Multiboot-compliant.

\item[should]
We use the term should, when any boot loader or OS image is recommended to
follow a rule, but it doesn't need to follow the rule.

\item[may]
We use the term may, when any boot loader or OS image is allowed to follow a
rule.

\item[boot loader]
Whatever program or set of programs loads the image of the final operating
system to be run on the machine. The boot loader may itself consist of several
stages, but that is an implementation detail not relevant to this specification.
Only the final stage of the boot loader\ --- the stage that eventually transfers
control to an operating system\ --- must follow the rules specified in this
document in order to be Multiboot-compliant; earlier boot loader stages may be
designed in whatever way is most convenient.

\item[OS image]
The initial binary image that a boot loader loads into memory and transfers
control to start an operating system. The OS image is typically an executable
containing the operating system kernel.

\item[boot module]
Other auxiliary files that a boot loader loads into memory along with an OS
image, but does not interpret in any way other than passing their locations to
the operating system when it is invoked.

\item[Multiboot-compliant]
A boot loader or an OS image which follows the rules defined as must is
Multiboot-compliant. When this specification specifies a rule as should or may,
a Multiboot-complaint boot loader/OS image doesn't need to follow the rule.

\item[u8]
The type of unsigned 8-bit data. 

\item[u16]
The type of unsigned 16-bit data. Because the target architecture is
little-endian, \term{u16} is coded in \emph{little-endian}.

\item[u32]
The type of unsigned 32-bit data. Because the target architecture is
little-endian, \term{u32} is coded in \emph{little-endian}.

\item[u64]
The type of unsigned 64-bit data. Because the target architecture is
little-endian, \term{u64} is coded in little-endian.

\end{description}

\secrel{The exact definitions of Multiboot Specification}\secdown

There are three main aspects of a boot loader/OS image interface:

\begin{enumerate}[nosep]
  \item 
The format of an OS image as seen by a boot loader.
  \item 
The state of a machine when a boot loader starts an operating system.
  \item 
The format of information passed by a boot loader to an operating system.
\end{enumerate}

\secrel{OS image format}\secdown

An OS image may be an ordinary 32-bit executable file in the standard format for
that particular operating system, except that it may be linked at a non-default
load address to avoid loading on top of the pc's I/O region or other reserved
areas, and of course it should not use shared libraries or other fancy features.

An OS image must contain an additional header called \term{Multiboot header},
besides the headers of the format used by the OS image. The Multiboot header
must be contained completely within the first 8192 bytes of the OS image, and
must be longword (32-bit) aligned. In general, it should come \emph{as early as
possible}, and may be embedded in the beginning of the text segment after the
real executable header.

\secrel{The layout of Multiboot header}\label{mbheader}

The layout of the Multiboot header must be as follows:

\bigskip
\begin{tabular}{l l l l}
Offset	&Type	&Field Name	&Note\\
\hline 
0	&u32	&magic	&required\\ 
4	&u32	&flags	&required\\ 
8	&u32	&checksum	&required\\ 
12	&u32	&header\_addr	&if flags[16] is set\\ 
16	&u32	&load\_addr	&if flags[16] is set\\ 
20	&u32	&load\_end\_addr	&if flags[16] is set\\ 
24	&u32	&bss\_end\_addr	&if flags[16] is set\\ 
28	&u32	&entry\_addr	&if flags[16] is set\\ 
32	&u32	&mode\_type	&if flags[2] is set\\ 
36	&u32	&width	&if flags[2] is set\\ 
40	&u32	&height	&if flags[2] is set\\ 
44	&u32	&depth	&if flags[2] is set\\
\end{tabular}
\bigskip

The fields ‘magic’, ‘flags’ and ‘checksum’ are defined in Header magic fields\
\ref{mbmagic}, the fields ‘header\_addr’, ‘load\_addr’, ‘load\_end\_addr’,
‘bss\_end\_addr’ and ‘entry\_addr’ are defined in Header address fields\
\ref{mbheader}, and the fields ‘mode\_type’, ‘width’, ‘height’ and ‘depth’ are
defined in Header graphics fields\ \ref{mbgraph}.

\secrel{The magic fields of Multiboot header}\label{mbmagic}

\begin{description}

\item[‘magic’]
The field ‘magic’ is the magic number identifying the header, which must be the
hexadecimal value 0x1BADB002.

\item[‘flags’]

The field ‘flags’ specifies features that the OS image requests or requires of
an boot loader. Bits 0-15 indicate requirements; if the boot loader sees any of
these bits set but doesn't understand the flag or can't fulfill the requirements
it indicates for some reason, it must notify the user and fail to load the OS
image. Bits 16-31 indicate optional features; if any bits in this range are set
but the boot loader doesn't understand them, it may simply ignore them and
proceed as usual. Naturally, all as-yet-undefined bits in the ‘flags’ word must
be set to zero in OS images. This way, the ‘flags’ fields serves for version
control as well as simple feature selection.

If bit 0 in the ‘flags’ word is set, then all boot modules loaded along with the
operating system must be aligned on page (4KB) boundaries. Some operating
systems expect to be able to map the pages containing boot modules directly into
a paged address space during startup, and thus need the boot modules to be
page-aligned.

If bit 1 in the ‘flags’ word is set, then information on available memory via at
least the ‘mem\_*’ fields of the Multiboot information structure (see Boot
information format\ \ref{mbbif}) must be included. If the boot loader is capable
of passing a memory map (the ‘mmap\_*’ fields) and one exists, then it may be
included as well.

If bit 2 in the ‘flags’ word is set, information about the video mode table (see
Boot information format\ \ref{mbbif}) must be available to the kernel.

If bit 16 in the ‘flags’ word is set, then the fields at offsets 12-28 in the
Multiboot header are valid, and the boot loader should use them instead of the
fields in the actual executable header to calculate where to load the OS image.
This information does not need to be provided if the kernel image is in elf
format, but it must be provided if the images is in a.out format or in some
other format. Compliant boot loaders must be able to load images that either are
in elf format or contain the load address information embedded in the Multiboot
header; they may also directly support other executable formats, such as
particular a.out variants, but are not required to.

\item[‘checksum’]
The field ‘checksum’ is a 32-bit unsigned value which, when added to the other
magic fields (i.e. ‘magic’ and ‘flags’), must have a 32-bit unsigned sum of
zero.

\end{description}

\secrel{The address fields of Multiboot header}\label{mbaddr}

All of the address fields enabled by flag bit 16 are physical addresses. The meaning of each is as follows:

\begin{description}
\item[header\_addr]
Contains the address corresponding to the beginning of the Multiboot header —
the physical memory location at which the magic value is supposed to be loaded.
This field serves to \emph{synchronize}\ the mapping between OS image offsets
and physical memory addresses.
\item[load\_addr]
Contains the physical address of the beginning of the text segment. The offset
in the OS image file at which to start loading is defined by the offset at which
the header was found, minus (header\_addr - load\_addr). load\_addr must be less
than or equal to header\_addr.
\item[load\_end\_addr]
Contains the physical address of the end of the data segment. (load\_end\_addr -
load\_addr) specifies how much data to load. This implies that the text and data
segments must be consecutive in the OS image; this is true for existing a.out
executable formats. If this field is zero, the boot loader assumes that the text
and data segments occupy the whole OS image file.
\item[bss\_end\_addr]
Contains the physical address of the end of the bss segment. The boot loader
initializes this area to zero, and reserves the memory it occupies to avoid
placing boot modules and other data relevant to the operating system in that
area. If this field is zero, the boot loader assumes that no bss segment is
present.
\item[entry\_addr]
The physical address to which the boot loader should jump in order to start
running the operating system.
\end{description}

\secrel{The graphics fields of Multiboot header}\label{mbgraph}

All of the graphics fields are enabled by flag bit 2. They specify the preferred
graphics mode. Note that that is only a recommended mode by the OS image. If the
mode exists, the boot loader should set it, when the user doesn't specify a mode
explicitly. Otherwise, the boot loader should fall back to a similar mode, if
available.

The meaning of each is as follows:

\begin{description}
\item[mode\_type]
Contains ‘0’ for linear graphics mode or ‘1’ for EGA-standard text mode.
Everything else is reserved for future expansion. Note that the boot loader may
set a text mode, even if this field contains ‘0’.
\item[width]
Contains the number of the columns. This is specified in pixels in a graphics
mode, and in characters in a text mode. The value zero indicates that the OS
image has no preference.
\item[height]
Contains the number of the lines. This is specified in pixels in a graphics
mode, and in characters in a text mode. The value zero indicates that the OS
image has no preference.
\item[depth]
Contains the number of bits per pixel in a graphics mode, and zero in a text
mode. The value zero indicates that the OS image has no
preference.
\end{description}

\secup

\secrel{Machine state}

When the boot loader invokes the 32-bit operating system, the machine must have
the following state:

\begin{description}
\item[‘EAX’]
Must contain the magic value ‘0x2BADB002’; the presence of this value indicates
to the operating system that it was loaded by a Multiboot-compliant boot loader
(e.g. as opposed to another type of boot loader that the operating system can
also be loaded from).
\item[‘EBX’]
Must contain the 32-bit physical address of the Multiboot information structure
provided by the boot loader (see Boot information format).
\item[‘CS’]
Must be a 32-bit read/execute code segment with an offset of ‘0’ and a limit of
‘0xFFFFFFFF’. The exact value is undefined.
\item[‘DS’]
\item[‘ES’]
\item[‘FS’]
\item[‘GS’]
\item[‘SS’]
Must be a 32-bit read/write data segment with an offset of ‘0’ and a limit of
‘0xFFFFFFFF’. The exact values are all undefined.
\item[‘A20 gate’]
Must be enabled. 
\item[‘CR0’]
Bit 31 (PG) must be cleared. Bit 0 (PE) must be set. Other bits are all
undefined.
\item[‘EFLAGS’]
Bit 17 (VM) must be cleared. Bit 9 (IF) must be cleared. Other bits are all
undefined.
\end{description}

All other processor registers and flag bits are undefined. This includes, in
particular:

\begin{description}
\item[‘ESP’]
The OS image must create its own stack as soon as it needs one. 
\item[‘GDTR’]
Even though the segment registers are set up as described above, the ‘GDTR’ may
be invalid, so the OS image must not load any segment registers (even just
reloading the same values!) until it sets up its own ‘GDT’.
\item[‘IDTR’]
The OS image must leave interrupts disabled until it sets up its own IDT.
\end{description}

However, other machine state should be left by the boot loader in normal working
order, i.e. as initialized by the bios (or DOS, if that's what the boot loader
runs from). In other words, the operating system should be able to make bios
calls and such after being loaded, as long as it does not overwrite the bios
data structures before doing so. Also, the boot loader must leave the pic
programmed with the normal bios/DOS values, even if it changed them during the
switch to 32-bit mode.

\secrel{Boot information format}

FIXME: Split this chapter like the chapter “OS image format”.

Upon entry to the operating system, the EBX register contains the physical
address of a Multiboot information data structure, through which the boot loader
communicates vital information to the operating system. The operating system can
use or ignore any parts of the structure as it chooses; all information passed
by the boot loader is advisory only.

The Multiboot information structure and its related substructures may be placed
anywhere in memory by the boot loader (with the exception of the memory reserved
for the kernel and boot modules, of course). It is the operating system's
responsibility to avoid overwriting this memory until it is done using it.

The format of the Multiboot information structure (as defined so far) follows:

\begin{verbatim}
             +-------------------+
     0       | flags             |    (required)
             +-------------------+
     4       | mem_lower         |    (present if flags[0] is set)
     8       | mem_upper         |    (present if flags[0] is set)
             +-------------------+
     12      | boot_device       |    (present if flags[1] is set)
             +-------------------+
     16      | cmdline           |    (present if flags[2] is set)
             +-------------------+
     20      | mods_count        |    (present if flags[3] is set)
     24      | mods_addr         |    (present if flags[3] is set)
             +-------------------+
     28 - 40 | syms              |    (present if flags[4] or
             |                   |                flags[5] is set)
             +-------------------+
     44      | mmap_length       |    (present if flags[6] is set)
     48      | mmap_addr         |    (present if flags[6] is set)
             +-------------------+
     52      | drives_length     |    (present if flags[7] is set)
     56      | drives_addr       |    (present if flags[7] is set)
             +-------------------+
     60      | config_table      |    (present if flags[8] is set)
             +-------------------+
     64      | boot_loader_name  |    (present if flags[9] is set)
             +-------------------+
     68      | apm_table         |    (present if flags[10] is set)
             +-------------------+
     72      | vbe_control_info  |    (present if flags[11] is set)
     76      | vbe_mode_info     |
     80      | vbe_mode          |
     82      | vbe_interface_seg |
     84      | vbe_interface_off |
     86      | vbe_interface_len |
             +-------------------+
\end{verbatim}

The first longword indicates the presence and validity of other fields in the
Multiboot information structure. All as-yet-undefined bits must be set to zero
by the boot loader. Any set bits that the operating system does not understand
should be ignored. Thus, the ‘flags’ field also functions as a version
indicator, allowing the Multiboot information structure to be expanded in the
future without breaking anything.

If bit 0 in the ‘flags’ word is set, then the ‘mem\_*’ fields are valid.
‘mem\_lower’ and ‘mem\_upper’ indicate the amount of lower and upper memory,
respectively, in kilobytes. Lower memory starts at address 0, and upper memory
starts at address 1 megabyte. The maximum possible value for lower memory is 640
kilobytes. The value returned for upper memory is maximally the address of the
first upper memory hole minus 1 megabyte. It is not guaranteed to be this value.

If bit 1 in the ‘flags’ word is set, then the ‘boot\_device’ field is valid, and
indicates which bios disk device the boot loader loaded the OS image from. If
the OS image was not loaded from a bios disk, then this field must not be
present (bit 3 must be clear). The operating system may use this field as a hint
for determining its own root device, but is not required to. The ‘boot\_device’
field is laid out in four one-byte subfields as follows:

\begin{verbatim}
     +-------+-------+-------+-------+
     | part3 | part2 | part1 | drive |
     +-------+-------+-------+-------+
\end{verbatim}

The first byte contains the bios drive number as understood by the bios INT 0x13
low-level disk interface: e.g. 0x00 for the first floppy disk or 0x80 for the
first hard disk.

The three remaining bytes specify the boot partition. ‘part1’ specifies the
top-level partition number, ‘part2’ specifies a sub-partition in the top-level
partition, etc. Partition numbers always start from zero. Unused partition bytes
must be set to 0xFF. For example, if the disk is partitioned using a simple
one-level DOS partitioning scheme, then ‘part1’ contains the DOS partition
number, and ‘part2’ and ‘part3’ are both 0xFF. As another example, if a disk is
partitioned first into DOS partitions, and then one of those DOS partitions is
subdivided into several BSD partitions using BSD's disklabel strategy, then
‘part1’ contains the DOS partition number, ‘part2’ contains the BSD
sub-partition within that DOS partition, and ‘part3’ is 0xFF.

DOS extended partitions are indicated as partition numbers starting from 4 and
increasing, rather than as nested sub-partitions, even though the underlying
disk layout of extended partitions is hierarchical in nature. For example, if
the boot loader boots from the second extended partition on a disk partitioned
in conventional DOS style, then ‘part1’ will be 5, and ‘part2’ and ‘part3’ will
both be 0xFF.

If bit 2 of the ‘flags’ longword is set, the ‘cmdline’ field is valid, and
contains the physical address of the command line to be passed to the kernel.
The command line is a normal C-style zero-terminated string.

If bit 3 of the ‘flags’ is set, then the ‘mods’ fields indicate to the kernel
what boot modules were loaded along with the kernel image, and where they can be
found. ‘mods\_count’ contains the number of modules loaded; ‘mods\_addr’
contains the physical address of the first module structure. ‘mods\_count’ may
be zero, indicating no boot modules were loaded, even if bit 1 of ‘flags’ is
set. Each module structure is formatted as follows:

\begin{verbatim}
             +-------------------+
     0       | mod_start         |
     4       | mod_end           |
             +-------------------+
     8       | string            |
             +-------------------+
     12      | reserved (0)      |
             +-------------------+
\end{verbatim}

The first two fields contain the start and end addresses of the boot module
itself. The ‘string’ field provides an arbitrary string to be associated with
that particular boot module; it is a zero-terminated ASCII string, just like the
kernel command line. The ‘string’ field may be 0 if there is no string
associated with the module. Typically the string might be a command line (e.g.
if the operating system treats boot modules as executable programs), or a
pathname (e.g. if the operating system treats boot modules as files in a file
system), but its exact use is specific to the operating system. The ‘reserved’
field must be set to 0 by the boot loader and ignored by the operating system.

\emph{Caution: Bits 4 \& 5 are mutually exclusive.}

If bit 4 in the ‘flags’ word is set, then the following fields in the Multiboot
information structure starting at byte 28 are valid:

\begin{verbatim}
             +-------------------+
     28      | tabsize           |
     32      | strsize           |
     36      | addr              |
     40      | reserved (0)      |
             +-------------------+
\end{verbatim}

These indicate where the symbol table from an a.out kernel image can be found.
‘addr’ is the physical address of the size (4-byte unsigned long) of an array of
a.out format nlist structures, followed immediately by the array itself, then
the size (4-byte unsigned long) of a set of zero-terminated ascii strings (plus
sizeof(unsigned long) in this case), and finally the set of strings itself.
‘tabsize’ is equal to its size parameter (found at the beginning of the symbol
section), and ‘strsize’ is equal to its size parameter (found at the beginning
of the string section) of the following string table to which the symbol table
refers. Note that ‘tabsize’ may be 0, indicating no symbols, even if bit 4 in
the ‘flags’ word is set.

If bit 5 in the ‘flags’ word is set, then the following fields in the Multiboot
information structure starting at byte 28 are valid:

\begin{verbatim}
             +-------------------+
     28      | num               |
     32      | size              |
     36      | addr              |
     40      | shndx             |
             +-------------------+
\end{verbatim}

These indicate where the section header table from an ELF kernel is, the size of
each entry, number of entries, and the string table used as the index of names.
They correspond to the ‘shdr\_*’ entries (‘shdr\_num’, etc.) in the Executable
and Linkable Format (elf) specification in the program header. All sections are
loaded, and the physical address fields of the elf section header then refer to
where the sections are in memory (refer to the i386 elf documentation for
details as to how to read the section header(s)). Note that ‘shdr\_num’ may be
0, indicating no symbols, even if bit 5 in the ‘flags’ word is set.

If bit 6 in the ‘flags’ word is set, then the ‘mmap\_*’ fields are valid, and
indicate the address and length of a buffer containing a memory map of the
machine provided by the bios. ‘mmap\_addr’ is the address, and ‘mmap\_length’ is
the total size of the buffer. The buffer consists of one or more of the
following size/structure pairs (‘size’ is really used for skipping to the next
pair):

\begin{verbatim}
             +-------------------+
     -4      | size              |
             +-------------------+
     0       | base_addr         |
     8       | length            |
     16      | type              |
             +-------------------+
\end{verbatim}

where ‘size’ is the size of the associated structure in bytes, which can be
greater than the minimum of 20 bytes. ‘base\_addr’ is the starting address.
‘length’ is the size of the memory region in bytes. ‘type’ is the variety of
address range represented, where a value of 1 indicates available ram, and all
other values currently indicated a reserved area.

The map provided is guaranteed to list all standard ram that should be available
for normal use.

If bit 7 in the ‘flags’ is set, then the ‘drives\_*’ fields are valid, and
indicate the address of the physical address of the first drive structure and
the size of drive structures. ‘drives\_addr’ is the address, and
‘drives\_length’ is the total size of drive structures. Note that
‘drives\_length’ may be zero. Each drive structure is formatted as follows:

\begin{verbatim}
             +-------------------+
     0       | size              |
             +-------------------+
     4       | drive_number      |
             +-------------------+
     5       | drive_mode        |
             +-------------------+
     6       | drive_cylinders   |
     8       | drive_heads       |
     9       | drive_sectors     |
             +-------------------+
     10 - xx | drive_ports       |
             +-------------------+
\end{verbatim}

The ‘size’ field specifies the size of this structure. The size varies,
depending on the number of ports. Note that the size may not be equal to (10 + 2
* the number of ports), because of an alignment.

The ‘drive\_number’ field contains the BIOS drive number. The ‘drive\_mode’
field represents the access mode used by the boot loader. Currently, the
following modes are defined:

\begin{description}
\item[‘0’]
CHS mode (traditional cylinder/head/sector addressing mode). 
\item[‘1’]
LBA mode (Logical Block Addressing mode).
\end{description}

The three fields, ‘drive\_cylinders’, ‘drive\_heads’ and ‘drive\_sectors’,
indicate the geometry of the drive detected by the bios. ‘drive\_cylinders’
contains the number of the cylinders. ‘drive\_heads’ contains the number of the
heads. ‘drive\_sectors’ contains the number of the sectors per track.

The ‘drive\_ports’ field contains the array of the I/O ports used for the drive
in the bios code. The array consists of zero or more unsigned two-bytes
integers, and is terminated with zero. Note that the array may contain any
number of I/O ports that are not related to the drive actually (such as dma
controller's ports).

If bit 8 in the ‘flags’ is set, then the ‘config\_table’ field is valid, and
indicates the address of the rom configuration table returned by the GET
CONFIGURATION bios call. If the bios call fails, then the size of the table must
be zero.

If bit 9 in the ‘flags’ is set, the ‘boot\_loader\_name’ field is valid, and
contains the physical address of the name of a boot loader booting the kernel.
The name is a normal C-style zero-terminated string.

If bit 10 in the ‘flags’ is set, the ‘apm\_table’ field is valid, and contains
the physical address of an apm table defined as below:

\secup

\secly{Examples}

\secly{History}

\secly{Index}


\secup