\secrel{\prog{os86}: низкоуровневое программирование i386}\secdown\label{os86}

Если вам по каким-то причинам не подходит одна из типовых распространенных ОС,
например требуется сделать систему управления жесткого реального
времени\note{или вы любитель гадить из прикладного ПО в аппаратные порты в обход
всех соглашений и средств защиты ОС}, информация в этом разделе поможет сделать
ОС-поделку для типового Wintel ПК.

\bigskip
Для компиляции кода вам потребуется специально собранный из исходников
кросс-\gnut\ для целевой архитектуры i386\ --- \term{триплет}\
\verb|TARGET=i386-pc-elf|. Процесс сборки подробно описан в отдельном разделе
\ref{cross}.

\bigskip

Для упрощения не будем завязываться на особенности конкретного ПК или
эмулятора \qemu\note{VMWare, VirtualPC}, все они вполне аппаратно совместимы
с любым i386 компьютером в базовой конфигурации, для которого мы и будем
рассматривать примеры кода:

\begin{itemize}[nosep]
  \item \verb|APP=bare| metal программирование, без базовой ОС
  \item \verb|HW=x86| типовой минимальный i386 компьютер 
\end{itemize}

\lst{os86/Makefile}{os86/Makefile}{make}

\secup