\secrel{Embedded-программирование с использованием \gnut\
\cite{kumar}}\label{kumaru}\secdown \copyright\ Vijay Kumar B.\\
\cp{http://bravegnu.org/gnu-eprog/}

\secrel{Введение}

GNU toolchain широко используется при разработки программного обеспечения для
встраиваемых систем. Этот тип разработки ПО также называют
\termdef{низкоуровневым программированием}, \termdef{standalone}{standalone}\
или \termdef{bare metal}{bare metal}\ программированием (на Си и \cpp).
Написание низкоуровневого кода на \cpp\ добавляет программисту новых проблем,
требующих глубокого понимания инструмента разработчика\ --- \gnut.
Руководства разработчика \gnut\ предоставляют отличную информацию по
инструментарию, но с точки зрения самого \gnut, чем с точки зрения решаемой
проблемы. Поэтому было написано это руководство, в котором будут описаны
типичные проблемы, с которыми сталкивается начинающий разработчик.

Этот учебник стремится занять свое место, объясняя использование \gnut\ с точки
зрения практического использования. Надеемся, что его будет достаточно для
разработчиков, собирающихся освоить и использовать \gnut\ в их embedded
проектах.

В иллюстративных целях была выбрана встроенная система на базе процессорного
ядра ARM, которая эмулируется в пакете \prog{Qemu}. С таким подходом вы сможете
освоить \gnut\ с комфортом на вашем рабочем компьютере, без необходимости
вкладываться в ``физическое''\ железо, и бороться со сложностями с его запуском.
Учебник не стремиться обучить работе с архитектурой ARM, для этого вам нужно
будет воспользоваться дополнительными книгами или онлайн-учебниками типа:

\begin{itemize}[nosep]
  \item ARM Assembler\ \url{http://www.heyrick.co.uk/assembler/}
  \item ARM Assembly Language Programming\
  \url{http://www.arm.com/miscPDFs/9658.pdf}
\end{itemize}

Но для удобства читателя, некоторое множество часто используемых ARM-инструкций
описано в приложении \ref{kumarB}.

\secrel{2. Setting up the ARM Lab}\secdown
\secrel{2.1. Qemu ARM}
\secrel{2.2. Installing Qemu in Debian}
\secrel{2.3. Installing GNU Toolchain for ARM}
\secup
\secrel{3. Hello ARM}\secdown
\secrel{3.1. Building the Binary}
\secrel{3.2. Executing in Qemu}
\secrel{3.3. More Monitor Commands}
\secup
\secrel{4. More Assembler Directives}\secdown
\secrel{4.1. Sum an Array}
\secrel{4.2. String Length}
\secup
\secrel{5. Using RAM}
\secrel{6. Linker}\secdown
\secrel{6.1. Symbol Resolution}
\secrel{6.2. Relocation}
\secup
\secrel{7. Linker Script File}\secdown
\secrel{7.1. Linker Script Example}
\secup
\secrel{8. Data in RAM, Example}\secdown
\secrel{8.1. RAM is Volatile!}
\secrel{8.2. Specifying Load Address}
\secrel{8.3. Copying .data to RAM}
\secup
\secrel{9. Exception Handling}
\secrel{10. C Startup}\secdown
\secrel{10.1. Stack}
\secrel{10.2. Global Variables}
\secrel{10.3. Read-only Data}
\secrel{10.4. Startup Code}
\secup
\secrel{11. Using the C Library}
\secrel{12. Inline Assembly}
\secrel{13. Contributing}
\secrel{14. Credits}\secdown
\secrel{14.1. People}
\secrel{14.2. Tools}
\secup
\secrel{15. Tutorial Copyright}
\secrel{A. ARM Programmer’s Model}
\secrel{B. ARM Instruction Set}\label{kumarB}
\secrel{C. ARM Stacks}
 
\secup 