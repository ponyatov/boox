\secrel{Высшая математика в упражнениях и задачах \cite{danko}}\secdown

В этом разделе будут размещены решения некторых задач из \cite{danko}\ в
``техническом''\ стиле: главное быстрый результат, а не точное
аналитическое решение, поэтому будем использовать системы компьютерной
математики.
Будут рассмотрены приемы применения OpenSource пакетов:

\begin{description}[nosep]
  \item[\maxima] \cite{maxima} символьная математика, аналог
  \prog{MathCAD}
  \item[\octave] \cite{octave} численная математика, аналог \prog{MATLAB}
  \item[\gnuplot] \cite{gnuplot} простейшее средство построения 3D/3D
  графиков
  \item[\wolfram] \url{http://www.wolframalpha.com/}\ бесплатная on-line
  система символьной математики и база знаний
  \item[\py] скриптовый язык программирования, в последнее время получил широкое
  применение в области численных методов, анализа данных и автоматизации,
  чаще всего применяется в комплекте с библиотеками:
\begin{description}[nosep]
\item[\prog{NumPy}] поддержка многомерных массивов (включая матрицы) и 
высокоуровневых математических функций, предназначенных для работы с ними
\item[\prog{SciPy}] библиотека предназначенная для выполнения научных и
инженерных расчётов: поиск минимумов и максимумов функций,
вычисление интегралов функций,
поддержка специальных функций,
обработка сигналов,
обработка изображений,
работа с генетическими алгоритмами,
решение обыкновенных дифференциальных уравнений,\ldots
\item[\prog{SymPy}] библиотека символьной
математики \url{https://en.wikipedia.org/wiki/SymPy}
\item[\prog{Matplotlib}] библиотека на языке программирования Python для
2D/3D визуализации данных.
Получаемые изображения могут быть использованы в качестве
иллюстраций в публикациях.
\end{description}
Подробно с применением \py\ при обработке
данных можно ознакомиться в \url{http://scipy-cookbook.readthedocs.org/}
\end{description}

\bigskip
Также этот раздел можно использовать как пример использования системы верстки
\LaTeX\ для научных публикаций\ --- смотрите \emph{исходные тексты}\ файла
\url{https://github.com/ponyatov/boox/tree/master/math/danko}\file{/danko.tex}. 

\secrel{Аналитическая геометрия на плоскости}\secdown

\secrel{Прямоугольные и полярные координаты}

\paragraph{1. Координаты на прямой. Деление отрезка в данном отношении}

Точку $M$ координатной оси $Ox$, имеющую \termdef{абсциссу}{абсцисса}\ $x$,
обозначают через $M(x)$.

Расстояние $d$ между точками $M_{1}(x_{1})$ и $M_{2}(x_{2})$ оси при любом
расположении точек на оси находятся по формуле:

\begin{equation}
d=|x_{2}-x_{1}|
\end{equation}

Пусть на произвольной прямой задан отрезок $AB$ ( $A$ --- начало отрезка, $B$
--- конец), тогда всякая третья точка $C$ этой прямой делить отрезок $AB$ в
некотором отношении $\lambda$, где $\lambda= \pm AC:CB$. Если отрезки $AC$ и
$CB$ направлены в одну сторону, то $\lambda$ приписывают знак ``плюс''; если же
отрезки $AC$ и $CB$ направлены в противоположные стороны, то $\lambda$
приписывающт знак ``минус''. Иными словами, $\lambda>0$ если точка $C$ лежит
между точками $A$ и $B$; $\lambda < 0$ если точка $C$ лежит вне отрезка $AB$.

Пусть точки $A$ и $B$ лежит на оси $Ox$, тогда \termdef{координата
точки}{координата точки} $C(\bar{x})$, делящей отрезок между точками $A(x_1)$ и
$B(x_2)$ в отношении $\lambda$, находится по формуле:

\begin{equation}
\bar x=\frac{x_1+\lambda x_2}{1+\lambda}
\end{equation}

В частности, при $\lambda=1$ получается формула для координаты середины отрезка:

\begin{equation}
\bar x = \frac{x_1+x_2}{2}
\end{equation}

\paragraph{1.}

Построить на прямой точки $A(3)$, $B(-2)$, $C(0)$, $D(\sqrt{2})$, $E(-3.5)$.

\bigskip\wolfram\bigskip\\
\verb|number line -3,2,0,sqrt(2),-3.5|
\fig{}{math/danko/w_1_1_1.png}{width=0.5\textwidth}

\bigskip\gnuplot
\begin{verbatim}
plot '-' 3 , -2 , 0 , sqrt(2) , -3.5
\end{verbatim}

\secup



\secup