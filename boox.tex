\input{../texheader/mini}
\author{\copyright\ Dmitry Ponyatov \email{dponyatov@gmail.com}}
\title{блогбук}

\begin{document}
\tableofcontents

\bigskip

\secly{Введение}

\prolog\ --- язык декларативного логического программирования. Детально
рассматривая его имя, получаем что это сокращение от \textsc{pro}gramming in
\textsc{log}ic: логическое программирование. Наследие \prolog а включает
исследования в области \term{автоматического доказательства теорем} и других
\term{дедуктивных систем}, разработанных в 1960-70хх гг. \term{Механизм вывода}
\prolog а базируется на принципе разрешения Робинсона (1965) и механизмах вывода
ответов, приложенных Грином (1968). Эти идеи используются вместе с продедурами
линейного разрешения. Процедуры точного целевого линейное разрешения, такие как
методы Kowalski / Kuehner (1971) и Kowalski (1974), дали толчок к разработке
систем логического программирования общего назначения. ``Первым'' \prolog ом был
``Марсельский \prolog'', реализация которого основана на работе Colmerauer
(1970). Первым делательным описанием языка \prolog\ было руководство к
интерпретатору Marseille Prolog (Roussel, 1975). Другим сильным влиянием на
природу этого первого \prolog а была адаптация этого интерпретатора к задачам
\term{обработки натуральных языков}.

\prolog\ является наиболее часто упоминаемым примеров языков программирования
четвертого поколения, которые поддерживают парадигму \emph{декларативного
программирования}. Японский проект Fifth-Generation Computer
Project\note{компьютерный проект пятого поколения}, анонсированный в 1981,
применял \prolog\ как язык разработки, и сосредоватчивал значительные усилия на
языке и его возможностях. Программы в этом учебнике написаны на ``стандартном''
\prolog е Эдинбургского университета\note{University of Edinburgh Prolog}, как
это сделано к классической книге по \prolog у под авторством Clocksin и Mellish
(1981,1992). Другой заметной версией \prolog а является семейство реализаций
\prolog II, которые являются ответсвлениями Марсельского \prologа. Справочник
Giannesini, et.al. (1986) использует версию \prolog II. Есть некоторые различия
бежду этими двумя вариантами \prolog а; часть различий в синтаксисе, и часть в
семантике. Тем не менее, студенты изучавшие одну из версий, впоследствии могут
легко адаптировать к другой.

Цель этого учебника\ --- помочь изучить самые необходимые, базовые концепции
языка \prolog. Примеры программ были особенно аккуратно выбраны для иллюстрации
программирования искуственного интлеллекта на \prolog е. \lisp\ и \prolog\
наиболее часто используемые языки символьного программирования для приложений
искуственного интеллекта. Они часто упоминаются как великолепные языки для
``исследовательского'' и ``прототипного программирования''.

Раздел \ref{fish1} рассматривает среду программирования на \prolog е для
начинающих.

Раздел \ref{fish2} объясняет синтаксис \prolog а и многие аспекты
программирования на нем через реализацию аккуратно выборанных программ-примеров.
Эти примеры организованы так, чтобы студент обучался через реализацию
\prolog-программ ``сверху вниз'' в декларативном стиле.
Были приняты меры к рассмотрению техник программирования на \prolog е, которые
очень важны для курса искуственного интеллекта. Фактически, \emph{этот учебник
может служить удобным, маленьким, кратким введением в применение \prolog а в
приложениях искуственного интеллекта}. Аспекты семантики языка \prolog\
рассмотриваются с самого начала с точки зрения концепции дерева условий
программы, которое используется для определения последовательностей спецификаций
\prolog-программы в абстрактном виде. Автор надеется что этот подход позволит
рассмотреть базовые принципы формальной верификации программ при
программировании на \prolog е. Последняя секция этого раздела приводит пример,
который показывает что \prolog\ может быть эффективно использован для
аккуратной, точной спецификации программных систем, несмотря на его репутацию
трудно документируемого языка, так что \prolog\ легко использовать как средство
прототипирования.

Раздел \ref{fish3}\ рассматривает работу внутренних механизмов \prolog-движка.
Раздел \ref{fish3}\ рекомендуется просмотреть сразу после того, как студент
изучил 2-3 примера программ из раздела \ref{fish2}. Последняя секция этого
радела рассматривает \term{мета-интерпретаторы \prolog а}.

Раздел \ref{fish4}\ дает краткий обзор основных встроенных предикатов, многие из
которых используются в разделе \ref{fish2}..

Раздел \ref{fish5}\ рассматривает разработку программ A*-поиска на \prolog е.
Раздел \ref{fish53}\ содержит программу $\alpha\beta$-поиска для игры 
tic tac toe.

Раздел \ref{fish6}\ представляет уникальное и обширное описание логического
мета-интерпретатора для нормальных логических баз правил.\note{Замечание от
9/4/2006: Я значительно отредактировал этот раздел, и обновил все ссылки на
секции.}

Раздел \ref{fish7}\ предствляет введение во встроенный в \prolog\ генератор
парсеров грамматики, и дает общий обзор приемов, с помощью которых \prolog\
может быть использован для разбора выражений натурального языка (английского).
Также эта секция описывает построение программных интерфейсов, использующих
идеоматически-простые натуральные языки.

Раздел \ref{fish8}\ показывает приемы реализации различных \prolog-прототипов.
Новый раздел \ref{fish84}\ раскрывает интерактивную связку между машиной вывода
\prolog а и Java GUI для игры tic tac toe. Рассмотренная простая модель связки
легко адаптируемая и применима.

Ранние версии частей этого учебника датируются 1988 годом. Вводный материал
изначально использовался для объяснения работы интерпретатора \prolog а,
разработанного автором\note{сейчас недоступен}\ для применения в учебном
процессе. Автор надеется что вводный материал, собранный в форме этого учебника,
может быть очень полезным для студентов, которые хотят быстрое, но при этом
хорошо сбалансированное, введение в программирование на \prolog е.

Для дальнейшего изучения \prolog а можно посоветовать книги
Clocksin и Mellish (1981,1992), O'Keefe (1990), Clocksin (1997, 2003),
или Sterling и Shapiro (1986).

Подробные заметки по истории \prolog\ и обработке натуральных языков с его
использованием содержатся в работе Pereira and Shieber (1987).

\copyright\ Помона, Калифорния\\ 1988-2015


\secrel{ML \& функциональное программирование}\secdown
\secrel{Основы Standard ML \copyright\ Michael P. Fourman}\secdown

\url{http://homepages.inf.ed.ac.uk/mfourman/teaching/mlCourse/notes/L01.pdf}

\secrel{Введение}

\ml\ обозначает “MetaLanguage”: МетаЯзык. У Robin Milnerа была идея создания
языка программирования, специально адаптированного для написания приложений для
обработки логических формул и доказательств. Этот язык должен быть метаязыком
для манипуляции объектами, представляющими формулы на логическом объектном
языке.

Первый \ml\ был метаязыком вспомогательного пакета автоматических доказательств
Edinburgh LCF. Оказалось что метаязык Милнера, с некоторыми дополнениями и
уточнениями, стал инновационным и универсальным языком программирования общего
назначения. Today there are many languages that adopt many or all of Milner’s
innovations. Standard ML (SML) is the most direct descendant of the original,
CAML is another, Haskell is a more-distant relative. In this note, we introduce
the SML language, and see how it can be used to compute some interesting results
with very little programming effort.

In the beginning, you were told that a program is a sequence of instructions
to be executed by a computer. This is not true! Giving a sequence of
instructions is just one way of programming a computer. A program is a text
specifying a computation. The degree to which this text can be viewed as a
sequence of instructions varies from programming language to programming
language. In these notes we will write programs in the language ML, which
is not quite so explicit as languages such as C and Pascal about the steps
needed to perform a desired computation. In many ways, ML is simpler than
Pascal and C. However, you may take some time to appreciate this.

ML is primarily a functional language: most ML programs are best viewed
as specifying the values we want to compute, without explicitly describing
the primitive steps used to achieve this. In particular, we will not generally
describe (or be concerned with) the way values stored in particular memory
locations change as the program executes. This will allow us to concentrate
on the organisation of data and computation, without becoming mired in
detailed housekeeping.

ML programs are fundamentally different from those you have become
accustomed to write in a traditional imperative language. It will not be
fruitful to begin by trying to translate between ML and the more-familiar
paradigm — resist this temptation!

We begin this chapter with a quick introduction to a small fragment of
ML. We then use this to investigate some functions that will be useful later
on. Finally, we give an overview of some important aspects of ML.

It is strongly recommended that you try some examples on the computer
as you read this text.\note{User input is terminated with a semicolon, “;”. In most systems, this must be followed
by <return>, to tell the system to send the line to ML. The examples have been run using
Abstract Hardware Limited’s Poly/ML system. The Poly/ML prompt is > — or, if input
is incomplete, \#.}

\secup

\secrel{Programming in Standard ML'97}\secdown
\addcontentsline{toc}{subsection}{An On-line Tutorial \copyright\ Stephen
Gilmore}

\url{http://homepages.inf.ed.ac.uk/stg/NOTES/}

\bigskip\noindent
\copyright\ Stephen Gilmore\\
Laboratory for Foundations of Computer Science\\
The University of Edinburgh

\secup

\secup

\secrel{Язык \prolog: логическое программирование\\
и искуственный интеллект}\secdown
\secrel{Учебник Фишера}\secdown

\copyright\ J.R.Fisher 's \prolog
Tutorial\ \cp{https://www.cpp.edu/~jrfisher/www/prolog\_tutorial/contents.html}

\bigskip

\secly{Введение}

\prolog\ --- язык декларативного логического программирования. Детально
рассматривая его имя, получаем что это сокращение от \textsc{pro}gramming in
\textsc{log}ic: логическое программирование. Наследие \prolog а включает
исследования в области \term{автоматического доказательства теорем} и других
\term{дедуктивных систем}, разработанных в 1960-70хх гг. \term{Механизм вывода}
\prolog а базируется на принципе разрешения Робинсона (1965) и механизмах вывода
ответов, приложенных Грином (1968). Эти идеи используются вместе с продедурами
линейного разрешения. Процедуры точного целевого линейное разрешения, такие как
методы Kowalski / Kuehner (1971) и Kowalski (1974), дали толчок к разработке
систем логического программирования общего назначения. ``Первым'' \prolog ом был
``Марсельский \prolog'', реализация которого основана на работе Colmerauer
(1970). Первым делательным описанием языка \prolog\ было руководство к
интерпретатору Marseille Prolog (Roussel, 1975). Другим сильным влиянием на
природу этого первого \prolog а была адаптация этого интерпретатора к задачам
\term{обработки натуральных языков}.

\prolog\ является наиболее часто упоминаемым примеров языков программирования
четвертого поколения, которые поддерживают парадигму \emph{декларативного
программирования}. Японский проект Fifth-Generation Computer
Project\note{компьютерный проект пятого поколения}, анонсированный в 1981,
применял \prolog\ как язык разработки, и сосредоватчивал значительные усилия на
языке и его возможностях. Программы в этом учебнике написаны на ``стандартном''
\prolog е Эдинбургского университета\note{University of Edinburgh Prolog}, как
это сделано к классической книге по \prolog у под авторством Clocksin и Mellish
(1981,1992). Другой заметной версией \prolog а является семейство реализаций
\prolog II, которые являются ответсвлениями Марсельского \prologа. Справочник
Giannesini, et.al. (1986) использует версию \prolog II. Есть некоторые различия
бежду этими двумя вариантами \prolog а; часть различий в синтаксисе, и часть в
семантике. Тем не менее, студенты изучавшие одну из версий, впоследствии могут
легко адаптировать к другой.

Цель этого учебника\ --- помочь изучить самые необходимые, базовые концепции
языка \prolog. Примеры программ были особенно аккуратно выбраны для иллюстрации
программирования искуственного интлеллекта на \prolog е. \lisp\ и \prolog\
наиболее часто используемые языки символьного программирования для приложений
искуственного интеллекта. Они часто упоминаются как великолепные языки для
``исследовательского'' и ``прототипного программирования''.

Раздел \ref{fish1} рассматривает среду программирования на \prolog е для
начинающих.

Раздел \ref{fish2} объясняет синтаксис \prolog а и многие аспекты
программирования на нем через реализацию аккуратно выборанных программ-примеров.
Эти примеры организованы так, чтобы студент обучался через реализацию
\prolog-программ ``сверху вниз'' в декларативном стиле.
Были приняты меры к рассмотрению техник программирования на \prolog е, которые
очень важны для курса искуственного интеллекта. Фактически, \emph{этот учебник
может служить удобным, маленьким, кратким введением в применение \prolog а в
приложениях искуственного интеллекта}. Аспекты семантики языка \prolog\
рассмотриваются с самого начала с точки зрения концепции дерева условий
программы, которое используется для определения последовательностей спецификаций
\prolog-программы в абстрактном виде. Автор надеется что этот подход позволит
рассмотреть базовые принципы формальной верификации программ при
программировании на \prolog е. Последняя секция этого раздела приводит пример,
который показывает что \prolog\ может быть эффективно использован для
аккуратной, точной спецификации программных систем, несмотря на его репутацию
трудно документируемого языка, так что \prolog\ легко использовать как средство
прототипирования.

Раздел \ref{fish3}\ рассматривает работу внутренних механизмов \prolog-движка.
Раздел \ref{fish3}\ рекомендуется просмотреть сразу после того, как студент
изучил 2-3 примера программ из раздела \ref{fish2}. Последняя секция этого
радела рассматривает \term{мета-интерпретаторы \prolog а}.

Раздел \ref{fish4}\ дает краткий обзор основных встроенных предикатов, многие из
которых используются в разделе \ref{fish2}..

Раздел \ref{fish5}\ рассматривает разработку программ A*-поиска на \prolog е.
Раздел \ref{fish53}\ содержит программу $\alpha\beta$-поиска для игры 
tic tac toe.

Раздел \ref{fish6}\ представляет уникальное и обширное описание логического
мета-интерпретатора для нормальных логических баз правил.\note{Замечание от
9/4/2006: Я значительно отредактировал этот раздел, и обновил все ссылки на
секции.}

Раздел \ref{fish7}\ предствляет введение во встроенный в \prolog\ генератор
парсеров грамматики, и дает общий обзор приемов, с помощью которых \prolog\
может быть использован для разбора выражений натурального языка (английского).
Также эта секция описывает построение программных интерфейсов, использующих
идеоматически-простые натуральные языки.

Раздел \ref{fish8}\ показывает приемы реализации различных \prolog-прототипов.
Новый раздел \ref{fish84}\ раскрывает интерактивную связку между машиной вывода
\prolog а и Java GUI для игры tic tac toe. Рассмотренная простая модель связки
легко адаптируемая и применима.

Ранние версии частей этого учебника датируются 1988 годом. Вводный материал
изначально использовался для объяснения работы интерпретатора \prolog а,
разработанного автором\note{сейчас недоступен}\ для применения в учебном
процессе. Автор надеется что вводный материал, собранный в форме этого учебника,
может быть очень полезным для студентов, которые хотят быстрое, но при этом
хорошо сбалансированное, введение в программирование на \prolog е.

Для дальнейшего изучения \prolog а можно посоветовать книги
Clocksin и Mellish (1981,1992), O'Keefe (1990), Clocksin (1997, 2003),
или Sterling и Shapiro (1986).

Подробные заметки по истории \prolog\ и обработке натуральных языков с его
использованием содержатся в работе Pereira and Shieber (1987).

\copyright\ Помона, Калифорния\\ 1988-2015


\secrel{Установка и запуск \prolog-системы}\label{fish1}

Примеры этого учебника \prolog а были подготовлены с использованием

\begin{itemize}[nosep]
  \item Quintus Prolog на компьютерах Digital Equipment Corporation MicroVAXes
  (далекая история)
  \item SWI Prolog на Sun Spark (давным давно)
  \item персональных компьютерах c \win
  \item или OS X на Macах  
\end{itemize}
 
Другие заметные \prolog-системы (Borland, XSB, LPA, Minerva \ldots)
использовались для разработки и тестирования последние 25 лет.
В этом учебнике запланирован новый раздел, в котором описано использование
любых \prolog-систем в общем, но пока этот раздел недоступен.

Сайт SWI-Prolog содержит много информации, ссылки на загрузку, и документацию:

\url{http://www.swi-prolog.org/}

Особо следуюет отметить возможность попробовать SWI Prolog on-line без 
регистрации и SMS: \url{http://swish.swi-prolog.org/}. Этот вариант особенно
удобен, так как не требует никакой установки ПО, административных прав, вы
можете работать с этим учебником даже в интернет-кафе.

\bigskip
Примеры в этом учебнике используют упрощенную форму взаимодействия в типичным
\prolog-интерпретатором, так что программы должны работать похоже в любой
\prolog-системе эдинбургского типа или интерактивном компиляторе.

Если в вашей UNIX-системе уже установлен SWI-Prolog, запустите окно
терминала, и начните интерактивную сессию командной:

\begin{verbatim}
user@computer$ swipl
\end{verbatim}

Мы не будем использовать команду запуска именно в такой форме все время:
при запуске могут быть указаны дополниительные параметры командной строки,
которые можно использовать в определенных случаях. Читатель должен расмотреть
эту возможность после освоения базовых приемов работы, чтобы получить больше
возможностей.

Если вы хотите установить SWI Prolog под Debian \linux, выполните команду:

\begin{verbatim}
sudo apt install swi-prolog
\end{verbatim}

\bigskip
Под \win\ инсталлятор SWI-Prolog добавляет иконку запуска
интерпретатора, который вы можете запустить простым двойным целчком по
иконке. При запуске интерпретатор создает свое собственное командное окно.
\bigskip

После запуска интерпретатора обычно появляется сообщение о версии, лицензии и
авторах, а затем выводится приглашение ввода \term{цели}\ типа

\begin{verbatim}
?- _
\end{verbatim}

Интерактивные \term{цели}\ в \prolog е вводятся пользователем за приглашением
\verb|?-|.

Многие \prolog-системы поддерживают предоставление документации по запросу из
командной строки. В SWI Prolog встроена подробная система помощи. Документация
индексирована, и помогает пользователю в процессе работы. Попробуйте ввести

\begin{verbatim}
?- help(help).
\end{verbatim}

Обратите внимание что должна быть введены все символы, и ввод завершен возвратом
каретки.

Для иллюстрации нескольких приемов взимодействия с \prolog ом рассмотрим
следующий пример сессии. Если приведена ссылка на файл, предполагается что это
локальный файл в пользовательском каталоге, который был создан пользователем,
получен копированием из другого публично доступного источника, или получен
сохранением текстового файла из веб-браузера. Способ достижения последнего\ ---
следователь URL-ссылке на файл и сохранить его, или выбрать кусок текста из
онлайн-учебника \prolog а, скопировать его, вставить в текстовый редактор, и
сохранить полученный файл из текстового редактора. Комментарии вида
\verb|/*...*/| после целей используются для описания этих целей.

\lstv{Лог типичной \prolog-сессии}{prolog/fisher/running.pl}


\secrel{2. Sample Programs -- Descriptions}\label{fish2}\secdown
\secrel{2.1 Map colorings} 

\secrel{Два определения факториала}\label{fish22}

Этот раздел вводит в вычисления математических функций используя \prolog.
Обсуждаются различные встроенные арифметические операции. Также обсуждается
концепция derivation дерева, и как derivation деревья связаны с трассировкой в
\prolog е.

В файле \file{2\_2.pl} находятся два определения предикатов, являющиеся
определением фукнции вычисления факториала:

\lst{первый вариант}{prolog/fisher/2_2.pl}{Prolog}
 
Эта программа состоит из двух clauses. Первое заключение\ --- формулировка
\termdef{факта}{факт} (unit clause) \emph{без тела}. Второе заключение\ ---
\termdef{правило}{правило}, так как \emph{у него есть тело}. Тело второго
заключения находится после \verb|:-|, которое можно читать как ``если''. Тело
содержит литералы, разделенные запятыми, каждую запятую можно читать как ``и``.
\termdef{Заголовок правила}{заголовок правила}\ --- весь текст \term{факта} или
часть текста до \verb|:-| в правиле. Рассматривая текст как декларативную
программу, первое (фактическое) предложение читается как ``факториал 0 есть
1''\note{или: 0 и 1 \term{связаны отношением} ``факториал'', но у объектов
одновременно могут быть и другие отношения, например биты(0,1) и целые(0,1)},
и второе предложение заявляет что ``факториал \var{N} есть \var{F}\note{точнее:
N и F связаны отношением ``факториал''} если \verb|N>0| и \var{N1} есть
\verb|N-1| , и факториал \var{N1} есть \var{F1}, и \var{F} есть \verb|N*F1|.

\termdef{\prolog-цель}{цель (Пролог)} (goal) для вычисления факториала от 3
дает ответ в W\ --- \termdef{переменной цели}{переменная цели}:

\begin{verbatim}
?-  factorial(3,W).  
W=6 . 
\end{verbatim}

Рассмотрим следующее clause дерево сконструированное для литерала\\
\verb|factorial(3,W)|. Как описано в предыдущей секции, clause дерево не
содержит никаких свободных переменных, вместо этого включает непосредственно их
значения. Каждое ветвление под узлом определяется clause оригинальной программы,
используя непосредственно вхождения значений переменных; узел задается
заголовком правила, а литералы теля становятся дочерними узлами.

\fig{}{prolog/fisher/f2_2.pdf}{width=0.95\textwidth}

\emph{Все арифметические листья \var|true|} при исполнении\note{в соответствие с
предполагаемой интерпретацией}, и самая нижная связь в дереве соответствует
самому первому clause в программе вычисленяи факториала. Первый clause может
быть записан как:

\begin{verbatim}
factorial(0,1) :- true. 
\end{verbatim}
и фактически \verb|?- true.| \prolog-цель которая всегда успешна
\note{\var{true} встроеннный предикат}. Для краткости, мы не отрисовали
\verb|true| для всех листьев, являющихся арифметическими литералами.

Программное clause дерево показывает значение цели в коорне дерева. Так,\\
\verb'factorial(3,6)' является консеквенцией \prolog-программы, так как
существует clause дерево с корнем \verb'factorial(3,6)', все листья которого
\verb|true|. С другой стороны литерал \verb'factorial(5,2)' не консеквенция,
так как такого дерева для него нет, а значением программы для литерала
\verb'factorial(5,2)' является \verb|false|:

\begin{verbatim}
?- factorial(3,6).  
true .
?- factorial(5,2).  
false . 
\end{verbatim}
как и следовало ожидать. Clause-деревья также называются AND-деревьями, так как
чтобы корень был консеквенцией программы, все его поддеревья также должны быть
консеквенциями. Позже clause деревья будут рассмотрены подробнее. Мы отметили
что \emph{clause дерево описывает семантику (значение) программы}. В разделе
\ref{fish6} мы рассмотрим другой подход к семантике программ. Clause-деревья
предоставляют интуитивный и корректный подход к описанию семантики.

\bigskip

Нам нужно отличать clause деревья программы и \termdef{деревья
вывода}{дерево вывода}. Сlause-деревья статичны, и могут быть нарисованы для
программмы или цели через механизм удовлетворения частичных (под)целей, как
описано выше. Грубо говоря, clause-деревья соответствуют декларативному чтению
программы.

\term{Деревья вывода} наоборот, имеют в виду механизм привязки переменных
\prolog а, и порядок в котором удовлетворяются вложенные частичные цели.
Подробнее деревья вывода описаны в разделе \ref{fish31}, но тем не менее
посмотрите анимацию, предоставляемую динамическим отладчиком, как описано ниже.

\termdef{Трассировка}{трассировка} исполнения \prolog-программы также показывает
как переменные привязываются при удовлетвормении целей. Следующий пример
показывает включение/выключение трассировки в типичной \prolog-системе.

\begin{verbatim}
?- trace. 
% The debugger will first creep -- showing everything (trace). 
 
true .
[trace] 
?- factorial(3,X). 
  (1) 0 Call: factorial(3,_8140) ? [Enter] creep 
  (1) 1 Head [2]: factorial(3,_8140) ? [Enter] creep 
  (2) 1 Call (built-in): 3>0 ?  creep
  (2) 1 Done (built-in): 3>0 ?  creep
  (3) 1 Call (built-in): _8256 is 3-1 ? creep 
  (3) 1 Done (built-in): 2 is 3-1 ?  creep
  (4) 1 Call: factorial(2, _8270) ?  creep
   ... 
  (1) 0 Exit: factorial(3,6) ? 
X=6 .
[trace] 
?- notrace. 
% The debugger is switched off 
 
true .
\end{verbatim}

The animated tree below gives another look at the derivation tree for the
\prolog goal \verb'factorial(3,X)'. To start (or to restart) the animation,
simply click on the \keys{Step} button.

\bigskip

Заголовок этого раздела говорит ``\emph{Два} определения факториала'', вот
второй вариант, использующий три переменых:

\lst{второй вариант}{prolog/fisher/2_2_2.pl}{Prolog}

Для этой версии используйте следующую цель-запрос:

\begin{verbatim}
?- factorial(5,1,F). 
F=120 .
\end{verbatim}

Второй параметр в определении называется \term{параметр-аккумулятор}, который
также хорошо известен в функциональном программировании. Эта версия факториала
определена с использованием \term{хвостовой рекурсии}. Важно чтобы вы выполнили
следующие упражнения:

\paragraph{Упражнение \ref{fish22}.1} Используя первый вариант программы
факториала, четко покажите что не существует clause-дерева с корнем \verb'factorial(5,2)',
имеющего все true листья.

\paragraph{Упражнение \ref{fish22}.2} Нарисуйте clause-дерево для цели
\verb'factorial(3,1,6)' со всеми true-листьями, в виде аналогичном ранее
описанному дереву для \verb'factorial(3,6)'.
Покажите, чем отличаются два варианта программы в процессе вычисления
факториала\,? Также, протрассируйте цель \verb'factorial(3,1,6)' используя
\prolog-систему.


\secrel{2.3 Towers of Hanoi puzzle} 
\secrel{2.4 Loading programs, editing programs} 
\secrel{2.5 Negation as failure} 
\secrel{2.6 Tree data and relations} 
\secrel{2.7 Prolog lists and sequences} 
\secrel{2.8 Change for a dollar} 
\secrel{2.9 Map coloring redux} 
\secrel{2.10 Simple I/O} 
\secrel{2.11 Chess queens challenge puzzle} 
\secrel{2.12 Finding all answers} 
\secrel{2.13 Truth table maker} 
\secrel{2.14 DFA parser} 
\secrel{2.15 Graph structures and paths} 
\secrel{2.16 Search} 
\secrel{2.17 Animal identification game} 
\secrel{2.18 Clauses as data} 
\secrel{2.19 Actions and plans}
\secup
\secrel{3. How Prolog Works}\label{fish3}\secdown
\secrel{3.1 Prolog derivation trees, choices and unification} 
\secrel{3.2 Cut} 
\secrel{3.3 Meta-interpreters in Prolog}
\secup
\secrel{4. Built-in Goals}\label{fish4}\secdown
\secrel{4.1 Utility goals }
\secrel{4.2 Universals (true and fail)} 
\secrel{4.3 Loading Prolog programs} 
\secrel{4.4 Arithmetic goals} 
\secrel{4.5 Testing types} 
\secrel{4.6 Equality of Prolog terms, unification} 
\secrel{4.7 Control} 
\secrel{4.8 Testing for variables} 
\secrel{4.9 Assert and retract} 
\secrel{4.10 Binding a variable to a numerical value} 
\secrel{4.11 Procedural negation, negation as failure }
\secrel{4.12 Input/output} 
\secrel{4.13 Prolog terms and clauses as data} 
\secrel{4.14 Prolog operators} 
\secrel{4.15 Finding all answers}
\secup
\secrel{5. Search in Prolog}\label{fish5}\secdown
\secrel{5.1 The A* algorithm in Prolog} 
\secrel{5.2 The 8-puzzle} 
\secrel{5.3 $\alpha\beta$ search in Prolog}\label{fish53}
\secup
\secrel{6. Logic Topics}\label{fish6}\secdown
\secrel{6.1 Chapter 6 notes} 
\secrel{6.2 Positive logic} 
\secrel{6.3 Convert first-order logic to normal form} 
\secrel{6.4 A normal rulebase goal interpreter} 
\secrel{6.5 Evidentiary soundness and completeness} 
\secrel{6.6 Rule tree visualization using Java}
\secup
\secrel{7. Introduction to Natural Language Processing}\label{fish7}\secdown
\secrel{7.1 Prolog grammar parser generator} 
\secrel{7.2 Prolog grammar for simple English phrase structures} 
\secrel{7.3 Idiomatic natural language command and question interfaces}
\secup
\secrel{8. Prototyping with Prolog}\label{fish8}\secdown
\secrel{8.1 Action specification for a simple calculator} 
\secrel{8.2 Animating the 8-puzzle (\$5.2) using character graphics} 
\secrel{8.3 Animating the blocks mover (\$2.19) using character graphics} 
\secrel{8.4 Java Tic-Tac-Toe GUI plays against Prolog opponent
(\$5.3)}\label{fish84}
\secrel{8.5 Structure diagrams and Prolog}
\secup
\secly{References}
\secup
\secrel{ASTLOG: Язык для анализа синтаксических деревьев}\secdown
\cp{\url{http://www.cs.nyu.edu/~lharris/papers/crew.pdf}}
\copyright\ Roger F. Crew \email{rfc@microsoft.com}\\
Microsoft Research
Microsoft Corporation
Redmond, WA 98052

\secly{Abstract}

We desired a facility for locating/analyzing syntactic artifacts in abstract
syntax trees of \ci/\cpp\ programs, similar to the facility \prog{grep} or
\prog{awk} provides for locating artifacts at the lexical level. \prolog, with
its implicit pattern-matching and backtracking capabilities, is a natural choice
for such an application. We have developed a \prolog\ variant that avoids the
overhead of translating the source syntactic structures into the form of a
\prolog\ database; this is crucial to obtaining acceptable performance on large
programs. An interpreter for this language has been implemented and used find
various kinds of syntactic bugs and other questionable constructs in real
programs like \prog{Microsoft SQL server} (450Klines) and \prog{Microsoft Word}
(2Mlines) in time comparable to the runtime of the actual compiler.

The model in which terms are matched against an implicit current object, rather
than simply proven against a database of facts, leads to a distinct ``inside-out
functional'' programming style that is quite unlike typical \prolog, but one
that is, in fact, well-suited to the examination of trees. Also, various second-order
\prolog\ set-predicates may be implemented via manipulation of the current
object, thus retaining an important feature without entailing that the database
be dynamically extensible as the usual implementation does.

\secrel{Introduction}\secdown

Tools like \prog{grep} and \prog{awk} are useful for finding and analyzing
lexical artifacts; e.g., a one-line command locates all occurences of a
particular string. Unfortunately, many simple facts about programs are less
accessible at the character/token level, such as the locations of assignments to
a particular \cpp\ class member. In general, reliably extracting such syntactic
constructs requires writing a parser or some fragment thereof. And after writing
one's twenty-seventh parser fragment, one might begin to yearn for a more
general tool capable of operating at the syntax-tree level.

Even given a compiler front-end that exposes the abstract syntax tree (AST)
representation for a given program, there remains the question of what exactly
to do with it. To be sure, supplying a \ci\ programmer with a sufficiently
complete interface to this representation generally solves any problem one might
care to pose about it. One may just as easily say that all problems at the
lexical level may be solved via proper use of the UNIX standard IO library
\verb|<stdio.h>|, a true, but utterly trivial and unsatisfying statement. The
question is rather that of building a simpler, more useful and flexible
interface: one that is less error-prone, more concise than writing in \ci, and
more directly suited to the task of exploring ASTs. We first consider a couple
of prior approaches.

\secrel{The \prog{awk} Approach}

One of the more popular approaches is to extend the \prog{awk} \cite{AKW86}
paradigm. An \prog{awk} script is a list of pairs, each being a
regular-expression with an accompanying statement in a C-like imperative
language. For each line in the input file, we consider each pair of the script
in turn; if the regular-expression matches the line, the corresponding statement
is executed.

Extending this to the AST domain is straightforward, though with numerous
variations. One defines a regular-expression-like language in which to express
tree patterns and an \prog{awk}-like imperative language for statements. The
tree nodes of the input program are traversed in some order (e.g., preorder),
and for each node the various pairs of the script are considered as before.

We have two objections to this approach, the first having to do with the
hardwired framework that usually implicit. In some cases (e. g., \prog{TAWK}
\cite{GA96}), the traversal order for the AST nodes is essentially fixed; using
a different order would be analogous to attempting to use plain \prog{awk} to
scan the lines of a text file in reverse order. In $A*$ \cite{LR95}, while the
user may define a general traversal order, only one traversal method may be
defined/active at any given time, making difficult any structure comparisons
between subtrees or other applications that require multiple concurrent
traversals. Since the imperative language is quite general in both cases, little
is deffinitively impossible, however for some applications one may be little
better off than when programming in straight \ci.

The second objection has to do with the kinds of pattern-abstraction available.
Inevitably there exist simply-described patterns that are a poor fit to a
regular-expression-like syntax. This tends to happen when said simple
descriptions are in terms of the idioms of a particular programming language;
most of the various tree-\prog{awk} pattern languages tend to be designed with
the intent of being language independent.

Suppose one wishes to find all consecutive occurrences of one statement
immediately preceding another, e. g., places where a given system call
\verb|syscall();| is followed immediately by an \verb|assert();| \note{on the
theory that testing of outcomes of system calls should be done in production
code rather than just debugging code}. A tree-regular-expression pattern of the
form

\begin{verbatim}
<syscall() pattern>; <assert() pattern>
\end{verbatim}

\noindent
(where \verb|;| is the regular-expression sequence operator) finds all instances
of the two calls occurring consecutively within a single block, but it misses
instances like

\begin{verbatim}
syscall();
{
    assert();
    ...
}
\end{verbatim}

and

\begin{verbatim}
if (...) {
    syscall();
}
else {
    ...
}
assert();
\end{verbatim}

While the tree-\prog{awk} languages allow one to write patterns to match each of
these cases, without a pattern-abstraction facility, we may be back at square
one when it comes time to look for some \emph{different} pair of consecutive
function calls. We prefer to write a single consecutive-statement pattern
constructor \emph{once} and then be able to use it for a variety of cases where
we need to find pairs of consecutive statements satisfying certain criteria,
invoking it as

\begin{verbatim}
follow_stmt(<syscall() pattern>, <assert() pattern>)
\end{verbatim}
for the above problem, or, if we instead want to be fiding all of the places
where a \ci\ switch-case falls through, as
\begin{verbatim}
follow_stmt(not(<unconditional-jump pattern>),
                <case-labeled stmt pattern>)
\end{verbatim}

One solution, used by \prog{TAWK}, is to use \prog{cpp}, the C preprocessor, to
preprocess the script, allowing for pattern-abstractions to be expressed as
\verb|#define| macros whose invocations are then expanded as needed. This is
unsatisfactory in a number of ways, whether one wants to consider the problem of
recursively-defined patterns, macros with large bodies that result in a
corresponding blow-up in the size of the script, or the difficulty of tracing
script errors that resulted from complex macro-expansions.

Another way out is to fall back on the procedural abstraction available in the
imperative language that the patterns invoke. One essentially uses a degenerate
pattern that always matches and then allows the imperative code to test whether
the given node is in fact the desired match, defining functions to test for
particular patterns. Once again, it seems we are back to programming in straight
C and not deriving as much benefit from having a pattern language available as
we could be.

In general, the philosophical underpinning of the \prog{awk} approach is that
the designer has already determined the kinds of searches the user will want to
do; the effort is put towards making those particular searches run efficiently.
There is also an assumption that the underlying imperative language for the
actions has all the abstraction facilities one will ever need, so that if the
pattern language is lacking in various ways, this is not deemed a serious
problem. While this is not an unreasonable approach, we have less confidence of
having identified all of the reasonable search possibilities, and thus would
prefer instead to make the pattern language more flexible and extensible, being
willing to sacrifice some efficiency to do so.

\secrel{The Logic Programming Approach}

Another common approach is to run an inference engine over a database of program
syntactic structures \cite{BCD88, BGV90, CMR92}. \prolog\ \cite{SS86} is a
convenient language for this sort of application. Backtracking and a form of
pattern matching are built in, the abstraction mechanisms to build up complex
predicates exist at a fundamental level, andfinally, \prolog\ allows for a more
declarative programming style.

The problems with using \prolog\ are two-fold. First there is the issue of
efficiency. Second, we must represent the AST for our source program in the
\prolog\ database. Large programs ($10^5..10^6$ lines) will result in
correspondingly large \prolog\ databases, most likely with a significant
performance penalty.

We finesse the second problem by not attempting to import the source program's
AST at all, instead opting to modify the interpretation of the predicates and
queries of \prolog\ so as to be applicable to external objects rather than just
facts provable in the existing database. Removing reasons that require the
database to grow beyond the initial script creates significant opportunities for
optimization. This, however, requires removing primitives like \verb|assert()|
and \verb|retract()| that allow for the dynamic (re)defiition or removal of
predicates, which in turn removes many higher-order logical features that are
defined in terms of them. Fortunately, some of the more essential ones can be
restored at relatively little cost.

\secup

\secrel{Elements of ASTLOG}\secdown

Section \ref{crewsyntax} gives the complete syntax for our language, ASTLOG. The
ASTLOG interpreter reads a script of user-defined predicate operator definitions
and then runs one or more queries.

As in \prolog, the definition of a user-defined predicate operator is composed
of one or more \term{clauses}. A compound term \verb|opname(term,...)| appearing
at top level in a clause body is interpreted as a predicate, whether
\term{opname} be primitive or user-defined. In the latter case, the script is
searched for a defining clause whose head terms successfully unify with the
respective operand terms of the given compound term, variables are bound
accordingly, and the terms of the clause body are likewise interpreted. The
clause \emph{succeeds} (i. e., is found to be true) if all of its body terms
succeed. Whenever a clause head fails to unify, or a clause body term
\emph{fails} (i. e., is found to be false), or any primitive term fails by the
rules of evaluation of that primitive, we backtrack to the last point where
there was a choice (e. g., of clauses to try for a given compound term) and
continue.

A \term{query} is a clause whose head terms are all variables. Ultimately,
whenever all terms of a query body succeed, the bindings of any variables listed
in the query head (\term{qhead}) are reported. Otherwise, we report failure.
Thus far, this is all exactly like \prolog.

\subsecly{Figure 1: Complete Syntax of ASTLOG}\label{crewsyntax} 

\begin{tabular}{l l l}
script & ::= named-clause* & script file syntax \\
query & ::= imports? ( varname* ) clause-body ; & query syntax \\
imports & ::= \{ varname+ \} &\\
named-clause & ::= opname anon-clause &\\
anon-clause & ::= ( term* ) clause-body? ; &\\
clause-body & ::= <- term+ &\\
\end{tabular}

\noindent
\begin{tabular}{l l l}
\hline
&Essential Term Syntax&\\
\hline
term & ::= literal & reference to denotable ob ject \\
& ::= varname &\\
& ::= opname ( term* ) & compound term \\
& ::= FN imports? ( anon-clause+ ) & anonymous predicate-operator-valued 
\\&&(``lambda'') term \\
& ::= ' opname arity-spec? & named predicate-operator-valued
\\&&(``function quote'') term \\
& ::= ( term )( term* ) & ``application'' term \\ 
\end{tabular}

\noindent
\begin{tabular}{l l l}
\hline
&Gratuitous Term Syntax&\\
\hline
&::= \# constname & named constant\\&&($\equiv$ corresponding literal number)\\
&::= \verb|[ term* ]| & \verb|[ ]| $\equiv$ nil(), \verb|[term]| $\equiv$
cons(term; nil()), etc\ldots\\
&::= \verb$[ term+ | term ]$ & \verb$[ term1 | term2 ]$ $\equiv$
cons(term1,term2), etc\ldots\\
arity-spec & ::= / integer &\\ 
\end{tabular}

\secrel{Objects}

ASTLOG refers to external objects. Given a \ci/\cpp\ compiler front end that
provides a (\cpp) interface to the syntactic/semantic data structures built
during the parse of a given program, it is simple to graft this onto the core of
ASTLOG so that it may recognize object references corresponding to
\begin{itemize}[nosep]
  \item whole C/C++ programs,
  \item single files,
  \item symbols,
  \item AST nodes (including statements, expressions, and declarations), and
  \item \ci/\cpp\ type descriptions.
\end{itemize}

For the purposes of ASTLOG, an \term{object} is simply some external entity that
is significant for its identity and for the primitive predicates that it may
satisfy. To simplify the language we regard the traditional
constants (integers, floats, and strings) to be references to ``external''
objects as well, though one could just as easily take the converse view in which
the universe of object references is just a (very large) pool of
constants\note{``atoms'' in the usual \prolog\ terminology}.

In any case, object references are terms in ASTLOG. Only references to equal
objects can unify, equality meaning numeric equality for numbers,
same-sequence-of-characters for strings, and identity for all other classes of
objects. Only objects that have denotations (numbers, strings and the unique
\verb|null object*|) can find their way into scripts.

\secrel{The Current Object}

The first significant departure from the \prolog\ model is that a query or
predicate term always evaluates under an ambient \term{current object}. Every
query and every term being evaluated as a predicate is not so much a standalone
statement that may or may not be intrinsically true (i. e., provable from the
``facts'' in the script) as it is a specification that may or may not be
satisfied by the current ob ject, or, alternatively, a \term{pattern} that may
or may not \term{match} the current object. For example, in \prolog
\begin{verbatim}
odd(3)
\end{verbatim}
always succeeds by virtue of 3 being odd or because the ``fact'' \verb|odd(3)|
exists in the script somewhere. By contrast, in ASTLOG
\begin{verbatim}
odd()
\end{verbatim}
succeeds if the current object happens to be the integer 3, fails if the current
object is 4, and raises an error if the current object is the string
\verb|"Hi mom"|. Another way to view this is that every predicate term takes an
extra, hidden current-object operand.

While one normally only expects to see compound (and application) terms in
predicate position, ASTLOG allows variables and ob ject references there as
well. The rules for matching are as follows:

\begin{itemize}
  \item 
An object reference matches the current object, if it references an equal
object.
  \item 
A bound variable matches according as whatever term it is bound to.
  \item 
An unbound variable gets bound to reference the current ob ject (and thus
automatically matches it).
\item 
A compound term whose operator is defined via clauses matches if there exists a
clause whose head operands unify with the term operands and whose body terms
themselves all match the current object.
\end{itemize}

Section \ref{crew31} describes the operator-valued and application terms.

The evaluation rules for compound terms having primitive operators are widely
varied, however the operands are usually treated one of two ways:

\begin{enumerate}
  \item 
\verb|(foo-pred)| requiring the operand to be match some object\note{which
becomes the current object for that evaluation}, not necessarily the same
current object as that which the full term is being matched against. For
example, the operand of \verb|strlen| (see \ref{crewfig2}) and the second
operand of \verb|with| are treated this way.
  \item 
\verb|(foo)| requiring the operand be an object reference, whether this be a
literal or an object-reference-bound variable. The operands of \verb|re|,
\verb|gt|, and the first operand of \verb|with| are treated this way.
\end{enumerate}

Most primitives also expect a current ob ject to be of a particular kind and
raise an error if confronted with something different.

The use of an implicit current object is not by itself an increase in
expressivity. If we had, in a \prolog\ database, terms representing the various
AST nodes, there would be a fairly straightforward translation of ASTLOG terms
into \prolog\ terms, one in which we simply modify all terms to make the current
object an explicit operand.

Nevertheless, ASTLOG programs exhibit a distinct style of programming. Consider
as an example that we might, in a typical functional language, write a function
call
\begin{verbatim}
strlen(string)
\end{verbatim}
to find the length of the string returned by the expression \verb|string|. Here
the length result is implicitly returned to the context of the call. In \prolog,
the natural style would be to express this as a relation
\begin{verbatim}
strlen(string, length)
\end{verbatim}
which stipulates that \verb|length| is in fact the length of \verb|string|. In
ASTLOG, we would write
\begin{verbatim}
strlen(length-pred)
\end{verbatim}
where now it is the \verb|string| argument that is implicitly supplied \emph{as
the current object} \textit{by} the context while the length result is returned
\textit{to} the subterm \verb|length-pred|, which in turn can be some arbitrary
term expecting a numeric current ob ject as its implicit argument. For example,
given an \verb|odd()| predicate as above, the term \verb|strlen(odd())| would
match any string consisting of an odd number of characters. It is this
``inside-out functional'' evaluation strategy that makes ASTLOG well-suited to
constructing anchored patterns to match tree-like structures.

\subsecly{Figure 2: Some core ASTLOG primitives}\label{crewfig2}

\begin{itemize}
  \item 
\verb|and(object-pred,... )|\\
The current object satisfies every \verb|object-pred| operand. 
  \item 
\verb|or(object-pred,... )|\\
The current object satisfies some \verb|object-pred| operand. 
  \item 
\verb|if(object-pred, then-pred, else-pred)|\\
The current object satisfies \verb|then-pred| or \verb|else-pred| according as
it satisfies or fails to satisfy \verb|object-pred| (once; if \verb|object-pred|
matches but \verb|then-pred| does not, we do not retry \verb|object-pred|). 
  \item 
\verb|not(object-pred)|\\
\verb|= if(object-pred, or(), and())|
  \item 
\verb|with(object, object-pred)|\\
\verb|object| satisfies \verb|object-pred| (outer current object is ignored).
  \item 
\verb|strlen(integer-pred)|\\
The current string object has length satisfying \verb|integer-pred|. 
  \item 
\verb|re(string)|\\
The regular expression \verb|string| matches the current string. 
  \item 
\verb|gt(integer)|\\
The current integer is greater than \verb|integer|. 
  \item 
\verb|minus(integer-pred, integer)|\\
\verb|integer-pred| matches the current integer \verb|+ integer|. 
  \item 
\verb|minus(integer, integer-pred)|\\
\verb|integer-pred| matches \verb|integer|\ --- the current integer.\\
(An error is raised if neither operand of a minus term
is an integer ob ject reference.) 
  \item 
\verb|plus(integer-pred, integer)|\\
\verb|integer-pred| matches the current integer\ --- \verb|integer|

\subsecly{Figure 3: Some primitive node and symbol predicates}
  \item 
\verb|parent(ast-pred)|\\
This AST node is not a root node and its parent satisfies \verb|ast-pred|.
  \item 
\verb|kid(integer-pred; ast-pred)|\\
This AST node has a child satisfying \verb|ast-pred| whose (0-based) index
satisfies \verb|integer-pred|.
  \item 
\verb|kidcount(integer-pred)|\\
The number of children of this AST node satifies \verb|integer-pred|. 
  \item 
\verb|op(integer-pred)|\\
The opcode of this AST node satisfies \verb|integer-pred|. 
  \item 
\verb|atype(type-pred)|\\
This AST node has a return type satisfying \verb|type-pred|. 
  \item 
\verb|asym(symbol-pred)|\\
This AST node is a symbol satisfying \verb|symbol-pred|. 
  \item 
\verb|aconst(const-pred)|\\
This AST node is a constant (integer, float or string) satisfying
\verb|const-pred|.
  \item 
\verb|sname(string-pred)|\\
This symbol's name satisfies \verb|string-pred|. 
\end{itemize}

There are named constants available for designating the opcodes of various kinds
of nodes for use in \verb|op()| terms, and the indices of particular children
for use in \verb|kid()|.

\secrel{Examples}

Given the set of AST node primitives in Figure 3, we could write
\begin{verbatim}
and(op(#=), kid(#LEFT, asym(sname("foo"))))
\end{verbatim}
which would be satisified by any AST node that is an assignment (=) expression
whose left-hand side is itself a symbol expression where the symbol name is
"foo". Here, \verb|#=| and \verb|#LEFT| are numeric literals for the assignment
node opcode and the assignment target's childindex, respectively.

To define a predicate \verb|assignment/2| to match assignment nodes, a script
could include the clause
\begin{verbatim}
assignment(target, value)
    <- op(#=),
        kid(#LEFT, target),
        kid(#RIGHT, value);
\end{verbatim}
which would then allow writing the previous term as
\begin{verbatim}
assignment(asym(sname("foo")), _)
\end{verbatim}

As in \prolog, the underscore \verb|(_)| is ``wild-card'' variable, i.e., one
that is internally given a distinct identity so as not to be conated with any
other instances of \verb|_|. Such a variable, being guaranteed to be unbound,
will match any o ject or unify with any term.

Defining a general purpose node-traversal predicate is also straightforward
\begin{verbatim}
somenode(pred)
    <- or(pred, kid(_ , somenode(pred)));
\end{verbatim}
Given this definition, an attempt to match \verb|somenode(test)| to a given node
will create an instance of the defining clause of \verb|somenode/1| above with
pred bound to \verb|test|. Satisfying the clause body requires that either
\verb|pred| match the current node, or, if (when) that fails, that
\verb|kid(_,somenode(pred))| match the current node. The latter in turn will
attempt to match the variable \verb|_| with 0 (easy) and the term
\verb|somenode(pred)| with the first child, and, when that fails, \verb|_| with
1 and \verb|somenode(pred)| with the second child, etc\ldots Making the
interpreter fail and backtrack after each hit (in the usual manner of \prolog)
eventually causes \verb|test| to be matched with the original node and all of
its descendants.

So, if we issue the query
\begin{verbatim}
(v) <- somenode(
    assignment(asym(sname("foo")), v)
        );
\end{verbatim}
on the root node of some function's AST, we obtain, via the successive bindings
reported for \verb|v| on each hit, all of the expressions assigned to variables
named \var{"foo"} within that function.

As an example that makes less trivial use of backtracking, consider the problem
of whether two trees have the same structure (i.e., root nodes have the same
opcode and all corresponding children have the same structure).
\begin{verbatim}
sametree(node)
    <- op(nodeop),
       with(node, op(nodeop)),
       not(and(with(node, kid(n, nkid)),
           kid(n, not(sametree(nkid)))));
\end{verbatim}

This defines a predicate \verb|sametree(node)| that holds if \verb|node| is a
reference to an AST node with the same structure as the current object. The
first line of the clause body binds the current node's opcode to \verb|nodeop|,
the second line compares that to the opcode of \verb|node|, while the remaining
lines search for children whose subtrees have distinct structure. The term
\verb|kid(n, nkid)| will match each child of \verb|node|, since both variables
are initially unbound. If \verb|sametree(nkid)| happens to be true of the
corresponding child of the current node, the inner \verb|not| fails and we go
back and try another child of \verb|node|. If \verb|sametree(nkid)| happens to
be true of \emph{every} corresponding child of the current node, then the
enclosing \verb|not| and thus the outer \verb|sametree(node)| invocation
succeeds.

The preceding version of \verb|sametree/1| is a purely
structural comparison; there is no attempt to take account
of the commutativity/associativity of the various
operators, e. g., \verb|a + b| and \verb|b + a| are not considered
the same. If, say, we \emph{did} want to consider commutativity,
we could define
\begin{verbatim}
csametree(node)
    <- op(nodeop),
       with(node,op(nodeop)),
       kidcount(if(with(nodeop,commutes()),
                    any_perm(perm),
                    id_perm(perm))),
       not(and(with(node,kid(corresp(perm,n),
                            nkid)),
                kid(n,not(csametree(nkid)))));
\end{verbatim}
along with suitable definitions of
\begin{description}
\item[commutes()]\ \\
the current integer is the opcode of a commutative operator,
\item[any\_perm(perm)]\ \\
\verb|perm| is any permutation of the sequence\\
\verb|0, ... , (<current-object> - 1)|,
\item[id\_perm(perm)]\ \\
\verb|perm| is the identity permutation of the sequence\\
\verb|0, ... , (<current-object> - 1)|,
\item[corresp(perm, n)]\ \\
permutation \verb|perm| takes the current integer to something matching
\verb|n|.
\end{description}
Here, permutations can be represented by list terms. Note that since all of the
commutative \ci/\cpp operators are, in fact, binary, this all simplifies
significantly.

It should, incidentally, be clear that there is nothing about the core language
that is specifically tailored for the examination of compiler-produced ASTs, let
alone ASTs for a given language. The language in fact lends itself to the
examination of a wide variety of external structures, e. g., hierarchical file
systems, or collections of web pages. All that is needed is a suitable
collection of primitive ASTLOG predicates for querying said structures.

\subsecly{Figure 4: Actual ASTLOG code for follow\_stmt}

Actual ASTLOG code for \verb|follow_stmt| and how one uses it to find case
statement fallthroughs. The cond operator is an if-then-elseif- construct, that
is, \verb|cond(p1, e1, p2, e2,..., e)| is equivalent to
\verb|if(p1, e1,if(p2,e2,..,e))|. \verb|sfa(emit(string))| always succeeds and,
as a side-effect, emits the source location of the current AST node in
grep-output form.

\lstx{\file{follow\_stmt.astlog}}{prolog/crew/follow.pl}

\subsecly{Figure 5: Definition of \file{flatten}}\label{crewfig5}

\begin{verbatim}
flatten(test, lst)
  <- flatten(test, lst, []);

flatten(test, head, tail)
  <- if(test,
        first(head, hrest),
        unify(head, hrest)),
     flattenkids(test, 0, hrest, tail);

flattenkids(test, n, head, tail)
  <- if(kid(n, flatten(test, head, mid)),
        and(with(n, minus(nplus1,1)),
            flattenkids(test, nplus1,
                        mid, tail)),
        unify(head, tail));

first([o|rest],rest) <- o;
unify(x,x);
\end{verbatim}

\subsecly{Figure 6: Parameterized version, \file{flatten2}}\label{crewfig6}

\begin{verbatim}
flatten2(test, lst)
  <- flatten2(test, lst, []);

flatten2(test, head, tail)
  <- if((test)(value),
        unify(head, [value|hrest]),
        unify(head, hrest)),
     flatten2kids(test, 0, hrest, tail);

flatten2kids(test, n, head, tail)
  <- if(kid(n, flatten2(test, head, mid)),
        and(with(n, minus(nplus1,1)),
            flatten2kids(test, nplus1,
                         mid, tail)),
        unify(head, tail));

unify(x,x);
\end{verbatim}

\secup %2

\secrel{Higher order features}\secdown %3

We have already included some of the non-1st-order features of \prolog, notably
``cut'' (in the form of \verb|if()|) and the corresponding notion of negation,
\verb|not()|. There are others that turn out to be essential as well.

\secrel{3.1 Lambdas and Applications}\label{crew31}

One may observe that, in \verb|somenode(test)|, because this is an existential
query, it does not matter that we are matching the same term \verb|test| to
every node of the tree. If variables in \verb|test| get bound as a result of
matching a given node, those bindings will be undone prior to advancing to the
next node.

If one instead wants to write a conjunctive predicate over all tree nodes, say
\begin{verbatim}
flatten(test, list)
\end{verbatim}
which holds if \verb|list| is a list of \emph{all} descendant nodes satisfying
\verb|test|,\ --- we give a definition in Figure 5\ --- this will not work
correctly if \verb|test| contains any variables that are bound during the course
of matching any node; said variables will \emph{stay} bound for the duration of
the \verb|flatten| evaluation.

Even in an existential query, there is the possibility that the \verb|test|
being passed in will itself need to take a parameter. For example, one might
imagine defining a version of \verb|sametree| that also requires an additional
user-specified \verb|test| to hold at each corresponding pair of nodes. If
\verb|test| is a mere compound term, it can be matched against one of the nodes,
but not both.

Thus we introduce \termdef{``application'' terms}{application term} and
operator-valued \termdef{``lambda'' terms}{lamdba term}. For an application
\verb|(fterm)(term;...)| to match the current object, the term \verb|fterm| must
be (or be a variable bound to) a predicate-operator-valued term, which will
either be
\begin{itemize}
  \item a reference, \verb|'foo/3| to a named predicate operator,
in which case the application evaluates exactly as the corresponding compound
term would, or
\item an anonymous predicate operator
\verb|FN{importvars ... }(anon-clauses ...)|, in which case the application
evaluates \emph{almost} exactly as if there were a named predicate-operator
defined by the given clauses and this were a compound term on that operator. The
difference is that any variables of those clauses that are also on the
\verb|{importvars... }| list are identified with the correspondingly-named
variables in the clause where the \verb|FN| term occurs lexically.
\end{itemize}

An \verb|FN| term with imports can be thought of as a kind of
\termdef{closure}{closure}.

The parameterized version of flatten, namely
\begin{verbatim}
flatten2(test, list)
\end{verbatim}
which holds iff list is a list of all x corresponding to
descendants that \verb|(test)(x)| matches, is defined in Figure
\ref{crefig6}.

\subsecly{Figure 7: Parameterized version of sametree}\label{crewfig7}
\begin{verbatim}
sametree(node,equiv)
<- unify(same,
FN{same,equiv}
((node)
<- op(nodeop),
with(node,op(nodeop)),
(equiv)(node),
not(and(with(node,kid(n,nkid)),
kid(n,not((same)(nkid))))))),
(same)(node);
\end{verbatim}

The parameterized version of sametree is invoked as
\begin{verbatim}
sametree(node, equiv)
\end{verbatim}
which holds iff node is a reference to an AST node with
the same tree structure as the current node and, for
every descendant n of node, the corresponding node in
the current tree satisfies \verb|equiv(n);| this predicate is defined in Figure
\ref{crewfig7}. This definition demonstrates the use
of import lists, both to define a recursive anonymous
predicate, and to make equiv available at once to all
evaluations of that predicate. Given that definition,
the following
\begin{verbatim}
sametree(node,
FN((n) <- if(aconst(c),
 with(n, aconst(c)),
  and());))
\end{verbatim}
would then test whether the current tree has the same
structure as underneath node and such that all corresponding
constants are the same.

\subsecly{Figure 8: Embedded Query State Primitives}\label{crewfig8}

\begin{description}
\item[query(fterm; query-pred)]\ \\
The embedded query state object created from fterm satisfies query-pred.
\item[qnext(pred; thisquery-pred; nextquery-pred)]\ \\
If the current embedded query state is a failure, pred is true, otherwise the
current object satisfies this query-pred and, after the embedded query is
advanced to the next hit or to failure, the resulting query state satisfies
nextquery-pred.
\item[qget(object-pred;::: )]\ \\
Each object-pred matches the ob ject bound to the corresponding variable in the
head of the embedded query corresponding to the current query state object.
An error will be raised if the embedded query has failed or if any head variable
is not bound to an object.
\end{description}

\secrel{Queries as Objects}

Sometimes one wishes to build a collection or some other kind of aggregate of
all ob jects found by a query. Unfortunately, when backtracking to get to the
next hit, information about the previous hit will generally be lost. One
solution is to rewrite the query into a conjunctive form, as we did in the
previous section converting writing flatten as a conjunctive version of somenode
(see Figure \ref{crewfig5}). We can already see that even in simple cases this
process can be non-trivial and is not readily generalized.

It may also be the case for some conjunctive queries
that they require memory proportional to the size of
the data structure being searched, instead of merely
memory proportional to the depth of the data structure.
Judicious use of if() | astlog's moral equivalent
of the cut operator | can avoid this, but this is
sometimes cumbersome to get right.

As it happens, Prolog provides a number of setpredicates for accumulating query
results. For example,
\begin{verbatim}
bagof(x, term, list)
\end{verbatim}
binds list to a list of the bindings of x corresponding to each instance where
term holds true. Unfortunately, this is usually implemented in terms of assert
and retract, meaning we would have to abandon the idea
of keeping our script small and fixed. Even just adding
this as a new primitive is dubious if we have to add,
say, another new primitive to merely count query hits,
and yet more new primitives for each accumulation
method anyone dreams up.

The key observation is that the execution model of
astlog allows for the possibility of treating some subset
of its own internal structures as ``external'' objects
which can then serve as the current ob ject of various
kinds of queries. To be sure, some care needs to be exercised,
since the internal structures of astlog are not
static the way the program asts are. We can however,
take a query whose hits we wish to accumulate, and
encapsulate its state after a given hit as an astlog
ob ject. Such an embedded query in a given state can
now be the current ob ject for the evaluation of some
other predicate term. We thus only need to provide
suitable primitive predicates applicable to query-state
ob jects that may be used in such a term. Figure \ref{crewfig8} lists
these primitives.

\subsecly{Figure 9: Query Accumulators qcount and qlist}\label{crewfig9}

\begin{verbatim}
qcount(n) <- qcount(0, n);
qcount(sofar, return)
<- qnext(unify(sofar, return),
with(sofar, minus(sofarp1,1)),
qcount(sofarp1, return));
qlist(lst)
<- qnext(unify(lst, []),
qget(first(lst,rest)),
qlist(rest));
// utilities
first([o|rest],rest) <- o;
unify(x,x);
\end{verbatim}

Using this mechanism, it is then possible to define
a wide variety of accumulators of query results. Given
an ast node, and a query to see if there exists a descendant
satisfying \verb|test(x)|
\begin{verbatim}
() <- somenode(test(x));
\end{verbatim}
the corresponding query to count the number of descendants
satisfying test(x) would be
\begin{verbatim}
(n) <- query(FN(() <- somenode(test(x)); ),
 qcount(n));
\end{verbatim}
where qcount/1 is defined as in Figure \ref{crewfig9}. Evaluating the
\verb|query()| term starts an embedded query corresponding to the first operand
and builds a query state ob ject representing the resulting first state (first
hit or failure). This ob ject then becomes the current object to which we try to
match qcount(n). It is the qnext() term therein that does the actual work. If
the query-state is a success state, we increment the count of hits thus far
(sofar), advance the embedded query, and recursively try to match a qcount term
to the new state. If the query-state is a failure, we unify the count of hits
thus far with the return variable.

To build a list of bindings for x corresponding to the
query hits, we can do
\begin{verbatim}
(list) <- query(FN((x) <- somenode(test(x)); ),
 qlist(list));
\end{verbatim}
which is essentially the same as before except that now
qlist(list) uses qget to examine the query state.
Since the embedded query has only one head variable
x, the qget term must likewise have at most one
operand.

Some care is required when using embedded queries
to phrase them so that the head variables will always
be bound to ob jects. qget() will in fact raise an error
if a head variable is not bound to an ob ject. This
requirement is crucial since, with a non-ob ject term,
there is no guarantee that said term will remain intact
when the embedded query backtracks to the next
state. Better to keep terms constructed by an embedded
query from polluting the outer world.

The mechanism is also somewhat impure in that
evaluating a qnext on a given query state ob ject essentially
destroys that ob ject. Subsequent attempts to
match additional terms against that query state will
raise an error since the state of a query is lost once we
advance it.

\secup %3

\secrel{Implementation}

astlog has been implemented as an interpreter in roughly 11,000 lines of \cpp\
for the core astlog interpreter and supporting utilities. Another 1100 lines
define the roughly 60 primitives and supporting structures to invoke the various
functions of the AST library. Coverage of the library API is in not entirely
complete, but it is sufficient to perform various interesting tasks:

\begin{itemize}
  \item 
Finding all instances of a simple assignment expression (=) occurring in any
boolean context, for example,
\begin{verbatim}
if ((major == SORTM)
|| (major == MEMORYM)
|| ((major == BUFFERM)
&& (minor = B_NOIO)))
\end{verbatim}
  \item 
Finding all instances of an equality-test (==) or dereference expression
occurring in any void context (i. e., where results are discarded); the converse
to the previous problem.
  \item 
Finding all case statement fall-throughs, i. e.,
where the preceding statement is not a break. 
  \item 
Finding various patterns of irreducible control-
ow in functions. 
  \item 
Obtaining all static call-graph edges. 
  \item 
Computing the McCabe cyclomatic complexity \cite{McC76} of a function. Our code
to do so looks like
\begin{verbatim}
mccabe(n) <- query(
FN(()<- somenode(
op(or(#IF,#FOR,#DO,
#WHILE,#CASE,
#?,#||,#&&)));)
qcount(minus(n,1))
);
\end{verbatim}
which might be compared with the 17-line version in Aria \cite{DR96}.
Admittedly, fairness would probably entail including the definitions of somenode
and qcount as well.
  \item 
Finding gaps (unused space due to alignment rules) in structure definitions;
this is a matter of traversing \ci\ type structures rather than asts.
\end{itemize}

A typical running time (on a 200MHz Pentium P6 with
64meg of RAM) for a one-pass search that evaluates
a simple predicate on every ast node in Microsoft
\prog{SQLserver} (roughly 450,000 lines, 4300 functions) is
roughly 10 minutes, of which 7.5 minutes are taken up
by the ast library building the actual trees. For Microsoft
Word (roughly 2,000,000 lines) the corresponding
times are 45-60 minutes of which about 30 minutes
is taken up by the tree builder.

Though this dreadfully slow in comparison with
grep, these times are arguably acceptable in comparison
with the times taken by the actual compiler |
what one might expect for a tool that requires the use
of compiler's data structures. One is, of course, free
to write arbitrarily non-linear programs in astlog, so
there are no guarantees. In any case we would doubtless
see a certain amount of speedup if we actually were
to attempt some kind of compilation of the astlog
code.

\secrel{Conclusions and Future Work}

We have described a language for doing syntax-level
analysis for C/C++ programs, though the core language
is, in fact, adaptable to many other kinds of
structures. As with previous such tools, the utility
to users who are thus no longer required to write
their own parse/semantic-analysis phase is apparent.
The contribution here is a pattern language sufficiently
powerful to provide traversal possibilites beyond what
is naturally available in prior awk-like frameworks
while avoiding some of the ineficiencies of importing
the entire program structure into a logical inference
engine. The Pan work \cite{BGV90} stressed the need to
partition code and data; this we have done in a rather
straightforward way. The surprise is that the \prolog\
with-an-ambient-current-object model turns out to be
so well suited to analyzing treelike structures.

To be sure, there are various rough edges:

\begin{enumerate}
  \item 
As already noted, embedded queries are slightly
unsafe; there may exist a more robust set of primitives
to use. Some form of type inference to detect
unsafe uses of qnext may also be worth considering.
More generally, there is the issue of typing
of astlog expressions to reduce the incidence of
unbound variables or ob jects of the wrong type
appearing as operands where ob ject-references of
a particular type are required.

\item
Occasionally, we run up against the generally
cumbersome nature of arithmetic in Prolog, which
is arguably worse in astlog. The ``inside-out
functional'' nature of astlog may be good for
ast patterns, but it can make arithmetic operations
like
\begin{verbatim}
with(n; divide(minus(x; 1); 2))
\end{verbatim}
downright unreadable. Algebraic syntax could help, e. g.,
\begin{verbatim}
with(n; (x - 1)=2) 
\end{verbatim}
but even so, one must stare at this pretty hard to
realize that n is being multiplied by 2 and then
incremented by 1.

One possibility is to complicate the language by
introducing actual ``forward'' functional operator
definitions. For example, with such forward operators
for addition and multiplication, one could
then write
\begin{verbatim}
with(2  n + 1; x)
\end{verbatim}
where the appearance of the + (plus) term in
a slot normally requiring an ob ject reference in- vokes the forward
return-value-to-context definition of the operator + to sum its operands
rather than the usual ``backward'' return-valueto-operand
definition (see Figure 2) in which one
operand is treated as a predicate.

\item
Though there is a surprising amount of mileage to
be had via instantiating terms with unbound variables
in them, there are those occasions when a
genuinely mutable data structure is required. Fortunately,
given the strong partition between the
script/database and the ob jects, having mutable
ob jects exist and primitives that side-effect them
when they match would not disrupt astlog's execution
model.

\item
Currently, new primitives need to be manually
written. Given the current collection of macros
available, this is not actually an arduous task.
Still, while language-independence was not one
of our priorities, given that the core language is
rather language-independent anyway, one would
hope for a more automatic means of adapting
astlog to work with other language parsers, perhaps
by adapting GENII \cite{Dev92} or some similar
tool to generate code for the basic primitive predicate
operators for a fresh language.
 
\end{enumerate}

\secrel{Acknowledgements}

ASTLOG would not have been possible without the existence
of an ast library for C/C++ implemented by
the members of Program Analysis group at Microsoft
Research, particularly Linda O'Gara, David Gay, Erik
Ruf and Bjarne Steensgaarde. I would also like to
thank Bruce Duba, Michael Ernst, Chris Ramming,
and the conference reviewers for much useful commentary
and discussion.

\secly{References}

\begin{description}
\item[AKW86] A. V. Aho, B. W. Kernighan, and P. J.
Weinberger. The AWK Programming Language.
Addison Wesley, Reading, MA,
1986.
\item[BCD88] P. Borras, D. Clement, Th. Despeyroux,
J. Incerpi, G. Kahn, B. Lang, and V. Pascual.
Centaur: The system. In Proceedings
of the SIGSOFT/SIGPLAN Software
Engineering Symposium on Practical Software
Development Environments, Boston,
MA, 1988.
\item[BGV90] Robert A. Ballance, Susan L. Graham,
and Michael L. Van De Vanter. The
pan language-based editing system for integrated
development environments. In Proceedings
of the 4th ACM SIGSOFT Symposium
on Software Development Environments,
pages 77..93, Irvine, CA, 1990.
\item[CMR92] Mariano Consens, Alberto Mendelzon, and
Arthur Ryman. Visualizing and querying
software structures. In Proceedings of the
Fourteenth International ACM Conference
on Software Engineering, pages 138..156,
1992.
\item[Dev92] Premkumar T. Devanbu. Genoa - a
customizable, language-and-front-end independent
code analyzer. In Proceedings of
the Fourteenth International ACM Conference
on Software Engineering, pages 307..319. ACM Press, 1992.
\item[DR96] Premkumar T. Devanbu and David S.
Rosenblum. Generating testing and analysis
tools with aria. ACM Transactions
on Software Engineering and Methodology, 5(1):42..62, January 1996.
\item[GA96] William G. Griswold and Darren C. Atkinson.
Fast, exible syntactic pattern matching
and processing. In Proceedings of the
IEEE Workshop on Program Comprehension.
ACM Press, 1996.
\item[LR95] David A. Ladd and J. Christopher Ramming.
A*: A language for implementing
language processors. IEEE Transactions
on Software Engineering, 21(11):894..901,
November 1995.
\item[McC76] T. McCabe. A complexity measure.
IEEE Transactions on Software Engineering,
2(4):308..320, December 1976.
\item[SS86] Leon Sterling and Ehud Shapiro. The Art
of Prolog: Advanced Programming Techniques.
MIT Press series in logic programming.
The MIT Press, Cambridge, MA,
1986.
\end{description}

\secly{Appendix}

For those who would prefer to see a slightly more formal
description, we include a brief outline of an operational
semantics for astlog in Figure 10, one that
bears some resemblance to the actual implementation.

For any given term that is not an ob ject reference,
one may imagine there being numerous instances of
that term in existence at any given time. We differentiate
the various instances by assigning each a unique
frame identifier (f ) which is only significant for its
identity. A variable v occurring within a given term t
may, for a particular instance hf ; \verb|[[t]]i| of that term,
be bound to some ob ject o or other term instance
\verb|hf 0 ; [[t|
\verb|0 ]]i|, this being indicated by having a binding,
i.e., one of \verb|hf ; [[v]]i fi o or hf ; [[v]]ifihf 0 ; [[t|
\verb|0 ]]i| present
in the current binding stack, which in turn is nothing
more than a list of bindings. The semantic function
\verb|vlookup(B ; hf ; [[t]]i)| returns
\begin{itemize}
  \item 
\verb| hf ; [[t]]i| itself if t is not a variable. 
  \item 
 ? if the variable t is not bound in B. 
  \item 
 o if hf ; \verb|[[t]]i | o is in B
  \item 
 \verb|vlookup(B ; hf 0 ; [[t|
\verb|0 ]]i) if hf ; [[t]]iffhf 0 ; [[t|
\verb|0]]i | is in B.
 \end{itemize}
 
 At any given time, the full state of our abstract machine
is described by a failure of the form B ` C :: F
which consists of

\begin{itemize}
  \item the current binding stack B,
  \item the current continuation C = (o; f ; g; C0), which
in turn consists of a current ob ject o, a current
frame identifier f , a current goal, usually a term,
but this can also be one of the auxiliary goals
``apply(\ldots)'' or ``cut(\ldots),'' and finally another
continuation C0 to which we advance if the goal
succeeds
\item the next failure F , to which we advance if the
current goal fails.
\end{itemize}

Note that unlike the case where the goal succeeds, failure
may involve undoing one or more bindings; thus,
a failure (F ) contains its own binding stack (a subset
of B) whereas the continuations (C, C0) do not.

The bottom half of Figure 10 (partially) defines a
transition relation between states of the abstract machine.
Given an initial current ob ject o and a query
[[query]] with n head variables, we take the initial state
to be

\verb|F0 = [] ` (o; f0; apply(f0; [[query]]; [[v1;:::; vn]]); yes) :: no|
If there is a sequence of transitions
\verb|F0 !|
\verb|B1 ` yes :: F\verb|
then we have a hit and the various query head bindings
are available as \verb|vlookup(B1; hf0; [[vi ]]i) for i = 1 :::n.| Likewise, if
\verb|Fk !|
\verb|Bk ` yes :: Fk+1|
then we have a |(k + 1)th| hit.

When we have \verb|a(k + 1)th| hit.
The semantic function
\verb|mgu(B; f; [[t1;:::;tn]]; f 0 ; [[t\verb|
\verb|01;:::;t0n]])|
returns an augmented binding stack that includes B
together with those additional bindings that make up
the most general unifier of the respective term instances
\verb|hf ; [[t1]]i with hf 0 ; [[t|
\verb|01]]i, etc:::| . If there is no
most general unifier, \verb|mgu()| returns \verb|ufail|.

In the actual implementation, because the script is
unified, we may precompute at load time mgus of all
pairs of same-operator-and-arity compound terms occurring
in the script, making clause invocation no more
expensive than a function call in many cases. We also
omit the ``occurs check'' \cite{SS86} for the run-time portion
of unification (i.e., where we're transitively following
variable bindings), with the usual increase in
speed and infinite-loop risk. Thus far, unification has
played a somewhat smaller role in astlog scripts than
expected, so there's some question whether we need to
be doing even this much.

As noted above ob jects only unify with equal objects.
The idea of allowing an ob ject to unify with
a compound predicate term that matches it has been
considered, but rejected due to the significant complications
it would introduce. Also, once one has subgoals
being attempted during the course of unification, the
user's control over evaluation order is drastically reduced,
something to be avoided if one is interested in
having users being able to write efficient scripts.

\subsecly{Figure 10: Outline of astlog Operational Semantics}\label{crewfig10}

\secup %crew
\secrel{Warren’s Abstract Machine\\Абстрактная машина
Варрена}\label{warren}\secdown

\cp{http://wambook.sourceforge.net/}

\copyright\ Hassan A\"it-Kaci \email{hak@cs.sfu.ca}

\copyright\ David H. D. Warren

\secly{Предисловие к репринтному изданию}

Этот докуент\ --- репринтное издание книги имеющей то же название, которая была
опубликована MIT Press, в 1991 году с кодом ISBN 0-262-51058-8 (мягкая обложка)
and ISBN 0-262-01123-9 (тканый переплет). Редакция книги MIT Press сейчас
не перездается, и права на издание были переданы автору.
Оригинальная версия\note{английская \url{http://wambook.sourceforge.net/}}\
была бесплатно доступна всем, кто хочет ее использовать в некоммерческих целях,
с веб-сайта автора:

\bigskip
\url{http://www.isg.sfu.ca/˜hak/documents/wam.html}
\bigskip

\textit{Сейчас ссылка недоступна, книга пеерехала на
\url{http://wambook.sourceforge.net/}}

\bigskip
Если вы используете ее, пожалуйста дайте мне знать кто вы и для каких целей
хотите ее использовать.

\bigskip
Thank you very much.

\bigskip
Hassan A\"it-Kaci

Burnaby, BC, Canada

May 1997

\secly{Предисловие}

Язык \prolog\ был задуман в начале 1970х Alain Colmerauer a и его коллегами из
Марсельского университета. Его реализация языка была первым практическим
воплощением концепции \term{логического программирования}, предложенной Robert
Kowalski. Ключевая идея логического программирования\ --- вычисления могут быть
выражены в виде конктролируемого вывода (дедукции) из набора декларативных
утверждений. Несмотря на то что эта область значительно развилась за последнее
время, \prolog\ остается наиболее фундаментальным и широко известным языком
логического программирования.

Первой реализацией \prolog а был интерпретатор, написанный на Фортране членами
группы Colmerauer а. Несмотря на очень ущербную в некотором смысле реализацию,
эта версия считается в некотором смысле первым камнем: она доказала
жизнеспособность \prolog а, помогла распространению языка, и заложила основные
принципы реализаций \prolog а. Следующим шагом возможно была \prolog-система для
PDPD-10, разработанная в Университете Эдинбурга мной и коллегами. Эта система
построена на базе техник Марсельской реализации, с добавлением понятия
компиляции \prolog а в низкоуровневый язык (в случае PDP-10 это машинный код), а
также различные техники экономии памяти. Позже я уточнил и абстрагировал
принципы реализации \prolog\ DEC-10 в то, что я называю \prog{WAM} (Warren
Abstract Machine).

\prog{WAM}\ --- абстрактная (виртуальная) машина с архитектурой памяти и набором
команд, заточенных под язык \prolog. Она может быть эффективно реализована на
широком наборе аппаратных архитектур, и служить целевой платформой для
переносимых компиляторов \prolog а. Сейчас она принимается как стандартный базис
при реализации \prolog а. Это конечно лично приятно, но неудобно в том, что WAM
слишком легко принимается как стандарт. Несмотря на то что WAM явилась
результатом длительной работы и большого опыта в реализации \prolog а, это
отнюдь не единственно возможный подход. Например, в то время как WAM применяет
\term{копирование структуры}\note{structure copying}\ для представления
\term{термов}\ \prolog а, метод \term{общих структур}\note{structure
sharing}, использованный в Марсельской и DEC-10 реализациях, все еще можно
рекомендовать к применению. Как бы то ни было, я считаю WAM хорошей отправной
точкой для изучения технологий реализации \prolog-машины.

К сожалению до сих пор не было хорошей книги для ознакомления с внутренним
устройством WAM. Мой оригинальный технический отчет слишком сложен, содержит
только скелетное описание \prolog-машины, и написан для опытного читателя.
Другие работы обсуждают WAM с различных точек зрения, но все же не могут быть
использованы в качестве хорошего вводного руководства.

Поэтому очень приятно видеть появление этого прекрасного учебника, написанного
Hassan A\"it-Kaci. Эту книгу приятно читать. Она объясняет WAM c большой
ясностью и элегантностью. Я думаю что читатели, интересующиеся информатикой,
найдут эту книгу очень стимулирующим введением в увлекательную тему\ ---
реализацию \prolog а. Я очень благодарен Хассану за донесение моей работы до
широкой аудитории.

\bigskip
\copyright\ David H. D. Warren

Бристоль, UK

Февраль 1991

\secrel{1 Введение 3}\secdown

В 1983 году Дэвид Варрэн разработал абстрактную машину для реализации языка
\prolog, содержащую специальную архитектуру памяти и набор инструкций
\cite{War83}. Эта разработка стала известка как Warren Abstract Machine (WAM)
и стала стандартом де-факто для реализаций компиляторов \prolog а. В
\cite{War83} Варрэн описан WAM в минималистичном стиле, который слишком сложен
для понимания неподготовленным читателем, даже заранее знакомым в операциями
\prolog а. Слишком многое было несказанным, и very little is justified in clear
terms\note{David H. D. Warren поделился в частной беседе что он ``чувствовал
что WAM важна, но к деталям ее реализации вряд ли будет широкий интерес, поэтому
он использовал стиль личных заметок''}. Это привело к очень скудному количеству
поклонников WAM, которые могли был похвастаться пониманием деталей ее работы.
Обычно это были реализаторы \prolog а, которые решили уделить необходимое время
для обучения через делание и кропотливого достижения просветления.

\secrel{1.1 Существующая литература 3}

Свидетельством недостатка понимания может служить тот факт, что за первые шесть
лет было крайне мало публикаций о WAM, не говоря о том чтобы формально доказать
ее корректность. Кроме оригинального герметического доклада Варрэна
\cite{War83}, практически не было никаких официальных публикаций о WAM.
Несколько лет спустя группой Аргонской Национальной Лаборатории был выпущен
единственный черновой стандарт \cite{GLLO85}. Но следует отметить что этот
манускрипт был еще менее понятен, чем оригинальный отчет Варрэна. Его
недостатком была цель описать готовую WAM как есть, а не как пошагово
трансформируемый и оптимизируемый проект.

Стиль пошагового улучшения фактически был использован в публикации David Maier и
David S. Warren\note{Это другой человек, а не разработчик WAM, работа которого
вдохновила S.Warren на исследования. В свою очередь достаточно интересно что
David H. D. Warren позже работал над параллельной архитектурой реализации
\prolog а, поддерживая некоторые идеи, независимо предложенные David S.
Warren.}\ \cite{MW88}. В этой работе можно найти описание техник компиляции
\prolog а похожие на принципы WAM\note{chap.9}. Тем не менее мы считаем что
эта похвальная попытка все еще страдает от нескольких недостатков, если его
рассматривать как окончательный учебник.
Прежде всего эта работа описывает собственный достаточно близкий вариант WAM, но
строго говоря не ее саму. Так что описаны не все особенности WAM.
Более того, объяснения ограничены иллюстративными примерами, и редко четко и
исчерпывающие очерчивают контекст, в котором применяются некоторые оптимизации.
Во-вторых, часть посвященная компиляции \prolog а, идет очень поздно\ --- в
предпоследней главе, полагаясь в деталях реализации на свердетализированные
процедуры на Паскакле, и структуры данных, последовательно улчшаемые в течение
предыдущих разделов. Мы чувствуем что это уводит и запутывает читателя,
интересующегося абстрактной машиной. Наконец, несмотря на то что публикация
содержит серию последовательно улучшаемых вариантов реализации, этот учебник не
отделяет независимые части \prolog а в процессе. Все представленные версии\ ---
полные \prolog-машины. В результате, читатель интересующися выбором и сравнением
отдельных техник, которые он хочет применить, не может различить отдельные
техники в тексте. По всей справедливости, книга Майера и С.Варрэна имеет амбиции
быть первой книгой по логическому программирования. Так что они совершили
подвиг, охватывая так много материала, как теоретического так и практического, и
даже включили техники компиляции \prolog а. Более важно, что их книга была
первой доступной официальной публикацией, содержащей реальный учебник по
техникам WAM.

After the preliminary version of this book had been completed, another recent
publication containing a tutorial on the WAM was brought to this author’s
attention. It is a book due to Patrice Boizumault \cite{Boi88} whose Chapter 9
is devoted to explaining the WAM. There again, its author does not use a gradual
presentation of partial \prolog\ machines. Besides, it is written in French\ ---
a somewhat restrictive trait as far as its readership is concerned. Still,
Boizumault’s book is very well conceived, and contains a detailed discussion
describing an explicit implementation technique for the \var{freeze}
meta-predicate\note{chap.10}.

Even more recently, a formal verification of the correctness of a slight
simplification of the WAM was carried out by David Russinoff \cite{Rus89}. That
work deserves justified praise as it methodically certifies correctness of most
of the WAM with respect to \prolog’s SLD resolution semantics. However, it is
definitely not a tutorial, although Russinoff defines most of the notions he
uses in order to keep his work self-contained. In spite of this effort,
understanding the details is considerably impeded without working familiarity
with the WAM as a prerequisite. At any rate, Russinoff’s contribution is
nevertheless a \emph{premi\`ere} as he is the first to establish rigorously
something that had been taken for granted thus far. Needless to say, that report
is not for the fainthearted.
 
\secrel{1.2 Этот учебник 5}\secdown

\secrel{1.2.1 Disclaimer and motivation 5}

The length of this monography has been kept deliberately short. Indeed, this
author feels that the typical expected reader of a tutorial on the WAM would
wish to get to the heart of the matter quickly and obtain complete but short
answers to questions. Also, for reasons pertaining to the specificity of the
topic covered, it was purposefully decided not to structure it as a real
textbook, with abundant exercises and lengthy comments. Our point is to make the
WAM explicit as it was conceived by David H. D. Warren and to justify its
workings to the reader with convincing, albeit informal, explanations. The few
proposed exercises are meant more as an aid for understanding than as food for
further thoughts.

The reader may find, at points, that some design decisions, clearly correct as
they may be, appear arbitrarily chosen among potentially many other
alternatives, some of which he or she might favor over what is described. Also,
one may feel that this or that detail could be ``simplified'' in some local or
global way. Regarding this, we wish to underscore two points: (1) we chose to
follow Warren’s original design and terminology, describing what he did as
faithfully as possible; and, (2) we warn against the casual thinking up of
alterations that, although that may appear to be “smarter” from a local
standpoint, will generally bear subtle global consequences interfering with
other decisions or optimizations made elsewhere in the design.
This being said, we did depart in some marginal way from a few original WAM
details. However, where our deviations from the original conception are
proposed, an explicit mention will be made and a justification given.

Our motivation to be so conservative is simple: our goal is not to teach the
world how to implement Prolog optimally, nor is it to provide a guide to the
state of the art on the subject. Indeed, having contributed little to the craft
of Prolog implementation, this author claims glaring incompetence for carrying
out such a task. Rather, this work’s intention is to explain in simpler terms,
and justify with informal discussions, David H. D. Warren’s abstract machine
\emph{specifically} and \emph{exclusively}. Our source is what he describes in
\cite{War83, War88}. The expected achievement is merely the long overdue filling
of a gap so far existing for whoever may be curious to acquire \emph{basic}
knowledge of Prolog implementation techniques, as well as to serve as a spring
board for the expert eager to contribute further to this field for which the WAM
is, in fact, just the tip of an iceberg. As such, it is hoped that this
monograph would constitute an interesting and self-contained complement to basic
textbooks for general courses on logic programming, as well as to those on
compiler design for more conventional programming languages. As a stand-alone
work, it could be a quick reference for the computer professional in need of
direct access to WAM concepts.

\secrel{1.2.2 Organization of presentation 6}

Our style of teaching the WAM makes a special effort to consider carefully each
feature of the WAM design in isolation by introducing separately and
incrementally distinct aspects of Prolog. This allows us to explain as limpidly
as possible specific principles proper to each. We then stitch and merge the
different patches into larger pieces, introducing independent optimizations one
at a time, converging eventually to the complete WAM design as described in
\cite{War83} or as overviewed in \cite{War88}. Thus, in \ref{warren2}, we
consider unification alone. Then, we look at flat resolution (that is, Prolog
without backtracking) in \ref{warren3}. Following that, we turn to disjunctive
definitions and backtracking in \ref{warren4}. At that point, we will have a
complete, albeit na\"ive, design for pure Prolog. In \ref{warren5}, this
first-cut design will be subjected to a series of transformations aiming at
optimizing its performance, the end product of which is the full WAM. We have
also prepared an index for quick reference to most critical concepts used in the
WAM, something without which no (real) tutorial could possibly be complete.

It is expected that the reader already has a basic understanding of the
operational semantics of \prolog\ --- in particular, of unification and
backtracking. Nevertheless, to make this work also profitable to readers lacking
this background, we have provided a quick summary of the necessary \prolog\
notions in \ref{warrenA}. As for notation, we implicitly use the syntax of
so-called Edinburgh Prolog (see, for instance, \cite{CM84}), which we also
recall in that appendix. Finally, \ref{warrenB} contains a recapitulation of all
explicit definitions implementing the full WAM instruction set and its
architecture so as to serve as a complete and concise summary.

\secup
\secup
\secrel{Унификация\ --- ясно и просто}\label{warren2}\secdown

Напомним что терм (первого порядка)\ ---
\termdef{переменная}{\prolog!переменная} (задается большой буквой в начале
имени), \termdef{константа}{\prolog!константа} (задается маленькой буквой в
начале имени) или \termdef{терм}{\prolog!терм}\ --- структура вида
$f(t_1,\ldots,t_n)$, где $f$ символ называемый
\termdef{функтором}{\prolog|функтор} (записывается аналогично константе, с
маленькой буквы), а элементы $t_i$ тоже термы первого порядка\ ---
\termdef{субтермы}{\prolog!субтерм}. Число субтермов для данного функтора
предопределено, и называется \termdef{\'{а}рностью}{\prolog!арность} функтора.
Для обеспечения возможности использовать один и тот же символ с разной арностью,
мы должны использовать запись $f/n$, что обозначает функтор $f$ с арностью $n$.
Таким образом, два функтора равны только в том случае, если они имеют
\emph{одинаковые символ $f$ и арность $n$}. Разрешая случай $n=0$ можно
рассматривать константу как особый случай терма: константе $c$ соответствует
функтор $c/0$ с нулевой арностью.

Мы рассмотрим очень простой низкоуровневый\note{IL\--- Intermediate Language}\
язык $\mathcal L_0$. На этом языке мы можем описать два вида объектов:
\termdef{терм программы}{\prolog!терм программы} и \termdef{терм
запроса}{\prolog!терм запроса}. Оба этих вида запросов являются термами первого
порядка, но не переменными. Семантика $\mathcal L_0$ равносильна вычислению
самого общего унификатора программы или запроса. Что касается синтаксиса,
$\mathcal L_0$ будет описывать программу как \verb|t| и запрос как \verb$?-t$
где \verb$t$ является термом. Область видимости переменных ограничена термом
программы/запроса. Таким образом, \emph{значение программы не зависит от имен ее
переменных}. Интерпретатор для $\mathcal L_0$ будет использовать определенное
представление данных для термов и использовать алгоритм унификации для ее
операционной семантики. Затем мы опишем $\mathcal M_0 = (\mathcal D_0,\mathcal
I_0)$ , дизайн абстрактной машины для $\mathcal L_0$ содержащий представление
данных $\mathcal D_0$, над которыми выполняется множество $\mathcal I_0$
машинных инструкций.

Идея достаточно проста: имея определенных программный терм $p$, мы можем
выполнить лююбой запрос \verb|?-q|, и выполнение запроса завершится с ошибкой
если $p$ и $q$ не унифицируются, или будет успешным с привязкой переменных
в $q$ полученной при унификации запроса с $p$.

\secrel{Представление термов}\label{warren21}

Для начала давайте определим внутреннее представление термов в языке $\mathcal
L_0$. Мы будем использовать глобальный блок хранения данных в форме адресуемой
\termdef{кучи}{\prolog!куча} который мы назовем \var{HEAP}: массив ячеек данных.
Адресом ячейки в куче является индекс элемента массива \var{HEAP}.

Для представления произвольных термов в \var{HEAP} будет достаточно закодировать
переменные и ``структуры'' имеющие форму $f(@_1,..,@_n)$ где $f/n$ функтор и
$@_i$ ссылки на адреса кучи для $n$ субтермов. Таким образом существует два вида
данных, хранимых в куче: переменные и структуры термов. Явно заданные
\termdef{тэги}{\prolog|тэг}, появляющиеся как часть внутреннего формата ячеек
кучи, будут использоваться для различения между этими двумя типами
данных.\note{интересно рассмотреть расширение тэгирования для реализации
ООП и динамического контроля типов}

Переменная будет индентифицироваться как указатель, и представляться как одна
ячейка кучи, так что мы должны говорить о \termdef{ячейках
переменных}{\prolog!ячейка переменной}. Ячейка переменной отмечается тэгом
\class{REF}, и обозначается как $<REF,k>$ где $k$ адрес хранения, т.е. индекс в
\var{HEAP}. Этот механизм предназначен для облегчения связывания переменных
через установление ссылки на терм в переменной, которая связывается с этим
термом. Таким образом при связывании переменной адресная часть
\class{REF}-ячейки получает значение соответствующего адреса терма. Соглашение о
представлении \termdef{несвязанной переменной}{\prolog!несвязанная переменная}\
--- адресная часть \class{REF}-ячейки указывает на саму переменную. Таким
образом \emph{несвязанные переменные представляются \class{REF}-ячейкой со
ссылкой на саму себя}.

Структуры\ --- термы не являющиеся переменными. Формат кучи для представления
структуры $f(t_1,..,t_n)$ содержит $n+2$ ячеек кучи. Первые две ячейки не
обязательно смежные. По сути первая их этих двух ячеек выступает в роли
сортированного указателя на вторую ячейку, и в то же время сама выступает как
представление функтора $f/n$.\note{причина использования этой кажущеся странной
косвенной адресации\ --- реализация разделяемых структур (structure sharing)\
--- будет вскоре ясна} Остальные $n$ ячеек предназначены для упорядоченного хранения
ссылок на корни соответствующих $n$ субтермов. 

Детальнее, первая из $n+2$ ячеек представляющих терм $f(t_1,..,t_n)$
форматирована как тэгированная \termdef{структурная ячейка}{\prolog!структурная
ячейка}, которую можно записать как $<STR,k>$, содержит тэг \var{STR} и
указатель $k$ на \termdef{ячейку функтора}{\prolog!ячейка функтора}, храняющую
представление функтора $f/n$. Важно отметить что \emph{непосредственно за
\term{ячейкой функтора} в смежных адресах всегда следуют $n$ \term{структурных
ячеек}, представляющих каждый из $t_i$ субтермов}. Так что если $HEAP[k]=f/n$
то $HEAP[k+1]$ будет ссылаться на первый субтерм $t_1$, а $HEAP[k+n]$ будет
ссылаться на последний субтерм $t_n$.

\fig{\\Фиг. 2.1: Представление кучи для терма
$p(Z,h(Z,W),f(W))$}{prolog/warren/fig21.pdf}{height=0.5\textheight}
\label{warrenfig21}

\begin{tabular}{l l l}
0 & STR & 1 \\
1 & $h/2$ \\
2 & REF & 2\\
3 & REF & 3\\
4 & STR & 5\\
5 & $f/1$\\
6 & REF & 3\\
7 & STR & 8\\
8 & $p/3$\\
9 & REF & 2\\
10 & STR & 1\\
11 & STR & 5\\
\end{tabular}

\bigskip
Например, рассмотрим представление кучи для терма $p(Z,h(Z,W),f(W))$, начальная
ячейка которого находится по адресу 7 (иллюстрация \ref{warrenfig21}).
Отметим что \emph{для каждой} непривязанной переменной существует только одно
вхождение, представленное как \class{REF}-ячейка, в то время как другие ее
вхождения в исходный терм представляются как ссылки на первое вхождение
($Z=HEAP[2]$, $W=HEAP[3]$). Также обратите внимание что за структурными ячейками
по адресам 0, 4 и 7 \emph{сразу} следуют их ячейки функторов, но это не так для
адресов 10 и 11.


\secrel{Компиляция $\mathcal L_0$ запросов}

Согласно операционной семантике $\mathcal L_0$ обработка запроса состоит из
подготовке в решению уравнения с одной стороны. А именно, терм запроса $q$
транлируется в последовательность инструкций, целью которой является построение
экземпляра $q$ на куче из текстового представления $q$. Таким образом, из-за
древовидной структуры терма и множествественных вхождениях переменных,
необходимо, чтобы при обработке части терма где-то временно сохранялись части
терма, которые еще предстоит обработать, или переменные которые могут
встретиться еще раз далее по ходу работы. Для этой цели виртуальная машина
$\mathcal M_0$ наделена достаточным количеством (изменяемых)
\termdef{регистров}{\prolog!регистр} $X_1$, $X_2$,\ldots которые используются
для временного хранения данных кучи по мере создания промежуточных термов. Таким
образом, содержимое каждого регистра должно иметь формат ячейки кучи. Эти
изменяемые регистры выделяются для терма по мере доступности, так что (1)
регистр $X_1$ всегда распределяется для охватывающего терма, и (2) тот же
регистр распределяется для всех вхождений определенной переменной.
Например регистры для переменных терма $p(Z,h(Z,W),f(W))$ распределяются
\begin{equation*}
\begin{split}
X_1 &= p(X_2,X_3,X_4)\\
X_2 &= Z\\
X_3 &= h(X_2,X_5)\\
X_4 &= f(X_5)\\
X_5 &= W
\end{split}
\end{equation*}

Это равносильно тому что терм рассматривается как сплющенный конъюктивный набор
уравнений в форме $X_i=X$ или $X_i=f(X_{i_1},..,X_{i_n}), (n \geqslant 0)$ ,
где члены $X_i$ различные новые имена переменных. Есть два последствия
распределения регистров: (1) все внешние имена переменных (такие как $Z$ and $W$
в нашем примере) могут быть забыты; и (2) терм запроса может быть
трансформирован в его \termdef{сплющенную форму}{\prolog!сплющенная форма},
т.е. последовательность назначений регистров только в форме 
$X_i=f(X_{i_1},..,X_{i_n})$. Эта форма\ --- то, что контролирует построение
представления терма в куче. Таким образом, чтобы генерация кода слева направо
была хорошо обоснована, необходимо сформировать сплющенный терм запроса, так
чтобы гарантировать что \emph{имена регистров не могут использоваться в правых
частях присвоений (например как субтерм) до их инициализации}\note{if it has one
(viz., being the lefthand side)}. Например сплющенная форма терма запроса
$p(Z,h(Z,W),f(W))$ это последовательность
$X_3=h(X_2,X_5)$, $X_4=f(X_5)$, $X_1=p(X_2,X_3,X_4)$\note{исключена привязка
переменных на регистры $X_2$, $X_5$}.

Сканируя сплющенный терм запроса слева направо, каждый компонент в форме
$X_i=f(X_{i_1},..,X_{i_n})$ токенизируется в последовательность
$X_i=f/n,X_{i_1},..,X_{i_n}$ такую что после регистра ассоциированного с n-арным
функтором идет последовательность $n$ имен регистров. Так что в потоке таких
токенов полученных в результате токенизации полного сплющенного терма,
существует три вида элементов для обработки:
\begin{enumerate}
  \item 
регистр ассоциированный со структурным функтором;
  \item 
регистр-аргумент который не был нигде равнее встречен в потоке;
  \item 
регистр-аргумент который уже был упомянут в потоке.
\end{enumerate}

Из такого потока легко получить представление кучи используя метод управляемого
потоком токенов синтеза. Для реализации этого нужно выполнить сооответствующие
действия для каждого типа токенов:
\begin{enumerate}
  \item 
создать на куче новую ячейку STR (и примыкающий функтор) и скопировать эту
ячейку в указаный регистр;
  \item 
создать на куче новую ячейку REF содержащую собственный адрес, и скопировать
ее в указанный регистр;
  \item 
создать на куче новую ячейку и копировать в нее значение регистра.
\end{enumerate}

Each of these three actions specifies the effect of respective instructions of
the machine $M_0$ that we note:
\begin{enumerate}
  \item 
put structure f n Xi
  \item 
set variable Xi
  \item 
set value Xi
\end{enumerate}
respectively.

From the preceding considerations, it has become clear that the heap is implicitly
used as a stack for building terms. Namely, term parts being constructed are
incrementally piled on top of what already exists in the heap. Therefore, it is
necessary to keep the address of the next free cell in the heap somewhere, precisely
as for a stack.\note{As a matter of fact, in [War83], Warren refers to the heap
as the \emph{global stack}.} Adding to M a global register H containing at all
times the next available address on the heap, these three instructions are given explicitly
in Figure 2.2. For example, given that registers are allocated as above, the
sequence of instructions to build the query term $pZh$
is shown in Figure 2.3.

\paragraph{Exercise 2.1} Verify that the effect of executing the sequence of
instructions shown in Figure 2.3 (starting with H 
 )
does indeed yield a correct heap representation
for the term $pZh$ the one shown earlier as Figure 2.1, in
fact.

\secrel{2.3 Compiling L
programs . . . . . . . . . . . . . . . . . . . . . . . 13}

\secrel{2.4 Argument registers . . . . . . . . . . . . . . . . . . . . . . . . .
19}

\secup
\secrel{Flat Resolution}\label{warren3}\secdown

We now extend the language L0 into a language L where procedures are no longer
reduced only to facts but may also have bodies. A body defines a procedure as a
conjunctive sequence of atoms. Said otherwise, L is Prolog without backtracking.

An L program is a set of procedure definitions or (definite) clauses, at most
one per predicate name, of the form ‘a :- a0000an0’ where n 0  and the ai’s are
atoms. As before, when n  , the clause is called a fact and written without
the ‘:-’ implication symbol. When n  , the clause is called a rule, the atom a
is called its head, the sequence of atoms a0000an is called its body and atoms
composing this body are called goals. A rule with exactly one body goal is
called a chain (rule). Other rules are called deep rules. L queries are
sequences of goals, of the form ‘?-g0000gk0’ where k 0 . When k  , the query
is called the empty query. As in Prolog, the scope of variables is limited to
the clause or query in which they appear.

Executing a query ‘?-g0000gk0’ in the context of a program made up of a set of
procedure-defining clauses consists of repeated application of leftmost
resolution until the empty query, or failure, is obtained. Leftmost resolution
amounts to unifying the goal g0 with its definition’s head (or failing if none
exists) and, if this succeeds, executing the query resulting from replacing g0
by its definition body, variables in scope bearing the binding side-effects of
unification. Thus, executing a query in L either terminates with success (i.e.,
it simplifies into the empty query), or terminates with failure, or never
terminates. The “result” of an L query whose execution terminates with success
is the (dereferenced) binding of its original variables after termination.

Note that a clause with a non-empty body can be viewed in fact as a conditional
query. That is, it behaves as a query provided that its head successfully
unifies with a predicate definition. Facts merely verify this condition, adding
nothing new to the query but a contingent binding constraint. Thus, as a first
approximation, since an L query (resp., clause body) is a conjunctive sequence
of atoms interpreted as procedure calls with unification as argument passing,
instructions for it may simply be the concatenation of the compiled code of each
goal as an L0 query making it up. As for a clause head, since the semantics
requires that it retrieves arguments by unification as did facts in L0,
instructions for L0’s fact unification are clearly sufficient.

Therefore, M0 unification instructions can be used for L clauses, but with two
measures of caution: one concerning continuation of execution of a goal
sequence, and one meant to avoid conflicting use of argument registers.

\secrel{Facts}

Let us first only consider L facts. Note that L0 is all contained in L.
Therefore, it is natural to expect that the exact same compilation scheme for
facts carries over untouched from L0 to L. This is true up to a wee detail
regarding the proceed instruction. It must be made to continue execution, after
successfully returning from a call to a fact, back to the instruction in the
goal sequence following the call. To do this correctly, we will use another
global register CP, along with P, set to contain the address (in the code area)
of the next instruction to follow up with upon successful return from a call
(i.e., set to P instruction size P at procedure call time). Then, having exited
the called procedure’s code sequence, execution could thus be resumed as
indicated by CP. Thus, for L’s facts, we need to alter the effect of M0’s call
pn to:
cal
\[pn  CP  P  instruction sizeP0\]
\[P  pn0\]
and that of proceed to:
\[P  CP\]
As before, when the procedure pn is not defined, execution fails.
In summary, with the simple foregoing adjustment, L facts are translated exactly
as were L0 facts.
\secrel{Rules and queries}

We now must think about translating rules. A query is a particular case of a rule
in the sense that it is one with no head. It is translated exactly the same way,
but without the instructions for the missing head. The idea is to use L0’s instructions,
treating the head as a fact, and each goal in the body as an L0 query term
in sequence; that is, roughly translate a rule ‘p0000
:- p00000000pn00000’
following the pattern:
\begin{verbatim}
get arguments of p
put arguments of p0
call p0
.
.
.
put arguments of pn
call pn
\end{verbatim}

Here, in addition to ensuring correct continuation of execution, we must arrange
for correct use of argument registers. Indeed, since the same registers are used by
each goal in a query or body sequence to pass its arguments to the procedure it
invokes, variables that occur in many different goals in the scope of the sequence
need to be protected from the side effects of put instructions. For example, consider
the rule ‘pX
Y 0 :- qX
Z0
rZ
Y 00’ If the variables YZ
were allowed
to be accessible only from an argument register, no guarantee could be made that
they still would be after performing the unifications required in executing the body
of p.

Therefore, it is necessary that variables of this kind be saved in an environment
associated with each activation of the procedure they appear in. Variables which
occur in more than one body goal are dubbed permanent as they have to outlive
the procedure call where they first appear. All other variables in a scope that are
not permanent are called temporary. We shall denote a permanent variable as Yi,
and use Xi as before for temporary variables. To determine whether a variable is
permanent or temporary in a rule, the head atom is considered to be part of the
first body goal. This is because get and unify instructions do not load registers
for further processing. Thus, the variable X in the example above is temporary as
it does not occur in more than one goal in the body (i.e., it is not affected by more
than one goal’s put instructions).

Clearly, permanent variables behave like conventional local variables in a procedure.
The situation is therefore quite familiar. As is customary in programming
languages, we protect a procedure’s local variables by maintaining a run-time
stack of procedure activation frames in which to save information needed for the
correct execution of what remains to be done after returning from a procedure call.
We call such a frame an environment frame. We will keep the address of the latest
environment on top of the stack in a global register E.\note{In [War83], this stack is called the local stack to distinguish it from the global stack (see
Footnote 1 at the bottom of Page 13).}

As for continuation of execution, the situation for rules is not as simple as that
for facts. Indeed, since a rule serves to invoke further procedures in its body, the
value of the program continuation register CP, which was saved at the point of
its call, will be overwritten. Therefore, it is necessary to preserve continuation
information by saving the value of CP along with permanent variables.

Let us recapitulate: M is an augmentation of M0 with the addition of a new data
area, along with the heap (HEAP), the code area (CODE), and the push-down list
(PDL). It is called the stack (STACK) and will contain procedure activation frames.
Stack frames are called environments. An environment is pushed onto STACK
upon a (non-fact) procedure entry call, and popped from STACK upon return.
Thus, an allocate/deallocate pair of instructions must bracket the code
generated for a rule in order to create and discard, respectively, such environment
frames on the stack. In addition, deallocate being the ultimate instruction
of the rule, it must connect to the appropriate next instruction as indicated by
the continuation pointer that had been saved upon entry in the environment being
discarded.

Since the size of an environment varies with each procedure in function of its
number of permanent variables, the stack is organized as a linked list through a
continuation environment slot; i.e., a cell in each environment frame bearing the
stack index of the environment previously pushed onto the stack.

To sum up, two new I instructions for M are added to the ones we defined for
I0:

1. allocate

2. deallocate

with effect, respectively:

1. to allocate a new environment on the stack, setting its continuation environment
field to the current value of E, and its continuation point field to that
of CP; and,

2. to remove the environment frame at stack location E from the stack and
proceed, resetting P to the value of its CP field and E to the value of its CE
field.

To have proper effect, an allocate instruction needs to have access to the size
of the current environment in order to increment the value of E by the right stack
offset. The necessary piece of information is a function of the calling clause (i.e.,
the number of permanent variables occurring in the calling clause). Therefore, it
is easily statically available at the time the code for the calling clause is generated.
Now, the problem is to transmit this information to the called procedure that, if
defined as a rule (i.e., starting with an allocate), will need it dynamically,
depending on which clause calls it. A simple solution is to save this offset in the
calling clause’s environment frame from where it can be retrieved by a callee that
needs it. Hence, in M, an additional slot in an environment is set by allocate
to contain the number of permanent variables in the clause in question.

Summing up again, an M stack environment frame contains:

1. the address in the code area of the next instruction to execute upon (successful)
return from the invoked procedure;

2. the stack address of the previous environment to reinstate upon return (i.e.,
where to pop the stack to);

3. the offset of this frame on the stack (the number of permanent variables);
and,

4. as many cells as there are permanent variables in the body of the invoked
procedure (possibly none).

Such an M environment frame pushed on top of the stack looks thus:
\[E CE continuation environment\]

This necessitates giving allocate an explicit argument that is the number of
permanent variables of the rule at hand, such that, in M:
\[allocate N =\]

Similarly, the explicit definition of M’s deallocate is:
\[deallocate  =\]

With this being set up, the general translation scheme into M instructions for an
L rule ‘p0000
:- p00000000pn00000’
with N permanent variables will follow
the pattern:
\[p allocate N\]

For example, for L clause ‘pX
Y 0 :- qX
Z0
rZ
Y 00’, the corresponding
M code is shown in Figure 3.1.

\paragraph{Figure 3.1: M machine code for rule pX
Y 0 :- qX
Z0
rZ
Y 00}
\begin{verbatim}

\end{verbatim}

\paragraph{Exercise 3.1} GiveM code for L facts qa0 b and rb0 c and L query
?-pU0 V , then trace the code shown in Figure 3.1 and verify that the solution produced is
U 
 a0 V 
 c.
C  



\secup
\secrel{4 Prolog 33}\label{warren4}\secdown

\secrel{4.1 Environment protection . . . . . . . . . . . . . . . . . . . . . . .
34}

\secrel{4.2 What’s in a choice point . . . . . . . . . . . . . . . . . . . . . .
36}

\secup

\secrel{Optimizing the Design}\label{warren5}\secdown

Now that the reader is hopefully convinced that the design we have reached forms
an adequate target language and architecture for compiling pure Prolog, we can
begin transforming it in order to recover Warren’s machine as an ultimate design.
Therefore, since all optimizations considered here are part of the definitive design,
we shall now refer to the abstract machine gradually being elaborated as the WAM.
In the process, we shall abide by a few principles of design pervasively motivating
all the conception features of the WAM. We will repeatedly invoke these principles
in design decisions as we progress toward the full WAM engine, as more evidence
justifying them accrues.

\textsc{WAM PRINCIPLE 1} Heap space is to be used as sparingly as possible, as
terms built on the heap turn out to be relatively persistent.

\textsc{WAM PRINCIPLE 2} Registers must be allocated in such a way as to avoid
unnecessary data movement, and minimize code size as well.

\textsc{WAM PRINCIPLE 3} Particular situations that occur very often, even
though correctly handled by general-case instructions, are to be accommodated by special
ones if space and/or time may be saved thanks to their specificity.

In the light of WAM Principles 1, 2, and 3, we may now improve on M.

\paragraph{Figure 5.1: Better heap representation for term pZ hZW f W}
\begin{verbatim}

h
 REF 
 REF
 f 
 REF
 p
 REF 
 STR
 STR
\end{verbatim}

\secrel{Heap representation}

As many readers of [AK90] did, this reader may have wondered about the necessity
of the extra level of indirection systematically introduced in the heap by
an STR cell for each functor symbol. In particular, Fernando Pereira [Per90]
suggested that instead of that shown in Figure 2.1 on Page 11, a more economical
heap representation for pZ
hZW
f W
ought to be that of Figure 5.1, where
reference to the term from elsewhere must be from a store (or register) cell of the
form h STR
 i. In other words, there is actually no need to allot a systematic STR
cell before each functor cell.

As it turns out, only one tiny modification of one instruction is needed in order to
accommodate this more compact representation. Namely, the put structure
instruction is simplified to:
\begin{verbatim}
put structure f n
Xi  HEAP[H]  f n
Xi  h STR
H i
H  H  
\end{verbatim}

Clearly, this is not only in complete congruence with WAM Principle 1, but it also
eliminates unnecessary levels of indirection and hence speeds up dereferencing.

The main reason for our not having used this better heap representation in Section
2.1 was essentially didactic, wishing to avoid having to mention references
from outside the heap (e.g., from registers) before due time. In addition, we did
not bother bringing up this optimization in [AK90] as we are doing here, as we
had not realized that so little was in fact needed to incorporate it.\note{After dire reflection seeded by discussions with Fernando Pereira, we eventually realized that
this optimization was indeed cheap—a fact that had escaped our attention. We are grateful to him
for pointing this out. However, he himself warns [Per90]:

“Now, this representation (which, I believe, is the one used by Quintus, SICStus
Prolog, etc.) has indeed some disadvantages:

1. If there aren’t enough tags to distinguish functor cells from the other
cells, garbage collection becomes trickier, because a pointed-to value does not in
general identify its own type (only the pointer does).

2. If you want to use [the Huet-Fages] circular term unification algorithm,
redirecting pointers becomes messy, for the [same] reason...
In fact, what [the term representation in Section 2.1 is] doing is enforcing a convention
that makes every functor application tagged as such by the appearance of a
STR cell just before the functor word.”}


\secrel{5.2 Constants, lists, and anonymous variables . . . . . . . . . . . . .
47}

\secrel{5.3 A note on set instructions . . . . . . . . . . . . . . . . . . . . .
52}

\secrel{5.4 Register allocation . . . . . . . . . . . . . . . . . . . . . . . .
. 54}

\secrel{5.5 Last call optimization . . . . . . . . . . . . . . . . . . . . . . .
. 56}

\secrel{5.6 Chain rules . . . . . . . . . . . . . . . . . . . . . . . . . . . .
. 57}

\secrel{5.7 Environment trimming . . . . . . . . . . . . . . . . . . . . . . .
58}

\secrel{5.8 Stack variables . . . . . . . . . . . . . . . . . . . . . . . . . .
. 60}\secdown

\secrel{5.8.1 Variable binding and memory layout . . . . . . . . . . . . 62}

\secrel{5.8.2 Unsafe variables . . . . . . . . . . . . . . . . . . . . . . 64}

\secrel{5.8.3 Nested stack references . . . . . . . . . . . . . . . . . . . 67}

\secup

\secrel{5.9 Variable classification revisited . . . . . . . . . . . . . . . . .
. . 69}

\secrel{5.10 Indexing . . . . . . . . . . . . . . . . . . . . . . . . . . . . .
. . 75}

\secrel{5.11 Cut . . . . . . . . . . . . . . . . . . . . . . . . . . . . . . . .
. . 83}


\secup

\secrel{6 Conclusion 89}
\secrel{A Prolog in a Nutshell 91}\label{warrenA}
\secrel{B The WAM at a glance 97}\label{warrenB}\secdown

\secrel{B.1 WAM instructions . . . . . . . . . . . . . . . . . . . . . . . . . .
97}

\secrel{B.2 WAM ancillary operations . . . . . . . . . . . . . . . . . . . . .
112}

\secrel{B.3 WAM memory layout and registers . . . . . . . . . . . . . . . . .
117}

\secup


\secup

%%\secrel{Learn Datalog Today}\secdown

\cp{\url{http://www.learndatalogtoday.org/}}

Learn Datalog Today\ --- интерактивный учебник разработанный для вашего обучения
использованию Datomic-диалекта языка Datalog. \term{Datalog}\ --- язык запросов
декларативных баз данных, вышедший из языка \prolog\ и логического
программирования. Datalog имеет выразительность похожую на SQL.

\prog{Datomic}\ --- база данных интересного и инновационного нового типа, дающая
своим пользователям уникальный набор возможностей. Почитать о Datomic подронее
вы можете на официальном сайте \url{http://datomic.com}, архитектура системы
подробнее описана в
\href{http://www.infoq.com/articles/Architecture-Datomic}{InfoQ article}.

\bigskip
This tutorial was written on rainy days for the Lisp In Summer Projects 2013. If
you find bugs, or have suggestions on how to improve the tutorial, please visit
the project on github.

Many thanks to Robert Stuttaford for his careful proof reading/editing. I'd also
like to thank everyone who has contributed by fixing bugs and spelling mistakes.

\url{www.learndatalogtoday.org} \copyright 2013 Jonas Enlund

\href{https://github.com/jonase/learndatalogtoday}{github}

\url{lispinsummerprojects.org}

\secrel{Extensible Data Notation}


\secrel{Basic Queries}


\secrel{Data Patterns}

\secrel{Parameterized Queries}

\secrel{More Queries}

\secrel{Predicates}

\secrel{Transformation Functions}

\secrel{Aggregates}

\secrel{Rules}

\secup % datomic
%%\secrel{Automatic Program Generation from Specifications Using \prolog}
\label{pelin}\secdown
\cp{\url{http://ntrs.nasa.gov/archive/nasa/casi.ntrs.nasa.gov/19880014812.pdf}}

\copyright\ Alex Pelin\\
School of Computer Science
Florida International University
Miami, Florida 33199
\bigskip

\copyright\ Paul Morrow\\
AFWL / SCR
Kirtland AFB, NM 87117

\secrel{Abstract}

This paper describes an automatic program generator which creates PROLOG
programs from input/output specifications. The generator takes as input
descriptions of the input and output data types, a set of tests, a set of
transformations and the input/output relation. Abstract data types are used as
models for data. They are defined as sets of terms satisfying a system of
equations. The tests, the transformations and the input/output relation are
also specified by equations.

\secup
\secrel{An Efficient Unification Martelli/Montanary Algorithm}
\label{mmalg}\secdown

\cp{\url{http://www.nsl.com/misc/papers/martelli-montanari.pdf}}

\noindent\copyright \
ALBERTO MARTELLI
Consiglio Nazionale delle Ricerche
\\and\\
UGO MONTANARI
Universita di Pisa
\note{
Authors' present addresses: A. Martelli, Istituto di Scienze della Informazione,
Universit~ di Torino, Corso M. d'Azeglio 42, 1-10125 Torino, Italy; U.
Montanari, Istituto di Scienze della Informazione, Universit\& di Pisa, Corso
Italia 40, 1-56100 Pisa, Italy.

Permission to copy without fee all or part of this material is granted provided
that the copies are not made or distributed for direct commercial advantage, the
ACM copyright notice and the title of the publication and its date appear, and
notice is given that copying is by permission of the Association for Computing
Machinery. To copy otherwise, or to republish, requires a fee and/or specific
permission.

\copyright\ 1982 ACM 0164-0925/82/0400-0258 \$00.75

ACM Transactions on Programming Languages and Systems, Vol. 4, No. 2, April
 1982, Pages 258-282.
}\bigskip

\subsecly{Abstract}

The unification problem in first-order predicate calculus is described in
general terms as the solution of a system of equations, and a nondeterministic
algorithm is given. A new unification algorithm, characterized by having the
acyclicity test efficiently embedded into it, is derived from the
nondeterministic one, and a PASCAL implementation is given. A comparison with
other well-known unification algorithms shows that the algorithm described here
performs well in all cases.

Categories and Subject Descriptors: F.2.2 [Analysis of Algorithms and Problem
Complexity]: Nonnumerical Algorithms and Problems--complexity of proof
procedures; F.4.1 [Mathematical Logic and Formal Languages]: Mathematical
Logic--mechanical theorem proving; 1.2.3 [Artificial Intelligence]: Deduction
and Theorem Proving--resolution

General Terms: Algorithms, Languages, Performance, Theory 

\secrel{INTRODUCTION}\label{mmalg1}

In its essence, the unification problem in first-order logic can be expressed as
follows: Given two terms containing some variables, find, if it exists, the
simplest substitution (i.e., an assignment of some term to every variable) which
makes the two terms equal. The resulting substitution is called the \term{most
general unifier} and is unique up to variable renaming.

Unification was first introduced by Robinson [17, 18] as the central step of the
inference rule called resolution. This single, powerful rule can replace all the
axioms and inference rules of the first-order predicate calculus and thus was
immediately recognized as especially suited to mechanical theorem provers. In
fact, a number of systems based on resolution were built and tried on a variety
of different applications [5]. Even though further research made it apparent
that resolution systems are difficult to direct during proof search and thus are
often prone to combinatorial explosion [6], new impetus to the research in this
area was given by Kowalski's idea of interpreting predicate logic as a
programming language [10]. Here predicate logic clauses are seen as procedure
declarations, and procedure invocation represents a resolution step. From this
viewpoint, theorem provers can be regarded as interpreters for programs written
in predicate logic, and this analogy suggests efficient implementations [3, 25].

Resolution, however, is not the only application of the unification algorithm.
In fact, its pattern matching nature can be exploited in many cases where
symbolic expressions are dealt with, such as, for instance, in interpreters for
equation languages [4, 11], in systems using a database organized in terms of
productions [19], in type checkers for programming languages with a complex type
structure [14], and in the computation of critical pairs for term rewriting
systems [9].

The unification algorithm constitutes the heart of all the applications listed
above, and thus its performance affects in a crucial way the global efficiency
of each. The unification algorithm as originally proposed can be extremely
inefficient; therefore, many attempts have been made to find more efficient
algorithms [2, 7, 13, 15, 16, 22]. Unification algorithms have also been
extended to the case of higher order logic [8] and to deal directly with
associativity and commutativity [20]. The problem was also tackled from a
computational complexity point of view, and linear algorithms were proposed
independently by Martelli and Montanari [13] and Paterson and Wegman [15].

In the next section we give some basic definitions by representing the
unification problem as the solution of a system of equations. A nondeterministic
algorithm, which comprehends as special cases most known algorithms, is then
defined and proved correct. In Section 3 we present a new version of this
algorithm obtained by grouping together all equations with some member in
common, and we derive from it a first version of our unification algorithm.

In Sections 4 and 5 we present the main ideas which make the algorithm
efficient, and the last details are described in Section 6 by means of a PASCAL
implementation.

Finally, in Section 7, the performance of this algorithm is compared with that
of two well-known algorithms, Huet's [7] and Paterson and Wegman's [15]. This
analysis shows that our algorithm has uniformly good performance for all classes
of data considered.


\secrel{UNIFICATION AS THE SOLUTION OF A SET OF EQUATIONS: A NONDETERMINISTIC
ALGORITHM}

In this section we introduce the basic definitions and give a few theorems which
are useful in proving the correctness of the algorithms. Our ay of stating the
unification problem is slightly more general than the classical one due to
Robinson [18] and directly suggests a number of possible solution methods.

Let
\[A= \bigcup_{i=0,1,..} A_i \quad (A_i \bigcap A_j = \varnothing, i \neq j)\] be
a ranked alphabet, where $A_i$ contains the $i$-adic function symbols
(the elements of $A_0$ are constant symbols). Furthermore, let $V$ be the
alphabet of the variables.
The \term{terms} are defined recursively as follows:

(1) constant symbols and variables are terms;

(2) if $t_1,..,t_n \ (n \geq 1)$ are terms and $f \in A_n$,
then $f(t_1,..,t_n)$ is a term.

A \term{substitution} $\vartheta$ is a mapping from variables to terms,
with $\vartheta(x)=x$ almost everywhere. A substitution can be represented
by a finite set of ordered pairs
$\vartheta={(t_1,x_1),(t_2,x_2),..,(t_m,x_m)}$
where $t_i$ are terms and $x_i$ are distinct variables,
$i = 1,..,m$. To apply a substitution $\vartheta$ to a term $t$, we
simultaneously substitute all occurrences in $t$ of every variable $x_i$ in a
pair $(t_i, x_i)$ of $\vartheta$ with the corresponding term $t_i$. We call the
resulting term $t_\vartheta$.

For instance, given a term $t = f(x_1, g(x_2, a)$ and a substitution 
$\vartheta = {(h(x_2),x_1),(b,x_2)}$, 
we have $t_\vartheta = f(f(x_2),g(b),a)$ and $t_{\vartheta\vartheta} =
f(h(b),g(b),a))$.

The standard unification problem can be written as an equation \[t'=t''\]

A solution of the equation, called a \term{unifier}, is any substitution
$\vartheta$, if it exists, which makes the two terms identical. For instance,
two unifiers of the equation $f(x_1,h(x_1),x_2)=f(g(x_3),x_4,x_3))$ are
$\vartheta_1={(g(x_3),x_1),(x_3,x_2),(h(g(x_3)),x_4)}$ and
$\vartheta_2={(g(a),x_1),(a,x_2),(a,x_3),(h(g(a)),x_4)}$.

In what follows it is convenient also to consider sets of equations
\[t'_j=t''_j, \quad j=1,..,k\]

Again, a \term{unifier} is any substitution which makes all pairs of terms
$t'_j,t''_j$ identical simultaneously.

Now we are interested in finding transformations which produce \emph{equivalent}
sets of equations, namely, transformations which preserve the sets of all
unifiers.
Let us introduce the following two transformations:

\paragraph{(1) Term Reduction.} Let

\begin{equation}\label{mm1}
f(t'_1,t'_2,..,t'_n)=f(t''_1,t''_2,..,t''_n), \quad f \in A_n
\end{equation} 
be an equation where both terms are not variables and where the two root
function symbols are equal. The new set of equations is obtained by replacing
this equation with the following ones: 
\begin{align}\label{mm2}
t'_1 &= t''_1\\
t'_2 &= t''_2\\
&.\\
&.\\
&.\\
t'_n &= t''_n
\end{align}
If $n = 0$, then $f$ is a constant symbol, and the equation is simply erased.

\paragraph{(2) Variable Elimination.} Let $x = t$ be an equation where $x$ is a
variable and $t$ is any term (variable or not). The new set of equations is
obtained by applying the substitution $\vartheta={(t,x)}$ to both terms of all
other equations in the set (without erasing $x = t$).

We can prove the following theorems: 

\paragraph{THEOREM 2.1.} \textit{Let $S$ be a set of equations and let
$f'(t'_1,..t'_n)=f''(t''_1,..,t''_n)$ be an equation of $S$. If $f' \neq f''$,
then $S$ has no unifier. Otherwise, the new set of equations $S'$, obtained by
applying term reduction to the given equation, is equivalent to $S$.}

\paragraph{PROOF.} If $f' \neq f''$, then no substitution can make the
two terms identical.
If $f' = f"$, any substitution which satisfies \ref{mm2} also satisfies
\ref{mm1}, and conversely for the recursive definition of term. $\square$

\paragraph{THEOREM 2.2.} \textit{Let $S$ be a set of equations, and let us apply
variable elimination to some equation $x = t$, getting a new set of equations
$S'$. If variable $x$ occurs in $t$ (but $t$ is not $x$), then $S$ has no
unifier; otherwise, $S$ and $S'$ are equivalent.}

\paragraph{PROOF.} Equation $x = t$ belongs both to $S$ and to $S'$, and thus
any unifier $\vartheta$ (if it exists) of $S$ or of $S'$ must unify $x$ and $t$;
that is, $x_\vartheta$ and $t_\vartheta$ are identical. Now let $t_1 = t_2$ be
any other equation of $S$, and let $t'_1 = t'_2$ be the corresponding equation
in $S'$. Since $t'_1$ and $t'_2$ have been obtained by substituting $t$ for
every occurrence of $x$ in $t_1$ and $t_2$, respectively, we have
$t_{1_\vartheta}=t'_{1_\vartheta}$ and $t_{2_\vartheta}=t'_{2_\vartheta}$. Thus,
any unifier of $S$ is also a unifier of $S'$ and vice versa. Furthermore, if
variable $x$ occurs in $t$ (but $t$ is not $x$), then no substitution
$\vartheta$ can make $x$ and $t$ identical, since $x_\vartheta$ becomes a
subterm of $t_\vartheta$, and thus $S$ has no unifier. $\square$

A set of equations is said to be \term{in solved form} iff it satisfies the
following conditions: 

(1) the equations are $x_j = t_j, j = 1,..,k$;

(2) every variable which is the left member of some equation occurs only there.

A set of equations in solved form has the obvious unifier
\[\vartheta = {(t_1,x_1),(t_2,x_2),..,(t_k,x_k)}\]

If there is any other unifier, it can be obtained as
\[0 = {(t,, x~), (t2, x2) .... , (tk, xk)} U a\]
where a is any substitution which does not rewrite variables xl .... , xk. Thus t~
is called a most general unifier (mgu ).  

The following nondeterministic algorithm shows how a set of equations can be
transformed into an equivalent set of equations in solved form. 

\paragraph{Algorithm 1 }\ \\

Given a set of equations, repeatedly perform any of the following transformations. If no
transformation applies, stop with success. 

(a) Select any equation of the form
\[t=x\]
where t is not a variable and x is a variable, and rewrite it as 
\[x=t.\]

(b) Select any equation of the form
\[X=X\] 
where x is variable, and erase it.  

(c) Select any equation of the form
\[t' = t"\]
where t' and t" are not variables. If the two root function symbols are different, stop with
failure; otherwise, apply term reduction. 

(d) Select any equation of the form
\[x=t\]
where x is a variable which occurs somewhere else in the set of equations and
where t \# x. If x occurs in t, then stop with failure; otherwise, apply variable elimination. 

As an example, let us consider the following set of equations:
\[g(x2) = xl;\]
\[f(xl, h(xl), x2) = f(g(x3), x4, x3).\]

By applying transformation (c) of Algorithm 1 to the second equation we get
\[g(x2) = xl;\]
\[xl = g(x3);\]
\[h(x~) = x4;\]
\[X2 =X3.\]

By applying transformation (d) to the second equation we get
\[g(x2) = g(xs);\]
\[xl = g(x3);\]
\[h(g(x3)) = x4;\]
\[X2 ~- X3.\]

We now apply transformation (c) to the first equation and transformation (a) to
the third equation:
\[X2 ~ X3\]
\[xl = g(x3);\]
\[Xa = h(g(x3));\]
\[X2 ----X3.\]

Finally, by applying transformation (d) to the first equation and transformation
(b) to the last equation, we get the set of equations in solved form:
\[X2 ~- X3 ;\]
\[xl = g(x3);\]
\[x4 = h(g(x3)).\]

Therefore, an mgu of the given system is
\[= {(g(x~), x~), (x3, x2), (h(g(x3)), x4)}.\] 

The following theorem proves the correctness of Algorithm 1. 

\paragraph{THEOREM 2.3}. Given a set of equations S,

(i) Algorithm 1 always terminates, no matter which choices are made.

(ii) If Algorithm 1 terminates with failure, S has no unifier. If Algorithm 1
terminates with success, the set S has been transformed into an equivalent
set in solved form. 
 
\paragraph{PROOF}.\\

(i) Let us define a function F mapping any set of equations S into a triple of
natural numbers (nl, n2, n3). The first number, n~, is the number of variables in
S which do not occur only once as the left-hand side of some equation. The
second number, n2, is the total number of occurrences of function symbols in S.
The third number, n3, is the sum of the numbers of equations in S of type x = x
and of type t = x, where x is a variable and t is not. Let us define a total ordering
on such triples as follows:
\begin{align*}
(n~, n~, n~) > (n~', n2, " n~')\ if\ &n~ > n~'\\
&or\ n~ = n~\ and\ n2 > n2\\
&or\ n~=n" 1 a n- 1 n2 ' ---- n2 "\ and\ n3 ' > n3.\\ 
\end{align*}   

With the above ordering, N 3 becomes a well-founded set, that is, a set where no
infinite decreasing sequence exists. Thus, if we prove that any transformation of
Algorithm 1 transforms a set S in a set S' such that F(S') < F(S), we have
proved the termination. In fact, transformations (a) and (b) always decrease n3
and, possibly, n~. Transformation (c) can possibly increase n3 and decrease nl,
but it surely decreases n2 (by two). Transformation (d) can possibly change n3
and increase n2, but it surely decreases n~. 

(ii) If Algorithm 1 terminates with failure, the thesis immediately follows from
Theorems 2.1 and 2.2. If Algorithm 1 terminates with success, the resulting set of
equations S' is equivalent to the given set S. In fact, transformations (a) and (b)
clearly do not change the set of unifiers, while for transformations (c) and (d) this
fact is stated in Theorems 2.1 and 2.2. Finally, S' is in solved form. In fact, if (a),
(b), and (c) cannot be applied, it means that the equations are all in the form
x = t, with t \# x. If (d) cannot be applied, that means that every v.arialSle
which is the left-hand side of some equation occurs only there.
$\square$\bigskip

The above nondeterministic algorithm provides a widely general version from
which most unification algorithms [2, 3, 7, 13, 15, 16, 18, 22-24] can be derived by
specifying the order in which the equations are selected and by defining suitable
concrete data structures. For instance, Robinson's algorithm [18] might be
obtained by considering the set of equations as a stack. 


\secrel{AN ALGORITHM WHICH EXPLOITS A PARTIAL ORDERING AMONG SETS
OF VARIABLES}\secdown
\secrel{Basic Definitions}

In this section we present an extension of the previous formalism to model our
algorithm more closely. We first introduce the concept of multiequation. A multiequation
is the generalization of an equation, and it allows us to group together
many terms which should be unified. To represent multiequations we use the
notation S -- M where the left-hand side S is a nonempty set of variables and the
right-hand side M is a multiset 1 of nonvariable terms. An example is
\[{xl, x2, x3} = (tl, t2).\]
\note{A multiset is a family of elements in which no ordering exists but in which many identical elements
may occur.}

The solution (unifier) of a multiequation is any substitution which makes all
terms in the left- and right-hand sides identical. 

A multiequation can be seen as a way of grouping many equations together.
For instance, the set of equations
\[Xl ---- X2;\]
\[X3 = Xl;\]
\[tl = Xl;\]
\[X2 ---- t2;\]
\[tl = t2\]
can be transformed into the above multiequation, since every unifier of this set
of equations makes the terms of all equations identical. To be more precise, given
a set of equations SE, let us define a relation RSE between pairs of terms as
follows: tl RSE t2 iff the equation tl = t2 belongs to SE. Let/tSE be the reflexive,
symmetric, and transitive closure of RSE. 

Now we can say that a set of equations SE corresponds to a multiequation
S = M iff all terms of SE belong to S U M and for every tr and ts E S U M we have
tr RSE t,. 

It is easy to see that many different sets of equations may correspond to a
given multiequation and that all these sets are equivalent. Thus the set of
solutions (unifiers) of a multiequation coincides with the set of solutions of any
corresponding set of equations. 

Similar definitions can be given for a set of multiequations Z by introducing a
relation Rz between pairs of terms which belong to the same multiequation. A set
of equations SE corresponds to a set of multiequations Z iff
\[ti/~SE tj ** ti Rz tj\]
for all terms t~, tj of SE or Z.  
 
\secup

\secup
\secup

\secrel{Дурдом на дереве}\secdown
\secrel{The Tree Processing Language\\Defining the structure and behaviour of a
tree}\label{papegaij}\secdown

\begin{tabular}{l l l}
\cp{\url{http://essay.utwente.nl/705/1/scriptie_Papegaaij.pdf}}&
\copyright & E. Papegaaij 
 \email{e.papegaaij@alumnus.utwente.nl} \\
&Supervisors &dr. ir. Theo C. Ruys\\
&&ir. Philip K.F. Hölzenspies\\
&&dr. ir. Arend Rensink\\
&Institute &University of Twente\\
&Chair &Formal Methods and Tools\\
\end{tabular}
\bigskip

Enschede, March 7, 2007

\secly{Abstract}

Tree structures are commonly used in many applications. One of these is a
compiler, in which the tree is called an abstract syntax tree (AST). Different
techniques have been developed for building and working with ASTs. However, many
of these techniques are limited in their applicability, require major effort to
implement or introduce maintenance problems in an evolving application.

This thesis introduces the Tree Processing Language, a language for defining the
structure of a tree and adding functionality to this tree. The compiler
\prog{TPLc} is used to produce the actual class hierarchy implementing the
specified tree. TPL provides a clear separation between the structure of a tree,
a \emph{tree definition}, and behaviour of a tree, \emph{logic specifications}.
Different aspects of the behaviour of a tree can be provided in separate logic
specifications, allowing a clear separation of concerns.

\prog{TPLc} generates a heterogeneous tree structure with strictly typed
children. Functionality in a logic specification is specified using the
inheritance pattern. To allow different inheritance trees in different logic
specifications, the inheritance pattern is enhanced with multiple inheritance.
For languages that do not support multiple inheritance, the inheritance pattern
with composition is developed.

To prove the applicability of TPL, \prog{TPLc} is written in TPL. When compared
with an implementation in Java, this implementation provides a better separation
of concerns and is easier to maintain.

\secly{Preface}

Compiler construction has always been one of my favourite fields of software
engineering. In the past few years I’ve written several parsers and compilers.
Of these compilers, the compiler for the functional programming language Tina
has been the most challenging. I used a hand-written heterogeneous abstract
syntax tree as underlying data structure. The most important algorithm applied
onto this AST, the transformation of Tina into a core lambda expression
language, was written as part of these AST node classes. However, the
overwhelming number of AST classes (almost 100) made this approach increasingly
difficult to maintain when other algorithms (such as a lambda lifter) where
added. At that moment, it became clear that a more structured approach was
required. To keep the development of an application, based on a heterogeneous
tree, maintainable, different algorithms needed to be separated in different
files. The development of tpl is an attempt to provide such an environment.

When I first approached Theo C. Ruys, my premiere supervisor, for an assignment,
I has no idea I would be solving this problem, which had bothered me for a long
time. At first, ambitious as I was, I proposed to design and implement a
completely new parser generator. Luckily, Theo slowed me down a bit and directed
me to focus on the real problem: the heterogeneous AST.

For his help in concreting the features of tpl, reading and correcting this
thesis and his patience during the endless discussions we had last year, I would
like to thank Theo C. Ruys, my premiere supervisor. His guiding helped me
structure my thoughts, to be able to write them down. I would also like to thank
Philip K.F.
H\"olzenspies for his help in writing and formatting this thesis. His knowledge
of the English language has proven to be far better than mine. Last, but not least,
I would like to thank Arend Rensink for having taken the time to examine this
thesis.

\bigskip
Emond Papegaaij\\
Enschede, March 7, 2007
\bigskip

\secrel{Introduction}\secdown

Tree structures have been, and probably will be for a considerable time in the
future, a widely used way of organising and working with data. Tree structures
are used to represent the structure of an input file\note{concrete and abstract
syntax trees}, user interface components, the representation of HTML
pages\note{the document object model}, XML and many more. Due its wide
acceptance, extensive research has been spent on working with tree structures.

This thesis is placed in the context of working with tree structures in an
object-oriented programming environment. The main focus is on defining the
runtime organisation of the tree and applying algorithms on this structure. The
origin of the tree\ --- the system responsible for constructing the tree
structure\ --- and the actual construction of the tree are discussed, but fall
outside the main research area.

In this chapter, an introduction on compiler construction is given, in
\ref{pape11}. This section shows how an abstract syntax tree is acquired, and
what the typical operations are that need to be performed on an AST.
\ref{pape12} describes the problem statement of this thesis. Finally, the
outline of this thesis is given in \ref{pape13}.

\secrel{Compiler Construction and Abstract Syntax Trees}\label{pape11}\secdown

A multi-pass compiler performs the compilation of a source file in several
stages. These stages will be discussed in this section. Compilation starts with
reading a source file, and recognising the syntax of the input. Next, an
abstract representation of this input is constructed. This is the abstract
syntax tree. This AST is used in subsequent phases to perform context checking
and code generation. More complex compilers might have more phases, such as
optimisers.

Abstract syntax trees are also commonly used in other disciplines, such as
communication (eg. a web browser) and source code refactoring in an integrated
development environment (IDE) \cite{Eclipse2007}. It is also possible that the
abstract syntax tree is not the result of a parser reading an input file, but
from speech, or from a graphical programming language. However, the most common
usage is a compiler, which reads an input language.

\secrel{Lexical Analysis and Parsing}

In the first stage, the lexical analysis, the compiler reads the input file and
produces a stream of tokens. Every token corresponds to a fragment, or
construct, found in the input file, such as identifiers, literals, operators and
keywords. These tokens are fed to a parser, which discovers (and checks) the
structure of the input.

Writing a lexer (or scanner) and parser by hand is tedious, difficult and error
prone. Many programs have been developed, which assist the developer in writing
the lexer and parser. These tools often take a syntax specification in (E)BNF,
and generate a lexer and parser from this specification. Therefore, these tools
are commonly called parser generators. Some of these tools are mentioned in
\ref{pape23}.

Different strategies exist, on how a parser matches the input language, such as
LALR and recursive descent parsing. However, a discussion of these is beyond the
scope of this thesis\note{An explanation of various parsing algorithms, such as
LR(k) and LL(k), can be found in \cite{Aho1986}.}.

\secrel{Construction of the AST}

In a multi-pass compiler, the task of the parser is to record the structure of
the parsed input in an abstract syntax tree. This tree contains all relevant
information from the input. What exactly is relevant information, depends on the
subsequent phases. Normally, tokens, such as comma’s and brackets, are
discarded. Also, nesting of parser production rules is removed.

AST construction is exemplified with the grammar presented in fragment 1.1.
This grammar matches simple expressions with addition and multiplication. The
actual values are represented by numbers and identifiers. Expressions can be
nested with brackets.

\bigskip

This grammar matches sentences such as ‘1’, ‘1+1’ and ‘(1+a)*b’. The parse
tree of the sentence ‘3+5*(a+b)’ is given in figure 1.1. This figure shows
how the complete sentence is matched as an hexpressioni. The hexpressioni
consists of a htermi, followed by the literal ‘+’, again followed by a htermi.
The left htermi is a simple hatomi, which in turn is a hnumberi. The right
htermi consists of two hatomis, separated by a ‘*’. This process is continued
until all tokens (the bottom line of the figure) are matched.

The parse tree clearly shows the structure of the parsed text, but this
structure is not very practical to work with. If an interpreter for this grammar
is needed, a set of four constructs is sufficient: addition, multiplication,
numbers and identifiers. The node adds the results of the left and right
operands. This node is created when a ‘’ is matched in hexpressioni. The node
multiplies the left operand with the right. It is created when a ‘’ is matched
in htermi. A node is created when a hnumberi is matched, and yields the value of
the number. Finally, the  node, which is created when an hidentifieri is
matched, resolves the value in a symbol table.



\secrel{Context Checking and Code Generation}

\secup

\secrel{Problem Statement}\label{pape12}

\secrel{Outline}\label{pape13}

\secup

\secup
\secup

\secrel{Язык \bi}\label{bi}\secdown
\secrel{DLR: Dynamic Language Runtime}\label{dlr}

\begin{framed}
DLR: Dynamic Language Runtime\ --- может использоваться как runtime-ядро
для реализации динамичеcких языков, или только в качестве библиотеки хранилища
данных
\end{framed}

\begin{description}
  \item[синтаксичеcкий парсер]
  для разбора текстовых данных, файлов конфигурации, скриптов и т.п.,
  необязателен. В результате разбора формируется синтаксическое дерево из
  динамических объектов DLR. По реализации может быть
  \begin{description}[nosep]
  \item[конфигурируемым в runtime]
  добавление/изменение/удаление правил правил грамматики в процессе работы
  программы
  \item[статическим] неизменный синтаксис, реализация в виде внешнего модуля,
  в самом простом случае достаточно использования \prog{flex}/\prog{bison}
  \end{description}
  \item[библиотека динамических типов данных] выполняет функции хранения
  данных, может быть реализована
  \begin{description}[nosep]
  \item[в \lisp-стиле]
  базовый набор скаляров \ref{biscalars}\ (символы, строки
  и числа) и тип \class{cons-ячейка}\ позволяющий конструировать составные 
  структуры данных
  \item[\bi-стиль]
  универсальный символьный тип \ref{ast}, позволяющий хранить как
  скаляры, так и вложенные элементы; в базовый тип \class{AST}\ заложено
  хранение типа данных \var{tag}, его значения \var{value}, и два способа
  вложенных хранилищ: плоский упорядоченный список \var{nest}\ и именованный
  неупорядоченный со строковыми ключами \var{pars}.
  
  От базового символьного типа наследуются
  \begin{description}[nosep]
  \item[скаляры] символ, строка, несколько вариантов чисел (целые, плавающие,
  машинные, комплексные)\note{критерием скалярности можно считать возможность
  распознавания элемента данных лексером}
  \item[композиты] структуры данных и объекты
  \item[функционалы] объекты, для которых определен \term{оператор аппликации}
  \end{description}
  \end{description}
  \item[библиотека операций над данными]
  для преобразования данных и символьных вычислений на списках, деревьях,
  комбинаторах и т.п.
  \begin{description}[nosep]
  \item[\lisp] стандартная библиотека функций языка \lisp
  \item[\bi] каждый тип данных имеет набор унарных и
  бинарных \term{операторов}, реализованых в виде виртуальных методов классов
  \end{description}
  
  \item[подсистема ООП] реализация механизмов ООП, наследования от класса и
  объекта-инстанса, вывод типов, преобразование объектных моделей
  \item[реализация механизмов функциональных языков] хвостовая рекурсия, pattern
  matching, динамическая компиляция, автоматическое распараллеливание на
  map/reduce
  \item[менеджер памяти со сборщиком мусора]
  \item[динамический компилятор] функциональных типов\ --- через библиотеку JIT
  LLVM
  \item[статический компилятор] \ \\
  \begin{description}[nosep]
  \item[в объектный код] через LLVM
  \item[кодогенератор \cpp]
  \end{description}
\end{description}

\paragraph{Расширенный функционал}

\begin{description}
  \item[подсистема облачных вычислений и кластеризации] расширение DLR на
  кластера: распределение объектов и процессов между вычислительными узлами.
  Варианты кластера с высокой связностью\note{аппаратная разделяемая память
  через сеть InfiniBand\ --- ``Сергей Королев''}, Beowulf\note{компьютеры
  общего назначени (офисные) с передачей сообщений по Gigabit Ethernet} с
  постоянным составом, интернет-облака с переменным составом: узлы асинхронно
  подключаются/отключаются, гомо/гетерогенные: по аппаратной платформе узлов и
  ОС/среде на каждом узле. Распределедение вычислений на одно- и
  многопроцессорных SMP-системах\note{многопоточные вычисления на одном
  многоядерном узле}
  \item[прикладные библиотеки] GUI, CAD/CAM/EDA, численные методы, цифровая
  обработка сигналов, сетевые сервера и протоколы,\ldots
  \item[подсистема кросс-трансляции] между ходовыми языками программирования
  (\cpp, \js, \py, PHP, Паскакаль) через связку: парсер входного
  языка $\rightarrow$\ система типов DLR $\rightarrow$\ кодогенератор выходного
  языка
  \item[интерактивная объектная среда] а-ля \st\ с виджетами и функционалом GUI,
  CAD, IDE и визуализациии данных
  \item[сервер приложений] обслуживающий тонких браузерных клиентов по HTTP/JS
\end{description}

\secrel{Система динамических типов}\label{bicore}\secdown
\secrel{\class{sym}: символ = Абстрактный Символьный Тип /AST/}\label{ast}

Использование класса \class{Sym}\ и виртуально наследованных от него
классов, позволяет реализовать на \cpp\ хранение и обработку \emph{любых}\
данных в виде деревьев\note{в этом АСТ близок к традиционной аббревиатуре AST: Abstract
Syntax Tree}. Прежде всего этот \termdef{символьный тип}{символьный тип}\
применяется при разборе текстовых форматов данных, и текстов программ.
\emph{Язык \bi\ построен как интерпретатор AST, примерно так же как язык \lisp\
использует списки}.

\bigskip

\begin{lstlisting}[language=C++]
// ============================================== ABSTRACT SYMBOLIC TYPE (AST)
struct Sym {
\end{lstlisting}

\begin{lstlisting}[language=C++,title=тип (класс) и значение элемента данных]
// ---------------------------------------------------------------------------
	string tag;							// data type / class
	string val;							// symbol value
\end{lstlisting}

\begin{lstlisting}[language=C++,title=конструкторы (токен используется в
лексере)]
// -------------------------------------------------------------- constructors
	Sym(string,string);					// <T:V>
	Sym(string);						// token
\end{lstlisting}


Хранение вложенных элементов реализовано через указатели на базовый тип
\class{Sym}. Благодаря виртуальному наследованию и использованию RTTI, этими
указателями можно пользоваться для работы с любыми другими наследованными типами
данных\note{числа, списки, высокоуровневые и скомпилированные функции,
элементы GUI,..}

\begin{lstlisting}[language=C++,
title=AST может хранить (и обрабатывать) вложенные элементы]
// --------------------------------------------------------- nest[]ed elements
	vector<Sym*> nest;
	void push(Sym*);
	void pop();
\end{lstlisting}

\begin{lstlisting}[language=C++,title=параметры (и поля класса)]
// -------------------------------------------------------------- par{}ameters
	map<string,Sym*> pars;
	void par(Sym*);						// add parameter
\end{lstlisting}

\begin{lstlisting}[language=C++,title=вывод дампа объекта в текстовом формате]
// ------------------------------------------------------------------- dumping
	virtual string dump(int depth=0);	// dump symbol object as text
	virtual string tagval();			// <T:V> header string
	string tagstr();					// <T:'V'> Str-like header string
	string pad(int);					// padding with tree decorators
\end{lstlisting}

\emph{Операции над \termdef{символами}{символ}\ выполняются через использование
набора \termdef{операторов}{оператор}:}

\begin{lstlisting}[language=C++,title=вычисление объекта]
// -------------------------------------------------------- compute (evaluate)
	virtual Sym* eval();
\end{lstlisting}

\begin{lstlisting}[language=C++,title=операторы]
// ----------------------------------------------------------------- operators
	virtual Sym* str();					// str(A)	string representation
	virtual Sym* eq(Sym*);				// A = B	assignment
	virtual Sym* inher(Sym*);			// A : B	inheritance
	virtual Sym* member(Sym*);			// A % B,C	named member (class slot)
	virtual Sym* at(Sym*);				// A @ B	apply
	virtual Sym* add(Sym*);				// A + B	add
	virtual Sym* div(Sym*);				// A / B	div
	virtual Sym* ins(Sym*);				// A += B	insert
};
\end{lstlisting}

\secrel{Скаляры}\label{biscalars}\secdown
\secrel{\class{str}: строка}
\secrel{\class{int}: целое число}
\secrel{\class{hex}: машинное hex}
\secrel{\class{bin}: бинарная строка}
\secrel{\class{num}: число с плавающей точкой}
\secup
\secrel{Композиты}\secdown
\secrel{\class{list}: плоский список}
\secrel{\class{cons}: cons-пара и списки в \lisp-стиле}
\secup
\secrel{Функционалы}\secdown
\secrel{\class{op}: оператор}
\secrel{\class{fn}: встроенная/скомпилированная функция}
\secrel{\class{lambda}: лямбда}
\secup
\secup


\secrel{Программирование в свободном синтаксисе: \prog{FSP}}\secdown

\secrel{Типичная структура проекта FSP: \textit{lexical skeleton}}\secdown

Скелет файловой структуры FSP-проекта = lexical skeleton = skelex 
\bigskip

\lstx{Создаем проект \prog{prog}\ из командной строки (\win):}{bi/fspskel.bat}
Создали каталог проекта, сгенерили набор пустых файлов (см. далее), и
запуститили батник-hepler который запустит \vim.

Для пользователей GitHub \verb|mkdir|\ надо заменить на 
\begin{verbatim}
git clone -o gh git@github.com:yourname/prog.git
cd prog
git gui &
...
\end{verbatim}

\bigskip

\begin{tabular}{l l l}
\file{src.src} & & исходный текст программы на вашем скриптовом языке \\
\file{log.log} & & лог работы ядра \bi\\
\file{ypp.ypp} & \prog{flex} & парсер \ref{biypp}\\
\file{lpp.lpp} & \prog{bison} & лексер \ref{bilpp}\\
\file{hpp.hpp} & \cpp & заголовочные файлы \ref{bihpp}\\
\file{cpp.cpp} & \cpp & код ядра \ref{bicpp}\\
\file{Makefile} & \prog{make} & зависимости между файлами и команды сборки (для
утилиты \prog{make})\\
\file{bat.bat} & \win & запускалка \prog{\vim}\ \ref{bibat}\\
\file{.gitignore} & \prog{git} & список масок временных и производных файлов
\ref{bigit}\\
\end{tabular}

\secrel{Настройки \vim}

При использовании редактора/IDE \prog{\vim}\ удобно настроить сочетания клавиш и
подсветку \emph{синтаксиса вашего скриптовго языка}\ так, как вам удобно. Для
этого нужно создать несколько файлов конфигурации .vim: по 2 файла\note{(1)
привязка расширения файла и (2) подсветка синтаксиса}\ для каждого диалекта
скрипт-языка\note{если вы пользуетесь сильно отличающимся синтаксисом, но
скорее всего через какое-то время практики FSP у вас выработается один диалект
для всех программ, соответсвующий именно вашим вкусам в синтаксисе, и в этом
случае его нужно будет описать только в файлах ~/.vim/(ftdetect|syntax).vim}, и
привязать их к расширениям через dot-файлы \vim\ в вашем домашнем каталоге.
Подробно конфигурирование \vim\ см. \ref{vim}. \bigskip 

\begin{tabular}{l l l}
filetype.vim & \vim & привязка расширений файлов (.src .log) к настройкам \vim
\\
syntax.vim & \vim & синтаксическая подсветка для скриптов \\
~/.vimrc & \linux & настройки для пользователя \\
~/vimrc & \win &\\
~/.vim/ftdetect/src.vim & \linux & привязка команд к расширению .src \\
~/vimfiles/ftdetect/src.vim & \win & \\
~/.vim/syntax/src.vim & \linux & синтаксис к расширению .src \\
~/vimfiles/syntax/src.vim & \win &\\
\end{tabular}

\secrel{Дополнительные файлы}

\begin{tabular}{l l l}
README.md & github & описание проекта для репоитория github \\
logo.png & github & логотип \\
logo.ico & \win & \\
rc.rc & \win & описание ресурсов: логотип, иконки приложения, меню,.. \\
\end{tabular}

% \file{rc.rc} & \ref{rc} & windres 
% 	& \\
% \file{logo.ico} && windres 
% 	& логотип в .ico формате \\
% \file{logo.png} &&
% 	& логотип в .png (для github README) \\
% \file{filetype.vim} & \ref{filetypevim} & (g)vim 
% 	& файлов cкриптов \\
% \file{syntax.vim} & \ref{syntaxvim} & (g)vim 
% 	&  \\
 
% \input{README.tex}

\secrel{Makefile}

Для сборки проекта используем команду \prog{make}\ или \prog{ming32-make}\ для
\win/\mingw. Прописываем в \file{Makefile}\ зависимости:

\lst{универсальный Makefile для fsp-проекта}{bi/Makefile}{make}

\begin{description}

\item{\file{./exe.exe}}\\ префикс ./ требуется для правильной работы
\prog{ming32-make}, поскольку в \linux\ исполняемый файл может иметь любое имя и
расширение, можем использовать .exe.

Для запуска транслятора используем простейший вариант\ --- перенаправление
потоков stdin/stdut на файлы, в этом случае не потребуется разбор параметров
командной строки, и получим подробную трассировку выполнения трансляции.

\item{переменные \var{C}\ и \var{H}} задают набор исходный файлов ядра
транслятора на \cpp:

\begin{description}
\item{\file{cpp.cpp}} реализация системы динамических типов данных,
наследованных от символьного типа AST \ref{ast}. Библиотека динамических классов
языка \bi\ \ref{bi}\ компактна, предоставляет достаточных набор типов
данных, и операций над ними. При необходимости вы можете легко написать свое
дерево классов, если вам достаточно только простого разбора.
\item{\file{hpp.hpp}} заголовочные файлы также используем из \bi\ \ref{bi}:
содержат декларации динамических типов и интерфейс лексического анализатора,
которые подходят для всех проектов
\item{ypp.tab.cpp ypp.tab.hpp} \cpp\ код синтаксического парсера, генерируемый
утилитой \prog{bison}\ \ref{parser}
\item{lex.yy.c} код лексического анализатора, генерируемый утилитой \prog{flex}\
\ref{flex}
\item{\var{CXXFLAGS}\verb| += gnu++11|} добавляем опцию диалекта \cpp,
необходимую для компиляции ядра \bi
\end{description}

\end{description}


% \secrel{bat.bat}\label{bat}
% 
% \lstx{bat.bat}{script/bat.bat}
% 
% \secrel{rc.rc}
% 
% \lstx{rc.rc}{script/rc.rc}
% 
% \bigskip\fig{}{../icons/hedgehog.png}{scale=2}

\secup
\secup
\secrel{Синтаксический анализ текстовых данных}\label{syntax}\secdown

\secrel{Универсальный \file{Makefile}}\label{lexmake}

Универсальный Makefile сделан на базе \ref{bimake}, с добавлением переменной
\var{APP}\ указывающий какой пример парсера следуует скомпилировать и выполнить.

Для хранения (и возможной обработки) отпарсенных данных используем ядро языка
\bi\ \ref{bicore}\ --- используем файлы \file{../bi/hpp.hpp}\ и
\file{../bi/cpp.cpp}. Ядро \emph{очень компактно}, но умеет работать со
скалярными, составными и функциональными данными, и содержит минимальную
реализацию \term{ядра динамического языка}.

\lstx{Универсальный \file{Makefile}}{parser/minimal.mk}

\secrel{\cpp\ интерфейс синтаксического анализатора}\label{lexinterface}

\begin{verbatim}
extern int yylex();             // получить код следующиго токена, и yylval.o 
extern int yylineno;            // номер текущей строки файла исходника
extern char* yytext;            // текст распознанного токена, asciiz
#define TOC(C,X) { yylval.o = new C(yytext); return X; }

extern int yyparse();           // отпарсить весь текущий входной поток токенов
extern void yyerror(string);    // callback вызывается при синтаксической ошибке
#include "ypp.tab.hpp"
\end{verbatim}

\secrel{Минимальный парсер}\label{miniparser}

Рассмотрим минимальный парсер, который может анализировать файлы текстовых
данных (например исходники программ), и вычленять из них последовательности
символов, которые можно отнести к \termdef{скалярам}: символ, строка и число.
\note{эти три типа можно назвать атомами computer science}

\lstx{Лексер \file{minimal.lpp}\
/\prog{flex}/}{parser/minimal.lpp}\label{minilexer}

\begin{description}
\item{(../bi/)\file{hpp.hpp}} содержит определения интерфейса лексера
\ref{lexinterface}, и ядра языка \bi\ \ref{bicore}\ для хранения результатов
разбора текстовых данных
\item{\var{noyywrap}} выключает использование функции \var{yywrap()}
\item{\var{yylineno}} включает отслеживание строки исходного файла, используется
при выводе сообщений об ошибках. В минимальном парсере не используется, но
требуется для сборки \bi-ядра.
\item{\verb|%%..%%|} набор правил группировки отдельных символов в
элементы данных\ --- \termdef{токены}{токен}, правила задаются с помощью \term{регулярных
выражений}
\item{\verb|TOC(Sym,SYM)|}\label{minisym} единственное правило, распознающее
любые группы сиволов как класс \class{bi::sym}: латинские буквы, цифры и символы
\_\ и .\ (точка)\note{точка добавлена, так часто используется в именах файлов}
\end{description}

\lstx{Парсер \file{minimal.ypp}\ /\prog{bison}/}{parser/minimal.ypp}

\begin{description}
\item{\file{hpp.hpp}} заголовок аналогичен лексеру \ref{minilexer}
\item{\verb|%defines %union|} указывает какие типы данных могут храниться в
узлах разобранного \termdef{синтаксического дерева}{синтаксическое дерево}.
Поскольку мы используем \bi-ядро, нам будет достаточно пользоваться
только классами языка \bi,
прежде всего универсальным символьным типом AST \ref{ast}\ и его прозводными
классами.
\item{\verb|%token|} описывает токены, которые может возвращать лексер
\ref{minlexer}, причем набор токенов должен быть согласованным между лексером и
парсером\note{определение токенов генерируется в файл \file{ypp.tab.hpp}}
\item{\verb|%type|} описывает типы синтаксических выражений, которые может
распознавать \termdef{грамматика}{грамматика} синтаксического анализатора, 
\item{\verb|REPL|} выражение, описывающее грамматику, аналогичную простейшему
варианту цикла REPL: Read Eval Print
Loop\note{чтение/вычисление/вывод/повторить}. В нашем случае часть вычисления
Eval не выполняется\note{разобранное выражение не вычисляется, хотя
используемое ядро \bi\ и поддерживает такой функционал}, а часть Print
выполняется через метод \verb|Sym.tagval()|, возвращающий котороткую строку
вида \verb|<класс:значение>|\ для найденного токена.
\item{\verb|ex|} (expression) универсальное символьное выражение языка \bi, в
нашем случае оно должно представлять только \verb|scalar|
\item{\verb|scalar|} выражение, представляющиее только распознаваемые скаляры:
\item{\verb|SYM|} символ, 
\item{\verb|STR|} строку \emph{или}
\item{\verb|NUM|} число\note{числа в грамматике языка \bi\ по типам не делятся,
токен соответствует как \class{int}, так и \class{num}}
\end{description}

В качестве тестового исходника возьмем \cpp\ код ядра языка \bi:
\file{../bi/cpp.cpp}:

\lstx{\file{minimal.src}: Тестовый исходник}{parser/minimal.src}

\lstx{\file{minimal.log}: Результат прогона}{parser/minimal.log}

Как видно по логу \file{minimal.log}, все группы сиволов, соответствующих
правилу лексера \verb|SYM|\ref{minisym}, распознались как объекты \bi, остальные
остались символами и попали в лог без изменений.


\secrel{Добавляем обработку комментариев}\label{minicomment}

В тестах программ и файлов конфигурации очень часто используются
\term{комментарии}. В языке \py, \bi\ и UNIX shell комментарием является все
от символа \#\ до конца строки.

Для обработки таких \termdef{строчных комментариев}{строчный комментарий}\
достаточно добавить одно правило лексера, \emph{обязательно первым правилом}:

\lstx{Лексер со строчными комментариями}{parser/comment.lpp}

Группа символов, начинающаяся с символа \#, затем идет ноль или более \verb|[]*|
любых символов не равных \verb|^|\ концу строки \verb|\n|.
Пустое тело правила: \cpp\ код в \verb|{}|\ скобках\ --- выполняется и ничего не
делает.

Тело правила SYM\ --- вызов макроса \verb|TOC(C,X)|, наоборот, при своем
выполнении создает токен, и возвращает код токена \verb|=SYM|.

\lstx{\file{comment.log}: Результат прогона}{parser/comment.log}

Как видно из лога, из вывода исчезли первые 2 строки, начинающиеся на \#, причем
концы этих строк остались (но не были как-либо распознаны).

\secrel{Разбор строк}\label{ministring}

Для разбора строк необходимо использовать лексер с применением
\termdef{состояний}{состояние лексера}. Строки имеют сильно отличающийся от
основного кода синтаксис, и для его обработки нужно \emph{переключать набор
правил лексера}.

\lstx{Лексер с состоянием для строк}{parser/string.lpp}\label{lexstring}

\begin{description}
\item{\verb|string LexString|} строковая буферная переменная, накапливающая
символы строки
\item{\verb|%x lexstring|} создание отдельного состояния лексера
\verb|lexstring|
\item{\verb|INITIAL|} основное состояние лексера
\item{\verb|<lexstring>\n|} правило конца строки позволяет использовать
многострочные строки\note{символ конца строки не распознается
метасимволом . (точка) в регулярном выражении, и требует явного указания}
\item{\verb|<lexstring>.|} любой символ в состоянии \verb|<lexstring>|
\end{description}

\lstx{Лог разбора со строками}{parser/string.log}

Обратите внимание, что ранее попадавшие в лог строки в двойных кавычках, типа
\verb|"]\n\n"|, стали распознаваться как строковые токены \verb|<str:']\n\n'>|.
\note{использованы 'одинарные кавычки' как в \py/\bi}

\secrel{Добавляем операторы}\label{miniops}

Для разбора языков программирования необходима поддержка операторов,
включим общепринятые одиночные операторы, операторы \cpp\ и \bi.
\emph{Скобки различного вида тоже будет рассматривать как операторы.}
Операторы реализованы в ядре \bi\ как отдельный класс \class{op}, зададим
пачку правил разбора операторов, создающих токены \verb|TOC(Op,XXX)|:

\lstx{Лексер с операторами}{parser/ops.lpp}

\lstx{Парсер с операторами}{parser/ops.ypp}

Лог уже стал нечитаем, переключаемся на древовидный вывод через метод
\verb|Sym.dump()|.

\lstx{Разбор с операторами}{parser/ops.log}

\secrel{Обработка вложенных структур (скобок)}

Обработка вложенных структур возможна только парсером, лексер оставляем
без изменений. Хранение вложенных структур в виде дерева\ --- главная фича
типа \bi\ AST\ref{ast}. Заменяем грамматическое выражение \verb|bracket|\ на
отдельные выражения для скобок:

\lstx{Парсер со скобками}{parser/brackets.ypp}

\lstx{Разбор со скобками}{parser/brackets.log}


\secup
\secrel{Синтаксический анализатор}\secdown

Синтаксис языка \bi\ был выбран алголо-подобным, более близким к современным
императивным языкам типа \cpp\ и \py. Использование типовых утилит-генераторов
позволяет легко описать синтаксис с инфиксными операторами и скобочной записью
для композитных типов\,\ref{bicompose}, и не заставлять пользователя
закапываться в море \lispовских скобок.

Инфиксный синтаксис для файлов конфигурации удобен неподготовленным
пользователям, а возможность определения пользовательских функций и объектная
система, встроенная в ядро \bi, дает богатейшие возможности по настройке
и кастомизации программ.

Единственной проблемой с точки зрения синтаксиса для начинающего пользователя
\bi\ может оказаться отказ от скобок при вызове функций, применение оператора
явной аппликации \verb|@|, и функциональные наклонности самого \bi,
претендующего на звание универсального \emph{объектного мета-языка}\ и
\emph{языка шаблонов}.

\secrel{\file{lpp.lpp}: лексер /flex/}\label{bilexer}
\secrel{\file{ypp.ypp}: парсер /bison/}\label{biparser}
\secup

\secup
\secrel{\prog{skelex}: скелет программы в свободном
синтаксисе}\label{skelex}\secdown

В этом разделе описана общая структура любого проекта, использующего принципы
\term{программирования в свободном синтаксисе}, в виде примера определения
синтаксиса и семантики языка \bi.

Материал дублирует другие разделы, но может быть использован как вариант
\emph{минимизированного} языкового ядра FSP-проекта: нет комментариев, лишних
классов, подробного описания работы ядра и т.п., \emph{только краткие пояснения
и минимальный код}.

\secly{Структура проекта}

\lstx{Создание проекта}{skelex/mkproject.rc}

\begin{tabular}{l l l}
src.src & \bi & текст программы в свободном синтаксисе\\
log.log & \bi & лог интерпретатора \\
ypp.ypp & \prog{bison} & парсер синтаксиса \\
lpp.lpp & \prog{flex} & лексер \\
hpp.hpp & \cpp & хедеры \\
cpp.cpp & \cpp & ядро интерпретатора \\
Makefile & \prog{make} & скрипты сборки проекта \\
.gitignore & \prog{git} & маски файлов, не попадающие в git-проект\\
bat.bat & \win & helper запуска \vim \\
\end{tabular}

\lstx{\file{.gitignore}}{skelex/git.ignore}
\lstx{\file{bat.bat}}{skelex/bat.bat}

\subsecly{\file{Makefile}}
\lst{\file{Makefile}}{skelex/skelex.mk}{make}

\begin{description}
\item[\var{MODULE}] имя программного модуля, в примере получается
автоматически из имени каталога проекта; при компиляции интерпретатора
добавляется как глобальная константа, и может быть использована в скриптах. 
\item[\var{TAIL}] \verb$= -n7|-n17|<none>$ при успешном выполнении
интерпретатора выводятся последние \verb|$(TAIL)| строк лога, при отладке
скриптов удобно добавлять \emph{в конец программы} вывод отладочной информации.
Конкретное значение параметра команды \prog{tail} выбирается в зависимости от
настроек вашей IDE, для \prog{eclipse} на старом 15"\ мониторе мне удобен
\verb|TAIL=-n7|, для \vim\ и командной строки можно увеличить до
\verb|TAIL=-n17|.
\item[\var{CURDIR}] полный путь для текущего каталога
\item[\var{\$(notdir \ldots)}] функция выделяет из полного пути 
последний /элемент
\end{description}

\subsecly{\file{ypp.ypp}: синтаксический парсер}

Весь код между \verb|%{...%}| будет скопирован в выходной сгенерированный файл
\file{ypp.tab.cpp}

\lstx{Заголовочная часть с \cpp\ кодом}{skelex/ypp/head.ypp}

\lstx{используем универсальный тип для хранения дерева
разбора}{skelex/ypp/union.ypp}

\lstx{токены для скалярных типов}{skelex/ypp/tokscalars.ypp}
\lstx{правило для скалярных типов}{skelex/ypp/scalar.ypp}

\emph{символ, число и строка\ --- атомы информатики}

\lstx{токены для скобок}{skelex/ypp/brackets.ypp}

[L]eft/[R]ight [P]arens, [Q]uad, [C]url

\lstx{пачка операторов\ \ref{lexops}}{skelex/ypp/ops.ypp}

\lstx{типы выражений}{skelex/ypp/type.ypp}

\lstx{правила парсера помещаются между}{skelex/lpp/pp.lpp}

\lstx{REPL-цикл интерпретатора}{skelex/ypp/repl.ypp}

\lstx{скаляры}{skelex/ypp/scalar.ypp}

\lstx{выражения}{skelex/ypp/ex.ypp}

\lstx{списки}{skelex/ypp/list.ypp}

\lstx{лямбда-определения}{skelex/ypp/lambda.ypp}

\subsecly{\file{lpp.lpp}: лексер}

Весь код между \verb|%{...%}| будет скопирован в выходной сгенерированный файл
\file{lex.yy.c}

\lstx{Заголовочная часть с \cpp\ кодом}{skelex/lpp/head.lpp}

определена дополнительная переменная \var{LexString}: буфер используемый при
разборе строк.

\lstx{опция}{skelex/lpp/yywrap.lpp}

подавляет вывод сообщений об отсутствии функции \var{yywrap} 

\lstx{опция включения счетчика нумерации строк}{skelex/lpp/lineno.lpp}

сохраняет в переменной \var{yylineno} номер текущей строки

\lstx{правила лексера помещаются между}{skelex/lpp/pp.lpp}

\lstx{строчные комментарии}{skelex/lpp/comment.lpp}

\lstx{разбор строк через специальное состояние лексера}{skelex/lpp/xstring.lpp}
\lstx{\ }{skelex/lpp/string.lpp}

\lstx{распознавание числел}{skelex/lpp/xnum.lpp}
\lstx{\ }{skelex/lpp/num.lpp}

\subsecly{\file{hpp.hpp}: хедеры}

\lst{\ }{skelex/hpp/head.hpp}{C++}
все остальное находится между препроцессорными ``скобками'',
блокирующими многократное включение кода
\lst{\ }{skelex/hpp/foot.hpp}{C++}

\lst{\var{\#include}}{skelex/hpp/include.hpp}{C++}

\lst{универсальный тип: Abstract Symbolic Type}{skelex/hpp/sym.hpp}{C++}

\lst{глобальная среда (таблица символов)}{skelex/hpp/env.hpp}{C++}

\lst{скаляры: строки}{skelex/hpp/string.hpp}{C++}
\lst{скаляры: числа}{skelex/hpp/num.hpp}{C++}

\lst{композиты}{skelex/hpp/list.hpp}{C++}

\lst{функционалы: оператор}{skelex/hpp/op.hpp}{C++}
\lst{встроенные функции}{skelex/hpp/fn.hpp}{C++}
\lst{лямбда-фукнции}{skelex/hpp/lambda.hpp}{C++}

\lst{интерфейс к лексеру/парсеру}{skelex/hpp/lex.hpp}{C++}

\subsecly{\file{cpp.cpp}: ядро интерпретатора}

\lst{\ }{skelex/cpp/hpp.cpp}{C++}
\lst{обработка ошибок синтаксического анализатора}{skelex/cpp/error.cpp}{C++}
\lst{функция \var{main()}}{skelex/cpp/main.cpp}{C++}
\lst{конструкторы AST}{skelex/cpp/sym.cpp}{C++}
\lst{дамп AST}{skelex/cpp/dump.cpp}{C++}
\lst{вычисление AST}{skelex/cpp/eval.cpp}{C++}

\lst{строки и \class{Sym::tagstr()}}{skelex/cpp/string.cpp}{C++}

\lst{числа}{skelex/cpp/num.cpp}{C++}

\lst{композиты}{skelex/cpp/list.cpp}{C++}

\lst{функционалы: оператор}{skelex/cpp/op.cpp}{C++}
\lst{встроенная функция}{skelex/cpp/fn.cpp}{C++}
\lst{лямбда-функция}{skelex/cpp/lambda.cpp}{C++}

\lst{глобальная таблица символов}{skelex/cpp/env.cpp}{C++}

\secly{Тестирование интерпретатора}\label{lextest}

\subsecly{Комментарии}

\lstx{\file{test/comment.src}}{skelex/src/comment.src}
\lstx{\file{test/comment.log}}{skelex/comment.log}

\subsecly{Скаляры и базовые композиты}

\lstx{\file{test/coretypes.src}}{skelex/src/coretypes.src}
\lstx{\file{test/coretypes.log}}{skelex/coretypes.log}

\secly{Операторы}\label{lexops}

\noindent\begin{tabular}{l l l}
\verb|A+B| & add & сложение \\
\verb|A-B| & sub & вычистание \\
\verb|A*B| & mul & умножение \\
\verb|A/B| & div & деление \\
\verb|A^B| & pow & возведение в степень \\
\verb|A>>B| & rsh & правый сдвиг \\
\verb|A<<B| & lsh & левый сдвиг \\
\hline
\end{tabular}

\noindent\begin{tabular}{l l l}
\verb|A>B| & great & больше \\
\verb|A=>B| & greateq & больше или равно \\
\verb|A<B| & less & меньше \\
\verb|A<=B| & lesseq & меньше или равно \\
\verb|A==B| & eq & равно \\
\verb|A!=B| & noteq & неравно\\
\verb|A&B| & and & и\\
\verb$A|B$ & or & или\\
\verb$A^B$ & xor & исключающее или\\
\verb$!A$ & not & не\\
\hline
\end{tabular}

\noindent\begin{tabular}{l l l}
\verb|A=B| & assign & назначение/присвоение переменной\\&&\emph{$A$
предварительно вычисляется}, результат является указателем на переменную\\
\verb|A@B| & apply & применение (функции) $A$ к (параметру) $B$\\
&&применимо не только к функциям: в общей случае $A$ может быть любым типом\\
&&в том числе классом: в роли конструктора объекта\\
\verb|~A| & quote & \emph{блокировка вычисления} выражения $A$ \\
\verb$A||B$ & map & применить распределенно $A$ \emph{к членам} $B$\\
&& функция \var{map}: $A$ функция, вычислить список $\rightarrow$ список\\
&& параллельное вычисление: $A$ constant-функция $f(x)=x$\\
&& \verb$A@B$ вычисляются параллельно при наличии поддержки в ядре
интерпретатора\\
\hline
\end{tabular}

\noindent\begin{tabular}{l l l}
-фукнци-фукнциии\verb|A%B| & member & вложить $B$ как член $A$ \\
&&чаще всего используется в определении (добавлении) членов класса\\
\verb|A:B| & inherit & наследовать $B$ от $A$ \\
&&если $A$ составное, выполняется множественное наследование в порядке
итерации\\
&&если $A$ \emph{не класс}, выполняется наследование копированием\\
\verb|A.B| & index & доступ по индексу: $B$-ый член $A$\\
&&$B$ может быть именем или числовым индексом вложенного элемента из $A$\\
\hline
\end{tabular}

\noindent\begin{tabular}{l l l}
\verb|A<>B| & symm & симметричное правило замены $A\leftrightarrow B$\\
\verb|A>>B| & is & одностороннее правило замены $A\rightarrow B$\\
\verb|A<!>B| & notsym & симмектричный запрет замены $A\cancel\leftrightarrow
B$\\
\verb|A!>B| & notis & односторонний запрет замены $A\cancel\rightarrow B$\\
\end{tabular}

\secup

% \secrel{Программирование в свободном синтаксисе: \prog{FSP}}\secdown

\secrel{Типичная структура проекта FSP: \textit{lexical skeleton}}\secdown

Скелет файловой структуры FSP-проекта = lexical skeleton = skelex 
\bigskip

\lstx{Создаем проект \prog{prog}\ из командной строки (\win):}{bi/fspskel.bat}
Создали каталог проекта, сгенерили набор пустых файлов (см. далее), и
запуститили батник-hepler который запустит \vim.

Для пользователей GitHub \verb|mkdir|\ надо заменить на 
\begin{verbatim}
git clone -o gh git@github.com:yourname/prog.git
cd prog
git gui &
...
\end{verbatim}

\bigskip

\begin{tabular}{l l l}
\file{src.src} & & исходный текст программы на вашем скриптовом языке \\
\file{log.log} & & лог работы ядра \bi\\
\file{ypp.ypp} & \prog{flex} & парсер \ref{biypp}\\
\file{lpp.lpp} & \prog{bison} & лексер \ref{bilpp}\\
\file{hpp.hpp} & \cpp & заголовочные файлы \ref{bihpp}\\
\file{cpp.cpp} & \cpp & код ядра \ref{bicpp}\\
\file{Makefile} & \prog{make} & зависимости между файлами и команды сборки (для
утилиты \prog{make})\\
\file{bat.bat} & \win & запускалка \prog{\vim}\ \ref{bibat}\\
\file{.gitignore} & \prog{git} & список масок временных и производных файлов
\ref{bigit}\\
\end{tabular}

\secrel{Настройки \vim}

При использовании редактора/IDE \prog{\vim}\ удобно настроить сочетания клавиш и
подсветку \emph{синтаксиса вашего скриптовго языка}\ так, как вам удобно. Для
этого нужно создать несколько файлов конфигурации .vim: по 2 файла\note{(1)
привязка расширения файла и (2) подсветка синтаксиса}\ для каждого диалекта
скрипт-языка\note{если вы пользуетесь сильно отличающимся синтаксисом, но
скорее всего через какое-то время практики FSP у вас выработается один диалект
для всех программ, соответсвующий именно вашим вкусам в синтаксисе, и в этом
случае его нужно будет описать только в файлах ~/.vim/(ftdetect|syntax).vim}, и
привязать их к расширениям через dot-файлы \vim\ в вашем домашнем каталоге.
Подробно конфигурирование \vim\ см. \ref{vim}. \bigskip 

\begin{tabular}{l l l}
filetype.vim & \vim & привязка расширений файлов (.src .log) к настройкам \vim
\\
syntax.vim & \vim & синтаксическая подсветка для скриптов \\
~/.vimrc & \linux & настройки для пользователя \\
~/vimrc & \win &\\
~/.vim/ftdetect/src.vim & \linux & привязка команд к расширению .src \\
~/vimfiles/ftdetect/src.vim & \win & \\
~/.vim/syntax/src.vim & \linux & синтаксис к расширению .src \\
~/vimfiles/syntax/src.vim & \win &\\
\end{tabular}

\secrel{Дополнительные файлы}

\begin{tabular}{l l l}
README.md & github & описание проекта для репоитория github \\
logo.png & github & логотип \\
logo.ico & \win & \\
rc.rc & \win & описание ресурсов: логотип, иконки приложения, меню,.. \\
\end{tabular}

% \file{rc.rc} & \ref{rc} & windres 
% 	& \\
% \file{logo.ico} && windres 
% 	& логотип в .ico формате \\
% \file{logo.png} &&
% 	& логотип в .png (для github README) \\
% \file{filetype.vim} & \ref{filetypevim} & (g)vim 
% 	& файлов cкриптов \\
% \file{syntax.vim} & \ref{syntaxvim} & (g)vim 
% 	&  \\
 
% \input{README.tex}

\secrel{Makefile}

Для сборки проекта используем команду \prog{make}\ или \prog{ming32-make}\ для
\win/\mingw. Прописываем в \file{Makefile}\ зависимости:

\lst{универсальный Makefile для fsp-проекта}{bi/Makefile}{make}

\begin{description}

\item{\file{./exe.exe}}\\ префикс ./ требуется для правильной работы
\prog{ming32-make}, поскольку в \linux\ исполняемый файл может иметь любое имя и
расширение, можем использовать .exe.

Для запуска транслятора используем простейший вариант\ --- перенаправление
потоков stdin/stdut на файлы, в этом случае не потребуется разбор параметров
командной строки, и получим подробную трассировку выполнения трансляции.

\item{переменные \var{C}\ и \var{H}} задают набор исходный файлов ядра
транслятора на \cpp:

\begin{description}
\item{\file{cpp.cpp}} реализация системы динамических типов данных,
наследованных от символьного типа AST \ref{ast}. Библиотека динамических классов
языка \bi\ \ref{bi}\ компактна, предоставляет достаточных набор типов
данных, и операций над ними. При необходимости вы можете легко написать свое
дерево классов, если вам достаточно только простого разбора.
\item{\file{hpp.hpp}} заголовочные файлы также используем из \bi\ \ref{bi}:
содержат декларации динамических типов и интерфейс лексического анализатора,
которые подходят для всех проектов
\item{ypp.tab.cpp ypp.tab.hpp} \cpp\ код синтаксического парсера, генерируемый
утилитой \prog{bison}\ \ref{parser}
\item{lex.yy.c} код лексического анализатора, генерируемый утилитой \prog{flex}\
\ref{flex}
\item{\var{CXXFLAGS}\verb| += gnu++11|} добавляем опцию диалекта \cpp,
необходимую для компиляции ядра \bi
\end{description}

\end{description}


% \secrel{bat.bat}\label{bat}
% 
% \lstx{bat.bat}{script/bat.bat}
% 
% \secrel{rc.rc}
% 
% \lstx{rc.rc}{script/rc.rc}
% 
% \bigskip\fig{}{../icons/hedgehog.png}{scale=2}

\secup
\secup
% \secrel{Синтаксический анализ текстовых данных}\label{syntax}\secdown

\secrel{Универсальный \file{Makefile}}\label{lexmake}

Универсальный Makefile сделан на базе \ref{bimake}, с добавлением переменной
\var{APP}\ указывающий какой пример парсера следуует скомпилировать и выполнить.

Для хранения (и возможной обработки) отпарсенных данных используем ядро языка
\bi\ \ref{bicore}\ --- используем файлы \file{../bi/hpp.hpp}\ и
\file{../bi/cpp.cpp}. Ядро \emph{очень компактно}, но умеет работать со
скалярными, составными и функциональными данными, и содержит минимальную
реализацию \term{ядра динамического языка}.

\lstx{Универсальный \file{Makefile}}{parser/minimal.mk}

\secrel{\cpp\ интерфейс синтаксического анализатора}\label{lexinterface}

\begin{verbatim}
extern int yylex();             // получить код следующиго токена, и yylval.o 
extern int yylineno;            // номер текущей строки файла исходника
extern char* yytext;            // текст распознанного токена, asciiz
#define TOC(C,X) { yylval.o = new C(yytext); return X; }

extern int yyparse();           // отпарсить весь текущий входной поток токенов
extern void yyerror(string);    // callback вызывается при синтаксической ошибке
#include "ypp.tab.hpp"
\end{verbatim}

\secrel{Минимальный парсер}\label{miniparser}

Рассмотрим минимальный парсер, который может анализировать файлы текстовых
данных (например исходники программ), и вычленять из них последовательности
символов, которые можно отнести к \termdef{скалярам}: символ, строка и число.
\note{эти три типа можно назвать атомами computer science}

\lstx{Лексер \file{minimal.lpp}\
/\prog{flex}/}{parser/minimal.lpp}\label{minilexer}

\begin{description}
\item{(../bi/)\file{hpp.hpp}} содержит определения интерфейса лексера
\ref{lexinterface}, и ядра языка \bi\ \ref{bicore}\ для хранения результатов
разбора текстовых данных
\item{\var{noyywrap}} выключает использование функции \var{yywrap()}
\item{\var{yylineno}} включает отслеживание строки исходного файла, используется
при выводе сообщений об ошибках. В минимальном парсере не используется, но
требуется для сборки \bi-ядра.
\item{\verb|%%..%%|} набор правил группировки отдельных символов в
элементы данных\ --- \termdef{токены}{токен}, правила задаются с помощью \term{регулярных
выражений}
\item{\verb|TOC(Sym,SYM)|}\label{minisym} единственное правило, распознающее
любые группы сиволов как класс \class{bi::sym}: латинские буквы, цифры и символы
\_\ и .\ (точка)\note{точка добавлена, так часто используется в именах файлов}
\end{description}

\lstx{Парсер \file{minimal.ypp}\ /\prog{bison}/}{parser/minimal.ypp}

\begin{description}
\item{\file{hpp.hpp}} заголовок аналогичен лексеру \ref{minilexer}
\item{\verb|%defines %union|} указывает какие типы данных могут храниться в
узлах разобранного \termdef{синтаксического дерева}{синтаксическое дерево}.
Поскольку мы используем \bi-ядро, нам будет достаточно пользоваться
только классами языка \bi,
прежде всего универсальным символьным типом AST \ref{ast}\ и его прозводными
классами.
\item{\verb|%token|} описывает токены, которые может возвращать лексер
\ref{minlexer}, причем набор токенов должен быть согласованным между лексером и
парсером\note{определение токенов генерируется в файл \file{ypp.tab.hpp}}
\item{\verb|%type|} описывает типы синтаксических выражений, которые может
распознавать \termdef{грамматика}{грамматика} синтаксического анализатора, 
\item{\verb|REPL|} выражение, описывающее грамматику, аналогичную простейшему
варианту цикла REPL: Read Eval Print
Loop\note{чтение/вычисление/вывод/повторить}. В нашем случае часть вычисления
Eval не выполняется\note{разобранное выражение не вычисляется, хотя
используемое ядро \bi\ и поддерживает такой функционал}, а часть Print
выполняется через метод \verb|Sym.tagval()|, возвращающий котороткую строку
вида \verb|<класс:значение>|\ для найденного токена.
\item{\verb|ex|} (expression) универсальное символьное выражение языка \bi, в
нашем случае оно должно представлять только \verb|scalar|
\item{\verb|scalar|} выражение, представляющиее только распознаваемые скаляры:
\item{\verb|SYM|} символ, 
\item{\verb|STR|} строку \emph{или}
\item{\verb|NUM|} число\note{числа в грамматике языка \bi\ по типам не делятся,
токен соответствует как \class{int}, так и \class{num}}
\end{description}

В качестве тестового исходника возьмем \cpp\ код ядра языка \bi:
\file{../bi/cpp.cpp}:

\lstx{\file{minimal.src}: Тестовый исходник}{parser/minimal.src}

\lstx{\file{minimal.log}: Результат прогона}{parser/minimal.log}

Как видно по логу \file{minimal.log}, все группы сиволов, соответствующих
правилу лексера \verb|SYM|\ref{minisym}, распознались как объекты \bi, остальные
остались символами и попали в лог без изменений.


\secrel{Добавляем обработку комментариев}\label{minicomment}

В тестах программ и файлов конфигурации очень часто используются
\term{комментарии}. В языке \py, \bi\ и UNIX shell комментарием является все
от символа \#\ до конца строки.

Для обработки таких \termdef{строчных комментариев}{строчный комментарий}\
достаточно добавить одно правило лексера, \emph{обязательно первым правилом}:

\lstx{Лексер со строчными комментариями}{parser/comment.lpp}

Группа символов, начинающаяся с символа \#, затем идет ноль или более \verb|[]*|
любых символов не равных \verb|^|\ концу строки \verb|\n|.
Пустое тело правила: \cpp\ код в \verb|{}|\ скобках\ --- выполняется и ничего не
делает.

Тело правила SYM\ --- вызов макроса \verb|TOC(C,X)|, наоборот, при своем
выполнении создает токен, и возвращает код токена \verb|=SYM|.

\lstx{\file{comment.log}: Результат прогона}{parser/comment.log}

Как видно из лога, из вывода исчезли первые 2 строки, начинающиеся на \#, причем
концы этих строк остались (но не были как-либо распознаны).

\secrel{Разбор строк}\label{ministring}

Для разбора строк необходимо использовать лексер с применением
\termdef{состояний}{состояние лексера}. Строки имеют сильно отличающийся от
основного кода синтаксис, и для его обработки нужно \emph{переключать набор
правил лексера}.

\lstx{Лексер с состоянием для строк}{parser/string.lpp}\label{lexstring}

\begin{description}
\item{\verb|string LexString|} строковая буферная переменная, накапливающая
символы строки
\item{\verb|%x lexstring|} создание отдельного состояния лексера
\verb|lexstring|
\item{\verb|INITIAL|} основное состояние лексера
\item{\verb|<lexstring>\n|} правило конца строки позволяет использовать
многострочные строки\note{символ конца строки не распознается
метасимволом . (точка) в регулярном выражении, и требует явного указания}
\item{\verb|<lexstring>.|} любой символ в состоянии \verb|<lexstring>|
\end{description}

\lstx{Лог разбора со строками}{parser/string.log}

Обратите внимание, что ранее попадавшие в лог строки в двойных кавычках, типа
\verb|"]\n\n"|, стали распознаваться как строковые токены \verb|<str:']\n\n'>|.
\note{использованы 'одинарные кавычки' как в \py/\bi}

\secrel{Добавляем операторы}\label{miniops}

Для разбора языков программирования необходима поддержка операторов,
включим общепринятые одиночные операторы, операторы \cpp\ и \bi.
\emph{Скобки различного вида тоже будет рассматривать как операторы.}
Операторы реализованы в ядре \bi\ как отдельный класс \class{op}, зададим
пачку правил разбора операторов, создающих токены \verb|TOC(Op,XXX)|:

\lstx{Лексер с операторами}{parser/ops.lpp}

\lstx{Парсер с операторами}{parser/ops.ypp}

Лог уже стал нечитаем, переключаемся на древовидный вывод через метод
\verb|Sym.dump()|.

\lstx{Разбор с операторами}{parser/ops.log}

\secrel{Обработка вложенных структур (скобок)}

Обработка вложенных структур возможна только парсером, лексер оставляем
без изменений. Хранение вложенных структур в виде дерева\ --- главная фича
типа \bi\ AST\ref{ast}. Заменяем грамматическое выражение \verb|bracket|\ на
отдельные выражения для скобок:

\lstx{Парсер со скобками}{parser/brackets.ypp}

\lstx{Разбор со скобками}{parser/brackets.log}


\secup
% \secrel{Синтаксический анализатор}\secdown

Синтаксис языка \bi\ был выбран алголо-подобным, более близким к современным
императивным языкам типа \cpp\ и \py. Использование типовых утилит-генераторов
позволяет легко описать синтаксис с инфиксными операторами и скобочной записью
для композитных типов\,\ref{bicompose}, и не заставлять пользователя
закапываться в море \lispовских скобок.

Инфиксный синтаксис для файлов конфигурации удобен неподготовленным
пользователям, а возможность определения пользовательских функций и объектная
система, встроенная в ядро \bi, дает богатейшие возможности по настройке
и кастомизации программ.

Единственной проблемой с точки зрения синтаксиса для начинающего пользователя
\bi\ может оказаться отказ от скобок при вызове функций, применение оператора
явной аппликации \verb|@|, и функциональные наклонности самого \bi,
претендующего на звание универсального \emph{объектного мета-языка}\ и
\emph{языка шаблонов}.

\secrel{\file{lpp.lpp}: лексер /flex/}\label{bilexer}
\secrel{\file{ypp.ypp}: парсер /bison/}\label{biparser}
\secup



\secrel{em\linux\ для встраиваемых систем}\secdown
\secly{Структура встраиваемого микро\linux а}

\begin{description}
  \item{\prog{syslinux}}\ Загрузчик

em\linux\ поставляется в виде двух файлов:

\begin{enumerate}[nosep]
  \item ядро \verb|$(HW)$(APP).kernel|
  \item сжатый образ корневой файловой системы \verb|$(HW)$(APP).rootfs|
\end{enumerate}

Загрузчик считывает их с одного из носителей данных, который поддерживается
загрузчиком\note{IDE/SATA HDD, CDROM, USB флешка, сетевая загрузка с
BOOTP-сервера по Ethernet}, распаковывает в память, включив защищенный режим
процессора, и передает управление ядру \linux.

\begin{framed}\noindent
Для работы em\linux\ не требуются какие-либо носители данных: вся
(виртуальная) файловая система располагается в ОЗУ. При необходимости к любому
из каталогов корневой ФС может быть \term{подмонтирована}\ любая существующая
дисковая или сетевая файловая система (FAT,NTFS,Samba,NFS,..), причем можно явно
запретить возможность записи на нее, защитив данные от
разрушения.

\emph{Использование rootfs в ОЗУ позволяет гарантировать защиту базовой ОС и
пользовательских исполняемых файлов от внезапных выключений питания и
ошибочной записи на диск. Еще большую защиту даст отключение драйверов
загрузочного носителя в ядре.} Если отключить драйвера SATA/IDE и грузиться
с USB флешки, практически невозможно испортить основную установку ОС и
пользовательские файлы на чужом компьютере.
\end{framed}
   
  \item{\prog{kernel}}\ Ядро \linux\ 3.13.xx
  \item{\prog{ulibc}}\ Базовая библиотека языка Си
  \item{\prog{busybox}}\ Ядро командной среды UNIX, базовые сетевые сервера
  \item{дополнительные библиотеки}
  \begin{description}[nosep]
    \item{\prog{zlib}} сжатие/распаковка gzip
    \item{\prog{???}} библиотека помехозащищенного кодирования
    \item{\prog{png}} библиотека чтения/записи графического формата .png 
    \item{\prog{freetype}} рендер векторных шрифтов (TTF) 
	\item{\prog{SDL}} полноэкранная (игровая) графика, аудио, джойстик
  \end{description}
  \item{кодеки аудио/видео форматов}: ogg vorbis, mp3, mpeg, ffmpeg/gsm
\end{description}

\begin{framed}\noindent
К базовой системе добавляются пользовательские
программы \dir{/usr/bin}\\ и динамические библиотеки \dir{/usr/lib}.
\end{framed}


\secly{Процедура сборки}
\secrel{\file{clock}: коридорные электронные часы = контроллер умного дурдома}
\secrel{\file{gambox}: игровая приставка}
\secup

\secrel{GNU Toolchain и \cpp\ для встраиваемых систем}\secdown
\secrel{Программирование встраиваемых систем с использованием \gnut\
\cite{kumar}}\label{kumaru}\secdown \copyright\ Vijay Kumar B.\ 
\cp{http://bravegnu.org/gnu-eprog/}
перевод \cp{https://github.com/ponyatov/gnu-eprog/blob/ru/gnu-eprog.asciidoc}

\secrel{Введение}

Пакет компиляторов GNU toolchain широко используется при разработке программного
обеспечения для встраиваемых систем. Этот тип разработки ПО также называют
\termdef{низкоуровневым}{низкоуровневое программирование},
\termdef{standalone}{standalone}\ или \termdef{bare metal}{bare metal}\
программированием (на \ci\ и \cpp). Написание низкоуровневого кода на \ci\
добавляет программисту новых проблем, требующих глубокого понимания инструмента
разработчика\ --- \gnut. Руководства разработчика \gnut\ предоставляют отличную
информацию по инструментарию, но с точки зрения самого \gnut, чем с точки зрения
решаемой проблемы. Поэтому было написано это руководство, в котором будут
описаны типичные проблемы, с которыми сталкивается начинающий разработчик.

Этот учебник стремится занять свое место, объясняя использование \gnut\ с точки
зрения практического использования. Надеемся, что его будет достаточно для
разработчиков, собирающихся освоить и использовать \gnut\ в их embedded
проектах.

В иллюстративных целях была выбрана встроенная система на базе процессорного
ядра ARM, которая эмулируется в пакете \qemu. С таким подходом вы сможете
освоить \gnut\ с комфортом на вашем рабочем компьютере, без необходимости
вкладываться в ``физическое''\ железо, и бороться со сложностями с его запуском.
Учебник не стремиться обучить работе с архитектурой ARM, для этого вам нужно
будет воспользоваться дополнительными книгами или онлайн-учебниками типа:

\begin{itemize}[nosep]
  \item ARM Assembler\ \url{http://www.heyrick.co.uk/assembler/}
  \item ARM Assembly Language Programming\
  \url{http://www.arm.com/miscPDFs/9658.pdf}
\end{itemize}

Но для удобства читателя, некоторое множество часто используемых ARM-инструкций
описано в приложении \ref{kumarB}.

\secrel{Настройка тестового стенда}\secdown

В этом разделе описано, как настроить на вашей рабочей станции простую среду
разработки и тестирования ПО для платформы ARM, используя \qemu\ и \gnut.
\qemu\ это программный\note{для i386\ --- программно-аппаратный, использует
средства виртуализации хост-компьютера} эмулятор нескольких распространенных
аппаратных платформ. Вы можете написать программу на ассемблере и \cpp,
скомпилировать ее используя \gnut\ и отладить ее в эмуляторе \qemu.

\secrel{\qemu\ ARM}

Будем использовать \qemu\ для эмуляции отладочной платы \prog{Gumstix connex}\ на базе
процессора PXA255. Для работы с этим учебником у вас должен быть установлен
\qemu\ версии не ниже 0.9.1.

Процессор\note{Точнее SoC: система-на-кристалле}\ PXA255 имеет ядро ARM c
набором инструкций ARMv5TE. PXA255 также имеет в своем составе несколько блоков
периферии. Некоторая периферия будет описана в этом курсе далее.

\secrel{Инсталляция \qemu\ на \debian}

Этот учебник требует \qemu\ версии не ниже 0.9.1. Пакет \qemu\ доступный для
современных дистрибутивов \debian, вполне удовлетворяет этим условиям, и
собирать свежий \qemu\ из исходников совсем не требуется\note{хотя может быть и
очень хочется}. Установим пакет командой:

\begin{verbatim}
$ sudo apt install qemu
\end{verbatim}

\secrel{Установка кросс-компилятора \gnut\ для ARM}

Если вы предпочитаете простые пути, установите пакет кросс-компилятора командной

\begin{verbatim}
sudo apt install gcc-arm-none-eabi
\end{verbatim}

или

\begin{enumerate}
  \item Годные чуваки из CodeSourcery\note{подразделение Mentor
  Graphics}\ уже давно запилили несколько вариантов \gnut ов для разных ходовых
  архитектур. Скачайте готовую бинарную бесплатную lite-сборку
  \href{http://www.mentor.com/embedded-software/sourcery-tools/sourcery-codebench/editions/lite-edition/}{\gnut-ARM}
  
\item Распакуйте tar-архив в каталог \dir{~/toolchains}:

\begin{verbatim}
$ mkdir ~/toolchains
$ cd ~/toolchains
$ tar -jxf ~/downloads/arm-2008q1-126-arm-none-eabi-i686-pc-linux-gnu.tar.bz2
\end{verbatim}

  \item Добавьте bin-каталог тулчейна в переменную среды \var{PATH}.

\begin{verbatim}
$ PATH=$HOME/toolchains/arm-2008q1/bin:$PATH 
\end{verbatim}

\item Чтобы каждый раз не выполнять предыдущую команду, вы можете прописать
ее в дот-файл \file{.bashrc}.

\end{enumerate}

Для совсем упертых подойдет рецепт сборки полного комплекта кросс-компилятора из
исходных тектов, описанный в \ref{cross}.

\secup
\secrel{Hello ARM}\secdown

В этом разделе вы научитесь пользоваться arm-ассемблером, и тестировать вашу
программу на голом железе\ --- эмуляторе платы \prog{connex}\ в \qemu.

Файл исходника ассемблерной программы состоит из последовательности инструкций,
по одной на каждую строку. Каждая инструкция имеет формат (каждый
компонент не обязателен):

\begin{verbatim}
<метка>:    <инструкция>         @ <комментарий>
\end{verbatim}

\begin{description}
\item[метка]\ --- типичный способ пометить адрес инструкции в памяти. Метка
может быть использована там, где требуется указать адрес, например как операнд
в команде перехода. Метка может состоять из латинских букв, цифр\note{не может
быть первым символом метки}, символов \verb|_|\ и \verb|$|.
\item[комментарий]\ начинается с символа \verb|@|\ --- все последующие символы
игнориуются до конца строки
\item[инструкция]\ может быть инструкцией процессора или директивой ассемблера,
начинающейся с точки ``.''
\end{description}

Вот пример простой ассемблерной программы \ref{kumarL1}\ для процессора ARM,
складывающей два числа:

\lstv{Сложение двух чисел}{kumar/add.s}\label{kumarL1}

\verb|.text|\ ассемблерная директива, указывающая что последующий код должен
быть \term{ассемблирован}\ в \termdef{секцию кода \var{.text}}, а не в секцию
\var{.data}. \term{Секции}\ будут подробно описаны далее.

\secrel{Сборка бинарника}

Сохраните программу в файл \file{add.s}\ \note{.s или .S стандарное расширение в
мире OpenSource, указывает что это файл с программной на ассемблере}.
Для ассемблирования файла вызовите ассемблер \prog{as}:

\begin{verbatim}
$ arm-none-eabi-as -o add.o add.s
\end{verbatim}

Опция \verb|-o|\ указывает выходной файл с \termdef{объектным кодом}{объектный
код}, имеющий стандартное расширение \verb|.o|\note{и внутренний формат ELF (как
завещал великий \linux)}.

\begin{framed}
\noindent Команды кросс-тулчейна всегда имеют префикс целевой архитектуры
(target triplet), для которой они были собраны, чтобы предотвратить
конфликт имен с хост-тулчейном для вашего рабочего компьютера. Далее утилиты
\gnut\ будут использоваться без префикса для лучшей читаемости. \emph{не
забывайте добавлять \prog{arm-none-eabi-}, иначе получите множество странных
ошибок типа ``unexpected command''}.
\end{framed}

\begin{verbatim}
$ (arm-none-eabi-)as -o add.o add.s
$ (arm-none-eabi-)objdump -x add.o
\end{verbatim}
\lstx{вывод команды \prog{arm-none-eabi-objdump\ -x}:  
ELF-заголовки в файле объектного кода}{kumar/add.objdump}

Секция \var{.text}\ имеет размер \verb|Size=0x0010|\ =16 байт, и содержит
\emph{машинный код}:

\lstx{машинный код из секции \var{.text}: \prog{objdump -d}}{kumar/add.disasm}

\bigskip
Для генерации \emph{исполняемого файла}\note{обычно тот же формат ELF.o,
слепленный из одного или нескольких объектных файлов, с некоторыми
модификациями см. опцию -T далее}\ вызовем \termdef{линкер}{линкер}\ \prog{ld}:

\begin{verbatim}
$ arm-none-eabi-ld -Ttext=0x0 -o add.elf add.o
\end{verbatim}

Опять, опция \verb|-o|\ задает выходной файл. \verb|-Ttext=0x0|\ явно указывает
адрес, от которого будут отсчитываться все метки, т.е. секция инструкций
начинается с адреса \verb|0x0000|. Для просмотра адресов, назначенных меткам,
можно использовать команду \verb|(arm-none-eabi-)nm|\ \note{NaMes}:

\begin{verbatim}
ponyatov@bs:/tmp$ arm-none-eabi-nm add.elf
...
00000000 t start
0000000c t stop
\end{verbatim}
* если вы забудете опцию -T, вы получите этот вывод с адресами \verb|00008xxx|\
--- эти адреса были заданы при компиляции \gnut-ARM, и могут не совпадать
с необходимыми вам. Проверяйте ваши .elfы с помощью \prog{nm}\ или
\prog{objdump}, если программы не запускаются, или \qemu\ ругается на ошибки
(защиты) памяти.
\bigskip

Обратите внимание на \termdef{назначение адресов}{назначение адресов}\ для меток
\var{start}\ и \var{stop}: адреса начанаются с 0x0. Это адрес первой инструкции.
Метка \var{stop}\ находится после третьей инструкции. Каждая инструкция занимает
4 байта\note{в множестве команд ARM-32, если вы компилируете код для
микроконтроллера \cm x в режиме команд Thumb или Thumb2, команды 16-битные, т.е.
2 байта}, так что \var{stop}\ находится по адресу 0x$C_{hex}=12_{dec}$.
\termdef{Линковка}{линковка}\ с другим \termdef{базовым адресом}{базовый
адрес}\ \verb|-Ttext=nnnn|\ приведет к сдвигу адресов, назначенных меткам.

\begin{verbatim}
$ arm-none-eabi-ld -Ttext=0x20000000 -o add.elf add.o
$ arm-none-eabi-nm add.elf
... clip ...
20000000 t start
2000000c t stop
\end{verbatim}

\bigskip
Выходной файл, созданный \prog{ld}\ имеет формат, который называется
\termdef{ELF}{ELF}. Существует множество форматов, предназначенных для хранения
выполняемого и объектного кода\note{можно отдельно отметить Microsoft COFF
(объектные файлы .obj) и PE (.exe)cutable}. Формат ELF применяется для хранения
машинного кода, если вы запускаете его в базовой ОС\note{прежде всего
``большой''\ или встраиваемый \linux}, но поскольку мы собираемся запускать нашу
программу на bare metal\note{голом железе}, мы должны сконвертировать полученный
.elf файл в более простой \termdef{бинарный формат}{бинарный формат}.

Файл в \term{бинарном формате}\ содержит последовательность байт, начинающуюся с
определенного адреса памяти, поэтому бинарный файл еще 
называют \term{образом памяти}. Этот
формат типичен для утилит программирования флеш-памяти микроконтроллеров, так
как все что требуется сделать\ --- последовательно скопировать каждый байт из
файла в FlashROM-память микроконтроллера, начиная с определенного начального
адреса.\note{Та же операция выполняется и для SoC-систем с NAND-флешем:
записать бинарный образ начиная с некоторого аппаратно фиксированного адреса.}

Команда \gnut\ \prog{objcopy}\ используется для конвертирования машинного кода
между разными объектными форматами. Типичное использование:

\begin{verbatim}
$ objcopy -O <выходной_формат> <входной_файл> <выходной_файл>
\end{verbatim}

Конвертируем \file{add.elf}\ в бинарный формат:

\begin{verbatim}
$ objcopy -O binary add.elf add.bin
\end{verbatim}

Проверим размер полученного бинарного файла, он должен быть равен тем же 16
байтам\note{4 инструкции по 4 байта каждая}:

\begin{verbatim}
$ ls -al add.bin
-rw-r--r-- 1 vijaykumar vijaykumar 16 2008-10-03 23:56 add.bin
\end{verbatim}

Если вы не доверяете \prog{ls}, можно дизассемблировать бинарный файл:

\begin{verbatim}
ponyatov@bs:/tmp$ arm-none-eabi-objdump -b binary -m arm -D add.bin

add.bin:     file format binary

Disassembly of section .data:

00000000 <.data>:
   0:   e3a00005        mov     r0, #5
   4:   e3a01004        mov     r1, #4
   8:   e0812000        add     r2, r1, r0
   c:   eafffffe        b       0xc
ponyatov@bs:/tmp$ 
\end{verbatim}
Опция \verb|-b|\ задает формат файла, опция \verb|-m|\ (machine) архитектуру
процессора, получить полный список сочетаний \verb|-b/-m|\ можно командной
\verb|arm-none-eabi-objdump -i|.

\secrel{Выполнение в \qemu}

Когда ARM-процессор сбрасывается, он начинает выполнять команды с адрсе 0x0.
На плате Commnex установлен флеш на 16 мегабайт, начинающийся с адрес 0x0. Таким
образом, при сбросе будут выполняться инструкции с начала флеша.

Когда \qemu\ эмулирует плату connex, в командной строке должен быть указан файл,
который будет считаться образом флеш-памяти. Формат флеша очень прост\ --- это
побайтный образ флеша без каких-либо полей или заголовков, т.е. это тот же самый
\term{бинарный формат}, описанный выше.

Для тестирования программы в эмуляторе Gumstix connex, сначала мы создаем
16-мегабайтный файл флеша, копируя 16М нулей из файла \file{/dev/zero}\ с
помощью команды \prog{dd}. Данные копируются 4Кбайтными блоками\note{опция bs=
(blocksize)} (4096 х 4К):

\begin{verbatim}
$ dd if=/dev/zero of=flash.bin bs=4K count=4K
4096+0 записей получено
4096+0 записей отправлено
 скопировано 16777216 байт (17 MB), 0,0153502 c, 1,1 GB/c

$ du -h flash.bin 
16M     flash.bin
\end{verbatim}

Затем переписываем начало \file{flash.bin}\ копируя в него содержимое
\file{add.bin}:

\begin{verbatim}
$ dd if=add.bin of=flash.bin bs=4K conv=notrunc
0+1 записей получено
0+1 записей отправлено
 скопировано 16 байт (16 B), 0,000173038 c, 92,5 kB/c
\end{verbatim}

После сброса процессор выполняет код с адреса 0x0, и будут выполняться
инструкции нашей программы. Команда запуска \qemu:

\begin{verbatim}
$ qemu-system-arm -M connex -pflash flash.bin -nographic -serial /dev/null

QEMU 2.1.2 monitor - type 'help' for more information
(qemu) 
\end{verbatim}

Опция \verb|-M connex|\ выбирает режим эмуляции: \qemu\ поддерживает эмуляцию
нескольких десятков вариантов железа на базе ARM процессоров. Опция \verb|-pflash|\
указывает файл образа флеша, который должен иметь определенный размер (16М).
\verb|-nographic|\ отключает эмуляцию графического дисплея (в отдельном окне).
Самая важная опция \verb|-serial /dev/null|\ подключает последовательный порт
платы на \file{/dev/null}, при этом в терминале после запуска \qemu\ вы получите
\emph{консоль монитора}.

\qemu\ выполняет инструкции, и останавливается в бесконечном цикле на
\var{stop}, выполняя команду \verb|stop: b stop|. Для просмотра содержимого
регистров процессора воспользуемся \termdef{монитором}{монитор \qemu}.
Монитор имеет интерфейс командной строки, который вы можете использовать для
контроля работы эмулируемой системы. Если вы запустите \qemu\ как указано выше,
монитор будет доступен через stdio.

Для просмотра регистров выполним команду \verb|info registers|:

\begin{verbatim}
(qemu) info registers
R00=00000005 R01=00000004 R02=00000009 R03=00000000
R04=00000000 R05=00000000 R06=00000000 R07=00000000
R08=00000000 R09=00000000 R10=00000000 R11=00000000
R12=00000000 R13=00000000 R14=00000000 R15=0000000c
PSR=400001d3 -Z-- A svc32
FPSCR: 00000000
\end{verbatim}

Обратите внимание на значения в регистрах r00..r02: 4, 5 и ожидаемый результат
9. Особое значение для ARM имеет регистр r15: он является указателем команд, и
содержит адрес текущей выполняемой машинной команды, т.е. \verb|0x000c: b stop|.

\secrel{Другие команды монитора}

Несколько полезных команд монитора:

\begin{tabular}{l l}
help & список доступных команд \\
quit & выход из эмулятора \\
xp /fmt addr & вывод содержимого физической памяти с адреса addr \\
system\_reset & перезапуск \\
\end{tabular}
\bigskip

Команда \verb|xp|\ требует некоторых пояснений. Аргумент \verb|/fmt|\ указывает
как будет выводиться содержимое памяти, и имеет синтаксис
\verb|<счетчик><формат><размер>|:

\begin{description}
\item[счетчик] число элементов данных
\item[size] размер одного элемента в битах: b=8 бит, h=16, w=32, g=64
\item[format] определяет формат вывода:
\begin{description}[nosep]
\item[x] hex
\item[d] десятичные целые со знаком
\item[u] десятичные без знака
\item[o] 8ричные
\item[c] символ (char)
\item[i] инструкции ассемблера
\end{description}
\end{description}

Команда \verb|xp|\ в формате \verb|i|\ будет дизассемблировать инструкции из
памяти. Выведем дамп с адреса 0x0 указав fmt=4iw: 4\ --- 4 , i\ ---
инструкции размером w=32 бита:

\begin{verbatim}
(qemu) xp /4wi 0x0
0x00000000:  e3a00005      mov  r0, #5  ; 0x5
0x00000004:  e3a01004      mov  r1, #4  ; 0x4
0x00000008:  e0812000      add  r2, r1, r0
0x0000000c:  eafffffe      b    0xc
\end{verbatim}

\secup
\secrel{Директивы ассемблера}\secdown

В этом разделе мы посмотрим несколько часто используемых директив ассемблера,
используя в качестве примера пару простых программ.

\secrel{Суммирование массива}\secdown

Следующий код \ref{kumarL2}\ вычисляет сумму массива байт и сохраняет результат
в \var{r3}:

\lstv{Сумма массива}{kumar/arrsum.s}\label{kumarL2}

В коде используются две новых ассемблерных директивы, описанных далее:
\verb|.byte|\ и \verb|.align|.

\secrel{\var{.byte}}

Аргументы директивы \verb|.byte|\ ассемблируются в последовательность байт в
памяти. Также существуют аналогичные директивы \verb|.2byte|\ и \verb|.4byte|\
для ассемблирования 16- и 32-битных констант:

\begin{verbatim}
.byte   exp1, exp2, ...
.2byte  exp1, exp2, ...
.4byte  exp1, exp2, ...
\end{verbatim}

Аргументом может быть целый числовой литерал: двоичный с префиксом 0b, 8-ричный
префикс 0, десятичный и hex 0x. Также может использоваться строковая константа в
одиночных кавычках, ассемблируемая в ASCII значения.

Также аргументом может быть \ci-выражение из литералов и других символов,
примеры:

\begin{verbatim}
pattern:  .byte 0b01010101, 0b00110011, 0b00001111
npattern: .byte npattern - pattern
halpha:   .byte 'A', 'B', 'C', 'D', 'E', 'F'
dummy:    .4byte 0xDEADBEEF
nalpha:   .byte 'Z' - 'A' + 1
\end{verbatim}

\secrel{\var{.align}}

Архитектура ARM требует чтобы инструкции находились в адресах памяти,
выровненных по границам 32-битного слова, т.е. в адресах с нулями в 2х младших
разрядах. Другими словами, адрес каждого первого байта из 4-байтной команды,
должен быть кратен 4. Для обеспечения этого предназначена директива
\verb|.align|, которая вставляет выравнивающие байты до следующего выровненного
адреса. Ее нужно использовать только если в код вставляются байты или неполные
32-битные слова.

\secup

\secrel{Вычисление длины строки}\secdown

Этот код вычисляет длину строки и помещает ее в \var{r1}:

\lstv{Длина строки}{kumar/strlen.s}

Код иллюстрирует применение директив \verb|.asciz|\ и \verb|.equ|.

\secrel{\var{.asciz}}

Директива \verb|.asciz|\ принимает аргумент: строковый литерал,
последовательность символов в двойных кавычках. Строковые литералы
ассемблируются в память последовательно, добавляя в конец нулевой символ
\verb|\0|\ (признак конца строки в языке \ci\ и стандарте POSIX).

Директива \verb|.ascii|\ аналогична \verb|.asciz|, но конец строки не
добавляется. Все символы\ --- 8-битные, кириллица может не поддерживаться.

\secrel{\var{.equ}}

Ассемблер при свой работе использует \termdef{таблицу символов}{таблица
символов}: она хранит соответствия меток и их адресов. Когда ассемблер встречает
очередное определение метки, он добавляет в таблицу новую запись. Когда
встречается упоминание метки, оно заменяется соответствующим адресом из
таблицы.

Использование директивы \verb|.equ|\ позволяет добавлять записи в
таблицу символов вручную, для привязки к именам любых числовых значений, не
обязательно адресов. Когда ассемблер встречает эти имена, они заменяются на их
значения. Эти имена-константы, и имена меток, называются
\termdef{сиволами}{символ}, а таблицы записанные в объектные файлы, или в
отдельные .sym файлы\ --- \termdef{таблицами символов}{таблица
символов}\note{также используются отладчиком, чтобы показывать адреса переходов
в виде понятных программисту сиволов, а не мутных числовых констант}.

Синтаксис директивы .equ:

\begin{verbatim}
.equ <имя>, <выражение>
\end{verbatim}

Имя символа имеет те же ограничения по используемым символам, что и метка.
Выражение может быть одиночным литералом или выражением как и в директиве
\verb|.byte|.

\begin{framed}
В отличие от .byte, директива .equ не выделяет никакой памяти под аргумент.
Она только добавляет значение в таблицу символов.  
\end{framed}

\secup

\secup
\secrel{Использование ОЗУ (адресного пространства процессора)}

Flash-память описанная ранее, в которой хранится машинный код программы, один из
видов EEPROM\note{Electrical Eraseable Programmable Read-Only Memory,
электрически стираемая перепрограммируемая память только-для-чтения}.
Это вторичный носитель данных, как например жесткий диск, но в
любом случае хранить данные и значения переменных во флеше неудобно как с точки
зрения возможности перезаписи, так и прежде всего со скоростью чтения флеша, и
кешированием.

В предыдущем примере мы использовали флеш как EEPROM для хранения константного
массива байт, но вообще переменные должны храниться в максимально быстрой и
неограниченно перезаписываемой RAM.

Плата connex имеет 64Mb ОЗУ начиная с адреса \verb|0xA0000000|, для хранения
данных программы. Карта памяти может быть представлена в виде диаграммы:

\fig{Карта памяти Gumstix connex}{kumar/memmap.png}{height=0.6\textheight}

\note{здесь адреса считаются сверху вниз, что нетипично, обычно на диаграммах
памяти адреса увеличиваются снизу вверх.}

Для настройки размещения переменных по нужным физическим адресам \emph{нужна}\
некоторая \emph{настройка адресного пространства}, особенно \emph{если вы
используете внешнюю память и переферийные устройства, подключемые к внешней
шине}\note{или используете малораспространенные клоны ARM-процессоров, типа
Миландровского 1986ВЕ9х ``чернобыль''}.

Для этого нужно понять, какую роль в распределении памяти играют ассемблер и
линкер.

\secrel{Линкер}\secdown

Линкер позволяет \termdef{скомпоновать}{компоновка}\ исполняемый файл программы
из нескольких объектных файлов, поделив ее на части. Чаще всего это нужно при
использовании нескольких компиляторов для разных языков программирования:
ассемблер, компиляторы \cpp, Фортрана и Паскакаля. 

Например, очень известная библиотека численных вычислений на базе матриц
BLAS/LAPACK написана на Фортране, и для ее использования с сишной программой
нужно слинковать program.o, blas.a и lapack.a\note{.a\ --- файлы архивов из пары
сотен отдельных .o файлов каждый, по одному .o файлу на каждый возможный вариант
функции линейной алгебры}\ в один исполняемый файл.

При написании многофайловой программы (еще это называют \termdef{инкрементной
компоновкой}{инкрементная компоновка}) каждый файл исходного кода ассемблируется
в индивидуальный файл объектного кода. Линкер\note{или компоновщик}\ собирает
объектные файлы в финальный исполняемый бинарник.

\bigskip
\fig{Роль линкера}{kumar/linkerrole.png}{height=0.25\textheight}
\bigskip

При сборке объектных файлов, линкер выполняет следующие операции:

\begin{itemize}[nosep]
  \item symbol resolution (разрешение символов)
  \item relocation (релокация) 
\end{itemize}

В этой секции мы детально рассмотрим эти операции.

\secrel{Разрешение символов}

В программе из одного файла при создании объектного файла все ссылки на метки
заменяются их адресами непосредственно ассемблером. Но в программе из нескольких
файлов существует множество ссылок на метки в других файлах, неизвестные на
момент ассемблирования/компиляции, и ассемблер помечает их ``unresolved''
(неразреш\'{е}нные). Когда эти объектные файлы обрабатываются линкером, он
определяет адреса этих меток по информации из других объектных файлов, и
корректирует код. Этот процесс называется \termdef{разрешением
символов}{разрешение символов}.

Пример суммирования массива разделен на два файла для демонстрации разрешения
символов, выполняемых линкером. Эти файлы ассемблируются отдельно, чтобы их
таблицы символов содержали неразрешенные ссылки.

Файл \file{sumsub.s}\ содержит процедуру суммирования, а \file{summain.s}\
вызывает процедуру с требуемыми аргументами:

\lstv{\file{summain.s}\ вызов внешней процедуры}{kumar/summain.s}
\lstv{\file{sumsub.s}\ код процедуры}{kumar/sumsub.s}
\note{в архитектуре ARM нет специальной команды возврата ret, ее функцию
выполняет прямое присваивание регистра указателя команд mov pc,lr} 

Применение директивы \verb|.global|\ обязательно. В \ci\ все функции и
переменные, определенные вне функций, считаются видимыми из других объектных
файлов, если они не определены с модификатором \verb|static|. В ассемблере
наоборот все метки считаются \term{статическими}\note{или локальными на уровне
файла}, если с помощью директивы \verb|.global|\ специально не указано, что они
должны быть видимы извне.

Ассемблируйте файлы, и посмотрите дамп их таблицы символов используя 
комунду \prog{nm}:

\begin{verbatim}
$ arm-none-eabi-as -o main.o main.s
$ arm-none-eabi-as -o sum-sub.o sum-sub.s
$ arm-none-eabi-nm main.o
00000004 t arr
00000007 t eoa
00000008 t start
00000018 t stop
         U sum 
$ arm-none-eabi-nm sum-sub.o
00000004 t loop
00000000 T sum
\end{verbatim}

Теперь сфокусируемся на букве во втором столбце, который указывает тип символа:
\verb|t|\ указывает что символ определен в секции кода \verb|.text|, \verb|u|\
указывает что символ не определен. Буква в верхнем регистре указывает что символ
глобальный и был указан в директиве \var{.global}.

Очевидно, что символ \var{sum}\ определ в \file{sum-sub.o}\ и еще не разрешен
в \file{main.o}. Вызов линкера разрешает символьные ссылки, и создает
исполняемый файл.
 
\secrel{Релокация}\secdown

\termdef{Релокация}{релокация символов}\ --- процесс изменения уже назначенных
меткам адресов.
Он также выполняет коррекцию всех ссылок, чтобы отразить заново назначенные
адреса меток. В общем, релокация выполняется по двум основным причинам:

\begin{enumerate}[nosep]
\item Объединение секций
\item Размещение секций
\end{enumerate}

Для понимания процеса релокации, нужно понимание самой концепции секций.

Код и данные отличаются по требованиям при исполнении. Например код может
размещаться в ROM-памяти, а данные требуют память открытую на запись. Очень
хорошо, если \emph{области кода и данных не пересекаются}. Для этого программы
делятся на секции. Большиство программ имеют как минимум две секции:
\var{.text}\ для кода и \var{.data}\ для данных. Ассемблерные директивы
\verb|.text|\ и \verb|.data|\ ожидаемо используются для переключения между этими
секциями.

Хорошо представить каждую секцию как ведро. Когда ассемблер натыкается на 
директиву секции, он начинает сливать код/данные в соответствующее ведро, так 
что они размещаются в смежных адресах. Эта диаграмма показывает как ассемблер
упорядочивает данные в секциях:

\fig{\\Секции}{../gnu-eprog/sections.png}{width=0.95\textwidth}

Теперь, когда у нас есть общее понимание
\termdef{секционирования}{секционирование}\ кода и данных, давайте посмотрим по
каким причинам выполняется релокация.

\secrel{Объединение секций}

Когда вы имеете дело с многофайловыми программами, секции в каждом объектном
файле имеют одинаковые имена (`.text`,\ldots), линкер отвечает за их объединение
в выполняемом файле. По умолчанию секции с одинаковыми именами из каждого
\file{.o}\ файла объединяются последовательно, и метки корректируются на новые
адреса.

Эффекты объединения секций можно посмотреть, анализируя таблицы символов
отдельно в объектных и исполняемом файлах. Многофайловая программа суммирования
может иллюстрировать объединение секций. Дампы таблиц символов:

\begin{verbatim}
$ arm-none-eabi-nm main.o
00000004 t arr
00000007 t eoa
00000008 t start
00000018 t stop
         U sum 
$ arm-none-eabi-nm sum-sub.o
00000004 t loop <1>
00000000 T sum
$ arm-none-eabi-ld -Ttext=0x0 -o sum.elf main.o sum-sub.o
$ arm-none-eabi-nm sum.elf
...
00000004 t arr
00000007 t eoa
00000008 t start
00000018 t stop
00000028 t loop <1>
00000024 T sum
\end{verbatim}
\begin{enumerate}
  \item символ \var{loop}\ имеет адрес \addr{0x4}\ в \file{sum-sub.o}, и
  \addr{0x28}\ в \file{sum.elf}, так как секция \var{.text}\ из
  \file{sum-sub.o}\ размещена сразу за секцией \var{.text}\ из \file{main.o}.
\end{enumerate}

\secrel{Размещение секций}

Когда программа ассемблируется, каждой секции назначается стартовый адрес
\addr{0x0}. Поэтому всем переменным назначаются адреса относитально начала
секции. Когда создается финальный исполнямый файл, секция размещаются по
некоторому адресу \addr{X}, и все адреса меток, назначенные в секции,
увеличиваются на \addr{X}, так что они указывают на новые адреса.

Размещение каждой секции по определенному месту в памяти и коррекцию всех ссылок
на метки в секции, выполняет линкер.

Эффект размещения секций можно увидеть по тому же дампу символов, описанному
выше. Для простоы используем объектный файл однофайловой программы суммирования
\file{sum.o}. Для увеличения заметности искуственно разместим секцию
\var{.text}\ по адресу \addr{0x100}:

\begin{verbatim}
$ arm-none-eabi-as -o sum.o sum.s
$ arm-none-eabi-nm -n sum.o
00000000 t entry <1>
00000004 t arr
00000007 t eoa
00000008 t start
00000014 t loop
00000024 t stop
$ arm-none-eabi-ld -Ttext=0x100 -o sum.elf sum.o <2>
$ arm-none-eabi-nm -n sum.elf
00000100 t entry <3>
00000104 t arr
00000107 t eoa
00000108 t start
00000114 t loop
00000124 t stop
...
\end{verbatim}

\begin{enumerate}[nosep]
\item Адреса меток назначаются с \addr{0}\ от начала секции.
\item Когда создается выполняемый файл, линкеру указано разместить секцию кода с 
адреса \addr{0x100}.
\item Адреса меток в \var{.text}\ переназначаются начиная с \addr{0x100}, и
все ссылки на метки корректируются.
\end{enumerate}

\bigskip
Процесс объединения и размещения секций в общем показаны на диаграмме:

\fig{\\Объединение и размещение
секций}{../gnu-eprog/relocation.png}{height=0.95\textheight}

\secup
\secup
\secrel{Скрипт линкера}\secdown

Как было описано в предыдущем разделе, объединение и размещение секций
выполняется линкером. Программист может контролировать этот процесс через
\termdef{скрипт линкера}{скрипт линкера}. Очень простой пример скрипта линкера:

\lstv{Простой скрипт линкера}{kumar/simple.lds}

\begin{enumerate}

\item Команда \var{SECTIONS}\ наиболее важная команда, она указывает как
секции объединяются, и по каким адресам они размещаются.

\item Внутри блока \var{SECTIONS}\ команда \var{.} (точка) представляет
\termdef{указатель адреса размещения}{указатель адреса размещения}. Указатель
адреса всегда инициализируется \addr{0x0}. Его можно модифицировать явно
присваивая новое значение. Показанная явная установка на \addr{0x0}\ на самом
деле не нужна.

\item Этот блок скрипта определяет что секция \var{.text}\ выходного файла
составляется из секций \var{.text}\ в файлах \file{abc.o}\ и \file{def.o},
причем именно в этом порядке.

\end{enumerate}

Скрипт линкера может быть значительно упрощен и универсализирован с помощью
использования символа шаблона \verb|*|\ вместо индивидуального указания имен
файлов:

\lstv{Шаблоны в скриптах линкера}{kumar/star.lds}

Если программа одновременно содержит секции `.text` и `.data`, объединение и 
размещение секций можно прописать вот так: 

\lstv{Несколько секций}{../gnu-eprog/code/sum-data.lds}

Здесь секция \var{.text}\ размещается по адресу \addr{0x0}, а секция
\var{.data}\ --- \addr{0x400}. Отметим, что если указателю размещения не
приваивать значения, секции \var{.text}\ и \var{.data}\ будут размещены в
смежных областях памяти.

\secrel{Пример скрипта линкера}

Для демонстрации использования скриптов линкера рассмотрим применение скрипта
\ref{linker1}\ для размещения секций \var{.text}\ и \var{.data}. Для этого
используем немного измененный пример программы суммирования массива, разделив
код и данные в отдельные секции:

\lstv{Программа суммирования массива}{../gnu-eprog/code/sum-data.s}

\begin{enumerate}
\item Изменения заключаются в выделении массива в секцию \var{.data}\ и
удалении директивы выравнивания \var{.align}.

\item Также не требуется инструкция перехода на метку \var{start}\ 
для обхода данных, так как линкер разместит секции отдельно. В результате
команды программы размещаются любым удобным способом, а скрипт линкера 
позаботится о правильном размещении сегментов в памяти.
 
\end{enumerate}

При линковке программы в командной строке нужно указать использования скрипта:

\begin{verbatim}
$ arm-none-eabi-as -o sum-data.o sum-data.s
$ arm-none-eabi-ld -T sum-data.lds -o sum-data.elf sum-data.o
\end{verbatim}

Опция \verb|-T sum-data.lds|\ указывает что используется скрипт
\file{sum-data.lds}. Выводим таблицу символов и видим размещение сегментов в
памяти:

\begin{verbatim}
$ arm-none-eabi-nm -n sum-data.elf
00000000 t start
0000000c t loop
0000001c t stop
00000400 d arr
00000403 d eoa
\end{verbatim}

Из таблицы символолов видно что секция \var{.text}\ размещена с адреса
\addr{0x0}, а секция \var{.data}\ c \addr{0x400}.

\secrel{Анализ объектного/исполняемого файла утилитой \prog{objdump}}

Более подробную информацию даст утилита \prog{objdump}:

\begin{verbatim}
$ arm-none-eabi-as -o sum-data.o sum-data.s
$ arm-none-eabi-ld -T sum-data.lds -o sum-data.elf sum-data.o
$ arm-none-eabi-objdump sum-data.elf
\end{verbatim}

\lstv{sum-data.objdump}{../gnu-eprog/code/sum-data.objdump}

\begin{enumerate}
\item указание на архитектуру, 

\item для которой предназначен исполняемый файл

\item стартовый адрес в секции \var{.text}, по умолчанию
\addr{0x0}\note{обязателен и фиксирован для прошивок микроконтроллеров, т.к. на
него перескакивает аппаратный сброс}

\item \termdef{ABI}{ABI}\ --- соглашения о передаче 

\item параметров в регистрах/стеке (для Си кода) 

\item приведена подробная информация о секциях

\item \var{.text}\ секция кода

\item \var{.data}\ секция данных

\item служебная информация

\item столбец \var{Size}\ указывает размер секции в байтах (hex)

\item \var{VMA}\note{Virtual Memory Address}\ указывает адрес размещения
сегмента

\item \var{Algn}\ (Align) автоматическое выравнивание содержимого сегмента в
памяти, в степени двойки \verb|2**n|: код выравнивается кратно \verb|2**2=4|\
байтам, данные не выравниваются \verb|2**0=1| 

\item Флаг \var{ALLOC}\ (Allocate) указывает что при загрузке программы под этот 
сегмент должна быть выделена память.

\item \var{LOAD}\ указывает что содержимое сегмента должно загружаться из
исполняемого файла в память при использовании ОС, а для микроконтроллеров
указывает программатору что сегмент нужно прошивать.

\item \var{READONLY}\ сегмент с константными неизменяемыми данными, которые
могут быть размещены в ROM, а при использовании ОС область памяти должна быть
помечена в таблице системы защиты памяти как R/O. Отсутствие флага
\var{READONLY}\ + наличие \var{LOAD}\ указывает что данные должны загружаться
\emph{только в ОЗУ}.

\item сегмент кода

\item сегмент данных

\item таблица символов

\item дизассемблированный код из секций, помеченных флагом \var{CODE}:
\var{.text}
 
\end{enumerate}

\secup
\secrel{Данные в RAM, пример}\secdown

Теперь мы знаем как писать скрипты линкера, и можем попытаться написать
программу, разместив данные в секции \var{.data}\ в ОЗУ.

Программа сложения модифицирована для загрузки значений из ОЗУ, и записи
результата обратно в ОЗУ: память для операндов и результат резмещена в секции
\var{.data}.

\lstv{Данные в ОЗУ}{../gnu-eprog/code/add-mem.s}

\lstv{Скрипт для линковки}{../gnu-eprog/code/flash-ram.lds}

Дамп таблицы символов:

\begin{verbatim}
$ arm-none-eabi-nm -n add-mem.elf
00000000 t start
0000001c t stop
a0000000 d val1
a0000001 d val2
a0000002 d result
\end{verbatim}

Срипт линкера решил проблему с размещением данных, но 
\emph{проблема с использованием ОЗУ еще не решена \,!}
 

\secrel{RAM энергозависима (volatile)\,!}

ОЗУ стирается при отключении питания, поэтому для использования ОЗУ недостаточно
разместить сегменты.

\begin{framed}
\emph{Во флеше должнен храниться}\ не только код, но \emph{и данные}, чтобы при
подаче питания специальный \termdef{startup код}{startup код}\ выполнил
\emph{инициалиацию ОЗУ}, копируя данные из флеша. Затем управление передается
основной программе.
\end{framed}

Поэтому секция \var{.data}\ имеет \emph{два адреса размещения}: 
\termdef{адрес хранения}{адрес хранения} во флеше \termdef{LMA}{LMA}\ и 
\termdef{адрес размещения}{адрес размещения} в ОЗУ \termdef{VMA}{VMA}.

TIP: как видно из раздела \ref{objdump}, в терминах \prog{ld}\ адрес
хранения (загрузки) называется \emph{LMA}\ (Load Memory Address), а адрес
размещения (времени выполнения) \emph{VMA}\ (Virtual Memory Address).

Нужно сделать следующие две мсодификации, чтобы программа работала корректно:

\begin{enumerate}
  \item модифицировать .lds чтобы для секции \var{.data}\ в нем учитывались оба
  адреса: LMA и VMA.
  \item написать небольшой кусочек кода, который будет \emph{инициализировать
  память данных}, копируя образ секции \var{.data}\ из флеша (из адреса хранения
  LMA) в ОЗУ (по адресу исполнения, VMA).
\end{enumerate}

\secrel{Спецификация адреса загрузки LMA}

VMA это адрес, который должен быть использован для вычисления адресов всех меток
при исполнении программы. В предыдущем линк-скрипте мы задали VMA секции
\var{.data}.
LDA не указан, и по умолчанию равен VMA. Это нормально для сегментов, 
размещаемых в ROM. Но если используются инициализируемые сегменты в ОЗУ, нужно
задать отдельно VMA и LMA.  

Адрес загрузки LMA, отличающийся от адреса выполнения VMA, задается с помощью
команды \verb|AT|. Модифицированный скрипт показан ниже:

\begin{verbatim}
SECTIONS {
	. = 0x00000000;
	.text : { * (.text); }
	etext = .; <1>

	. = 0xA0000000;
	.data : AT (etext) { * (.data); } <2>
}
\end{verbatim}

\begin{enumerate}
  \item 
В блоках описания секций можно создавать символы, назначая им значения:
числовой адрес или текущую позицию с помощью точки ".". Символу \var{etext}\
назначается адрес флеша, следующий сразу за концом кода. Отметим что
\var{etext}\ сам по себе не выделяет никакой памяти, а только помечает адрес LMA
сегмента .data в таблице символов.

  \item 
При настройке сегмента \var{.data}\ с помощью ключевого слова
\verb|AT (etext)|\ назначается LMA для хранения содержимого сегмента данных.
Команде \var{AT}\ может быть передан любой адрес или символ\note{значением которого
является валидный адрес}. Так что в результате мы настроили адрес хранения 
\var{.data}\ на область флеша, помеченную символов \var{etext}.

\end{enumerate}

\secrel{Копирование `.data` в ОЗУ}

Для копирования данных инициализации из флеша в ОЗУ требуется следующая 
информация: 

\begin{enumerate}[nosep]
  \item 
  Адрес данных во флеше (\var{flash\_sdata})
  \item 
  Адрес данных в ОЗУ (\var{ram\_sdata})
  \item 
  Размер секции \var{.data}\ (\var{data\_size})
\end{enumerate}

Имея эту информацию, сегмент \var{.data}\ может быть инициализирован
может быть скопирован следующим стартовым кодом:

\begin{verbatim}
	ldr   r0, =flash_sdata
	ldr   r1, =ram_sdata
	ldr   r2, =data_size
copy:	
	ldrb  r4, [r0], #1
	strb  r4, [r1], #1
	subs  r2, r2, #1
	bne   copy
\end{verbatim}

Для получения такой информации скрипт линкера нужно немного модифицировать:

\lstv{Скрипт линкера с символами для копирования секции
\var{.data}}{kumar/ramcopy.lds}

\begin{enumerate}
  \item 
Начало данных во флеше сразу за секцией кода.
  \item 
Начало данных\ --- базовый адрес ОЗУ в адресном пространстве процессора.
  \item 
Получение размера непросто: размер вычисляется вычитанием адресов
метод начала и конца данных. Да, простые выражения тоже можно использовать в
скрипте линкера.
\end{enumerate}

Полный листинг программы с добавленной инициализацией данных:

\lstv{Инициализация ОЗУ}{../gnu-eprog/code/add-ram.s}

\lstv{\file{add-ram.objdump}}{../gnu-eprog/code/add-ram.objdump}

Программа была ассемблирована и слинкована используя .lds в \ref{linker2}.

Запуск и тестирование программы в Qemu:

\begin{verbatim}
qemu-system-arm -M connex -pflash flash.bin -nographic -serial /dev/null
(qemu) xp /4dw 0xA0000000
a0000000:         10         30         40          0
\end{verbatim}

\begin{framed}
На реальной физической системе с SDRAM, память не может использована 
сразу. Сначала нужно инициализировать контроллер памяти, и только затем
обращаться к ОЗУ. Наш код работает потому, что симулятор не требует
инициализации контроллера.
\end{framed}

\secup

\secrel{Обработка аппаратных исключений}

Все примеры программ, приведенные выше, содержат гигантский баг: \emph{первые
8 машинных слов в адресном пространстве зарезервированы для векторов
исключений}. Когда возникает исключение, выполняется аппаратный переход
на один из этих жестко заданных меток. Исключения и их адреса приведены
в следующей таблице:

\paragraph{Адреса векторов исключений\\}

\begin{tabular}{l l l}
Исключение && Адрес \\
\hline
 Сброс & Reset                   & 0x00 \\
 Неопределенная инструкция & Undefined Instruction	  & 0x04 \\
 Программное прерывание (SWI) & Software Interrupt (SWI)& 0x08 \\
 Ошибка предвыборки & Prefetch Abort	  & 0x0C \\
 Ошибка данных & Data Abort		  & 0x10 \\
 Резерв, не используется & Reserved, not used	  & 0x14 \\
 Аппаратное прерывание & IRQ			  & 0x18 \\
 Быстрое прерывание & FIQ			  & 0x1C \\
\end{tabular}

Предполагается что по этим адресам находятся команды перехода, которые
передадут управление на соответствующий произвольный адрес обработчика
исключения. Во всех примерах ранее бы не вставляли таблицу обработчиков
исключений, так как мы предполагали что эти исключения не случатся. Все
эти программы можно скорректировать, слинковав их со следующим кодом:  

\begin{verbatim}
	.section "vectors"
reset:	b     start
undef:  b     undef
swi:	b     swi
pabt:	b     pabt
dabt:	b     dabt
	nop
irq:	b     irq
fiq:	b     fiq
\end{verbatim}

Только обработчик \var{reset}\ векторизован на отдельный адрес \var{start}. Все
остальные исключения векторизованы сами на себя. Таким образом если случится любое
исключение, процессор зациклится на соответствущем векторе. В этом случае
возникшее исключение может быть идентифицирвовано в отладчике (мониторе Qemu,
в нашем случае) по адресу указателя команд \var{pc=r15}.

В ассемблерном коде видно применение директивы \var{.section}\ которая позволяет
создавать секции с произвольными именами, чтобы прописать для них отдельную
обработку в скрипте линкера.

Чтобы обеспечить правильное размещение таблицы обработчиков, нужно
скорректировать скрипт линкера: 

\begin{verbatim}
SECTIONS {
	. = 0x00000000;
	.text : {
		* (vectors);
		* (.text); 
		...
	}
	...
}
\end{verbatim}

Обратите внимание что секция \var{vectors}\ размещена сразу за инициализацией
указателя размещения на первом месте, до всего остального кода, что гарантирует
что таблица векторов будет находится по жесткому адресу \verb|0x0|.


\secrel{10. C Startup}\secdown
\secrel{10.1. Stack}
\secrel{10.2. Global Variables}
\secrel{10.3. Read-only Data}
\secrel{10.4. Startup Code}
\secup
\secrel{11. Using the C Library}
\secrel{12. Inline Assembly}
\secrel{13. Contributing}
\secrel{14. Credits}\secdown
\secrel{14.1. People}
\secrel{14.2. Tools}
\secup
\secrel{15. Tutorial Copyright}
\secrel{A. ARM Programmer’s Model}
\secrel{B. ARM Instruction Set}\label{kumarB}
\secrel{C. ARM Stacks}
 
\secup 

\secrel{Embedded Systems Programming in \cpp\ \cite{bogo}}\label{bogoru}\secdown
\cp{http://www.bogotobogo.com/cplusplus/embeddedSystemsProgramming.php} 
\secup

\secrel{Сборка кросс-компилятора \gnut\ из исходных
текстов}\label{cross}\secdown

Если вам по каким-то причинам не подходит одна из типовых сборок
кросс-компиляторов, поставляемых в виде готовых бинарных пакетов из репозитория
вашего дистрибутива \linux, \gnut\ можно легко скопилировать \emph{из исходных
текстов}\ и установить в систему, даже имея только пользовательские права
доступа.

\bigskip
Сборка \gnut\ из исходников может понадобиться, если вы хотите:
\begin{itemize}[nosep]
  \item самую свежую или какую-то конкретную версию \gnut
  \item опции компиляции: малораспространенный \var{target}-процессор,
  \emph{нетиповой формат файлов объектного кода}\note{например для i386
  может понадобится сборка кросс-компилятора с
  \var{--target=i486-none-elf}\ \ref{os86}\ или \var{i686-linux-uclibc}\ вместо
  типовой компиляции для \linux\ типа \var{i486-linux-gnu}}\ или
  экспериментальные оптимизаторы, не включенные в бинарные пакеты из
  дистрибутива \linux
  \item полпроцента ускорения работы компилятора благодаря жесткой оптимизации
  его машинного кода точно под ваш рабочий компьютер
  (\verb|-march=native -mtune=native -O3|)
\end{itemize}

\bigskip
При сборке используется утилита \prog{make}\ \ref{make}, которой можно передать
набор переменных конфигурирования. В таблице перечислен набор переменных
конфигурирования сборки с указанием их значения по умолчанию\note{также
приведены часто используемые варианты значения}\ и имя mk-файла, где оно
задано:

\bigskip
\begin{tabular}{l l l l}
\var{APP} & \verb|cross| & \verb|Makefile| & приложение: условное
имя проекта\\&&&(только латиница, буквы a-z)\\
\var{HW} & \verb|x86| & \verb|Makefile| & \var{qemu vmware virtualpc}\\
	&&&\var{x86 pc686 amd64}\\
	&&&\var{cortexM avr8}\\
\var{CPU} & \verb|i386| & \verb|hw/$(HW).mk| &\\
\var{ARCH} & \verb|i386| & \verb|cpu/$(CPU).mk| &\\
\var{TARGET} & \verb|$(CPU)-pc-elf| & \verb|hw/$(HW).mk| &
\var{i686-linux-uclibc x86\_64-linux-gnu}\\
	&&&\var{i386-pc-elf arm-none-eabi avr-none-elf}\\
\end{tabular}

\secly{\var{APP}/\var{HW}: приложение/платформа}

Для сборки необходимо выбрать имя проекта\note{только латиница, буквы a-z}\ и
аппаратную платформу, для которой будет настраиваться пакет кросс-компилятора.

Особенно это важно для варианта сборки, когда собирается не только
кросс-компилятор, но и базовая ОС\ --- минимальная \linux-система из ядра,
libc и дополнительных прикладных библиотек. В этом случае \prog{APP/HW}
используются для формирования имен файлов ядра \verb|$(APP)$(HW).kernel|,
названия и состава загрузочного образа \verb|$(APP)$(HW).rootfs|, и внутренних
настроек.

\secly{Подготовка \var{BUILD}-системы: необходимое ПО}

Для сборки необходимо установить следующие пакеты:

\begin{verbatim}
sudo apt install gcc g++ make flex bison m4 bc bzip2 xz-utils libncurses-dev 
\end{verbatim}

\secly{\prog{dirs}: создание структуры каталогов}

\begin{verbatim}
user@bs:~/boox/cross$ make dirs
mkdir -p
    /home/user/boox/cross/gz /home/user/boox/cross/src /home/user/boox/cross/tmp
    /home/user/boox/cross/toolchain /home/user/boox/cross/root
\end{verbatim}

Командной \verb|make dirs|\ создается набор вспомогательных каталогов:

\bigskip
\begin{tabular}{l l l}
\var{TC} & \verb|$(CWD)/$(APP)$(ROOT).cross| & каталог установки
кросс-компилятора \\
\var{ROOT} & \verb|$(CWD)/$(APP)$(ROOT)| & каталог файловой системы для целевого
em\linux\\
\hline
\end{tabular}

\begin{tabular}{l l l}
\var{CWD} & \verb|$(CURDIR)| & текущий каталог \\
\var{GZ} & \verb|$(CWD)/gz| & архивы исходных текстов GNU Toolchain, загрузчика,
и библиотек\\
\var{SRC} & \verb|$(CWD)/src| & каталог для распаковки исходников \\
\var{TMP} & \verb|$(CWD)/tmp| & каталог для out-of-tree сборки GNU toolchain \\
\end{tabular}
\bigskip

\lst{mk/dirs.mk}{cross/mk/dirs.mk}{make}

\secly{Сборка в ОЗУ на ramdiskе}

Если у вас есть админские права и достаточный объем RAM, после выполнения
\verb|make dirs|\ рекомендуется примонировать на каталоги \var{SRC} и \var{TMP}
файловую систему tmpfs\ --- это значительно ускорит компиляцию, т.к. все
временные файлы будут хранится только в ОЗУ:

\lstx{/etc/fstab}{cross/fstab.txt}

Если вы прописали монтирование \term{ramdisk}ов в \file{/etc/fstab}, или
сделали \verb|mount -t tmpfs|\ вручную, может оказаться нужным запускать
\prog{make} с явным указанием значений переменных SRC/TMP:

\begin{verbatim}
make blablabla SRC=/home/user/src TMP=/home/user/tmp
\end{verbatim}

\secly{Пакеты системы кросс-компиляции}

\begin{description}
\item[\gnut] \ \\\lstx{mk/pack\_cross.mk}{cross/mk/pack_cross.mk}{make}
\item[\prog{newlib}] стандартная библиотека \prog{libc}\\
\end{description}

\secly{\prog{gz}: загрузка исходного кода для пакетов}

\begin{verbatim}
user@bs$ make APP=cross HW=x86 GZ=/home/user/gz gz
\end{verbatim}

В примере команды показано два обязательных параметра \var{APP/HW}\note{по ним
могут закачиваться дополнительные файлы исходников, зависящие от платформы\ ---
например исходник загрузчика или бинарные файлы (блобы) драйверов от
производителя железки}\ и необязательный \var{GZ}: поскольку я собираю
кросс-компиляторы для нескольких целевых платформ, я создал каталог
\verb|$(HOME)/gz|\ и загружаю туда архивы исходников \emph{для всех проектов
сразу}\note{а не в /gz каждого проекта, нет смысла дублировать исходники
\gnut\ одной и той же версии}. Более простой способ\ -- просто сделать симлинк
\verb|ln -s ~/gz project/gz|\ и не переопределять переменную \verb|GZ|\ явно.

\lst{mk/gz.mk}{cross/mk/gz.mk}{make}

\secly{Макро-правила для автоматической распаковки исходников}

\lst{mk/src.mk}{cross/mk/src.mk}{make}

\secly{Общие параметры для \prog{./configure}}

\lst{mk/cfg.mk}{cross/mk/cfg.mk}{make}

\secrel{Сборка кросс-компилятора}\secdown

Для пакетов кросс-компилятора существуют два варианта сборки пакетов:

\begin{description}

\item[Пакеты с 0 в конце имени] задают сборку программ, которые будут
выполняться на \var{BUILD}-компьютере, и компилировать код для \var{TARGET}-системы, т.е. это
простейший вариант кросс-компиляции.

\item[Пакеты без 0], которые могут появиться в будущем\ --- \emph{относятся
только к сборке em\linux}, собирают кросс-компилятор \termdef{канадским
крестом}{канадский крест}:

\begin{itemize}[nosep]
\item пакет собирается на \var{BUILD}-системе\ --- ваш рабочий компьютер,
\item выполняется на \var{HOST}-системе\ --- например PC104 или роутер с
em\linux,
\item и компилирует код для \var{TARGET}-микропроцессора\ --- модуль
ввода/вывода на USB, подключенный к PC104)
\end{itemize}
 
\end{description}

\secrel{\prog{binutils0}: ассемблер и линкер}

Чтобы побыстрее получить результат, который можно сразу потестировать, соберем
сначала кросс-\prog{bintuils}, а потом все что относится к \ci шному
компилятору\note{на самом деле \prog{binutils0}\ надо собирать после
\prog{cclibs0}, так как есть зависимость от библиотек \prog{isl0}\ и
\prog{cloog0}}.

\begin{description}
\item[\var{--target}] триплет целевой системы, например \var{i386-pc-elf}
\item[\var{CFG\_ARCH CFG\_CPU}] задаются в файлах \file{arch/\$(ARCH).mk}\ и
\file{cpu/\$(CPU).mk}, и определяют опции сборки \prog{binutils/gcc} для
конкретного процессора\note{например \var{--without-fpu}\ для
\file{cpu/i486sx.mk}}
\item[\var{--with-sysroot}] каталог где должны храниться файлы для целевой
системы: откомпилированные библиотеки и каталог \file{include}
\item[\var{--with-native-system-header-dir}] имя каталога с
\file{include}-файлами, относительно \file{sysroot}
\end{description}

\lst{arch/i386.mk}{cross/arch/i386.mk}{make}
\lst{cpu/i386.mk}{cross/cpu/i386.mk}{make}
\lst{mk/bintuils.mk}{cross/mk/binutils.mk}{make}

\lstx{Файлы \prog{binutils0}\ с 
\var{TARGET}-префиксами и типовые скрипты линкера}{cross/crossbinutils.txt}

\secrel{\prog{cclibs0}: сборка библиотек поддержки \prog{gcc}}
\secrel{\prog{gmp0}: библиотека целых чисел произвольной точности}

\secrel{\prog{gcc0}}
\secup

\secrel{Сборка стандартной библиотеки \prog{newlib} (libc)}

\secrel{Поддерживаемые платформы}\secdown
\secrel{\prog{i386}: ПК и промышленные PC104}
\lst{arch/i386.mk}{cross/arch/i386.mk}{make}
\secrel{\prog{x86\_64}: серверные системы}
\lst{arch/x86\_64.mk}{cross/arch/x86_64.mk}{make}
\secrel{\prog{AVR}: Atmel AVR Mega}
\lst{arch/avr.mk}{cross/arch/avr.mk}{make}
\secrel{\prog{arm}: процессоры ARM \cm x}\label{crossarm}
\lst{arch/arm.mk}{cross/arch/arm.mk}{make}
\secrel{\prog{armhf}: SoCи Cortex-A, PXA270,..}
\lst{arch/armhf.mk}{cross/arch/armhf.mk}{make}
\secup

\secrel{Целевые аппаратные системы}\secdown
\secrel{\prog{x86}: типовой компьютер на процессоре i386+}
\lst{hw/x86.mk}{cross/hw/x86.mk}{make}
\secup

\secup

\secrel{Porting The GNU Tools To Embedded Systems}\label{portingnu}

Embed With GNU

Porting The GNU Tools To Embedded Systems

Spring 1995

Very *Rough* Draft

Rob Savoye - Cygnus Support

\url{http://ieee.uwaterloo.ca/coldfire/gcc-doc/docs/porting\_toc.html}
\secrel{Оптимизация кода}\secdown
\secrel{PGO
опитимизация}\label{pgo}\cp{https://habrahabr.ru/post/138132/}
\secup
\secup

\secrel{Микроконтроллеры \cm x}

\secrel{\prog{os86}: низкоуровневое программирование i386}\secdown\label{os86}

Если вам по каким-то причинам не подходит одна из типовых распространенных ОС,
например требуется сделать систему управления жесткого реального
времени\note{или вы любитель гадить из прикладного ПО в аппаратные порты в обход
всех соглашений и средств защиты ОС}, информация в этом разделе поможет сделать
ОС-поделку для типового Wintel ПК.

\secly{Специализированный \gnut\ для \prog{i386-pc-gnu}}

Для компиляции кода вам потребуется специально собранный из исходников
кросс-\gnut\ для целевой архитектуры i386\ --- \term{триплет}\
\verb|TARGET=i386-pc-elf|. Процесс сборки подробно описан в отдельном разделе
\ref{cross}.

\bigskip

Для упрощения не будем завязываться на особенности конкретного ПК или
эмулятора \qemu\note{VMWare, VirtualPC}, все они вполне аппаратно совместимы
с любым i386 компьютером в базовой конфигурации, для которого мы и будем
рассматривать примеры кода:

\begin{itemize}[nosep]
  \item \verb|APP=bare| metal программирование, без базовой ОС
  \item \verb|HW=x86| типовой минимальный i386 компьютер 
\end{itemize}

\lst{os86/Makefile}{os86/Makefile}{make}

\secly{Использование загрузчика (спецификация
\prog{MUltiBoot})}\label{multiboot}

\secup

\secrel{Технологии}

\secrel{Сетевое обучение}

\secrel{Базовая теоретическая подготовка}\secdown
\secrel{Математика}\secdown
\secrel{Высшая математика в упражнениях и задачах \cite{danko}}\secdown

В этом разделе будут размещены решения некторых задач из \cite{danko}\ в
``техническом''\ стиле: главное быстрый результат, а не точное
аналитическое решение, поэтому будем использовать системы компьютерной
математики.
Будут рассмотрены приемы применения OpenSource пакетов:

\begin{description}[nosep]
  \item[\maxima] \cite{maxima} символьная математика, аналог
  \prog{MathCAD}
  \item[\octave] \cite{octave} численная математика, аналог \prog{MATLAB}
  \item[\gnuplot] \cite{gnuplot} простейшее средство построения 3D/3D
  графиков
  \item[\wolfram] \url{http://www.wolframalpha.com/}\ бесплатная on-line
  система символьной математики и база знаний, примеры даны в синтаксисе
  коммерческого пакета \prog{Mathematica}, на случай если ваш ВУЗ купил учебную
  лицензию
  \item[\py] скриптовый язык программирования, в последнее время получил широкое
  применение в области численных методов, анализа данных и автоматизации,
  чаще всего применяется в комплекте с библиотеками:
\begin{description}[nosep]
\item[\prog{NumPy}] поддержка многомерных массивов (включая матрицы) и 
высокоуровневых математических функций, предназначенных для работы с ними
\item[\prog{SciPy}] библиотека предназначенная для выполнения научных и
инженерных расчётов: поиск минимумов и максимумов функций,
вычисление интегралов функций,
поддержка специальных функций,
обработка сигналов,
обработка изображений,
работа с генетическими алгоритмами,
решение обыкновенных дифференциальных уравнений,\ldots
\item[\prog{SymPy}] библиотека символьной
математики \url{https://en.wikipedia.org/wiki/SymPy}
\item[\prog{Matplotlib}] библиотека на языке программирования Python для
2D/3D визуализации данных.
Получаемые изображения могут быть использованы в качестве
иллюстраций в публикациях.
\end{description}
Подробно с применением \py\ при обработке
данных можно ознакомиться в \url{http://scipy-cookbook.readthedocs.org/}
\end{description}

\bigskip
Также этот раздел можно использовать как пример использования системы верстки
\LaTeX\ для научных публикаций\ --- смотрите \emph{исходные тексты}\ файла
\url{https://github.com/ponyatov/boox/tree/master/math/danko}\file{/danko.tex}. 

\secrel{Аналитическая геометрия на плоскости}\secdown

\secrel{Прямоугольные и полярные координаты}

\paragraph{1. Координаты на прямой. Деление отрезка в данном отношении}

Точку $M$ координатной оси $Ox$, имеющую \termdef{абсциссу}{абсцисса}\ $x$,
обозначают через $M(x)$.

Расстояние $d$ между точками $M_{1}(x_{1})$ и $M_{2}(x_{2})$ оси при любом
расположении точек на оси находятся по формуле:

\begin{equation}
d=|x_{2}-x_{1}|
\end{equation}

Пусть на произвольной прямой задан отрезок $AB$ ( $A$ --- начало отрезка, $B$
--- конец), тогда всякая третья точка $C$ этой прямой делить отрезок $AB$ в
некотором отношении $\lambda$, где $\lambda= \pm AC:CB$. Если отрезки $AC$ и
$CB$ направлены в одну сторону, то $\lambda$ приписывают знак ``плюс''; если же
отрезки $AC$ и $CB$ направлены в противоположные стороны, то $\lambda$
приписывающт знак ``минус''. Иными словами, $\lambda>0$ если точка $C$ лежит
между точками $A$ и $B$; $\lambda < 0$ если точка $C$ лежит вне отрезка $AB$.

Пусть точки $A$ и $B$ лежит на оси $Ox$, тогда \termdef{координата
точки}{координата точки} $C(\bar{x})$, делящей отрезок между точками $A(x_1)$ и
$B(x_2)$ в отношении $\lambda$, находится по формуле:

\begin{equation}
\bar x=\frac{x_1+\lambda x_2}{1+\lambda}
\end{equation}

В частности, при $\lambda=1$ получается формула для координаты середины отрезка:

\begin{equation}
\bar x = \frac{x_1+x_2}{2}
\end{equation}

\paragraph{1.}

Построить на прямой точки $A(3)$, $B(-2)$, $C(0)$, $D(\sqrt{2})$, $E(-3.5)$.

\bigskip\wolfram\bigskip\\
\verb|number line {3 , -2 , 0 , sqrt[2] , -3.5}|
\fig{}{math/danko/w111.png}{width=0.5\textwidth}

\bigskip\gnuplot
\begin{verbatim}
plot '-' 3 , -2 , 0 , sqrt(2) , -3.5
\end{verbatim}

\secup



\secup
\secup

\secup

\secrel{Прочее}\secdown
\secly{Ф.И.Атауллаханов об учебниках США и России}

\copyright\ Доктор биологических наук Фазли Иноятович Атауллаханов.\\
МГУ им. М. В. Ломоносова, Университет Пенсильвании, США

\bigskip 
\url{http://www.nkj.ru/archive/articles/19054/}
\bigskip 

\ldots

У необходимости рекламировать науку есть важная обратная сторона: каждый
американский учёный непрерывно, с первых шагов и всегда, учится излагать свои
мысли внятно и популярно. В России традиции быть понятными у учёных нет. Как
пример я люблю приводить двух великих физиков: русского Ландау и американца
Фейнмана. Каждый написал многотомный учебник по физике. Первый\ --- знаменитый
``Ландау-Лифшиц'', второй\ --- ``Лекции по физике''. Так вот, ``Ландау-Лифшиц''
прекрасный справочник, но представляет собой полное издевательство над
читателем. Это типичный памятник автору, который был, мягко говоря, малоприятным
человеком. Он излагает то, что излагает, абсолютно пренебрегая своим читателем и
даже издеваясь над ним. А у нас целые поколения выросли на этой книге, и
считается, что всё нормально, кто справился, тот молодец. Когда я столкнулся с
``Лекциями по физике''\ Фейнмана, я просто обалдел: оказывается, можно
по-человечески разговаривать со своими коллегами, со студентами, с аспирантами.
Учебник Ландау\ --- пример того, как устроена у нас вся наука. Берёшь текст
русской статьи, читаешь с самого начала и ничего не можешь понять, а иногда
сомневаешься, понимает ли автор сам себя. Конечно, крупицы осмысленного и
разумного и оттуда можно вынуть. Но автор явно считает, что это твоя работа\ ---
их оттуда извлечь. Не потому, что он не хочет быть понятым, а потому, что его не
научили правильно писать. Не учат у нас человека ни писать, ни говорить понятно,
это считается неважным.

\ldots

Думаю, американская наука в целом устроена именно так: она продаёт не просто
себя, а всю свою страну. Сегодня американцы дороги не метут, сапоги не тачают,
даже телевизоры не собирают, за них это делает весь остальной мир. А что же
делают американцы\,? Самая богатая страна в мире\,? Они объяснили, в первую
очередь самим себе, а заодно и всему миру, что они\ --- мозг планеты. Они
изобретают. ``Мы придумываем продукты, а вы их делайте. В том числе и для нас''.
Это прекрасно работает, поэтому они очень ценят науку.

\ldots
\secrel{Настройка редактора/IDE \prog{\vim}}\label{vim}\secdown

При использовании редактора/IDE \prog{\vim}\ удобно настроить сочетания клавиш и
подсветку синтаксиса языков, которые вы использете так, как вам удобно.

\secrel{для вашего собственного скриптового языка}

Через какое-то время практики FSP у вас выработается один диалект скриптов для
всех программ, соответсвующий именно вашим вкусам в синтаксисе, и в этом случае
его нужно будет описать только в файлах ~/.vim/(ftdetect|syntax).vim, и
привязать их к расширениям через dot-файлы \vim\ в вашем домашнем
каталоге:\bigskip

\begin{tabular}{l l l}
filetype.vim & \vim & привязка расширений файлов (.src .log) к настройкам \vim
\\
syntax.vim & \vim & синтаксическая подсветка для скриптов \\
~/.vimrc & \linux & настройки для пользователя \\
~/vimrc & \win &\\
~/.vim/ftdetect/src.vim & \linux & привязка команд к расширению .src \\
~/vimfiles/ftdetect/src.vim & \win & \\
~/.vim/syntax/src.vim & \linux & синтаксис к расширению .src \\
~/vimfiles/syntax/src.vim & \win &\\
\end{tabular}



\secup
\secup

\addcontentsline{toc}{part}{Литература}
\begin{thebibliography}{99}

\bibitem{dragon} \textbf{\emph{Dragon Book}: Компиляторы}

Ахо, Сети, Ульман

\bibitem{sicp} \textbf{\emph{SICP}: Структура и интерпретация компьютерных
программ}

\end{thebibliography}

\addcontentsline{toc}{chapter}{Индекс}\printindex
\end{document}
