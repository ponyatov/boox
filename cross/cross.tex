\secrel{Сборка кросс-компилятора \gnut\ из исходных
текстов}\label{cross}\secdown

Если вам по каким-то причинам не подходит одна из типовых сборок
кросс-компиляторов, поставляемых в виде готовых бинарных пакетов из репозитория
вашего дистрибутива \linux, \gnut\ можно легко скопилировать \emph{из исходных
текстов}\ и установить в систему, даже имея только пользовательские права
доступа.

\bigskip
Сборка \gnut\ из исходников может понадобиться, если вы хотите:
\begin{itemize}[nosep]
  \item самую свежую или какую-то конкретную версию \gnut
  \item опции компиляции: малораспространенный \var{target}-процессор,
  \emph{нетиповой формат файлов объектного кода}\note{например для i386
  может понадобится сборка кросс-компилятора с
  \var{--target=i486-none-elf}\ \ref{os86}\ или \var{i686-linux-uclibc}\ вместо
  типовой компиляции для \linux\ типа \var{i486-linux-gnu}}\ или
  экспериментальные оптимизаторы, не включенные в бинарные пакеты из
  дистрибутива \linux
  \item полпроцента ускорения работы компилятора благодаря жесткой оптимизации
  его машинного кода точно под ваш рабочий компьютер
  (\verb|-march=native -mtune=native -O3|)
\end{itemize}

\bigskip
При сборке используется утилита \prog{make}\ \ref{make}, которой можно передать
набор переменных конфигурирования. В таблице перечислен набор переменных
конфигурирования сборки с указанием их значения по умолчанию\note{также
приведены часто используемые варианты значения}\ и имя mk-файла, где оно
задано:

\bigskip
\begin{tabular}{l l l l}
\var{HW} & \verb|x86| & \verb|Makefile| & \var{qemu vmware virtualpc}\\
	&&&\var{x86 pc686 amd64}\\
	&&&\var{cortexM avr8}\\
\var{CPU} & \verb|i386| & \verb|hw/$(HW).mk| &\\
\var{ARCH} & \verb|i386| & \verb|cpu/$(CPU).mk| &\\
\underline{\var{TARGET}} & \verb|$(CPU)-pc-elf| & \verb|hw/$(HW).mk| &
\var{i686-linux-uclibc x86\_64-linux-gnu}\\
	&&&\var{i386-pc-elf arm-none-eabi avr-none-elf}\\
\end{tabular}

\secly{Подготовка \var{BUILD}-системы: необходимое ПО}

Для сборки необходимо установить следующие пакеты:

\begin{verbatim}
sudo apt install gcc g++ make flex bison m4 bc bzip2 xz-utils libncurses-dev 
\end{verbatim}

\secly{\prog{dirs}: создание структуры каталогов}

\begin{verbatim}
user@bs:~/boox/cross$ make dirs
mkdir -p
    /home/user/boox/cross/gz /home/user/boox/cross/src /home/user/boox/cross/tmp
    /home/user/boox/cross/toolchain /home/user/boox/cross/root
\end{verbatim}

Командной \verb|make dirs|\ создается набор вспомогательных каталогов:

\bigskip
\begin{tabular}{l l l}
\var{TC} & \verb|$(CWD)/toolchain| & каталог установки кросс-компилятора \\
\var{ROOT} & \verb|$(CWD)/root| & каталог файловой системы для целевого
em\linux\\
\hline
\var{CWD} & \verb|$(CURDIR)| & текущий каталог \\
\var{GZ} & \verb|$(CWD)/gz| & архивы исходных текстов GNU Toolchain, загрузчика,
и библиотек\\
\var{SRC} & \verb|$(CWD)/src| & каталог для распаковки исходников \\
\var{TMP} & \verb|$(CWD)/tmp| & каталог для out-of-tree сборки GNU toolchain \\
\end{tabular}
\bigskip

\lst{mk/dirs.mk}{cross/mk/dirs.mk}{make}

\secly{Сборка в ОЗУ на ramdiskе}

Если у вас есть админские права и достаточный объем RAM, после выполнения
\verb|make dirs|\ рекомендуется примонировать на каталоги \var{SRC} и \var{TMP}
файловую систему tmpfs\ --- это значительно ускорит компиляцию, т.к. все
временные файлы будут хранится только в ОЗУ:

\lstx{/etc/fstab}{cross/fstab.txt}

Если вы прописали монтирование \term{ramdisk}ов в \file{/etc/fstab}, или
сделали \verb|mount -t tmpfs|\ вручную, может оказаться нужным запускать
\prog{make} с явным указанием значений переменных SRC/TMP:

\begin{verbatim}
make blablabla SRC=/home/yourlogin/src TMP=/home/yourlogin/tmp
\end{verbatim}

\secly{Пакеты системы кросс-компиляции}

\begin{description}
\item[\gnut] \ \\\lstx{mk/pack\_cross.mk}{cross/mk/pack_cross.mk}{make}
\item[\prog{newlib}] стандартная библиотека \prog{libc}\\
\end{description}

\secly{\prog{gz}: загрузка исходного кода для пакетов}



\secrel{Поддерживаемые платформы}\secdown
\secrel{\prog{i386}: ПК и промышленные PC104}
\lst{arch/i386.mk}{cross/arch/i386.mk}{make}
\secrel{\prog{x86\_64}: серверные системы}
\lst{arch/x86\_64.mk}{cross/arch/x86_64.mk}{make}
\secrel{\prog{AVR}: Atmel AVR Mega}
\lst{arch/avr.mk}{cross/arch/avr.mk}{make}
\secrel{\prog{arm}: процессоры ARM \cm x}\label{crossarm}
\lst{arch/arm.mk}{cross/arch/arm.mk}{make}
\secrel{\prog{armhf}: SoCи Cortex-A, PXA270,..}
\lst{arch/armhf.mk}{cross/arch/armhf.mk}{make}
\secup

\secrel{Целевые аппаратные системы}\secdown
\secrel{\prog{x86}: типовой компьютер на процессоре i386+}
\lst{hw/x86.mk}{cross/hw/x86.mk}{make}
\secup

\secrel{\prog{cclibs0}: сборка библиотек поддержки \prog{gcc}}
\secrel{\prog{binutils0}}
\secrel{\prog{gcc0}}
\secup

