\secrel{Сборка кросс-компилятора \gnut\ из исходных
текстов}\label{cross}\secdown

Если вам по каким-то причинам не подходит одна из типовых сборок
кросс-компиляторов, поставляемых в виде готовых бинарных пакетов из репозитория
вашего дистрибутива \linux, \gnut\ можно легко скопилировать \emph{из исходных
текстов}\ и установить в систему, даже имея только пользовательские права
доступа.

\bigskip
Сборка \gnut\ из исходников может понадобиться, если вы хотите:
\begin{itemize}[nosep]
  \item самую свежую или какую-то конкретную версию \gnut
  \item опции компиляции: малораспространенный \var{target}-процессор,
  \emph{нетиповой формат файлов объектного кода}\note{например для i386
  может понадобится сборка кросс-компилятора с
  \var{--target=i486-none-elf}\ \ref{os86}\ или \var{i686-linux-uclibc}\ вместо
  типовой компиляции для \linux\ типа \var{i486-linux-gnu}}\ или
  экспериментальные оптимизаторы, не включенные в бинарные пакеты из
  дистрибутива \linux
  \item полпроцента ускорения работы компилятора благодаря жесткой оптимизации
  его машинного кода точно под ваш рабочий компьютер
  (\verb|-march=native -mtune=native -O3|)
\end{itemize}

\bigskip
При сборке используется утилита \prog{make}\ \ref{make}, которой можно передать
набор переменных конфигурирования. В таблице перечислен набор переменных
конфигурирования сборки с указанием их значения по умолчанию\note{также
приведены часто используемые варианты значения}\ и имя mk-файла, где оно
задано:

\bigskip
\begin{tabular}{l l l l}
\var{APP} & \verb|cross| & \verb|Makefile| & приложение: условное
имя проекта\\&&&(только латиница, буквы a-z)\\
\var{HW} & \verb|x86| & \verb|Makefile| & \var{qemu vmware virtualpc}\\
	&&&\var{x86 pc686 amd64}\\
	&&&\var{cortexM avr8}\\
\var{CPU} & \verb|i386| & \verb|hw/$(HW).mk| &\\
\var{ARCH} & \verb|i386| & \verb|cpu/$(CPU).mk| &\\
\var{TARGET} & \verb|$(CPU)-pc-elf| & \verb|hw/$(HW).mk| &
\var{i686-linux-uclibc x86\_64-linux-gnu}\\
	&&&\var{i386-pc-elf arm-none-eabi avr-none-elf}\\
\end{tabular}

\secly{\var{APP}/\var{HW}: приложение/платформа}

Для сборки необходимо выбрать имя проекта\note{только латиница, буквы a-z}\ и
аппаратную платформу, для которой будет настраиваться пакет кросс-компилятора.

Особенно это важно для варианта сборки, когда собирается не только
кросс-компилятор, но и базовая ОС\ --- минимальная \linux-система из ядра,
libc и дополнительных прикладных библиотек. В этом случае \prog{APP/HW}
используются для формирования имен файлов ядра \verb|$(APP)$(HW).kernel|,
названия и состава загрузочного образа \verb|$(APP)$(HW).rootfs|, и внутренних
настроек.

\secly{Подготовка \var{BUILD}-системы: необходимое ПО}

Для сборки необходимо установить следующие пакеты:

\begin{verbatim}
sudo apt install gcc g++ make flex bison m4 bc bzip2 xz-utils libncurses-dev 
\end{verbatim}

\secly{\prog{dirs}: создание структуры каталогов}

\begin{verbatim}
user@bs:~/boox/cross$ make dirs
mkdir -p
    /home/user/boox/cross/gz /home/user/boox/cross/src /home/user/boox/cross/tmp
    /home/user/boox/cross/toolchain /home/user/boox/cross/root
\end{verbatim}

Командной \verb|make dirs|\ создается набор вспомогательных каталогов:

\bigskip
\begin{tabular}{l l l}
\var{TC} & \verb|$(CWD)/$(APP)$(ROOT).cross| & каталог установки
кросс-компилятора \\
\var{ROOT} & \verb|$(CWD)/$(APP)$(ROOT)| & каталог файловой системы для целевого
em\linux\\
\hline
\end{tabular}

\begin{tabular}{l l l}
\var{CWD} & \verb|$(CURDIR)| & текущий каталог \\
\var{GZ} & \verb|$(CWD)/gz| & архивы исходных текстов GNU Toolchain, загрузчика,
и библиотек\\
\var{SRC} & \verb|$(CWD)/src| & каталог для распаковки исходников \\
\var{TMP} & \verb|$(CWD)/tmp| & каталог для out-of-tree сборки GNU toolchain \\
\end{tabular}
\bigskip

\lst{mk/dirs.mk}{cross/mk/dirs.mk}{make}

\secly{Сборка в ОЗУ на ramdiskе}

Если у вас есть админские права и достаточный объем RAM, после выполнения
\verb|make dirs|\ рекомендуется примонировать на каталоги \var{SRC} и \var{TMP}
файловую систему tmpfs\ --- это значительно ускорит компиляцию, т.к. все
временные файлы будут хранится только в ОЗУ:

\lstx{/etc/fstab}{cross/fstab.txt}

Если вы прописали монтирование \term{ramdisk}ов в \file{/etc/fstab}, или
сделали \verb|mount -t tmpfs|\ вручную, может оказаться нужным запускать
\prog{make} с явным указанием значений переменных SRC/TMP:

\begin{verbatim}
make blablabla SRC=/home/user/src TMP=/home/user/tmp
\end{verbatim}

\secly{Пакеты системы кросс-компиляции}

\begin{description}
\item[\gnut] \ \\\lstx{mk/pack\_cross.mk}{cross/mk/pack_cross.mk}{make}
\item[\prog{newlib}] стандартная библиотека \prog{libc}\\
\end{description}

\secly{\prog{gz}: загрузка исходного кода для пакетов}

\begin{verbatim}
user@bs$ make APP=cross HW=x86 GZ=/home/user/gz gz
\end{verbatim}

В примере команды показано два обязательных параметра \var{APP/HW}\note{по ним
могут закачиваться дополнительные файлы исходников, зависящие от платформы\ ---
например исходник загрузчика или бинарные файлы (блобы) драйверов от
производителя железки}\ и необязательный \var{GZ}: поскольку я собираю
кросс-компиляторы для нескольких целевых платформ, я создал каталог
\verb|$(HOME)/gz|\ и загружаю туда архивы исходников \emph{для всех проектов
сразу}\note{а не в /gz каждого проекта, нет смысла дублировать исходники
\gnut\ одной и той же версии}. Более простой способ\ -- просто сделать симлинк
\verb|ln -s ~/gz project/gz|\ и не переопределять переменную \verb|GZ|\ явно.

\lst{mk/gz.mk}{cross/mk/gz.mk}{make}

\secly{Макро-правила для автоматической распаковки исходников}

\lst{mk/src.mk}{cross/mk/src.mk}{make}

\secly{Общие параметры для \prog{./configure}}

\lst{mk/cfg.mk}{cross/mk/cfg.mk}{make}

\secrel{Сборка кросс-компилятора}\secdown

Для пакетов кросс-компилятора существуют два варианта сборки пакетов:

\begin{description}

\item[Пакеты с 0 в конце имени] задают сборку программ, которые будут
выполняться на \var{BUILD}-компьютере, и компилировать код для \var{TARGET}-системы, т.е. это
простейший вариант кросс-компиляции.

\item[Пакеты без 0], которые могут появиться в будущем\ --- \emph{относятся
только к сборке em\linux}, собирают кросс-компилятор \termdef{канадским
крестом}{канадский крест}:

\begin{itemize}[nosep]
\item пакет собирается на \var{BUILD}-системе\ --- ваш рабочий компьютер,
\item выполняется на \var{HOST}-системе\ --- например PC104 или роутер с
em\linux,
\item и компилирует код для \var{TARGET}-микропроцессора\ --- модуль
ввода/вывода на USB, подключенный к PC104)
\end{itemize}
 
\end{description}

\secrel{\prog{binutils0}: ассемблер и линкер}

Чтобы побыстрее получить результат, который можно сразу потестировать, соберем
сначала кросс-\prog{bintuils}, а потом все что относится к \ci шному
компилятору\note{на самом деле \prog{binutils0}\ надо собирать после
\prog{cclibs0}, так как есть зависимость от библиотек \prog{isl0}\ и
\prog{cloog0}}.

Из важных опций можно указать

\begin{description}
\item[\var{--target}] триплет целевой системы, например \var{i386-pc-elf}
\item[\var{CFG\_ARCH CFG\_CPU}] задаются в файлах \file{arch/\$(ARCH).mk}\ и
\file{cpu/\$(CPU).mk}, и определяют опции сборки \prog{binutils/gcc} для
конкретного процессора\note{например \var{--without-fpu}\ для
\file{cpu/i486sx.mk}}
\end{description}

\lst{arch/i386.mk}{cross/arch/i386.mk}{make}
\lst{cpu/i386.mk}{cross/cpu/i386.mk}{make}
\lst{mk/bintuils.mk}{cross/mk/binutils.mk}{make}

\lstx{Файлы \prog{binutils0}\ с 
\var{TARGET}-префиксами и типовые скрипты линкера}{cross/crossbinutils.txt}

\secrel{\prog{cclibs0}: сборка библиотек поддержки \prog{gcc}}
\secrel{\prog{gmp0}: библиотека целых чисел произвольной точности}

\secrel{\prog{gcc0}}
\secup

\secrel{Сборка стандартной библиотеки \prog{newlib} (libc)}

\secrel{Поддерживаемые платформы}\secdown
\secrel{\prog{i386}: ПК и промышленные PC104}
\lst{arch/i386.mk}{cross/arch/i386.mk}{make}
\secrel{\prog{x86\_64}: серверные системы}
\lst{arch/x86\_64.mk}{cross/arch/x86_64.mk}{make}
\secrel{\prog{AVR}: Atmel AVR Mega}
\lst{arch/avr.mk}{cross/arch/avr.mk}{make}
\secrel{\prog{arm}: процессоры ARM \cm x}\label{crossarm}
\lst{arch/arm.mk}{cross/arch/arm.mk}{make}
\secrel{\prog{armhf}: SoCи Cortex-A, PXA270,..}
\lst{arch/armhf.mk}{cross/arch/armhf.mk}{make}
\secup

\secrel{Целевые аппаратные системы}\secdown
\secrel{\prog{x86}: типовой компьютер на процессоре i386+}
\lst{hw/x86.mk}{cross/hw/x86.mk}{make}
\secup

\secup

\secrel{Porting The GNU Tools To Embedded Systems}\label{portingnu}

Embed With GNU

Porting The GNU Tools To Embedded Systems

Spring 1995

Very *Rough* Draft

Rob Savoye - Cygnus Support

\url{http://ieee.uwaterloo.ca/coldfire/gcc-doc/docs/porting\_toc.html}