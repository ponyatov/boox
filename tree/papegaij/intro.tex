\secrel{Introduction}\secdown

Tree structures have been, and probably will be for a considerable time in the
future, a widely used way of organising and working with data. Tree structures
are used to represent the structure of an input file\note{concrete and abstract
syntax trees}, user interface components, the representation of HTML
pages\note{the document object model}, XML and many more. Due its wide
acceptance, extensive research has been spent on working with tree structures.

This thesis is placed in the context of working with tree structures in an
object-oriented programming environment. The main focus is on defining the
runtime organisation of the tree and applying algorithms on this structure. The
origin of the tree\ --- the system responsible for constructing the tree
structure\ --- and the actual construction of the tree are discussed, but fall
outside the main research area.

In this chapter, an introduction on compiler construction is given, in
\ref{pape11}. This section shows how an abstract syntax tree is acquired, and
what the typical operations are that need to be performed on an AST.
\ref{pape12} describes the problem statement of this thesis. Finally, the
outline of this thesis is given in \ref{pape13}.

\secrel{Compiler Construction and Abstract Syntax Trees}\label{pape11}\secdown

\secrel{Lexical Analysis and Parsing}

\secrel{Construction of the AST}

\secrel{Context Checking and Code Generation}

\secup

\secrel{Problem Statement}\label{pape12}

\secrel{Outline}\label{pape13}

\secup