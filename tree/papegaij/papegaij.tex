\secrel{The Tree Processing Language\\Defining the structure and behaviour of a
tree}\label{papegaij}\secdown

\begin{tabular}{l l l}
\cp{\url{http://essay.utwente.nl/705/1/scriptie_Papegaaij.pdf}}&
\copyright & E. Papegaaij 
 \email{e.papegaaij@alumnus.utwente.nl} \\
&Supervisors &dr. ir. Theo C. Ruys\\
&&ir. Philip K.F. Hölzenspies\\
&&dr. ir. Arend Rensink\\
&Institute &University of Twente\\
&Chair &Formal Methods and Tools\\
\end{tabular}
\bigskip

Enschede, March 7, 2007

\secly{Abstract}

Tree structures are commonly used in many applications. One of these is a
compiler, in which the tree is called an abstract syntax tree (AST). Different
techniques have been developed for building and working with ASTs. However, many
of these techniques are limited in their applicability, require major effort to
implement or introduce maintenance problems in an evolving application.

This thesis introduces the Tree Processing Language, a language for defining the
structure of a tree and adding functionality to this tree. The compiler
\prog{TPLc} is used to produce the actual class hierarchy implementing the
specified tree. TPL provides a clear separation between the structure of a tree,
a \emph{tree definition}, and behaviour of a tree, \emph{logic specifications}.
Different aspects of the behaviour of a tree can be provided in separate logic
specifications, allowing a clear separation of concerns.

\prog{TPLc} generates a heterogeneous tree structure with strictly typed
children. Functionality in a logic specification is specified using the
inheritance pattern. To allow different inheritance trees in different logic
specifications, the inheritance pattern is enhanced with multiple inheritance.
For languages that do not support multiple inheritance, the inheritance pattern
with composition is developed.

To prove the applicability of TPL, \prog{TPLc} is written in TPL. When compared
with an implementation in Java, this implementation provides a better separation
of concerns and is easier to maintain.

\secly{Preface}

Compiler construction has always been one of my favourite fields of software
engineering. In the past few years I’ve written several parsers and compilers.
Of these compilers, the compiler for the functional programming language Tina
has been the most challenging. I used a hand-written heterogeneous abstract
syntax tree as underlying data structure. The most important algorithm applied
onto this AST, the transformation of Tina into a core lambda expression
language, was written as part of these AST node classes. However, the
overwhelming number of AST classes (almost 100) made this approach increasingly
difficult to maintain when other algorithms (such as a lambda lifter) where
added. At that moment, it became clear that a more structured approach was
required. To keep the development of an application, based on a heterogeneous
tree, maintainable, different algorithms needed to be separated in different
files. The development of tpl is an attempt to provide such an environment.

When I first approached Theo C. Ruys, my premiere supervisor, for an assignment,
I has no idea I would be solving this problem, which had bothered me for a long
time. At first, ambitious as I was, I proposed to design and implement a
completely new parser generator. Luckily, Theo slowed me down a bit and directed
me to focus on the real problem: the heterogeneous AST.

For his help in concreting the features of tpl, reading and correcting this
thesis and his patience during the endless discussions we had last year, I would
like to thank Theo C. Ruys, my premiere supervisor. His guiding helped me
structure my thoughts, to be able to write them down. I would also like to thank
Philip K.F.
H\"olzenspies for his help in writing and formatting this thesis. His knowledge
of the English language has proven to be far better than mine. Last, but not least,
I would like to thank Arend Rensink for having taken the time to examine this
thesis.

\bigskip
Emond Papegaaij\\
Enschede, March 7, 2007
\bigskip

\secrel{Introduction}\secdown

Tree structures have been, and probably will be for a considerable time in the
future, a widely used way of organising and working with data. Tree structures
are used to represent the structure of an input file\note{concrete and abstract
syntax trees}, user interface components, the representation of HTML
pages\note{the document object model}, XML and many more. Due its wide
acceptance, extensive research has been spent on working with tree structures.

This thesis is placed in the context of working with tree structures in an
object-oriented programming environment. The main focus is on defining the
runtime organisation of the tree and applying algorithms on this structure. The
origin of the tree\ --- the system responsible for constructing the tree
structure\ --- and the actual construction of the tree are discussed, but fall
outside the main research area.

In this chapter, an introduction on compiler construction is given, in
\ref{pape11}. This section shows how an abstract syntax tree is acquired, and
what the typical operations are that need to be performed on an AST.
\ref{pape12} describes the problem statement of this thesis. Finally, the
outline of this thesis is given in \ref{pape13}.

\secrel{Compiler Construction and Abstract Syntax Trees}\label{pape11}\secdown

\secrel{Lexical Analysis and Parsing}

\secrel{Construction of the AST}

\secrel{Context Checking and Code Generation}

\secup

\secrel{Problem Statement}\label{pape12}

\secrel{Outline}\label{pape13}

\secup

\secup