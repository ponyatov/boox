\secrel{Раскраска карт}

Этот раздел использует известную математическую проблему\ --- \emph{раскраска
географических карт}\ --- в качестве иллюстрации применения набора фактов и
логических правил. Рассмотренная \prolog-программа показывает представление
смежных регионов карты, ее раскраски, и определение конфликтов раскраски: когда
\emph{два смежных региона имеют одинаковый цвет}.  Секция также показывает
применение концепции \termdef{семантического дерева}{семантическое деревj} и его
применение в логическом программировании.

Согласно формулировке известной математической задачи по раскраске смежных
плоских регионов\note{таких как географические карты}, необходимо подобрать
минимум цветов раскраски, и цвета регионов, так что никакие два смежных региона
не имеют один цвет. Два региона являются смежными, если они имеют некоторый
общий сегмент границы, например\note{упрощенно, только прямоугольные области}:

\fig{}{prolog/fisher/f2_1_1.png}{height=0.3\textheight}

По данным численным именам регионов строим представление в виде \termdef{графа
смежности}{граф смежности}:

\fig{}{prolog/fisher/f2_1_2.pdf}{height=0.3\textheight}

