\secrel{Раскраска карт}

Этот раздел использует известную математическую проблему\ --- \emph{раскраска
географических карт}\ --- в качестве иллюстрации применения набора фактов и
логических правил. Рассмотренная \prolog-программа показывает представление
смежных регионов карты, ее раскраски, и определение конфликтов раскраски: когда
\emph{два смежных региона имеют одинаковый цвет}.  Секция также показывает
применение концепции \termdef{семантического дерева}{семантическое деревj} и его
применение в логическом программировании.

Согласно формулировке известной математической задачи по раскраске смежных
плоских регионов\note{таких как географические карты}, необходимо подобрать
минимум цветов раскраски, и цвета регионов, так что никакие два смежных региона
не имеют один цвет. Два региона являются смежными, если они имеют некоторый
общий сегмент границы, например\note{упрощенно, только прямоугольные области}.
По данным численным именам регионов строим представление в виде \termdef{графа
смежности}{граф смежности}:

\begin{tabular}{p{0.4\textwidth} p{0.4\textwidth}}
\fig{}{prolog/fisher/f2_1_1.png}{height=0.3\textheight}&
\fig{}{prolog/fisher/f2_1_2.pdf}{height=0.35\textheight}\\
\end{tabular}

Мы удалили все границы, и нарисовали дугу между именами каждых двух смежных
областей. Фактически граф смежности содержит полную оригинальную информацию о
смежности областей. Для представления информации о смежности в синтаксисе
\prolog а запишем следующее:

\lst{\ }{prolog/fisher/f21_1.pl}{Prolog}

это набор выражений устанавливает факт смежности $A \rightarrow B$:
\verb|adjacent(A,B)|.\bigskip

Если загрузить этот файл в \prolog-систему, можно проверить работу целей:

\begin{verbatim}
?- adjacent(2,3). 
true .
?- adjacent(5,3). 
false . 
?- adjacent(3,R). 
R = 1 ; 
R = 2 ; 
R = 4 ; 
false . 
\end{verbatim}

Аналогично можно задать два набора раскраски регионов используя единичные
заключения: вариант \var{a} и вариант \var{b}:

\lst{\ }{prolog/fisher/f21_2.pl}{Prolog}
в форме
\begin{verbatim}
<имя отношения:color> (
    <номер зоны/узла графа>,
    <присвоенный цвет>,
    <имя раскраски>
).
\end{verbatim}

Что обозначает \termdef{факт}{факт}: ``имеется отношение color между номером
узла, цветом и именем раскраски''\note{причем не указывается какой
элемент главный или подчиненный, все элементы отношения равноправны}.\bigskip

Теперь мы хотим написать \prolog-определение конфликта раскрасок, имея в виду
совпадение цветов для двух регионов, например:

\lst{\ }{prolog/fisher/f21_3.pl}{Prolog}

Например,

\begin{verbatim}
?- conflict(a). 
false . 
?- conflict(b). 
true . 
?- conflict(Which). 
Which = b .
\end{verbatim}

Запрашивая отношение с неким значением-константой, или переменной\note{имя с
большой буквы} (последний случай), мы получаем от \prolog-системы заключение:
выполняется ли запрошенное отношение-\term{цель} и при каких значениях
переменных, имея в виду ранее определенный \term{набор фактов и отношений}\note{которые 
являются \term{базой знаний}, или \term{экспертной системой}}. В случае
использования переменной \prolog\ выдаст нам \emph{все} значения переменных, для
которых запрос истинен.

\bigskip
Можно определить другое отношение с тем же именем \verb|conflict| но с другим
количеством логических параметров:

\lst{\ }{prolog/fisher/f21_4.pl}{Prolog}

\prolog\ позволяет отличать два отношения с одинаковым именем: одно имеет один
параметр \verb|conflict/1|, а другой\ --- \verb|conflict/3|.\note{/цифра имеет
название \term{арность}}

\begin{verbatim}
?- conflict(R1,R2,b). 
R1 = 2   R2 = 4 
?- conflict(R1,R2,b),color(R1,C,b). 
R1 = 2   R2 = 4   C = blue 
\end{verbatim}

Последняя \term{цель} значит что регионы 2 и 4 связаны (adjacent) и оба синие
(blue). \term{Обоснованные} случаи, такие как \verb|conflict(2,4,b)|, называются
\termdef{консеквенцией}{консеквенция} или \termdef{выводом}{вывод
\prolog-программы} \prolog-программы. Один из способов демонстрации
консеквенции\ --- нарисовать \termdef{дерево заключений}{дерево заключений},
которое имеет консеквенцию в корне дерева, используя заключения программы для
обхода дерева, получая в результате конечное дерево, в котором все листья имеют
истинное значение. Например следующее дерево заключений может быть построено
используя полностью обоснованные заключения программы без переменных:

\noindent\fig{\ }{prolog/fisher/f2_1_3.pdf}{width=0.7\textwidth}

Нижняя левая ветка дерева соответствует unit clause:

\begin{verbatim}
adjacent(2,4). 
\end{verbatim}

которая в \prolog е эквивалента clause

\begin{verbatim}
adjacent(2,4) :- true. 
\end{verbatim}

С другой стороны \verb'conflict(1,3,b)' не является consequence в
\prolog-программе, так как невозможно construct finite clause tree используя
grounded clauses \var{P} содержащие все \verb'true' листья. Аналогично
\verb'conflict(a)' не консеквенция, как можно ожидать. В последующих секциях
clause деревья в subsequent sections описаны более подробно.

Мы повторно рассмотрим проблему раскраски карт в \ref{fish29}, где мы
разработаем \prolog-программу которая генерирует все возможные схемы
раскраски\note{given colors to color with}. Известная Гипотеза Четырех Цветов
гласит что любая плоская карта требует для раскраски не более 4х цветов. Это
было доказано в работе Appel и Haken (1976). Решение использовало компьютерную
программу\note{не на \prolog е} для проверки всевоможных карт, с целью выявить
возможные проблемные случаи. Следующая схема раскраски например требует не менее
4х цветов:

\fig{}{prolog/fisher/f2_1_4.png}{height=0.3\textheight}

\paragraph{Упражнение 2.1} Если карта имеет \var{N} регионов, определите сколько
вычислений должно быть выполнено для определения есть ли конфликт раскраски.
Аргументируйте используя program clause деревья.