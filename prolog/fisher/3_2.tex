\secrel{3.2 Cut}\label{fish32}

The Prolog \termdef{cut}{cut} predicate, or \verb'!', eliminates choices is a
Prolog derivation tree. To illustrate, first consider a cut in a goal. For
example, consider goal \verb|?-p(X),!.| for the program P used in section
\ref{fish31}.

\begin{framed}
The cut goal succeeds whenever it is the current goal, AND the derivation tree
is trimmed of all other choices on the way back to and including the point in
the derivation tree where the cut was introduced into the sequence of goals.
\end{framed}

For the previous derivation tree, this means that the branches labeled \#2 and
\#3 are eliminated, and hence the entire subtrees below these two edges are also
cut off. This then corresponds to only producing the first answer X=a:

\begin{verbatim}
?- p(X),!. 
X=a ; 
no 
\end{verbatim}

Here we tried to get Prolog to find some more answers using ';' but they have
already been cut off. Consider also:

\begin{verbatim}
?- r(X),!,s(Y). 
X=a Y=a ; 
X=a Y=b ; 
X=a Y=c ; 
no 
\end{verbatim}

Note that there is no backtracking into that first goal. Also,
\begin{verbatim}
?- r(X), s(Y), !. 
X=a Y=a ; 
no
\end{verbatim}
as expected.

Suppose that a cut occurs in the body of the program. The cut rule (above) still
applies when the cut appears as a called subgoal. The cut is used in the body of
a given clause so as to avoid using clauses appearing after the given clause in
the program. To illustrate, consider the following program:

\begin{verbatim}
part(a). part(b). part(c). 
red(a). black(b). 
color(P,red) :- red(P),!. 
color(P,black) :- black(P),!. 
color(P,unknown).
\end{verbatim}

The intention is to determine a color for a part based upon specific stored
information, or else conclude that the color is 'unknown' otherwise.

A derivation tree for goal \verb|?- color(a,C)| is

\fig{}{prolog/fisher/f3_2_1.png}{width=0.7\textwidth}

which corresponds with

\begin{verbatim}
?- color(a,C). 
C = red 
\end{verbatim}

and a derivation tree for goal \verb|?- color(c,C)| is

\fig{}{prolog/fisher/f3_2_2.png}{width=0.95\textwidth}

which corresponds with

\begin{verbatim}
?- color(c,C). 
C = unknown 
\end{verbatim}

The Prolog cut is a procedural device for controlling goal satisfaction. The use
of cut affects the meanings of programs. For example, in the 'color' program,
the following program clause tree says that 'color(a,unknown)' should be a
consequence of the program:

\fig{}{prolog/fisher/f3_2_3.png}{width=0.3\textwidth}

\begin{verbatim}
\end{verbatim}

\begin{verbatim}
\end{verbatim}

\begin{verbatim}
\end{verbatim}

\begin{verbatim}
\end{verbatim}

\begin{verbatim}
\end{verbatim}

\begin{verbatim}
\end{verbatim}

