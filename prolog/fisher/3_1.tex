\secrel{Деривационные деревья, выборы и унификация}\label{fish31}

Для иллюстрации того, как \prolog-программа создает ответы, рассмотрим следующую
простую программу регистрации данных (это не функции):

\lst{}{prolog/fisher/3_1.pl}{Prolog}

\paragraph{Упражнение \ref{fish31}.1} Загрузите программу \prog{P}\ в \prolog\ и
посмотрите что случится при вводе цели \verb|?-p(X)|. Когда будет выведен ответ,
нажимайте \keys{;} чтобы \prolog\ продолжил трассировку и нашел все ответы.

\paragraph{Упражнение \ref{fish31}.2} Загрузите программы, включите
трассировку, и посмотрите что происходит при вводе той же цели.
Нажимайте \keys{Enter} в каждой строке трассировки, и \keys{;} в
конце строки ответа, чтобы найти все остветы. Используйте \verb|?-help(trace)|
если необходимо.

\lstv{Трассировка}{prolog/fisher/3_1.trace}

Следующая диаграмма показывает полное \termdef{дерево вывода}{дерево вывода} для
цели \verb|?-p(X)|. Ребра помечены номером утверждения в исходномо файле
программы \prog{P}, которое было использовано для подмены цели подцелями.
Прямые потомки под каждой (под)целью в дереве вывода соответствуют
\term{вариантам выбора}. Например корневая цель \verb|p(X)|
\termdef{унифицируется}{унификация} заголовками утверждений \#1, \#2, \#3,
порождая три выбора.

\fig{}{prolog/fisher/f3_1_1.png}{width=0.9\textwidth}

Трассировка упражнения \ref{fish31}.2 для цели \verb|?-p(X)| соответствует
обходу дерева вывода вглубь. Каждый узел дерева вывода \prolog а в определенный
момент времени становится текущей целью. Аналогично каждый узел\ ---
последовательность субцелей. Ребра сразу ниже узла соотвутсвуют доступным
выборам замены для текущего узла. Текущий side clause, номер которого отмечает
дугу в дереве вывода, тестируется следующим способом: если самая левая подцель
текущего узла\note{отмечена как \var{g1} в небольшой диаграмме ниже}
унифицируется головой side clause\note{отмечена как \var{h} в диаграмме}, затем
самая левая подцель заменяется телом side clause\note{ \var{b1,b2,...,bn}}.
Графически мы можем это показать вот так:

\fig{}{prolog/fisher/f3_1_2.png}{width=0.5\textwidth}

Одна важная вещь не показана в диаграмме\ --- логические переменные в
результирующей цели \verb|b1,b2,...,bn,g2,g3,...| были привязаны в результате
унификации, и \prolog\ требует отслеживать эти унифицирующие подстановки, в
процессе роста дерева вывода вниз, во всех ветках.

Итак, обход дерева вывода вглубь значат что альтернативные варианты выбора будут
проверены тогда, когда поиск возвратиться в точку ветвления, содержащую этот
выбор. Процесс называется \termdef{backtracking}{backtracking}.

Естественно, если хвост цели пуст, самая левая подцель 
эффективно удаляется. Если все подцели могут быть удалены по одному из путей
дерева вывода, то находится ответ, и возвраается резщультат \verb|true|. В этой
точке привязки переменных могут быть использованы для дачи ответа на
оригинальный запрос.

\subsecly{Унификация термов \prolog а}

Prolog unification matches two Prolog terms T1 and T2 by finding a substitution
of variables mapping M such that if M is applied T1 and M is applied to T2 then
the results are equal.

For example, Prolog uses unification in order to satisfy equations (T1=T2) ...

\begin{verbatim}
?- p(X,f(Y),a) = p(a,f(a),Y).
X = a  Y = a

?- p(X,f(Y),a) = p(a,f(b),Y).
No
\end{verbatim}

In the first case the successful substituton is \verb|{X/a, Y/b}|, and for the
second example there is no substitution that would result in equal terms. In
some cases the unification does not bind variables to ground terms but result in
variables sharing references \ldots

\begin{verbatim}
?- p(X,f(Y),a) = p(Z,f(b),a).
X = _G182   Y = b   Z = _G182 
\end{verbatim}

In this case the unifying substitution is \verb|{X/_G182, Y/b, Z/_G182}|, and X
and Z share reference, as can be illustrated by the next goal ...

\begin{verbatim}
?- p(X,f(Y),a) = p(Z,f(b),a), X is d.
X = d   Y = b   Z = d 
\end{verbatim}

\verb|{X/_G182, Y/b, Z/_G182}| was the most general unifying substitution for
the previous goal, and the instance \verb|{X/d, Y/b, Z/d}| is specialized to
satisfy the last goal.

Prolog does not perform an occurs check when binding a variable to another term,
in case the other term might also contain the variable. For example (SWI-Prolog)
...

\begin{verbatim}
?- X=f(X).
X = f(**)
\end{verbatim}

The circular reference is flagged (**) in this example, but the goal does
succeed \verb|{X/f(f(f(...)))}|. However ...

\begin{verbatim}
?- X=f(X), X=a.
No
\end{verbatim}

The circular reference is checked by the binding, so the goal fails. "a canNOT
be unified with \verb|f(_Anything)"| ...

\begin{verbatim}
?- a \=f(_).
Yes
\end{verbatim}

Some Prologs have an occurs-check version of unification available for use. For
example, in SWI-Prolog ...

\begin{verbatim}
?- unify_with_occurs_check(X,f(X)).
No
\end{verbatim}

The Prolog goal satisfation algoritm, which attempts to unify the current goal
with the head of a program clause, uses an instance form of the clause which
does not share any of the variables in the goal. Thus the occurs-check is not
needed for that.

The only possibility for an occurs-check error will arise from the processing of
Prolog terms (in user programs) that have unintended circular reference of
variables which the programmer believes should lead to failed goals when they
occur . Some Prologs might succeed on these circular bindings, some might fail,
others may actually continue to record the bindings in an infinite loop, and
thus generate a run-time error (out of memory). These rare situations need
careful programming.

It is possible to mimic the general unification algorithm in Prolog. But here we
present a specialized version of unification, whose computational complexity is
linear in the size of the input terms, and simply matches terms left-to-right.
The match(+GeneralTerm,+GroundedTerm) predicate attempts to match its first
argument (which may contain variables) against its second argument (which must
be grounded). This little program should be considered just as an illustration,
or a programming exercise, although we do know of cogent applications for the LR
matching algorithm in situations where general unification in not needed. We
would not use match, however, in a Prolog application because built-in
unification would be so much faster; we would simply have to ensure that the
presuppositions for match are appropriately checked when built-in unification is
used. The reference Apt and Etalle (1993) discusses the question in general
regarding how much of general unification is actually NOT needed by Prolog.

\begin{verbatim}
%%%%%%%%%%%%%%%%%%%%%%%%%%%%%
%% match(U,V) : 
%%   U may contain variables
%%   V must be ground
%%%%%%%%%%%%%%%%%%%%%%%%%%%%%
% match a variable with a ground term
match(U,V) :- 
   var(U), 
   ground(V),
   U = V.  % U assigned value V

% match grounded terms
match(U,V) :- 
   ground(U), 
   ground(V),
   U == V.

% match compound terms
match(U,V) :- 
   \+var(U), 
   ground(V),
   functor(U,Functor,Arity),
   functor(V,Functor,Arity),
   matchargs(U,V,1,Arity).

% match arguments, left-to-right
matchargs(_,_,N,Arity) :- 
   N > Arity.
matchargs(U,V,N,Arity) :- 
   N =< Arity,
   arg(N,U,ArgU),
   arg(N,V,ArgV), 
   match(ArgU,ArgV), 
   N1 is N+1, 
   matchargs(U,V,N1,Arity).
\end{verbatim}
