\secrel{Учебник Фишера}\secdown

\copyright\ J.R.Fisher 's \prolog
Tutorial\ \cp{https://www.cpp.edu/~jrfisher/www/prolog\_tutorial/contents.html}

\bigskip

\secly{Введение}

\prolog\ --- язык декларативного логического программирования. Детально
рассматривая его имя, получаем что это сокращение от \emph{pro}gramming in
\emph{log}ic: логическое программирование.

The name itself, \prolog, is short for PROgramming in LOGic. Prolog's heritage includes the
research on theorem provers and other automated deduction systems developed in
the 1960s and 1970s. The inference mechanism of Prolog is based upon Robinson's
resolution principle (1965) together with mechanisms for extracting answers
proposed by Green (1968). These ideas came together forcefully with the advent
of linear resolution procedures. Explicit goal-directed linear resolution
procedures, such as those of Kowalski and Kuehner (1971) and Kowalski (1974),
gave impetus to the development of a general purpose logic programming system.
The "first" Prolog was "Marseille Prolog" based on work by Colmerauer (1970).
The first detailed description of the Prolog language was the manual for the
Marseille Prolog interpreter (Roussel, 1975). The other major influence on the
nature of this first Prolog was that it was designed to facilitate natural
language processing.

Prolog is the major example of a fourth generation programming language
supporting the declarative programming paradigm. The Japanese Fifth-Generation
Computer Project, announced in 1981, adopted Prolog as a development language,
and thereby focused considerable attention on the language and its capabilities.
The programs in this tutorial are written in "standard" (University of)
Edinburgh Prolog, as specified in the classic Prolog textbook by authors
Clocksin and Mellish (1981,1992). The other major kind of Prolog is the PrologII
family of Prologs which are the descendants of Marseille Prolog. The reference
to Giannesini, et.al. (1986) uses a version of PrologII. There are differences
between these two varieties of Prolog; part of the difference is syntax, and
part is semantics. However, students who learn either kind of Prolog can easily
adapt to the other kind.

This tutorial is intended to be used to help learn the essential, basic concepts
of Prolog. The sample programs have been especially chosen to help promote the
use of Prolog programming in an artificial intelligence course. Lisp and Prolog
are the most frequently used symbolic programming languages for artificial
intelligence. They are widely regarded as excellent languages for "exploratory"
and "prototype programming".

Chapter 1 explains the Prolog programming environment for the beginner.

Chapter 2 explains Prolog syntax and many essentials of Prolog programming
through the use of carefully chosen sample programs. The sample programs are
arranged to guide the student through the development of Prolog programs that
are constructed in a top-down, declarative fashion. Care has been taken to cover
Prolog programming techniques that are very useful in an artificial intelligence
course. In fact, this primer can serve as a convenient, small, concise Prolog
introduction for such a course. Semantic issues have been addressed by
introducing early the concept of a program clause tree that is used to define in
an abstract way what are supposed to be the consequences of a Prolog program
specification. The author believes this to be a viable way to promote the basic
semantic issues of software verification for Prolog programming. The last
section of this chapter introduces an example that shows that Prolog can be
effectively used to give careful, precise specifications of systems, contrary to
its usual reputation as being hard to document because it is easy to use as an
exploratory programming tool.

Chapter 3 explains the operation of the underlying inference engine of Prolog.
Chapter 3 should be first read after the student has studied two or three of the
sample programs in Chapter 2. The last section in this chapter introduces Prolog
meta-interpreters.

Chapter 4 gives an outlined view of the major built-in predicates of Prolog,
many of which are exemplified in Chapter 2.

Chapter 5 gives an outline for developing A* search programs in prolog.  Section
5.3 has an $\alpha\beta$ search program for the game of tic tac toe.

Chapter 6 presents a unique and extensive presentation of a logic
meta-interpreter for normal logical rulebases.\note{Note 9/4/2006: I have
edited this chapter heavily, and the section links are all new.}

Chapter 7 gives an introduction to Prolog's built-in grammar parser generator,
and a brief overview of how Prolog can be used to parse English (natural
language) sentences. Also, there is a section covering the construction of
simple idiomatic natural language interfaces to programs.

Chapter 8 shows how to implement varios prolog prototypes. A new section (§8.4)
develops an interactive connection between prolog (inference engine) and Java
(GUI) to play tic tac toe. The simple connection model is quite widely adaptable
and applicable.

Earlier versions of portions of this tutorial date back to 1988. The
introductory material was originally used to help explain a Prolog interpreter
developed by the author (no longer available) for use in his courses. The author
believes that the introductory material, gathered together in the form given
here might be very useful for the student who wants a quick, but well-tailored,
introduction to Prolog.

For fuller treatments of Prolog the student is advised to see the textbooks by
Clocksin and Mellish (1981,1992), by O'Keefe (1990), by Clocksin (1997, 2003),
or by Sterling and Shapiro (1986).

For excellent historical notes regarding Prolog and natural language processing
using Prolog the text by Pereira and Shieber (1987) is recommended.

\copyright\ Pomona, California\\ 1988-2015


\secrel{Установка и запуск \prolog-системы}\label{fish1}

Примеры этого учебника \prolog а были подготовлены с использованием

\begin{itemize}[nosep]
  \item Quintus Prolog на компьютерах Digital Equipment Corporation MicroVAXes
  (далекая история)
  \item SWI Prolog на Sun Spark (давным давно)
  \item персональных компьютерах c \win
  \item или OS X на Macах  
\end{itemize}
 
Другие заметные \prolog-системы (Borland, XSB, LPA, Minerva \ldots)
использовались для разработки и тестирования последние 25 лет.
В этом учебнике запланирован новый раздел, в котором описано использование
любых \prolog-систем в общем, но пока этот раздел недоступен.

Сайт SWI-Prolog содержит много информации, ссылки на загрузку, и документацию:

\url{http://www.swi-prolog.org/}

Особо следуюет отметить возможность попробовать SWI Prolog on-line без 
регистрации и SMS: \url{http://swish.swi-prolog.org/}. Этот вариант особенно
удобен, так как не требует никакой установки ПО, административных прав, вы
можете работать с этим учебником даже в интернет-кафе.

\bigskip
Примеры в этом учебнике используют упрощенную форму взаимодействия в типичным
\prolog-интерпретатором, так что программы должны работать похоже в любой
\prolog-системе эдинбургского типа или интерактивном компиляторе.

Если в вашей UNIX-системе уже установлен SWI-Prolog, запустите окно
терминала, и начните интерактивную сессию командной:

\begin{verbatim}
user@computer$ swipl
\end{verbatim}

Мы не будем использовать команду запуска именно в такой форме все время:
при запуске могут быть указаны дополниительные параметры командной строки,
которые можно использовать в определенных случаях. Читатель должен расмотреть
эту возможность после освоения базовых приемов работы, чтобы получить больше
возможностей.

Если вы хотите установить SWI Prolog под Debian \linux, выполните команду:

\begin{verbatim}
sudo apt install swi-prolog
\end{verbatim}

\bigskip
Под \win\ инсталлятор SWI-Prolog добавляет иконку запуска
интерпретатора, который вы можете запустить простым двойным целчком по
иконке. При запуске интерпретатор создает свое собственное командное окно.
\bigskip

После запуска интерпретатора обычно появляется сообщение о версии, лицензии и
авторах, а затем выводится приглашение ввода \term{цели}\ типа

\begin{verbatim}
?- _
\end{verbatim}

Интерактивные \term{цели}\ в \prolog е вводятся пользователем за приглашением
\verb|?-|.

Многие \prolog-системы поддерживают предоставление документации по запросу из
командной строки. В SWI Prolog встроена подробная система помощи. Документация
индексирована, и помогает пользователю в процессе работы. Попробуйте ввести

\begin{verbatim}
?- help(help).
\end{verbatim}

Обратите внимание что должна быть введены все символы, и ввод завершен возвратом
каретки.

Для иллюстрации нескольких приемов взимодействия с \prolog ом рассмотрим
следующий пример сессии. Если приведена ссылка на файл, предполагается что это
локальный файл в пользовательском каталоге, который был создан пользователем,
получен копированием из другого публично доступного источника, или получен
сохранением текстового файла из веб-браузера. Способ достижения последнего\ ---
следователь URL-ссылке на файл и сохранить его, или выбрать кусок текста из
онлайн-учебника \prolog а, скопировать его, вставить в текстовый редактор, и
сохранить полученный файл из текстового редактора. Комментарии вида
\verb|/*...*/| после целей используются для описания этих целей.

\lstv{Лог типичной \prolog-сессии}{prolog/fisher/running.pl}

Комментарии в правой части были добавлены в текстовом редакторе. Они отмечают
некоторые вещи, перечисленные ниже:

\begin{enumerate}
  \item
Определение \term{цели} \prolog а завершается точкой \verb|.|\ .
В этом случае цель была загружена в внешнего файла с исходным тестом программы.
Этот скобочный стиль записи программы унаследован из самых первых реализаций
\prolog а. Можно загрузить несколько файлов сразу, указав их имена в одиночных
кавычках, разделяя запятыми. В нашем случае имя файла \file{2\_2.pl}, программа
содержит два программы на \prolog е для вычисления факториала от положительного
целого. Подробно эта программа описана в разделе \ref{fisher22}. Файл программы
ищется в текущем каталоге. Если поиск неуспешен, нужно явно указать полный путь
обычным для вашей ОС способом.

\item 
Встроенный предикат \verb|listing|\ выводит листинг программы из ОЗУ\ --- в
нашем случае программу вычисления факториала, загруженную ранее. Внешний вид
этого литсинга несколько отличается от исходного кода в файле из \ref{fisher22}.
Заметим, что \prog{Quintus Prolog} компилирует программу, если отдельно не
указано что определенные предикаты являются динамическими. Скомпилированные
предикаты не могут быть выведены через \verb|listing|, поэтому если у вас он не
срабатывает, возможно требуется дополнить исходник декларацией динамического
предиката, чтобы пример сработал. В \prog{SWI Prolog}\ этот пример работает без
модификации.

  \item 
Эта цель \verb|factorial(10,What)| говорит ``факториал 10ти что?''. Слово
\verb|What| начинается с большой буквы, указывающей что это \termdef{логическая
переменная}{логическая переменная}. \prolog\ удовлетворяет цель находя все
возможные значения переменной \verb|What|.

  \item
Теперь в памяти находятся обе программы из файлов \file{2\_1.pl} и
\file{2\_7.pl}. Файл \file{2\_7.pl} содержит несколько определений обработки
списков (см. \ref{fisher27}).

  \item 
Только что загруженная программа (\file{2\_7.pl} содержит определение предиката
\verb|takeout|. Цель \verb|takeout(X,[1,2,3,4],Y)| запрашивает поиск всех таких
\verb|X| что значение взятое из списка \verb|[1,2,3,4]| оставляет остаток в
переменной \verb|Y|, для всех возможных случаев. Существует четыре способа
сделать это, как показано в результате.  Предикат \verb|takeout| обсуждается в
разделе \ref{fisher27}. Таким образом, \emph{в \prolog\ заложен поиск всех
возможных ответов}: после того как будет выведен очередной ответ, \prolog\
ожидает реакции пользователя мигая курсором в конце строки с ответом. Если
пользователель нажмет \keys{;}, \prolog\ будет выполнять поиск следующего
ответа. Если пользователь просто нажмет \keys{Enter}, \prolog\ остановит поиск.

  \item
Составная, или \termdef{конъюнктивная цель}{конъюнктивная цель}, определяет
одновременное удовлетворение \emph{двух} отдельных целей. Отметим что
используется арифметическая цель (встроенное отношение) \verb|X>3|.
\prolog\ будет пытаться удовлетворить эти цели \emph{слева направо}, в порядке
чтения. В нашем случае существует единственный ответ. Отметим использование
в цели \termdef{анонимной переменной}{анонимная переменная} \verb|_|\ ,
\termdef{биндинг}{биндинг логической переменной} (\termdef{привязка}{привязка
логической переменной}) для которой не выводится (переменная ``не важно'').

  \item 
Цель \verb|halt| всегда успешна и завершает работу интерпретатора.
\end{enumerate}

\secrel{2. Sample Programs -- Descriptions}\label{fish2}\secdown
\secrel{2.1 Map colorings} 
\secrel{2.2 Two factorial definitions} 
\secrel{2.3 Towers of Hanoi puzzle} 
\secrel{2.4 Loading programs, editing programs} 
\secrel{2.5 Negation as failure} 
\secrel{2.6 Tree data and relations} 
\secrel{2.7 Prolog lists and sequences} 
\secrel{2.8 Change for a dollar} 
\secrel{2.9 Map coloring redux} 
\secrel{2.10 Simple I/O} 
\secrel{2.11 Chess queens challenge puzzle} 
\secrel{2.12 Finding all answers} 
\secrel{2.13 Truth table maker} 
\secrel{2.14 DFA parser} 
\secrel{2.15 Graph structures and paths} 
\secrel{2.16 Search} 
\secrel{2.17 Animal identification game} 
\secrel{2.18 Clauses as data} 
\secrel{2.19 Actions and plans}
\secup
\secrel{3. How Prolog Works}\label{fish3}\secdown
\secrel{3.1 Prolog derivation trees, choices and unification} 
\secrel{3.2 Cut} 
\secrel{3.3 Meta-interpreters in Prolog}
\secup
\secrel{4. Built-in Goals}\label{fish4}\secdown
\secrel{4.1 Utility goals }
\secrel{4.2 Universals (true and fail)} 
\secrel{4.3 Loading Prolog programs} 
\secrel{4.4 Arithmetic goals} 
\secrel{4.5 Testing types} 
\secrel{4.6 Equality of Prolog terms, unification} 
\secrel{4.7 Control} 
\secrel{4.8 Testing for variables} 
\secrel{4.9 Assert and retract} 
\secrel{4.10 Binding a variable to a numerical value} 
\secrel{4.11 Procedural negation, negation as failure }
\secrel{4.12 Input/output} 
\secrel{4.13 Prolog terms and clauses as data} 
\secrel{4.14 Prolog operators} 
\secrel{4.15 Finding all answers}
\secup
\secrel{5. Search in Prolog}\label{fish5}\secdown
\secrel{5.1 The A* algorithm in Prolog} 
\secrel{5.2 The 8-puzzle} 
\secrel{5.3 $\alpha\beta$ search in Prolog}\label{fish53}
\secup
\secrel{6. Logic Topics}\label{fish6}\secdown
\secrel{6.1 Chapter 6 notes} 
\secrel{6.2 Positive logic} 
\secrel{6.3 Convert first-order logic to normal form} 
\secrel{6.4 A normal rulebase goal interpreter} 
\secrel{6.5 Evidentiary soundness and completeness} 
\secrel{6.6 Rule tree visualization using Java}
\secup
\secrel{7. Introduction to Natural Language Processing}\label{fish7}\secdown
\secrel{7.1 Prolog grammar parser generator} 
\secrel{7.2 Prolog grammar for simple English phrase structures} 
\secrel{7.3 Idiomatic natural language command and question interfaces}
\secup
\secrel{8. Prototyping with Prolog}\label{fish8}\secdown
\secrel{8.1 Action specification for a simple calculator} 
\secrel{8.2 Animating the 8-puzzle (\$5.2) using character graphics} 
\secrel{8.3 Animating the blocks mover (\$2.19) using character graphics} 
\secrel{8.4 Java Tic-Tac-Toe GUI plays against Prolog opponent
(\$5.3)}\label{fish84}
\secrel{8.5 Structure diagrams and Prolog}
\secup
\secly{References}
\secup