\secrel{Учебник Фишера}\secdown

\copyright\ J.R.Fisher 's \prolog
Tutorial\ \cp{https://www.cpp.edu/~jrfisher/www/prolog\_tutorial/contents.html}

\bigskip

\secly{Введение}

\prolog\ --- язык декларативного логического программирования. Детально
рассматривая его имя, получаем что это сокращение от \textsc{pro}gramming in
\textsc{log}ic: логическое программирование. Наследие \prolog а включает
исследования в области \term{автоматического доказательства теорем} и других
\term{дедуктивных систем}, разработанных в 1960-70хх гг. \term{Механизм вывода}
\prolog а базируется на принципе разрешения Робинсона (1965) и механизмах вывода
ответов, приложенных Грином (1968). Эти идеи используются вместе с продедурами
линейного разрешения. Процедуры точного целевого линейное разрешения, такие как
методы Kowalski / Kuehner (1971) и Kowalski (1974), дали толчок к разработке
систем логического программирования общего назначения. ``Первым'' \prolog ом был
``Марсельский \prolog'', реализация которого основана на работе Colmerauer
(1970). Первым делательным описанием языка \prolog\ было руководство к
интерпретатору Marseille Prolog (Roussel, 1975). Другим сильным влиянием на
природу этого первого \prolog а была адаптация этого интерпретатора к задачам
\term{обработки натуральных языков}.

\prolog\ является наиболее часто упоминаемым примеров языков программирования
четвертого поколения, которые поддерживают парадигму \emph{декларативного
программирования}. Японский проект Fifth-Generation Computer
Project\note{компьютерный проект пятого поколения}, анонсированный в 1981,
применял \prolog\ как язык разработки, и сосредоватчивал значительные усилия на
языке и его возможностях. Программы в этом учебнике написаны на ``стандартном''
\prolog е Эдинбургского университета\note{University of Edinburgh Prolog}, как
это сделано к классической книге по \prolog у под авторством Clocksin и Mellish
(1981,1992). Другой заметной версией \prolog а является семейство реализаций
\prolog II, которые являются ответсвлениями Марсельского \prologа. Справочник
Giannesini, et.al. (1986) использует версию \prolog II. Есть некоторые различия
бежду этими двумя вариантами \prolog а; часть различий в синтаксисе, и часть в
семантике. Тем не менее, студенты изучавшие одну из версий, впоследствии могут
легко адаптировать к другой.

Цель этого учебника\ --- помочь изучить самые необходимые, базовые концепции
языка \prolog. Примеры программ были особенно аккуратно выбраны для иллюстрации
программирования искуственного интлеллекта на \prolog е. \lisp\ и \prolog\
наиболее часто используемые языки символьного программирования для приложений
искуственного интеллекта. Они часто упоминаются как великолепные языки для
``исследовательского'' и ``прототипного программирования''.

Раздел \ref{fish1} рассматривает среду программирования на \prolog е для
начинающих.

Раздел \ref{fish2} объясняет синтаксис \prolog а и многие аспекты
программирования на нем через реализацию аккуратно выборанных программ-примеров.
Эти примеры организованы так, чтобы студент обучался через реализацию
\prolog-программ ``сверху вниз'' в декларативном стиле.
Были приняты меры к рассмотрению техник программирования на \prolog е, которые
очень важны для курса искуственного интеллекта. Фактически, \emph{этот учебник
может служить удобным, маленьким, кратким введением в применение \prolog а в
приложениях искуственного интеллекта}. Аспекты семантики языка \prolog\
рассмотриваются с самого начала с точки зрения концепции дерева условий
программы, которое используется для определения последовательностей спецификаций
\prolog-программы в абстрактном виде. Автор надеется что этот подход позволит
рассмотреть базовые принципы формальной верификации программ при
программировании на \prolog е. Последняя секция этого раздела приводит пример,
который показывает что \prolog\ может быть эффективно использован для
аккуратной, точной спецификации программных систем, несмотря на его репутацию
трудно документируемого языка, так что \prolog\ легко использовать как средство
прототипирования.

Раздел \ref{fish3}\ рассматривает работу внутренних механизмов \prolog-движка.
Раздел \ref{fish3}\ рекомендуется просмотреть сразу после того, как студент
изучил 2-3 примера программ из раздела \ref{fish2}. Последняя секция этого
радела рассматривает \term{мета-интерпретаторы \prolog а}.

Раздел \ref{fish4}\ дает краткий обзор основных встроенных предикатов, многие из
которых используются в разделе \ref{fish2}..

Раздел \ref{fish5}\ рассматривает разработку программ A*-поиска на \prolog е.
Раздел \ref{fish53}\ содержит программу $\alpha\beta$-поиска для игры 
tic tac toe.

Раздел \ref{fish6}\ представляет уникальное и обширное описание логического
мета-интерпретатора для нормальных логических баз правил.\note{Замечание от
9/4/2006: Я значительно отредактировал этот раздел, и обновил все ссылки на
секции.}

Раздел \ref{fish7}\ предствляет введение во встроенный в \prolog\ генератор
парсеров грамматики, и дает общий обзор приемов, с помощью которых \prolog\
может быть использован для разбора выражений натурального языка (английского).
Также эта секция описывает построение программных интерфейсов, использующих
идеоматически-простые натуральные языки.

Раздел \ref{fish8}\ показывает приемы реализации различных \prolog-прототипов.
Новый раздел \ref{fish84}\ раскрывает интерактивную связку между машиной вывода
\prolog а и Java GUI для игры tic tac toe. Рассмотренная простая модель связки
легко адаптируемая и применима.

Ранние версии частей этого учебника датируются 1988 годом. Вводный материал
изначально использовался для объяснения работы интерпретатора \prolog а,
разработанного автором\note{сейчас недоступен}\ для применения в учебном
процессе. Автор надеется что вводный материал, собранный в форме этого учебника,
может быть очень полезным для студентов, которые хотят быстрое, но при этом
хорошо сбалансированное, введение в программирование на \prolog е.

Для дальнейшего изучения \prolog а можно посоветовать книги
Clocksin и Mellish (1981,1992), O'Keefe (1990), Clocksin (1997, 2003),
или Sterling и Shapiro (1986).

Подробные заметки по истории \prolog\ и обработке натуральных языков с его
использованием содержатся в работе Pereira and Shieber (1987).

\copyright\ Помона, Калифорния\\ 1988-2015


\secrel{Установка и запуск \prolog-системы}\label{fish1}

Примеры этого учебника \prolog а были подготовлены с использованием

\begin{itemize}[nosep]
  \item Quintus Prolog на компьютерах Digital Equipment Corporation MicroVAXes
  (далекая история)
  \item SWI Prolog на Sun Spark (давным давно)
  \item персональных компьютерах c \win
  \item или OS X на Macах  
\end{itemize}
 
Другие заметные \prolog-системы (Borland, XSB, LPA, Minerva \ldots)
использовались для разработки и тестирования последние 25 лет.
В этом учебнике запланирован новый раздел, в котором описано использование
любых \prolog-систем в общем, но пока этот раздел недоступен.

Сайт SWI-Prolog содержит много информации, ссылки на загрузку, и документацию:

\url{http://www.swi-prolog.org/}

Особо следуюет отметить возможность попробовать SWI Prolog on-line без 
регистрации и SMS: \url{http://swish.swi-prolog.org/}. Этот вариант особенно
удобен, так как не требует никакой установки ПО, административных прав, вы
можете работать с этим учебником даже в интернет-кафе.

\bigskip
Примеры в этом учебнике используют упрощенную форму взаимодействия в типичным
\prolog-интерпретатором, так что программы должны работать похоже в любой
\prolog-системе эдинбургского типа или интерактивном компиляторе.

Если в вашей UNIX-системе уже установлен SWI-Prolog, запустите окно
терминала, и начните интерактивную сессию командной:

\begin{verbatim}
user@computer$ swipl
\end{verbatim}

Мы не будем использовать команду запуска именно в такой форме все время:
при запуске могут быть указаны дополниительные параметры командной строки,
которые можно использовать в определенных случаях. Читатель должен расмотреть
эту возможность после освоения базовых приемов работы, чтобы получить больше
возможностей.

Если вы хотите установить SWI Prolog под Debian \linux, выполните команду:

\begin{verbatim}
sudo apt install swi-prolog
\end{verbatim}

\bigskip
Под \win\ инсталлятор SWI-Prolog добавляет иконку запуска
интерпретатора, который вы можете запустить простым двойным целчком по
иконке. При запуске интерпретатор создает свое собственное командное окно.
\bigskip

После запуска интерпретатора обычно появляется сообщение о версии, лицензии и
авторах, а затем выводится приглашение ввода \term{цели}\ типа

\begin{verbatim}
?- _
\end{verbatim}

Интерактивные \term{цели}\ в \prolog е вводятся пользователем за приглашением
\verb|?-|.

Многие \prolog-системы поддерживают предоставление документации по запросу из
командной строки. В SWI Prolog встроена подробная система помощи. Документация
индексирована, и помогает пользователю в процессе работы. Попробуйте ввести

\begin{verbatim}
?- help(help).
\end{verbatim}

Обратите внимание что должна быть введены все символы, и ввод завершен возвратом
каретки.

Для иллюстрации нескольких приемов взимодействия с \prolog ом рассмотрим
следующий пример сессии. Если приведена ссылка на файл, предполагается что это
локальный файл в пользовательском каталоге, который был создан пользователем,
получен копированием из другого публично доступного источника, или получен
сохранением текстового файла из веб-браузера. Способ достижения последнего\ ---
следователь URL-ссылке на файл и сохранить его, или выбрать кусок текста из
онлайн-учебника \prolog а, скопировать его, вставить в текстовый редактор, и
сохранить полученный файл из текстового редактора. Комментарии вида
\verb|/*...*/| после целей используются для описания этих целей.

\lstv{Лог типичной \prolog-сессии}{prolog/fisher/running.pl}


\secrel{2. Sample Programs -- Descriptions}\label{fish2}\secdown
\secrel{2.1 Map colorings} 

\secrel{Два определения факториала}\label{fish22}

Этот раздел вводит в вычисления математических функций используя \prolog.
Обсуждаются различные встроенные арифметические операции. Также обсуждается
концепция derivation дерева, и как derivation деревья связаны с трассировкой в
\prolog е.

В файле \file{2\_2.pl} находятся два определения предикатов, являющиеся
определением фукнции вычисления факториала:

\lst{первый вариант}{prolog/fisher/2_2.pl}{Prolog}
 
Эта программа состоит из двух clauses. Первое заключение\ --- формулировка
\termdef{факта}{факт} (unit clause) \emph{без тела}. Второе заключение\ ---
\termdef{правило}{правило}, так как \emph{у него есть тело}. Тело второго
заключения находится после \verb|:-|, которое можно читать как ``если''. Тело
содержит литералы, разделенные запятыми, каждую запятую можно читать как ``и``.
\termdef{Заголовок правила}{заголовок правила}\ --- весь текст \term{факта} или
часть текста до \verb|:-| в правиле. Рассматривая текст как декларативную
программу, первое (фактическое) предложение читается как ``факториал 0 есть
1''\note{или: 0 и 1 \term{связаны отношением} ``факториал'', но у объектов
одновременно могут быть и другие отношения, например биты(0,1) и целые(0,1)},
и второе предложение заявляет что ``факториал \var{N} есть \var{F}\note{точнее:
N и F связаны отношением ``факториал''} если \verb|N>0| и \var{N1} есть
\verb|N-1| , и факториал \var{N1} есть \var{F1}, и \var{F} есть \verb|N*F1|.

\termdef{\prolog-цель}{цель (Пролог)} (goal) для вычисления факториала от 3
дает ответ в W\ --- \termdef{переменной цели}{переменная цели}:

\begin{verbatim}
?-  factorial(3,W).  
W=6 . 
\end{verbatim}

Рассмотрим следующее clause дерево сконструированное для литерала\\
\verb|factorial(3,W)|. Как описано в предыдущей секции, clause дерево не
содержит никаких свободных переменных, вместо этого включает непосредственно их
значения. Каждое ветвление под узлом определяется clause оригинальной программы,
используя непосредственно вхождения значений переменных; узел задается
заголовком правила, а литералы теля становятся дочерними узлами.

\fig{}{prolog/fisher/f2_2.pdf}{width=0.95\textwidth}

\emph{Все арифметические листья \var|true|} при исполнении\note{в соответствие с
предполагаемой интерпретацией}, и самая нижная связь в дереве соответствует
самому первому clause в программе вычисленяи факториала. Первый clause может
быть записан как:

\begin{verbatim}
factorial(0,1) :- true. 
\end{verbatim}
и фактически \verb|?- true.| \prolog-цель которая всегда успешна
\note{\var{true} встроеннный предикат}. Для краткости, мы не отрисовали
\verb|true| для всех листьев, являющихся арифметическими литералами.

Программное clause дерево показывает значение цели в коорне дерева. Так,\\
\verb'factorial(3,6)' является консеквенцией \prolog-программы, так как
существует clause дерево с корнем \verb'factorial(3,6)', все листья которого
\verb|true|. С другой стороны литерал \verb'factorial(5,2)' не консеквенция,
так как такого дерева для него нет, а значением программы для литерала
\verb'factorial(5,2)' является \verb|false|:

\begin{verbatim}
?- factorial(3,6).  
true .
?- factorial(5,2).  
false . 
\end{verbatim}
как и следовало ожидать. Clause-деревья также называются AND-деревьями, так как
чтобы корень был консеквенцией программы, все его поддеревья также должны быть
консеквенциями. Позже clause деревья будут рассмотрены подробнее. Мы отметили
что \emph{clause дерево описывает семантику (значение) программы}. В разделе
\ref{fish6} мы рассмотрим другой подход к семантике программ. Clause-деревья
предоставляют интуитивный и корректный подход к описанию семантики.

\bigskip

Нам нужно отличать clause деревья программы и \termdef{деревья
вывода}{дерево вывода}. Сlause-деревья статичны, и могут быть нарисованы для
программмы или цели через механизм удовлетворения частичных (под)целей, как
описано выше. Грубо говоря, clause-деревья соответствуют декларативному чтению
программы.

\term{Деревья вывода} наоборот, имеют в виду механизм привязки переменных
\prolog а, и порядок в котором удовлетворяются вложенные частичные цели.
Подробнее деревья вывода описаны в разделе \ref{fish31}, но тем не менее
посмотрите анимацию, предоставляемую динамическим отладчиком, как описано ниже.

\termdef{Трассировка}{трассировка} исполнения \prolog-программы также показывает
как переменные привязываются при удовлетвормении целей. Следующий пример
показывает включение/выключение трассировки в типичной \prolog-системе.

\begin{verbatim}
?- trace. 
% The debugger will first creep -- showing everything (trace). 
 
true .
[trace] 
?- factorial(3,X). 
  (1) 0 Call: factorial(3,_8140) ? [Enter] creep 
  (1) 1 Head [2]: factorial(3,_8140) ? [Enter] creep 
  (2) 1 Call (built-in): 3>0 ?  creep
  (2) 1 Done (built-in): 3>0 ?  creep
  (3) 1 Call (built-in): _8256 is 3-1 ? creep 
  (3) 1 Done (built-in): 2 is 3-1 ?  creep
  (4) 1 Call: factorial(2, _8270) ?  creep
   ... 
  (1) 0 Exit: factorial(3,6) ? 
X=6 .
[trace] 
?- notrace. 
% The debugger is switched off 
 
true .
\end{verbatim}

The animated tree below gives another look at the derivation tree for the
\prolog goal \verb'factorial(3,X)'. To start (or to restart) the animation,
simply click on the \keys{Step} button.

\bigskip

Заголовок этого раздела говорит ``\emph{Два} определения факториала'', вот
второй вариант, использующий три переменых:

\lst{второй вариант}{prolog/fisher/2_2_2.pl}{Prolog}

Для этой версии используйте следующую цель-запрос:

\begin{verbatim}
?- factorial(5,1,F). 
F=120 .
\end{verbatim}

Второй параметр в определении называется \term{параметр-аккумулятор}, который
также хорошо известен в функциональном программировании. Эта версия факториала
определена с использованием \term{хвостовой рекурсии}. Важно чтобы вы выполнили
следующие упражнения:

\paragraph{Упражнение \ref{fish22}.1} Используя первый вариант программы
факториала, четко покажите что не существует clause-дерева с корнем \verb'factorial(5,2)',
имеющего все true листья.

\paragraph{Упражнение \ref{fish22}.2} Нарисуйте clause-дерево для цели
\verb'factorial(3,1,6)' со всеми true-листьями, в виде аналогичном ранее
описанному дереву для \verb'factorial(3,6)'.
Покажите, чем отличаются два варианта программы в процессе вычисления
факториала\,? Также, протрассируйте цель \verb'factorial(3,1,6)' используя
\prolog-систему.


\secrel{2.3 Towers of Hanoi puzzle} 
\secrel{2.4 Loading programs, editing programs} 
\secrel{2.5 Negation as failure} 
\secrel{2.6 Tree data and relations} 
\secrel{2.7 Prolog lists and sequences} 
\secrel{2.8 Change for a dollar} 
\secrel{2.9 Map coloring redux} 
\secrel{2.10 Simple I/O} 
\secrel{2.11 Chess queens challenge puzzle} 
\secrel{2.12 Finding all answers} 
\secrel{2.13 Truth table maker} 
\secrel{2.14 DFA parser} 
\secrel{2.15 Graph structures and paths} 
\secrel{2.16 Search} 
\secrel{2.17 Animal identification game} 
\secrel{2.18 Clauses as data} 
\secrel{2.19 Actions and plans}
\secup
\secrel{3. How Prolog Works}\label{fish3}\secdown
\secrel{3.1 Prolog derivation trees, choices and unification} 
\secrel{3.2 Cut} 
\secrel{3.3 Meta-interpreters in Prolog}
\secup
\secrel{4. Built-in Goals}\label{fish4}\secdown
\secrel{4.1 Utility goals }
\secrel{4.2 Universals (true and fail)} 
\secrel{4.3 Loading Prolog programs} 
\secrel{4.4 Arithmetic goals} 
\secrel{4.5 Testing types} 
\secrel{4.6 Equality of Prolog terms, unification} 
\secrel{4.7 Control} 
\secrel{4.8 Testing for variables} 
\secrel{4.9 Assert and retract} 
\secrel{4.10 Binding a variable to a numerical value} 
\secrel{4.11 Procedural negation, negation as failure }
\secrel{4.12 Input/output} 
\secrel{4.13 Prolog terms and clauses as data} 
\secrel{4.14 Prolog operators} 
\secrel{4.15 Finding all answers}
\secup
\secrel{5. Search in Prolog}\label{fish5}\secdown
\secrel{5.1 The A* algorithm in Prolog} 
\secrel{5.2 The 8-puzzle} 
\secrel{5.3 $\alpha\beta$ search in Prolog}\label{fish53}
\secup
\secrel{6. Logic Topics}\label{fish6}\secdown
\secrel{6.1 Chapter 6 notes} 
\secrel{6.2 Positive logic} 
\secrel{6.3 Convert first-order logic to normal form} 
\secrel{6.4 A normal rulebase goal interpreter} 
\secrel{6.5 Evidentiary soundness and completeness} 
\secrel{6.6 Rule tree visualization using Java}
\secup
\secrel{7. Introduction to Natural Language Processing}\label{fish7}\secdown
\secrel{7.1 Prolog grammar parser generator} 
\secrel{7.2 Prolog grammar for simple English phrase structures} 
\secrel{7.3 Idiomatic natural language command and question interfaces}
\secup
\secrel{8. Prototyping with Prolog}\label{fish8}\secdown
\secrel{8.1 Action specification for a simple calculator} 
\secrel{8.2 Animating the 8-puzzle (\$5.2) using character graphics} 
\secrel{8.3 Animating the blocks mover (\$2.19) using character graphics} 
\secrel{8.4 Java Tic-Tac-Toe GUI plays against Prolog opponent
(\$5.3)}\label{fish84}
\secrel{8.5 Structure diagrams and Prolog}
\secup
\secly{References}
\secup