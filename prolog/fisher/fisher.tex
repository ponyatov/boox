\secrel{Учебник Фишера}\secdown

\copyright\ J.R.Fisher 's \prolog
Tutorial\ \cp{https://www.cpp.edu/~jrfisher/www/prolog\_tutorial/contents.html}

\bigskip

\secly{Введение}

\prolog\ --- язык декларативного логического программирования. Детально
рассматривая его имя, получаем что это сокращение от \textsc{pro}gramming in
\textsc{log}ic: логическое программирование. Наследие \prolog а включает
исследования в области \term{автоматического доказательства теорем} и других
\term{дедуктивных систем}, разработанных в 1960-70хх гг. \term{Механизм вывода}
\prolog а базируется на принципе разрешения Робинсона (1965) и механизмах вывода
ответов, приложенных Грином (1968). Эти идеи используются вместе с продедурами
линейного разрешения. Процедуры точного целевого линейное разрешения, такие как
методы Kowalski / Kuehner (1971) и Kowalski (1974), дали толчок к разработке
систем логического программирования общего назначения. ``Первым'' \prolog ом был
``Марсельский \prolog'', реализация которого основана на работе Colmerauer
(1970). Первым делательным описанием языка \prolog\ было руководство к
интерпретатору Marseille Prolog (Roussel, 1975). Другим сильным влиянием на
природу этого первого \prolog а была адаптация этого интерпретатора к задачам
\term{обработки натуральных языков}.

\prolog\ является наиболее часто упоминаемым примеров языков программирования
четвертого поколения, которые поддерживают парадигму \emph{декларативного
программирования}. Японский проект Fifth-Generation Computer
Project\note{компьютерный проект пятого поколения}, анонсированный в 1981,
применял \prolog\ как язык разработки, и сосредоватчивал значительные усилия на
языке и его возможностях. Программы в этом учебнике написаны на ``стандартном''
\prolog е Эдинбургского университета\note{University of Edinburgh Prolog}, как
это сделано к классической книге по \prolog у под авторством Clocksin и Mellish
(1981,1992). Другой заметной версией \prolog а является семейство реализаций
\prolog II, которые являются ответсвлениями Марсельского \prologа. Справочник
Giannesini, et.al. (1986) использует версию \prolog II. Есть некоторые различия
бежду этими двумя вариантами \prolog а; часть различий в синтаксисе, и часть в
семантике. Тем не менее, студенты изучавшие одну из версий, впоследствии могут
легко адаптировать к другой.

Цель этого учебника\ --- помочь изучить самые необходимые, базовые концепции
языка \prolog. Примеры программ были особенно аккуратно выбраны для иллюстрации
программирования искуственного интлеллекта на \prolog е. \lisp\ и \prolog\
наиболее часто используемые языки символьного программирования для приложений
искуственного интеллекта. Они часто упоминаются как великолепные языки для
``исследовательского'' и ``прототипного программирования''.

Раздел \ref{fish1} рассматривает среду программирования на \prolog е для
начинающих.

Раздел \ref{fish2} объясняет синтаксис \prolog а и многие аспекты
программирования на нем через реализацию аккуратно выборанных программ-примеров.
Эти примеры организованы так, чтобы студент обучался через реализацию
\prolog-программ ``сверху вниз'' в декларативном стиле.
Были приняты меры к рассмотрению техник программирования на \prolog е, которые
очень важны для курса искуственного интеллекта. Фактически, \emph{этот учебник
может служить удобным, маленьким, кратким введением в применение \prolog а в
приложениях искуственного интеллекта}. Аспекты семантики языка \prolog\
рассмотриваются с самого начала с точки зрения концепции дерева условий
программы, которое используется для определения последовательностей спецификаций
\prolog-программы в абстрактном виде. Автор надеется что этот подход позволит
рассмотреть базовые принципы формальной верификации программ при
программировании на \prolog е. Последняя секция этого раздела приводит пример,
который показывает что \prolog\ может быть эффективно использован для
аккуратной, точной спецификации программных систем, несмотря на его репутацию
трудно документируемого языка, так что \prolog\ легко использовать как средство
прототипирования.

Раздел \ref{fish3}\ рассматривает работу внутренних механизмов \prolog-движка.
Раздел \ref{fish3}\ рекомендуется просмотреть сразу после того, как студент
изучил 2-3 примера программ из раздела \ref{fish2}. Последняя секция этого
радела рассматривает \term{мета-интерпретаторы \prolog а}.

Раздел \ref{fish4}\ дает краткий обзор основных встроенных предикатов, многие из
которых используются в разделе \ref{fish2}..

Раздел \ref{fish5}\ рассматривает разработку программ A*-поиска на \prolog е.
Раздел \ref{fish53}\ содержит программу $\alpha\beta$-поиска для игры 
tic tac toe.

Раздел \ref{fish6}\ представляет уникальное и обширное описание логического
мета-интерпретатора для нормальных логических баз правил.\note{Замечание от
9/4/2006: Я значительно отредактировал этот раздел, и обновил все ссылки на
секции.}

Раздел \ref{fish7}\ предствляет введение во встроенный в \prolog\ генератор
парсеров грамматики, и дает общий обзор приемов, с помощью которых \prolog\
может быть использован для разбора выражений натурального языка (английского).
Также эта секция описывает построение программных интерфейсов, использующих
идеоматически-простые натуральные языки.

Раздел \ref{fish8}\ показывает приемы реализации различных \prolog-прототипов.
Новый раздел \ref{fish84}\ раскрывает интерактивную связку между машиной вывода
\prolog а и Java GUI для игры tic tac toe. Рассмотренная простая модель связки
легко адаптируемая и применима.

Ранние версии частей этого учебника датируются 1988 годом. Вводный материал
изначально использовался для объяснения работы интерпретатора \prolog а,
разработанного автором\note{сейчас недоступен}\ для применения в учебном
процессе. Автор надеется что вводный материал, собранный в форме этого учебника,
может быть очень полезным для студентов, которые хотят быстрое, но при этом
хорошо сбалансированное, введение в программирование на \prolog е.

Для дальнейшего изучения \prolog а можно посоветовать книги
Clocksin и Mellish (1981,1992), O'Keefe (1990), Clocksin (1997, 2003),
или Sterling и Shapiro (1986).

Подробные заметки по истории \prolog\ и обработке натуральных языков с его
использованием содержатся в работе Pereira and Shieber (1987).

\copyright\ Помона, Калифорния\\ 1988-2015


\secrel{Установка и запуск \prolog-системы}\label{fish1}

Примеры этого учебника \prolog а были подготовлены с использованием

\begin{itemize}[nosep]
  \item Quintus Prolog на компьютерах Digital Equipment Corporation MicroVAXes
  (далекая история)
  \item SWI Prolog на Sun Spark (давным давно)
  \item персональных компьютерах c \win
  \item или OS X на Macах  
\end{itemize}
 
Другие заметные \prolog-системы (Borland, XSB, LPA, Minerva \ldots)
использовались для разработки и тестирования последние 25 лет.
В этом учебнике запланирован новый раздел, в котором описано использование
любых \prolog-систем в общем, но пока этот раздел недоступен.

Сайт SWI-Prolog содержит много информации, ссылки на загрузку, и документацию:

\url{http://www.swi-prolog.org/}

Особо следуюет отметить возможность попробовать SWI Prolog on-line без 
регистрации и SMS: \url{http://swish.swi-prolog.org/}. Этот вариант особенно
удобен, так как не требует никакой установки ПО, административных прав, вы
можете работать с этим учебником даже в интернет-кафе.

\bigskip
Примеры в этом учебнике используют упрощенную форму взаимодействия в типичным
\prolog-интерпретатором, так что программы должны работать похоже в любой
\prolog-системе эдинбургского типа или интерактивном компиляторе.

Если в вашей UNIX-системе уже установлен SWI-Prolog, запустите окно
терминала, и начните интерактивную сессию командной:

\begin{verbatim}
user@computer$ swipl
\end{verbatim}

Мы не будем использовать команду запуска именно в такой форме все время:
при запуске могут быть указаны дополниительные параметры командной строки,
которые можно использовать в определенных случаях. Читатель должен расмотреть
эту возможность после освоения базовых приемов работы, чтобы получить больше
возможностей.

Если вы хотите установить SWI Prolog под Debian \linux, выполните команду:

\begin{verbatim}
sudo apt install swi-prolog
\end{verbatim}

\bigskip
Под \win\ инсталлятор SWI-Prolog добавляет иконку запуска
интерпретатора, который вы можете запустить простым двойным целчком по
иконке. При запуске интерпретатор создает свое собственное командное окно.
\bigskip

После запуска интерпретатора обычно появляется сообщение о версии, лицензии и
авторах, а затем выводится приглашение ввода \term{цели}\ типа

\begin{verbatim}
?- _
\end{verbatim}

Интерактивные \term{цели}\ в \prolog е вводятся пользователем за приглашением
\verb|?-|.

Многие \prolog-системы поддерживают предоставление документации по запросу из
командной строки. В SWI Prolog встроена подробная система помощи. Документация
индексирована, и помогает пользователю в процессе работы. Попробуйте ввести

\begin{verbatim}
?- help(help).
\end{verbatim}

Обратите внимание что должна быть введены все символы, и ввод завершен возвратом
каретки.

Для иллюстрации нескольких приемов взимодействия с \prolog ом рассмотрим
следующий пример сессии. Если приведена ссылка на файл, предполагается что это
локальный файл в пользовательском каталоге, который был создан пользователем,
получен копированием из другого публично доступного источника, или получен
сохранением текстового файла из веб-браузера. Способ достижения последнего\ ---
следователь URL-ссылке на файл и сохранить его, или выбрать кусок текста из
онлайн-учебника \prolog а, скопировать его, вставить в текстовый редактор, и
сохранить полученный файл из текстового редактора. Комментарии вида
\verb|/*...*/| после целей используются для описания этих целей.

\lstv{Лог типичной \prolog-сессии}{prolog/fisher/running.pl}


\secrel{Разбор примеров программ}\label{fish2}\secdown

В этом разделе мы рассмотрим несколько специально подобраных примеров программ
на \prolog е. Порядок примеров специально выбран от наиболее простых до более
сложных. Ключевая цель\ --- показать основные приемы \term{представления знаний}
и методов декларативного программирования. 

\secrel{Раскраска карт}

Этот раздел использует известную математическую проблему\ --- \emph{раскраска
географических карт}\ --- в качестве иллюстрации применения набора фактов и
логических правил. Рассмотренная \prolog-программа показывает представление
смежных регионов карты, ее раскраски, и определение конфликтов раскраски: когда
\emph{два смежных региона имеют одинаковый цвет}.  Секция также показывает
применение концепции \termdef{семантического дерева}{семантическое деревj} и его
применение в логическом программировании.

Согласно формулировке известной математической задачи по раскраске смежных
плоских регионов\note{таких как географические карты}, необходимо подобрать
минимум цветов раскраски, и цвета регионов, так что никакие два смежных региона
не имеют один цвет. Два региона являются смежными, если они имеют некоторый
общий сегмент границы, например\note{упрощенно, только прямоугольные области}.
По данным численным именам регионов строим представление в виде \termdef{графа
смежности}{граф смежности}:

\begin{tabular}{p{0.4\textwidth} p{0.4\textwidth}}
\fig{}{prolog/fisher/f2_1_1.png}{height=0.3\textheight}&
\fig{}{prolog/fisher/f2_1_2.pdf}{height=0.35\textheight}\\
\end{tabular}

Мы удалили все границы, и нарисовали дугу между именами каждых двух смежных
областей. Фактически граф смежности содержит полную оригинальную информацию о
смежности областей. Для представления информации о смежности в синтаксисе
\prolog а запишем следующее:

\lst{\ }{prolog/fisher/f21_1.pl}{Prolog}

это набор выражений устанавливает факт смежности $A \rightarrow B$:
\verb|adjacent(A,B)|.\bigskip

Если загрузить этот файл в \prolog-систему, можно проверить работу целей:

\begin{verbatim}
?- adjacent(2,3). 
true .
?- adjacent(5,3). 
false . 
?- adjacent(3,R). 
R = 1 ; 
R = 2 ; 
R = 4 ; 
false . 
\end{verbatim}

Аналогично можно задать два набора раскраски регионов используя единичные
заключения: вариант \var{a} и вариант \var{b}:

\lst{\ }{prolog/fisher/f21_2.pl}{Prolog}
в форме
\begin{verbatim}
<имя отношения:color> (
    <номер зоны/узла графа>,
    <присвоенный цвет>,
    <имя раскраски>
).
\end{verbatim}

Что обозначает \termdef{факт}{факт}: ``имеется отношение color между номером
узла, цветом и именем раскраски''\note{причем не указывается какой
элемент главный или подчиненный, все элементы отношения равноправны}.\bigskip

Теперь мы хотим написать \prolog-определение конфликта раскрасок, имея в виду
совпадение цветов для двух регионов, например:

\lst{\ }{prolog/fisher/f21_3.pl}{Prolog}

Например,

\begin{verbatim}
?- conflict(a). 
false . 
?- conflict(b). 
true . 
?- conflict(Which). 
Which = b .
\end{verbatim}

Запрашивая отношение с неким значением-константой, или переменной\note{имя с
большой буквы} (последний случай), мы получаем от \prolog-системы заключение:
выполняется ли запрошенное отношение-\term{цель} и при каких значениях
переменных, имея в виду ранее определенный \term{набор фактов и отношений}\note{которые 
являются \term{базой знаний}, или \term{экспертной системой}}. В случае
использования переменной \prolog\ выдаст нам \emph{все} значения переменных, для
которых запрос истинен.

\bigskip
Можно определить другое отношение с тем же именем \verb|conflict| но с другим
количеством логических параметров:

\lst{\ }{prolog/fisher/f21_4.pl}{Prolog}

\prolog\ позволяет отличать два отношения с одинаковым именем: одно имеет один
параметр \verb|conflict/1|, а другой\ --- \verb|conflict/3|.\note{/цифра имеет
название \term{арность}}

\begin{verbatim}
?- conflict(R1,R2,b). 
R1 = 2   R2 = 4 
?- conflict(R1,R2,b),color(R1,C,b). 
R1 = 2   R2 = 4   C = blue 
\end{verbatim}

Последняя \term{цель} значит что регионы 2 и 4 связаны (adjacent) и оба синие
(blue). \term{Обоснованные} случаи, такие как \verb|conflict(2,4,b)|, называются
\termdef{консеквенцией}{консеквенция} или \termdef{выводом}{вывод
\prolog-программы} \prolog-программы. Один из способов демонстрации
консеквенции\ --- нарисовать \termdef{дерево заключений}{дерево заключений},
которое имеет консеквенцию в корне дерева, используя заключения программы для
обхода дерева, получая в результате конечное дерево, в котором все листья имеют
истинное значение. Например следующее дерево заключений может быть построено
используя полностью обоснованные заключения программы без переменных:

\noindent\fig{\ }{prolog/fisher/f2_1_3.pdf}{width=\textwidth}

\secrel{Два определения факториала}\label{fish22}

Этот раздел вводит в вычисления математических функций используя \prolog.
Обсуждаются различные встроенные арифметические операции. Также обсуждается
концепция derivation дерева, и как derivation деревья связаны с трассировкой в
\prolog е.

В файле \file{2\_2.pl} находятся два определения предикатов, являющиеся
определением фукнции вычисления факториала:

\lst{первый вариант}{prolog/fisher/2_2.pl}{Prolog}
 
Эта программа состоит из двух clauses. Первое заключение\ --- формулировка
\termdef{факта}{факт} (unit clause) \emph{без тела}. Второе заключение\ ---
\termdef{правило}{правило}, так как \emph{у него есть тело}. Тело второго
заключения находится после \verb|:-|, которое можно читать как ``если''. Тело
содержит литералы, разделенные запятыми, каждую запятую можно читать как ``и``.
\termdef{Заголовок правила}{заголовок правила}\ --- весь текст \term{факта} или
часть текста до \verb|:-| в правиле. Рассматривая текст как декларативную
программу, первое (фактическое) предложение читается как ``факториал 0 есть
1''\note{или: 0 и 1 \term{связаны отношением} ``факториал'', но у объектов
одновременно могут быть и другие отношения, например биты(0,1) и целые(0,1)},
и второе предложение заявляет что ``факториал \var{N} есть \var{F}\note{точнее:
N и F связаны отношением ``факториал''} если \verb|N>0| и \var{N1} есть
\verb|N-1| , и факториал \var{N1} есть \var{F1}, и \var{F} есть \verb|N*F1|.

\termdef{\prolog-цель}{цель (Пролог)} (goal) для вычисления факториала от 3
дает ответ в W\ --- \termdef{переменной цели}{переменная цели}:

\begin{verbatim}
?-  factorial(3,W).  
W=6 . 
\end{verbatim}

Рассмотрим следующее clause дерево сконструированное для литерала\\
\verb|factorial(3,W)|. Как описано в предыдущей секции, clause дерево не
содержит никаких свободных переменных, вместо этого включает непосредственно их
значения. Каждое ветвление под узлом определяется clause оригинальной программы,
используя непосредственно вхождения значений переменных; узел задается
заголовком правила, а литералы теля становятся дочерними узлами.

\fig{}{prolog/fisher/f2_2.pdf}{width=0.95\textwidth}

\emph{Все арифметические листья \var|true|} при исполнении\note{в соответствие с
предполагаемой интерпретацией}, и самая нижная связь в дереве соответствует
самому первому clause в программе вычисленяи факториала. Первый clause может
быть записан как:

\begin{verbatim}
factorial(0,1) :- true. 
\end{verbatim}
и фактически \verb|?- true.| \prolog-цель которая всегда успешна
\note{\var{true} встроеннный предикат}. Для краткости, мы не отрисовали
\verb|true| для всех листьев, являющихся арифметическими литералами.

Программное clause дерево показывает значение цели в коорне дерева. Так,\\
\verb'factorial(3,6)' является консеквенцией \prolog-программы, так как
существует clause дерево с корнем \verb'factorial(3,6)', все листья которого
\verb|true|. С другой стороны литерал \verb'factorial(5,2)' не консеквенция,
так как такого дерева для него нет, а значением программы для литерала
\verb'factorial(5,2)' является \verb|false|:

\begin{verbatim}
?- factorial(3,6).  
true .
?- factorial(5,2).  
false . 
\end{verbatim}
как и следовало ожидать. Clause-деревья также называются AND-деревьями, так как
чтобы корень был консеквенцией программы, все его поддеревья также должны быть
консеквенциями. Позже clause деревья будут рассмотрены подробнее. Мы отметили
что \emph{clause дерево описывает семантику (значение) программы}. В разделе
\ref{fish6} мы рассмотрим другой подход к семантике программ. Clause-деревья
предоставляют интуитивный и корректный подход к описанию семантики.

\bigskip

Нам нужно отличать clause деревья программы и \termdef{деревья
вывода}{дерево вывода}. Сlause-деревья статичны, и могут быть нарисованы для
программмы или цели через механизм удовлетворения частичных (под)целей, как
описано выше. Грубо говоря, clause-деревья соответствуют декларативному чтению
программы.

\term{Деревья вывода} наоборот, имеют в виду механизм привязки переменных
\prolog а, и порядок в котором удовлетворяются вложенные частичные цели.
Подробнее деревья вывода описаны в разделе \ref{fish31}, но тем не менее
посмотрите анимацию, предоставляемую динамическим отладчиком, как описано ниже.

\termdef{Трассировка}{трассировка} исполнения \prolog-программы также показывает
как переменные привязываются при удовлетвормении целей. Следующий пример
показывает включение/выключение трассировки в типичной \prolog-системе.

\begin{verbatim}
?- trace. 
% The debugger will first creep -- showing everything (trace). 
 
true .
[trace] 
?- factorial(3,X). 
  (1) 0 Call: factorial(3,_8140) ? [Enter] creep 
  (1) 1 Head [2]: factorial(3,_8140) ? [Enter] creep 
  (2) 1 Call (built-in): 3>0 ?  creep
  (2) 1 Done (built-in): 3>0 ?  creep
  (3) 1 Call (built-in): _8256 is 3-1 ? creep 
  (3) 1 Done (built-in): 2 is 3-1 ?  creep
  (4) 1 Call: factorial(2, _8270) ?  creep
   ... 
  (1) 0 Exit: factorial(3,6) ? 
X=6 .
[trace] 
?- notrace. 
% The debugger is switched off 
 
true .
\end{verbatim}

The animated tree below gives another look at the derivation tree for the
\prolog goal \verb'factorial(3,X)'. To start (or to restart) the animation,
simply click on the \keys{Step} button.

\bigskip

Заголовок этого раздела говорит ``\emph{Два} определения факториала'', вот
второй вариант, использующий три переменых:

\lst{второй вариант}{prolog/fisher/2_2_2.pl}{Prolog}

Для этой версии используйте следующую цель-запрос:

\begin{verbatim}
?- factorial(5,1,F). 
F=120 .
\end{verbatim}

Второй параметр в определении называется \term{параметр-аккумулятор}, который
также хорошо известен в функциональном программировании. Эта версия факториала
определена с использованием \term{хвостовой рекурсии}. Важно чтобы вы выполнили
следующие упражнения:

\paragraph{Упражнение \ref{fish22}.1} Используя первый вариант программы
факториала, четко покажите что не существует clause-дерева с корнем \verb'factorial(5,2)',
имеющего все true листья.

\paragraph{Упражнение \ref{fish22}.2} Нарисуйте clause-дерево для цели
\verb'factorial(3,1,6)' со всеми true-листьями, в виде аналогичном ранее
описанному дереву для \verb'factorial(3,6)'.
Покажите, чем отличаются два варианта программы в процессе вычисления
факториала\,? Также, протрассируйте цель \verb'factorial(3,1,6)' используя
\prolog-систему.


\secrel{Классическая задача ``Ханойские башни''}\label{fish23}

Показано формулирование и решение классической задача на \prolog е. Рассмотрены
декларативные и процедурные подходы к программированию. Решение задачи выводится
на экран.

Цель известной головоломки\ --- переместить \var{N} дисков с левого штыря на
правый, используя центральный штырь как дополнительное храненилище. Требование:
\emph{нельзя класть б\'{о}льший диск на м\'{е}ньший}. Следующая диаграмма
показывает начальное положение для \verb|N=3| дисков.

Регурсивная программа на \prolog е, решающая головоломку, состоит из двух
утверждений:

\noindent\begin{tabular}{p{0.4\textwidth} p{0.5\textwidth}}
\fig{}{prolog/fisher/f2_3.png}{width=0.4\textwidth}&
\lst{Ханойские башни}{prolog/fisher/2_3.pl}{Prolog}\\
\end{tabular}

Переменная \verb'_' (или любое другое имя начинающееся с подчеркивания)\ --- 
переменные \verb|don't-care| (не важно). \prolog\ позволяет использовать
эти перемененные как обычные в любых структурах, но для них \emph{не выполняется
привязка}.

Вот что выводится при решении задачи при \verb|N=3|:

\begin{verbatim}
?-  move(3,left,right,center). 
Move top disk from left to right 
Move top disk from left to center 
Move top disk from right to center 
Move top disk from left to right 
Move top disk from center to left 
Move top disk from center to right 
Move top disk from left to right 
true .
\end{verbatim}

Первое предложение программы описывает перемещение одного диска. Второе
предложение описывает как можно получить решение рекурсивно. Например,
декларативное чтение второго предложения для случая \verb|N=3, X=left, Y=right|,
и \verb|Z=center| приводит к следующему:

\begin{verbatim}
move(3,left,right,center) если 
    move(2,left,center,right) и ] * 
    move(1,left,right,center) и 
    move(2,center,right,left). ] ** 
\end{verbatim}

Это декларативное чтение очевидно правильно. Процедурное чтение тесно связано с
декларативной интерпретацией рекурсивного утверждения, оно должно выглядеть
как-то так:

\begin{verbatim}
удовлетворить цель ?-move(2,left,center,right), и потом 
удовлетворить цель ?-move(1,left,right,center), и потом 
удовлетворить цель ?-move(2,center,right,left). 
\end{verbatim}

Аналогично мы можем записать декларативное прочтение для случая \verb|N=2|:

\begin{verbatim}
move(2,left,center,right) если ] * 
move(1,left,right,center) и 
move(1,left,center,right) и 
move(1,right,center,left). 
move(2,center,right,left) если ] ** 
move(1,center,left,right) и
move(1,center,right,left) и 
move(1,left,right,center). 
\end{verbatim}

Теперь подставим содержимое последних двух implications и увидим решение которое
сгенерирует \prolog:

\begin{verbatim}
move(3,left,right,center) если 
move(1,left,right,center) и 
move(1,left,center,right) и * 
move(1,right,center,left) и 
--------------------------- 
move(1,left,right,center) и 
--------------------------- 
move(1,center,left,right) и 
move(1,center,right,left) и ** 
move(1,left,right,center). 
\end{verbatim}

Процедурное прочтение последних двух больших implication должно быть очевидно.
Этот пример показывает при основных операции \prolog а:
\begin{enumerate}[nosep]
  \item
Цели сопоставляются с головой правила, и
  \item 
тело правила (с соответствующе привязанными переменными) становится новой
последовательностью целей; процесс повторяется 
  \item 
пока не будет удовлетворена основная цель или условие, или не будет выполнено
простое действие, например выведен текст.
\end{enumerate}

\begin{framed}\noindent
Процесс сопоставления переменных с образцом (variable matching)\\
называется \termdef{унификацией}{унификация}.
\end{framed}

\paragraph{Упражнение \ref{fish23}.1} Нарисуйте clause-дерево для цели
\verb'move(3,left,right,center)', покажите что это консеквенция программы. Как
полученное дерево связано с процессом подстановки, поисанным выше\,?

\paragraph{Exercise \ref{fish23}.2} Попробуйте \prolog-цель
\verb|?-move(3,left,right,left)|. Что не так\,? Предложите способ
исправления, и проследите процесс работы исправления.


\secrel{Загрузка, редактирование, хранение программ}\label{fish24}

Примеры показывают различные способы хранения и загрузки \prolog-программ, и
пример вызова системного редактора. Читаютелю предлагается предварительно
заглянуть в разделы \ref{fish31}, \ref{fish32} чтобы иметь представление о том,
как работает \prolog.

Стандартные предикаты для загрузки программ это \verb'consult',
\verb'reconsult', и скобочная нотация загрузки \verb'[ ...]'. Например цель
\verb|?- consult('lists.pro').| открывает файл \file{lists.pro} и загружает из
него предложения в память.

Существует два способа, которыми \prolog-программа может быть неправильна:
\begin{enumerate}[nosep]
\item исходный код имеет синтаксические ошибки, в этом случае при загрузке будут
выводиться сообщения об ошибках, и
\item в программе есть какие-то логические ошибки, которые программист должен
найти через тестирование программы.
\end{enumerate}
Программа в ее текущей версии должна рассматриваться как прототип корректной
версии в будущем, и принята обычная практика редактирования текущей версии, и ее
перезагрузка с повторным тестированием. Существуют хорошие приемы быстрого
прототипирования, чтобы программист уделял все время и усилия на анализ
проблемы. Интересно что если подход быстрого прототипирования кажется ошибочным,
это отличный сигнал взять ручку и бумагу, еще раз проанализировать требования, и
начать сначала\,!

Мы можем вызывать редактор непосредственно в \prolog е:
\begin{verbatim}
?- edit('lists.pro'). %% редактор определенный пользователем, см.ниже ...
\end{verbatim}
и после возврата из редактора\note{предполагается что новая версия файла была
сохранена в том же файле} использовать цель
\begin{verbatim}
? reconsult('lists.pro').
\end{verbatim}
для перезагрузки утверждений программы в память, автоматически замещая
предыдущие определения. Если использовать \verb'consult' вместо
\verb'reconsult', старая\note{и скорее всего неправильная} версия утверждений
программы останется в памяти наряду с новыми определениями\note{это поведение
зависит от конкретной версии \prolog-системы}.
 
Если в память было загружено несколько файлов, и требуется перезагрузить только
один, используйте \verb'reconsult'. Если перегружаемый файл определяет
предикаты, которые не определяются в остальных файлах, перезагрузка не повлияет
на кляузы, которые были загружены в остальных файлах.

Скобочная нотация очень удобна, например

\begin{verbatim}
?- ['file1.pro',file2.pro',file3.pro'].
\end{verbatim}
загрузит (точнее \verb|reconsult|) все три файла в память \prolog-системы.

Многие \prolog-системы оставляют программисту определение любимого текстового
редактора. Здесь описан пример программы, которая вызывает \prog{TextEdit}
на \prog{Mac(OSX)}\note{это просто пример; мы не используем конкретно
\prog{TextEdit}}.

\begin{verbatim}
edit(File) :- 
   name(File,FileString), 
   name('open -e ', TextEditString), %% укажите ваш любимый редактор
   append(TextEditString,FileString,EDIT),
   name(E,EDIT), 
   shell(E).
\end{verbatim}

Для использования этого редактора, этот код должен быть
загружен\note{предполагаем локальную \prolog-сессию}

\begin{verbatim}
?- [edit]. 
yes 
\end{verbatim}

и цель \verb'edit' может быть использована\note{опять же предполагаем что
файл для редактирования локален для сессии}

\begin{verbatim}
?- edit('p.pl'). 

{ TextEdit запускается с файлом, редактируйте его...}
{ и сохраните измененную программу с тем же именем файла  ... } 
\end{verbatim}

\noindent\fig{\\Вызов внешнего
редактора}{prolog/fisher/edit_snap.jpg}{width=0.7\textwidth}

После редактирования и сохранения мы можем перезагрузить новую версию

\begin{verbatim}
?- reconsult('p.pl'). 

{ наша prolog-сессия перезагружает программу для тестирования ...}
\end{verbatim}

Для редактирования утверждений, введеных пользователем интерактивно, используйте
цели

\begin{verbatim}
?-consult(user).
?-reconsult(user).
?-[user].  
\end{verbatim}
Пользователь вводит предложения интерактивно, используя символ останова \verb'.'
в конце набора устверждений, и сочетание клавиш \keys{Ctrl+D} для окончания
ввода.

\paragraph{Упражнение \ref{fish24}} Проанализируйте как работает
редактирование программы. Сначала попробуйте цели

\begin{verbatim}
?-name('name',NameString). 
\end{verbatim}
и
\begin{verbatim}
?- name(Name,"name").
\end{verbatim}
\verb|name/2| описана в разделе \ref{fish413}.

\bigskip
Теперь хороший момент для читателя немного заглянуть вперед и попробовать
почитать первые две секции из раздела \ref{fish3}\ ``Как работает \prolog'',
и затем вернуться к остальным примерам программ. Необходимо чтобы вы понимали
как работат машина вывода \prolog а, чтобы понять как конструируются следующие
примеры программ.
 
\secrel{2.5 Negation as failure} 

The section gives an introduction to \prolog\'s negation-as-failure feature, with
some simple examples. Further examples show some of the difficulties that can be
encountered for programs with negation as failure.

\secrel{2.6 Tree data and relations}

This section shows \prolog\ operator definitions for a simple tree structure. Tree
processing relations are defined and corresponding goals are studied.

\secrel{2.7 Prolog lists and sequences}\label{fish27}

This section contains some of the most useful Prolog list accessing and
processing relations. Prolog's primary dynamic structure is the list, and this
structure will be used repeatedly in later sections.
 
 
\secrel{2.8 Change for a dollar}

A simple change maker program is studied. The important observation here is how
a \prolog\ predicate like 'member' can be used to generate choices, the choices
are checked to see whether they solve the problem, and then backtracking on
'member' generates additional choices. This fundamental generate and test
strategy is very natural in \prolog.

\secrel{2.9 Map coloring redux} 

We take another look at the map coloring problem introduced in Section 2.1. This
time, the data representing region adjacency is stored in a list, colors are
supplied in a list, and the program generates colorings which are then checked
for correctness.

\secrel{2.10 Simple I/O}

This section discusses opening and closing files, reading and writing of \prolog\
data.
 
\secrel{2.11 Chess queens challenge puzzle}

This familiar puzzle is formulate in \prolog\ using a permutation generation
program from Section 2.7. Backtracking on permutations produces all solutions.
 
\secrel{2.12 Finding all answers} 

\prolog\'s 'setof' and 'bagof' predicates are presented. An implementation of
'bagof' using 'assert' and 'retract' is given.

\secrel{2.13 Truth table maker} 

This section designs a recursive evaluator for infix Boolean expressions, and a
program which prints a truth table for a Boolean expression. The variables are
extracted from the expression and the truth assignments are automatically
generated.

\secrel{2.14 DFA parser}

A generic DFA parser is designed. Particular DFAs are represented as \prolog\
data.
 
\secrel{2.15 Graph structures and paths}

This section designs a path generator for graphs represented using a static
\prolog\ representation. This section serves as an introduction to and motivation
for the next section, where dynamic search grows the search graph as it works.
 
\secrel{2.16 Search} 

The previous section discussed path generation in a static graph. This section
develops a general \prolog\ framework for graph searching, where the search graph
is constructed as the search proceeds. This can be the basis for some of the
more sophisticated graph searching techniques in A.I.

\secrel{2.17 Animal identification game} 

This is a toy program for animal identification that has appeared in several
references in some form or another. We take the opportunity to give a unique
formulation using \prolog\ clauses as the rule database. The implementation of
verification of askable goals (questions) is especially clean. This example is a
good motivation for expert systems, which are studied in Chapter 6.

\secrel{2.18 Clauses as data} 

This section develops a \prolog\ program analysis tool. The program analyses a
\prolog\ program to determine which procedures (predicates) use, or call, which
other procedures in the program. The program to be analyzed is loaded
dynamically and its clauses are processed as first-class data.

\secrel{2.19 Actions and plans}

An interesting prototype for action specifications and plan generation is
presented, using the toy blocks world. This important subject is continued and
expanded in Chapter 7.

\secup
\secrel{3. How \prolog\ Works}\label{fish3}\secdown
\secrel{3.1 \prolog\ derivation trees, choices and unification}\label{fish31} 
\secrel{3.2 Cut}\label{fish32}
\secrel{3.3 Meta-interpreters in \prolog}\label{fish33}
\secup
\secrel{4. Built-in Goals}\label{fish4}\secdown
\secrel{4.1 Utility goals }
\secrel{4.2 Universals (true and fail)} 
\secrel{4.3 Loading \prolog\ programs} 
\secrel{4.4 Arithmetic goals} 
\secrel{4.5 Testing types} 
\secrel{4.6 Equality of \prolog\ terms, unification} 
\secrel{4.7 Control} 
\secrel{4.8 Testing for variables} 
\secrel{4.9 Assert and retract} 
\secrel{4.10 Binding a variable to a numerical value} 
\secrel{4.11 Procedural negation, negation as failure }
\secrel{4.12 Input/output} 
\secrel{4.13 \prolog\ terms and clauses as data}\label{fish413} 
\secrel{4.14 \prolog\ operators} 
\secrel{4.15 Finding all answers}
\secup
\secrel{5. Search in \prolog}\label{fish5}\secdown
\secrel{5.1 The A* algorithm in \prolog} 
\secrel{5.2 The 8-puzzle} 
\secrel{5.3 $\alpha\beta$ search in \prolog}\label{fish53}
\secup
\secrel{6. Logic Topics}\label{fish6}\secdown
\secrel{6.1 Chapter 6 notes} 
\secrel{6.2 Positive logic} 
\secrel{6.3 Convert first-order logic to normal form} 
\secrel{6.4 A normal rulebase goal interpreter} 
\secrel{6.5 Evidentiary soundness and completeness} 
\secrel{6.6 Rule tree visualization using Java}
\secup
\secrel{7. Introduction to Natural Language Processing}\label{fish7}\secdown
\secrel{7.1 \prolog\ grammar parser generator} 
\secrel{7.2 \prolog\ grammar for simple English phrase structures} 
\secrel{7.3 Idiomatic natural language command and question interfaces}
\secup
\secrel{8. Prototyping with \prolog}\label{fish8}\secdown
\secrel{8.1 Action specification for a simple calculator} 
\secrel{8.2 Animating the 8-puzzle (\$5.2) using character graphics} 
\secrel{8.3 Animating the blocks mover (\$2.19) using character graphics} 
\secrel{8.4 Java Tic-Tac-Toe GUI plays against \prolog\ opponent
(\$5.3)}\label{fish84}
\secrel{8.5 Structure diagrams and \prolog}
\secup
\secly{References}
\secup