\secrel{Загрузка, редактирование, хранение программ}\label{fish24}

Примеры показывают различные способы хранения и загрузки \prolog-программ, и
пример вызова системного редактора. Читаютелю предлагается предварительно
заглянуть в разделы \ref{fish31}, \ref{fish32} чтобы иметь представление о том,
как работает \prolog.

Стандартные предикаты для загрузки программ это \verb'consult',
\verb'reconsult', и скобочная нотация загрузки \verb'[ ...]'. Например цель
\verb|?- consult('lists.pro').| открывает файл \file{lists.pro} и загружает из
него предложения в память.

Существует два способа, которыми \prolog-программа может быть неправильна:
\begin{enumerate}[nosep]
\item исходный код имеет синтаксические ошибки, в этом случае при загрузке будут
выводиться сообщения об ошибках, и
\item в программе есть какие-то логические ошибки, которые программист должен
найти через тестирование программы.
\end{enumerate}
Программа в ее текущей версии должна рассматриваться как прототип корректной
версии в будущем, и принята обычная практика редактирования текущей версии, и ее
перезагрузка с повторным тестированием. Существуют хорошие приемы быстрого
прототипирования, чтобы программист уделял все время и усилия на анализ
проблемы. Интересно что если подход быстрого прототипирования кажется ошибочным,
это отличный сигнал взять ручку и бумагу, еще раз проанализировать требования, и
начать сначала\,!

Мы можем вызывать редактор непосредственно в \prolog е:
\begin{verbatim}
?- edit('lists.pro'). %% редактор определенный пользователем, см.ниже ...
\end{verbatim}
и после возврата из редактора\note{предполагается что новая версия файла была
сохранена в том же файле} использовать цель
\begin{verbatim}
? reconsult('lists.pro').
\end{verbatim}
для перезагрузки утверждений программы в память, автоматически замещая
предыдущие определения. Если использовать \verb'consult' вместо
\verb'reconsult', старая\note{и скорее всего неправильная} версия утверждений
программы останется в памяти наряду с новыми определениями\note{это поведение
зависит от конкретной версии \prolog-системы}.
 
Если в память было загружено несколько файлов, и требуется перезагрузить только
один, используйте \verb'reconsult'. Если перегружаемый файл определяет
предикаты, которые не определяются в остальных файлах, перезагрузка не повлияет
на кляузы, которые были загружены в остальных файлах.

Скобочная нотация очень удобна, например

\begin{verbatim}
?- ['file1.pro',file2.pro',file3.pro'].
\end{verbatim}
загрузит (точнее \verb|reconsult|) все три файла в память \prolog-системы.

Многие \prolog-системы оставляют программисту определение любимого текстового
редактора. Здесь описан пример программы, которая вызывает \prog{TextEdit}
на \prog{Mac(OSX)}\note{это просто пример; мы не используем конкретно
\prog{TextEdit}}.

\begin{verbatim}
edit(File) :- 
   name(File,FileString), 
   name('open -e ', TextEditString), %% укажите ваш любимый редактор
   append(TextEditString,FileString,EDIT),
   name(E,EDIT), 
   shell(E).
\end{verbatim}

Для использования этого редактора, этот код должен быть
загружен\note{предполагаем локальную \prolog-сессию}

\begin{verbatim}
?- [edit]. 
yes 
\end{verbatim}

и цель \verb'edit' может быть использована\note{опять же предполагаем что
файл для редактирования локален для сессии}

\begin{verbatim}
?- edit('p.pl'). 

{ TextEdit запускается с файлом, редактируйте его...}
{ и сохраните измененную программу с тем же именем файла  ... } 
\end{verbatim}

\noindent\fig{\\Вызов внешнего
редактора}{prolog/fisher/edit_snap.jpg}{width=0.7\textwidth}

После редактирования и сохранения мы можем перезагрузить новую версию

\begin{verbatim}
?- reconsult('p.pl'). 

{ наша prolog-сессия перезагружает программу для тестирования ...}
\end{verbatim}

Для редактирования утверждений, введеных пользователем интерактивно, используйте
цели

\begin{verbatim}
?-consult(user).
?-reconsult(user).
?-[user].  
\end{verbatim}
Пользователь вводит предложения интерактивно, используя символ останова \verb'.'
в конце набора устверждений, и сочетание клавиш \keys{Ctrl+D} для окончания
ввода.

\paragraph{Упражнение \ref{fish24}} Проанализируйте как работает
редактирование программы. Сначала попробуйте цели

\begin{verbatim}
?-name('name',NameString). 
\end{verbatim}
и
\begin{verbatim}
?- name(Name,"name").
\end{verbatim}
\verb|name/2| описана в разделе \ref{fish413}.

\bigskip
Теперь хороший момент для читателя немного заглянуть вперед и попробовать
почитать первые две секции из раздела \ref{fish3}\ ``Как работает \prolog'',
и затем вернуться к остальным примерам программ. Необходимо чтобы вы понимали
как работат машина вывода \prolog а, чтобы понять как конструируются следующие
примеры программ.