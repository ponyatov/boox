\secrel{Установка и запуск \prolog-системы}\label{fish1}

Примеры этого учебника \prolog а были подготовлены с использованием

\begin{itemize}[nosep]
  \item Quintus Prolog на компьютерах Digital Equipment Corporation MicroVAXes
  (далекая история)
  \item SWI Prolog на Sun Spark (давным давно)
  \item персональных компьютерах c \win
  \item или OS X на Macах  
\end{itemize}
 
Другие заметные \prolog-системы (Borland, XSB, LPA, Minerva \ldots)
использовались для разработки и тестирования последние 25 лет.
В этом учебнике запланирован новый раздел, в котором описано использование
любых \prolog-систем в общем, но пока этот раздел недоступен.

Сайт SWI-Prolog содержит много информации, ссылки на загрузку, и документацию:

\url{http://www.swi-prolog.org/}

Особо следуюет отметить возможность попробовать SWI Prolog on-line без 
регистрации и SMS: \url{http://swish.swi-prolog.org/}. Этот вариант особенно
удобен, так как не требует никакой установки ПО, административных прав, вы
можете работать с этим учебником даже в интернет-кафе.

\bigskip
Примеры в этом учебнике используют упрощенную форму взаимодействия в типичным
\prolog-интерпретатором, так что программы должны работать похоже в любой
\prolog-системе эдинбургского типа или интерактивном компиляторе.

Если в вашей UNIX-системе уже установлен SWI-Prolog, запустите окно
терминала, и начните интерактивную сессию командной:

\begin{verbatim}
user@computer$ swipl
\end{verbatim}

Мы не будем использовать команду запуска именно в такой форме все время:
при запуске могут быть указаны дополниительные параметры командной строки,
которые можно использовать в определенных случаях. Читатель должен расмотреть
эту возможность после освоения базовых приемов работы, чтобы получить больше
возможностей.

Если вы хотите установить SWI Prolog под Debian \linux, выполните команду:

\begin{verbatim}
sudo apt install swi-prolog
\end{verbatim}

\bigskip
Под \win\ инсталлятор SWI-Prolog добавляет иконку запуска
интерпретатора, который вы можете запустить простым двойным целчком по
иконке. При запуске интерпретатор создает свое собственное командное окно.
\bigskip

После запуска интерпретатора обычно появляется сообщение о версии, лицензии и
авторах, а затем выводится приглашение ввода \term{цели}\ типа

\begin{verbatim}
?- _
\end{verbatim}

Интерактивные \term{цели}\ в \prolog е вводятся пользователем за приглашением
\verb|?-|.

Многие \prolog-системы поддерживают предоставление документации по запросу из
командной строки. В SWI Prolog встроена подробная система помощи. Документация
индексирована, и помогает пользователю в процессе работы. Попробуйте ввести

\begin{verbatim}
?- help(help).
\end{verbatim}

Обратите внимание что должна быть введены все символы, и ввод завершен возвратом
каретки.

Для иллюстрации нескольких приемов взимодействия с \prolog ом рассмотрим
следующий пример сессии. Если приведена ссылка на файл, предполагается что это
локальный файл в пользовательском каталоге, который был создан пользователем,
получен копированием из другого публично доступного источника, или получен
сохранением текстового файла из веб-браузера. Способ достижения последнего\ ---
следователь URL-ссылке на файл и сохранить его, или выбрать кусок текста из
онлайн-учебника \prolog а, скопировать его, вставить в текстовый редактор, и
сохранить полученный файл из текстового редактора. Комментарии вида
\verb|/*...*/| после целей используются для описания этих целей.

\lstv{Лог типичной \prolog-сессии}{prolog/fisher/running.pl}

Комментарии в правой части были добавлены в текстовом редакторе. Они отмечают
некоторые вещи, перечисленные ниже:

\begin{enumerate}
  \item
Определение \term{цели} \prolog а завершается точкой \verb|.|\ .
В этом случае цель была загружена в внешнего файла с исходным тестом программы.
Этот скобочный стиль записи программы унаследован из самых первых реализаций
\prolog а. Можно загрузить несколько файлов сразу, указав их имена в одиночных
кавычках, разделяя запятыми. В нашем случае имя файла \file{2\_2.pl}, программа
содержит два программы на \prolog е для вычисления факториала от положительного
целого. Подробно эта программа описана в разделе \ref{fisher22}. Файл программы
ищется в текущем каталоге. Если поиск неуспешен, нужно явно указать полный путь
обычным для вашей ОС способом.

\item 
Встроенный предикат \verb|listing|\ выводит листинг программы из ОЗУ\ --- в
нашем случае программу вычисления факториала, загруженную ранее. Внешний вид
этого литсинга несколько отличается от исходного кода в файле из \ref{fisher22}.
Заметим, что \prog{Quintus Prolog} компилирует программу, если отдельно не
указано что определенные предикаты являются динамическими. Скомпилированные
предикаты не могут быть выведены через \verb|listing|, поэтому если у вас он не
срабатывает, возможно требуется дополнить исходник декларацией динамического
предиката, чтобы пример сработал. В \prog{SWI Prolog}\ этот пример работает без
модификации.

  \item 
Эта цель \verb|factorial(10,What)| говорит ``факториал 10ти что?''. Слово
\verb|What| начинается с большой буквы, указывающей что это \termdef{логическая
переменная}{логическая переменная}. \prolog\ удовлетворяет цель находя все
возможные значения переменной \verb|What|.

  \item
Теперь в памяти находятся обе программы из файлов \file{2\_1.pl} и
\file{2\_7.pl}. Файл \file{2\_7.pl} содержит несколько определений обработки
списков (см. \ref{fisher27}).

  \item 
Только что загруженная программа (\file{2\_7.pl} содержит определение предиката
\verb|takeout|. Цель \verb|takeout(X,[1,2,3,4],Y)| запрашивает поиск всех таких
\verb|X| что значение взятое из списка \verb|[1,2,3,4]| оставляет остаток в
переменной \verb|Y|, для всех возможных случаев. Существует четыре способа
сделать это, как показано в результате.  Предикат \verb|takeout| обсуждается в
разделе \ref{fisher27}. Таким образом, \emph{в \prolog\ заложен поиск всех
возможных ответов}: после того как будет выведен очередной ответ, \prolog\
ожидает реакции пользователя мигая курсором в конце строки с ответом. Если
пользователель нажмет \keys{;}, \prolog\ будет выполнять поиск следующего
ответа. Если пользователь просто нажмет \keys{Enter}, \prolog\ остановит поиск.

  \item
Составная, или \termdef{конъюнктивная цель}{конъюнктивная цель}, определяет
одновременное удовлетворение \emph{двух} отдельных целей. Отметим что
используется арифметическая цель (встроенное отношение) \verb|X>3|.
\prolog\ будет пытаться удовлетворить эти цели \emph{слева направо}, в порядке
чтения. В нашем случае существует единственный ответ. Отметим использование
в цели \termdef{анонимной переменной}{анонимная переменная} \verb|_|\ ,
\termdef{биндинг}{биндинг логической переменной} (\termdef{привязка}{привязка
логической переменной}) для которой не выводится (переменная ``не важно'').

  \item 
Цель \verb|halt| всегда успешна и завершает работу интерпретатора.
\end{enumerate}