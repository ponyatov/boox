\secrel{Установка и запуск \prolog-системы}\label{fish1}

Примеры этого учебника \prolog а были подготовлены с использованием

\begin{itemize}[nosep]
  \item Quintus Prolog на компьютерах Digital Equipment Corporation MicroVAXes
  (далекая история)
  \item SWI Prolog на Sun Spark (давным давно)
  \item персональных компьютерах c \win
  \item или OS X на Macах  
\end{itemize}
 
Другие заметные \prolog-системы (Borland, XSB, LPA, Minerva \ldots)
использовались для разработки и тестирования последние 25 лет.
В этом учебнике запланирован новый раздел, в котором описано использование
любых \prolog-систем в общем, но пока этот раздел недоступен.

Сайт SWI-Prolog содержит много информации, ссылки на загрузку, и документацию:

\url{http://www.swi-prolog.org/}

Особо следуюет отметить возможность попробовать SWI Prolog on-line без 
регистрации и SMS: \url{http://swish.swi-prolog.org/}. Этот вариант особенно
удобен, так как не требует никакой установки ПО, административных прав, вы
можете работать с этим учебником даже в интернет-кафе.

\bigskip
Примеры в этом учебнике используют упрощенную форму взаимодействия в типичным
\prolog-интерпретатором, так что программы должны работать похоже в любой
\prolog-системе эдинбургского типа или интерактивном компиляторе.

Если в вашей UNIX-системе уже установлен SWI-Prolog, запустите окно
терминала, и начните интерактивную сессию командной:

\begin{verbatim}
user@computer$ swipl
\end{verbatim}

Мы не будем использовать команду запуска именно в такой форме все время:
при запуске могут быть указаны дополниительные параметры командной строки,
которые можно использовать в определенных случаях. Читатель должен расмотреть
эту возможность после освоения базовых приемов работы, чтобы получить больше
возможностей.

Если вы хотите установить SWI Prolog под Debian \linux, выполните команду:

\begin{verbatim}
sudo apt install swi-prolog
\end{verbatim}

\bigskip
Под \win\ инсталлятор SWI-Prolog добавляет иконку запуска
интерпретатора, который вы можете запустить простым двойным целчком по
иконке. При запуске интерпретатор создает свое собственное командное окно.
\bigskip

После запуска интерпретатора обычно появляется сообщение о версии, лицензии и
авторах, а затем выводится приглашение ввода \term{цели}\ типа

\begin{verbatim}
?- _
\end{verbatim}

Интерактивные \term{цели}\ в \prolog е вводятся пользователем за приглашением
\verb|?-|.

Многие \prolog-системы поддерживают предоставление документации по запросу из
командной строки. В SWI Prolog встроена подробная система помощи. Документация
индексирована, и помогает пользователю в процессе работы. Попробуйте ввести

\begin{verbatim}
?- help(help).
\end{verbatim}

Обратите внимание что должна быть введены все символы, и ввод завершен возвратом
каретки.

Для иллюстрации нескольких приемов взимодействия с \prolog ом рассмотрим
следующий пример сессии. Если приведена ссылка на файл, предполагается что это
локальный файл в пользовательском каталоге, который был создан пользователем,
получен копированием из другого публично доступного источника, или получен
сохранением текстового файла из веб-браузера. Способ достижения последнего\ ---
следователь URL-ссылке на файл и сохранить его, или выбрать кусок текста из
онлайн-учебника \prolog а, скопировать его, вставить в текстовый редактор, и
сохранить полученный файл из текстового редактора. Комментарии вида
\verb|/*...*/| после целей используются для описания этих целей.

\lstv{Лог типичной \prolog-сессии}{prolog/fisher/running.pl}

