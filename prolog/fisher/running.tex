\secrel{Установка и запуск \prolog-системы}\label{fish1}

Примеры этого учебника \prolog а были подготовлены с использованием

\begin{itemize}[nosep]
  \item Quintus Prolog на компьютерах Digital Equipment Corporation MicroVAXes
  (далекая история)
  \item SWI Prolog на Sun Spark (давным давно)
  \item персональных компьютерах c \win
  \item или OS X на Macах  
\end{itemize}
 
Другие заметные \prolog-системы (Borland, XSB, LPA, Minerva \ldots)
использовались для разработки и тестирования последние 25 лет.
В этом учебнике запланирован новый раздел, в котором описано использование
любых \prolog-систем в общем, но пока этот раздел недоступен.

Сайт SWI-Prolog содержит много информации, ссылки на загрузку, и документацию:

\url{http://www.swi-prolog.org/}

Особо следуюет отметить возможность попробовать SWI Prolog on-line без 
регистрации и SMS: \url{http://swish.swi-prolog.org/}  
