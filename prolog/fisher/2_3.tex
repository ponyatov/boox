\secrel{Классическая задача ``Ханойские башни''}\label{fish23}

Показано формулирование и решение классической задача на \prolog е. Рассмотрены
декларативные и процедурные подходы к программированию. Решение задачи выводится
на экран.

Цель известной головоломки\ --- переместить \var{N} дисков с левого штыря на
правый, используя центральный штырь как дополнительное храненилище. Требование:
\emph{нельзя класть б\'{о}льший диск на м\'{е}ньший}. Следующая диаграмма
показывает начальное положение для \verb|N=3| дисков.

Регурсивная программа на \prolog е, решающая головоломку, состоит из двух
утверждений:

\noindent\begin{tabular}{p{0.4\textwidth} p{0.5\textwidth}}
\fig{}{prolog/fisher/f2_3.png}{width=0.4\textwidth}&
\lst{Ханойские башни}{prolog/fisher/2_3.pl}{Prolog}\\
\end{tabular}

Переменная \verb'_' (или любое другое имя начинающееся с подчеркивания)\ --- 
переменные \verb|don't-care| (не важно). \prolog\ позволяет использовать
эти перемененные как обычные в любых структурах, но для них \emph{не выполняется
привязка}.

Вот что выводится при решении задачи при \verb|N=3|:

\begin{verbatim}
?-  move(3,left,right,center). 
Move top disk from left to right 
Move top disk from left to center 
Move top disk from right to center 
Move top disk from left to right 
Move top disk from center to left 
Move top disk from center to right 
Move top disk from left to right 
true .
\end{verbatim}

Первое предложение программы описывает перемещение одного диска. Второе
предложение описывает как можно получить решение рекурсивно. Например,
декларативное чтение второго предложения для случая \verb|N=3, X=left, Y=right|,
и \verb|Z=center| приводит к следующему:

\begin{verbatim}
move(3,left,right,center) если 
    move(2,left,center,right) и ] * 
    move(1,left,right,center) и 
    move(2,center,right,left). ] ** 
\end{verbatim}

Это декларативное чтение очевидно правильно. Процедурное чтение тесно связано с
декларативной интерпретацией рекурсивного утверждения, оно должно выглядеть
как-то так:

\begin{verbatim}
удовлетворить цель ?-move(2,left,center,right), и потом 
удовлетворить цель ?-move(1,left,right,center), и потом 
удовлетворить цель ?-move(2,center,right,left). 
\end{verbatim}

Аналогично мы можем записать декларативное прочтение для случая \verb|N=2|:

\begin{verbatim}
move(2,left,center,right) если ] * 
move(1,left,right,center) и 
move(1,left,center,right) и 
move(1,right,center,left). 
move(2,center,right,left) если ] ** 
move(1,center,left,right) и
move(1,center,right,left) и 
move(1,left,right,center). 
\end{verbatim}

Теперь подставим содержимое последних двух implications и увидим решение которое
сгенерирует \prolog:

\begin{verbatim}
move(3,left,right,center) если 
move(1,left,right,center) и 
move(1,left,center,right) и * 
move(1,right,center,left) и 
--------------------------- 
move(1,left,right,center) и 
--------------------------- 
move(1,center,left,right) и 
move(1,center,right,left) и ** 
move(1,left,right,center). 
\end{verbatim}

Процедурное прочтение последних двух больших implication должно быть очевидно.
Этот пример показывает при основных операции \prolog а:
\begin{enumerate}[nosep]
  \item
Цели сопоставляются с головой правила, и
  \item 
тело правила (с соответствующе привязанными переменными) становится новой
последовательностью целей; процесс повторяется 
  \item 
пока не будет удовлетворена основная цель или условие, или не будет выполнено
простое действие, например выведен текст.
\end{enumerate}

\begin{framed}\noindent
Процесс сопоставления переменных с образцом (variable matching)\\
называется \termdef{унификацией}{унификация}.
\end{framed}

\paragraph{Упражнение \ref{fish23}.1} Нарисуйте clause-дерево для цели
\verb'move(3,left,right,center)', покажите что это консеквенция программы. Как
полученное дерево связано с процессом подстановки, поисанным выше\,?

\paragraph{Exercise \ref{fish23}.2} Попробуйте \prolog-цель
\verb|?-move(3,left,right,left)|. Что не так\,? Предложите способ
исправления, и проследите процесс работы исправления.
