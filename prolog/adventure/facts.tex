\secrel{Facts}\label{adv2}\secdown

This chapter describes the basic Prolog facts. They are the simplest form of
Prolog predicates, and are \emph{similar to records in a relational database}.
As we will see in the next chapter they can be queried like database records.

The syntax for a fact is
\begin{verbatim}
pred(arg1, arg2, ...  argN).
\end{verbatim}
where
\begin{description}[nosep]
\item[pred]\ \\The name of the predicate
\item[arg1,..]\ \\The arguments
\item[N]\ \\The arity
\item[.]\ \\The syntactic end of all Prolog clauses
\end{description}

\bigskip
A predicate \verb'pred/0' of arity 0 is simply
\begin{verbatim}
pred.
\end{verbatim}

The arguments can be any legal Prolog \term{term}. The basic Prolog terms are
\begin{description}[nosep]
\item[integer]\ \\A positive or negative number whose absolute value is less
than some implementation-specific power of 2
\item[atom]\ \\A text constant beginning with a lowercase letter
\item[variable]\ \\Begins with an uppercase letter or underscore (\_)
\item[structure]\ \\Complex terms, which will be covered in chapter \ref{adv9}
\end{description}

Various Prolog implementations enhance this basic list with other data types,
such as floating point numbers, or strings.

The Prolog character set is made up of
\begin{itemize}[nosep]
  \item 
Uppercase letters, \verb|A-Z|
  \item 
Lowercase letters, \verb|a-z|
  \item 
Digits, \verb|0-9|
  \item 
Symbols, \verb|+ - * / \ ^ , . ~ : . ? @ # $ &|
\end{itemize}
Integers are made from digits. Other numerical types\note{0x\ldots hex numbers
or floating point numbers} are allowed in some Prolog implementations.

Atoms are usually made from letters and digits with the first character being a
\emph{lowercase} letter, such as
\begin{verbatim}
hello
twoWordsTogether
x14
\end{verbatim}

For readability, the underscore (\_), \emph{but not the hyphen (-)}, can be used
as a separator in longer names. So the following are legal.
\begin{verbatim}
a_long_atom_name
z_23
\end{verbatim}

The following are not legal atoms.
\begin{verbatim}
no-embedded-hyphens
123nodigitsatbeginning
_nounderscorefirst
Nocapsfirst
\end{verbatim}

Use single quotes to make any character combination a legal atom as follows.
\begin{verbatim}
'this-hyphen-is-ok'
'UpperCase'
'embedded blanks'
\end{verbatim}

Do not use double quotes (\verb|""|) to build atoms. This is a special syntax
that causes the character string to be treated as a list of ASCII character
codes (string).

Atoms can also be legally made from symbols, as follows.
\begin{verbatim}
-->
++
\end{verbatim}

Variables are similar to atoms, but are distinguished by beginning with either
an \emph{uppercase} letter or the underscore (\_).
\begin{verbatim}
X
Input_List
_4th_argument
Z56
\end{verbatim}

Using these building blocks, we can start to code facts. The predicate name
follows the rules for atoms. The arguments can be any Prolog terms.

Facts are often used to store the data a program is using. For example, a
business application might have \verb'customer/3'.
\begin{verbatim}
customer('John Jones', boston, good_credit).
customer('Sally Smith', chicago, good_credit).
\end{verbatim}

The single quotes are needed around the names because they begin with uppercase
letters and because they have embedded blanks.

Another example is a windowing system that uses facts to store data about the
various windows. In this example the arguments give the window name and
coordinates of the upper left and lower right corners.
\begin{verbatim}
window(main, 2, 2, 20, 72).
window(errors, 15, 40, 20, 78).
\end{verbatim}

A medical diagnostic expert system might have \verb'disease/2'.
\begin{verbatim}
disease(plague, infectious).
\end{verbatim}

A Prolog listener provides the means for dynamically recording facts and rules
in the logicbase, as well as the means to \termdef{query}{\prolog!query}
(\termdef{call}{\prolog!call}) them. The logicbase is updated by
\verb'consult'ing or \verb'reconsult'ing program source. Predicates can also be
typed directly into the listener, but they are not saved between sessions.

\secrel{Nani Search}

We will now begin to develop \prog{Nani Search} by defining the basic facts that
are meaningful for the game. These include
\begin{itemize}[nosep]
  \item 
The rooms and their connections
  \item 
The things and their locations
  \item 
The properties of various things
  \item 
Where the player is at the beginning of the game
\end{itemize}
\fig{\\Figure 2.1. The rooms of \prog{Nani
Search}}{prolog/adventure/advfig21.png}{width=0.6\textwidth}

Open a new source file and save it as \file{myadven.pro}\note{or \file{nani.pl},
.pl is ProLog program extension}, or whatever name you feel is appropriate. You
will make your changes to the program in that source file. (A completed version
of \file{nanisrch.pro} is in the Prolog \dir{samples/}
directory,\\\file{samples/prolog/misc\_one\_file}.)

First we define the \emph{rooms} with the predicate \verb|room/1|, which has
five clauses, all of which are facts. They are based on the game map in figure
2.1.
\begin{verbatim}
room(kitchen).
room(office).
room(hall).
room('dining room').
room(cellar).
\end{verbatim}

We define the \emph{locations of things} with a two-argument predicate
\verb'location/2'. The \emph{first} argument will mean the \emph{thing} and the
\emph{second} will mean its \emph{location}. To begin with, we will add the
following things\note{is can be thinked as \term{relation}s in traditional
RDBMS}.
\begin{verbatim}
location(desk, office).
location(apple, kitchen).
location(flashlight, desk).
location('washing machine', cellar).
location(nani, 'washing machine').
location(broccoli, kitchen).
location(crackers, kitchen).
location(computer, office).
\end{verbatim}

The symbols we have chosen, such as \var{kitchen} and \var{desk} have meaning to
us, but none to Prolog. The relationship between the arguments should also
accurately reflect our meaning.

For example, the meaning we attach to \verb'location/2' is ``The first argument
is located in the second argument``. Fortunately Prolog considers
\verb'location(sink, kitchen)' and \verb'location(kitchen, sink)' to be
different. Therefore, as long as we are consistent in our use of arguments, we
can accurately represent our meaning and avoid the potentially ambiguous
interpretation of the kitchen being in the sink.

We are not as lucky when we try to represent the \emph{connections between
rooms}. Let's start, however, with \verb'door/2', which will contain facts such
as
\begin{verbatim}
door(office, hall).
\end{verbatim}

We would like this to mean ``There is a connection from the office to the hall,
\emph{or} from the hall to the office''.

Unfortunately, Prolog considers \verb'door(office, hall)' to be different from
\verb'door(hall, office)'. If we want to accurately represent a two-way
connection, we would have to define \verb'door/2' twice for each
connection\note{it is \term{symmetric relation}}.
\begin{verbatim}
door(office, hall).
door(hall, office).
\end{verbatim}
The strictness about order serves our purpose well for location, but it creates
this problem for connections between rooms. If the office is connected to the
hall, then we would like the reverse to be true as well.

For now, we will just add \emph{one-way doors} to the program; we will address
the symmetry problem again in the next chapter \ref{adv3} and resolve it in
chapter 5 \ref{adv5}.
\begin{verbatim}
door(office, hall).
door(kitchen, office).
door(hall, 'dining room').
door(kitchen, cellar).
door('dining room', kitchen).
\end{verbatim}

Here are some other facts about properties of things the game player might try
to eat.
\begin{verbatim}
edible(apple).
edible(crackers).

tastes_yucky(broccoli).
\end{verbatim}

Finally we define the initial status of the flashlight, and the player's
location at the beginning of the game.
\begin{verbatim}
turned_off(flashlight).
here(kitchen).
\end{verbatim}
We have now seen how to use basic facts to represent data in a Prolog program.

\secrel{Exercises}

During the course of completing the exercises you will develop three Prolog
applications in addition to \prog{Nani Search}. The exercises from each chapter
will build on the work of previous chapters. Suggested solutions to the
exercises are contained in the Prolog source files listed in the appendix, and
are also included in \file{samples/prolog/misc\_one\_file}. The files
\begin{description}
\item[\file{gene}.pl]\ \\A genealogical intelligent logicbase
\item[\file{custord}.pl]\ \\A customer order entry application
\item[\file{birds}.pl]\ \\An expert system that identifies birds
\end{description}

\secup