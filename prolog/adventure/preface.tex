\secly{Preface}

I was working for an aerospace company in the 1970s when someone got a copy of
the original \prog{Adventure} game and installed it on our mainframe computer.
For the next month my lunch hours, evenings and weekends, as well as normal work
hours, were consumed with fighting the fierce green dragon and escaping from the
twisty little passages. Finally, with a few hints about the plover's egg and
dynamite, I had proudly earned all the points in the game.

My elation turned to terror as I realized it was time for my performance review.
My boss was a stern man, who was more comfortable with machines than with
people. He opened up a large computer printout containing a log of the hours
each of his programmers spent on the mainframe computer. He said he noticed that
recently I had been working evenings and weekends and that he admired that type
of dedication in his employees. He gave me the maximum raise and told me to keep
up the good work.

Ever since I've had a warm spot in my heart for adventure games. Years later,
when I got my first home computer, I immediately started to write my own
adventure game in \ci. First came the tools, a simple dynamic database to keep
track of the game state and pattern matching functions to search that database.
Then came a natural language parser for the front end. Functions implemented the
various rules of the game.

At around the same time I joined the Boston Computer Society and attended a
lecture of the newly formed Artificial Intelligence group. The lecture was about
\prolog. I was amazed\ --- here was a language that included all of the tools
needed for building adventure games and more.

It had a much richer dynamic database and more powerful pattern matcher than the
one I had written, plus its syntax was rules, which are much more natural for
coding the specification of the game. It had a built-in search engine and, to
top it all off, had tools for natural language processing.

I learned \prolog\ from the classic Clocksin and Mellish \cite{clocksin} text
and started writing adventure games anew.

I went on to use \prolog\ for a number of expert system applications at my then
current job, including a mainframe database performance tuning system and
installation expert. This got others interested in the language and I began
teaching it as well.

While the applications we were using Prolog for were serious and performed a key
role in improving technical support for the growing company, I still found the
adventure game to be an excellent showcase for teaching the language.

This book is the result of that work. It takes a pragmatic, rather than
theoretical, approach to the language and is designed for programmers interested
in adding this powerful language to their bag of tools.

I offer my thanks to Will Crowther and Don Woods for writing the first (and in
my opinion still the best) adventure game and to the Boston Computer Society for
testing the ideas in the book. Thanks also to Ray Reeves, who speaks fluent
\prolog, and Nancy Wilson, who speaks fluent English, for their careful reading
of the text.

\bigskip\noindent\copyright\ Dennis Merritt\\Stow, Massachusetts, April 1996
