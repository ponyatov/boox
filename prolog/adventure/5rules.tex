\secrel{Rules}\label{adv5}\secdown

We said earlier a predicate is defined by clauses, which may be facts or rules. A rule is no more than a stored query. Its syntax is
\begin{verbatim}
head :- body.
\end{verbatim}
where
\begin{description}
\item[head]\ \\
a predicate definition (just like a fact)
\item[:-]\ \\
the neck symbol, sometimes read as "if"
\item[body]\ \\
one or more goals (a query)
\end{description}

For example, the compound query that finds out where the good things to eat are
can be stored as a rule with the predicate name \verb'where_food/2'.
\begin{verbatim}
where_food(X,Y) :-  
  location(X,Y),
  edible(X).
\end{verbatim}
It states ``There is something X to eat in room Y if X is located in Y, and X
is edible''.

We can now use the new rule directly in a query to find things to eat in a room.
As before, the semicolon (;) after an answer is used to find all the answers.
\begin{verbatim}
?- where_food(X, kitchen).
X = apple ;
X = crackers ;
no
\end{verbatim}
\begin{verbatim}
?- where_food(Thing, 'dining room').
no
\end{verbatim}
Or it can check on specific things.
\begin{verbatim}
?- where_food(apple, kitchen).
yes
\end{verbatim}
Or it can tell us everything.
\begin{verbatim}
?- where_food(Thing, Room).
Thing = apple
Room = kitchen ;
\end{verbatim}
\begin{verbatim}
Thing = crackers
Room = kitchen ;
no
\end{verbatim}

Just as we had multiple facts defining a predicate, we can have multiple rules
for a predicate. For example, we might want to have the broccoli included in
where_food/2. (Prolog doesn't have an opinion on whether or not broccoli is
legitimate food. It just matches patterns.) To do this we add another
where_food/2 clause for things that 'taste_yucky.'
\begin{verbatim}
where_food(X,Y) :-
  location(X,Y),
  edible(X).
where_food(X,Y) :-
  location(X,Y),
  tastes_yucky(X).
\end{verbatim}
  
\secup