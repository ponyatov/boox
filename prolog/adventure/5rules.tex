\secrel{Rules}\label{adv5}\secdown

We said earlier a predicate is defined by clauses, which may be facts or rules. A rule is no more than a stored query. Its syntax is
\begin{verbatim}
head :- body.
\end{verbatim}
where
\begin{description}
\item[head]\ \\
a predicate definition (just like a fact)
\item[:-]\ \\
the neck symbol, sometimes read as "if"
\item[body]\ \\
one or more goals (a query)
\end{description}

For example, the compound query that finds out where the good things to eat are
can be stored as a rule with the predicate name \verb'where_food/2'.
\begin{verbatim}
where_food(X,Y) :-  
  location(X,Y),
  edible(X).
\end{verbatim}
It states ``There is something X to eat in room Y if X is located in Y, and X
is edible''.

\secup