\secrel{Arithmetic}\secdown

Prolog must be able to handle arithmetic in order to be a useful general purpose
programming language. However, arithmetic does not fit nicely into the logical
scheme of things.

That is, the concept of evaluating an arithmetic expression is in contrast to
the straight pattern matching we have seen so far. For this reason, Prolog
provides the built-in predicate \verb'is' that evaluates arithmetic expressions.
Its syntax calls for the use of operators, which will be described in more
detail in chapter 12.

\begin{verbatim}
X is <arithmetic expression>
\end{verbatim}
The variable \var{X} is set to the value of the arithmetic expression. On
backtracking it is unassigned.

The arithmetic expression looks like an arithmetic expression in any other
programming language.

Here is how to use Prolog as a calculator.

\begin{verbatim}
?- X is 2 + 2.
X = 4
\end{verbatim}

\begin{verbatim}
?- X is 3 * 4 + 2.
X = 14
\end{verbatim}
Parentheses clarify precedence.

\begin{verbatim}
?- X is 3 * (4 + 2).
X = 18
\end{verbatim}

\begin{verbatim}
?- X is (8 / 4) / 2.
X = 1
\end{verbatim}
In addition to \verb'is', Prolog provides a number of operators that compare two
numbers. These include \verb'greater than', \verb'less than',
\verb'greater or equal than', and \verb'less or equal than'. They behave more
logically, and succeed or fail according to whether the comparison is true or
false. Notice the order of the symbols in the greater or equal than and less
than or equal operators. They are specifically constructed not to look like an
arrow, so that you can use arrow symbols in your programs without confusion.

\begin{verbatim}
X > Y
X < Y
X >= Y
X =< Y
\end{verbatim}
Here are a few examples of their use.

\begin{verbatim}
?- 4 > 3.
yes
\end{verbatim}

\begin{verbatim}
?- 4 < 3.
no
\end{verbatim}

\begin{verbatim}
?- X is 2 + 2, X > 3.
X = 4
\end{verbatim}

\begin{verbatim}
?- X is 2 + 2, 3 >= X.
no
\end{verbatim}

\begin{verbatim}
?- 3+4 > 3*2.
yes
\end{verbatim}
They can be used in rules as well. Here are two example predicates. One converts
centigrade temperatures to Fahrenheit, the other checks if a temperature is
below freezing.

\begin{verbatim}
c_to_f(C,F) :-
  F is C * 9 / 5 + 32.

freezing(F) :-
  F =< 32.
\end{verbatim}
Here are some examples of their use.

\begin{verbatim}
?- c_to_f(100,X).
X = 212
yes

?- freezing(15).
yes

?- freezing(45).
no
\end{verbatim}

\secrel{Exercises}\secdown

\secrel{Customer Order Entry}

1- Write a predicate \verb'valid_order/3' that checks whether a customer order
is valid. The arguments should be customer, item, and quantity. The predicate
should succeed only if the customer is a valid customer with a good credit
rating, the item is in stock, and the quantity ordered is less than the quantity
in stock.

2- Write a \verb'reorder/1' predicate which checks inventory levels in the
inventory record against the reorder quantity in the item record. It should
write a message indicating whether or not it's time to reorder.

\secup
\secup