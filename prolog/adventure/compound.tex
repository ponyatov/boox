\secrel{Compound Queries}\label{adv4}\secdown

Simple goals can be combined to form compound queries. For example, we might
want to know if there is anything good to eat in the kitchen. In Prolog we might
ask
\begin{verbatim}
?- location(X, kitchen), edible(X).
\end{verbatim}
Whereas a simple query had a single goal, the compound query has a
\emph{conjunction} of goals. The comma separating the goals is read as
\verb|and|.

Logically (declaratively) the example means ``Is there an \var{X} such that
\var{X} is located in the \verb|kitchen| and \var{X} is \verb|edible|\,?'' If
the \emph{same variable} name appears \emph{more than once} in a query,
\emph{it must have the same value in all places} it appears. The query in the
above example will only succeed if there is a single value of \var{X} that can
satisfy both goals.

However, the variable name has no significance to any other query, or clause in
the logicbase. If \var{X} appears in other queries or clauses, that query or
clause gets its own copy of the variable. We say the \emph{scope of a logical
variable is a query}.

Trying the sample query we get
\begin{verbatim}
?- location(X, kitchen), edible(X).
X = apple ;
X = crackers ;
no
\end{verbatim}
The \verb'broccoli' does not show up as an answer because we did not include it
in the \verb|edible/1| predicate.

This logical query can also be interpreted procedurally, using an understanding
of Prolog's execution strategy. The procedural interpretation is: ``First find
an \var{X} located in the \verb|kitchen|, and then test to see if it is
\verb|edible|. If it is not, go back and find another \var{X} in the
\verb|kitchen| and test it. Repeat until successful, or until there are no more
\var{X}s in the \verb|kitchen|''.

To understand the execution of a compound query, think of the goals as being
arranged from left to right. Also think of a separate table which is kept for
the current variable bindings. The flow of control moves back and forth through
the goals as Prolog attempts to find variable bindings that satisfy the query.

Each goal can be entered from either the left or the right, and can be left from
either the left or the right. These are the ports of the goal as seen in the
last chapter.

A compound query begins by calling the first goal on the left. If it succeeds,
the next goal is called with the variable bindings as set from the previous
goal. If the query finishes via the exit port of the rightmost goal, it
succeeds, and the listener prints the values in the variable table.

If the user types semicolon \keys{;} after an answer, the query is re-entered at
the redo port of the rightmost goal. Only the variable bindings that were set in
that goal are undone.

If the query finishes via the fail port of the leftmost goal, the query fails.
Figure 4.1 shows a compound query with the listener interaction on the ending
ports.

\fig{Figure 4.1. Compound
queries}{prolog/adventure/fig41.png}{width=0.6\textwidth}

\secdown
\secrel{Annotated trace of compound query}

Next log contains the annotated trace of the sample query. Make sure you
understand it before proceeding.

\bigskip

\begin{verbatim}
?- location(X, kitchen), edible(X).
\end{verbatim}
The trace has a new feature, which is a number in the first column that
indicates the goal being worked on.

First the goal \verb|location(X, kitchen)| is called, and the trace
indicates that pattern matches the second clause of location.

\begin{verbatim}
1 CALL location(X, kitchen)
\end{verbatim}

It succeeds, and results in the binding of \var{X} to \verb|apple|.

\begin{verbatim}
1 EXIT (2)location(apple, kitchen)
\end{verbatim}

Next, the second goal \verb|edible(X)| is called. However, \var{X} is now bound
to apple, so it is called as \verb|edible(apple)|.

\begin{verbatim}
2 CALL edible(apple)
\end{verbatim}

It succeeds on the first clause of \verb|edible/1|, thus exiting the query
successfully.

\begin{verbatim}
2 EXIT (1) edible(apple)
    X = apple ;
\end{verbatim}

Entering semicolon (;) causes the listener to backtrack into the rightmost goal
of the query.

\begin{verbatim}
2 REDO edible(apple)
\end{verbatim}

There are no other clauses that match this pattern, so it fails.

\begin{verbatim}
2 FAIL edible(apple)
\end{verbatim}

Leaving the fail port of the second goal causes the listener to enter the redo
port of the first goal. In so doing, the variable binding that was established
by that goal is undone, leaving \var{X} unbound.

\begin{verbatim}
1 REDO location(X, kitchen)
\end{verbatim}

It now succeeds at the sixth clause, rebinding \var{X} to \verb|broccoli|.

\begin{verbatim}
1 EXIT (6) location(broccoli, kitchen)
\end{verbatim}

The second goal is called again with the new variable binding. This is a fresh
call, just as the first one was, and causes the search for a match to begin at
the first clause

\begin{verbatim}
2 CALL edible(broccoli)
\end{verbatim}

There is no clause for \verb|edible(broccoli)|, so it fails.

\begin{verbatim}
2 FAIL edible(broccoli)
\end{verbatim}

The first goal is then re-entered at the redo port, undoing the variable
binding.

\begin{verbatim}
1 REDO location(X, kitchen)
\end{verbatim}

It succeeds with a new variable binding.

\begin{verbatim}
1 EXIT (7) location(crackers, kitchen)
\end{verbatim}

This leads to the second solution to the query.

\begin{verbatim}
2 CALL edible(crackers)
2 EXIT (2) edible(crackers)
    X = crackers ;
\end{verbatim}

Typing semicolon (;) initiates backtracking again, which fails through both
goals and leads to the ultimate failure of the query.

\begin{verbatim}
2 REDO edible(crackers)
2 FAIL edible(crackers)
1 REDO location(X, kitchen)
1 FAIL location(X, kitchen)
     no
\end{verbatim}

\secup

In this example we had a single variable, which was bound (given a value) by the
first goal and tested in the second goal. We will now look at a more general
example with two variables. It is attempting to ask for all the things located
in rooms adjacent to the kitchen.

In logical terms, the query says ``Find a \var{T} and \var{R} such that there is
a door from the kitchen to \var{R} and \var{T} is located in \var{R}''. In
procedural terms it says ``First find an R with a door from the kitchen to R.
Use that value of R to look for a T located in R``.
\begin{verbatim}
?- door(kitchen, R), location(T,R).
R = office
T = desk ;

R = office
T = computer ;

R = cellar
T = 'washing machine' ;
no
\end{verbatim}

In this query, the backtracking is more complex. Figure 4.3 shows its trace.

\lstx{Figure 4.3. Trace of a compound query}{prolog/adventure/fig43.log}

Notice that the variable \var{R} is bound by the first goal and \var{T} is bound
by the second. Likewise, the two variables are unbound by entering the redo port
of the goal that bound them. After \var{R} is first bound to office, that
binding sticks during backtracking through the second goal. Only when the
listener backtracks into the first goal does R get unbound.

\secrel{Built-in Predicates}

Up to this point we have been satisfied with the format Prolog uses to give us
answers. We will now see how to generate output that is customized to our needs.
The example will be a query that lists all of the items in the kitchen. This
will require performing I/O and forcing the listener to automatically backtrack
to find all solutions.

To do this, we need to understand the concept of the \term{built-in} (evaluable)
predicate. A built-in predicate is predefined by Prolog. There are no clauses in
the logicbase for built-in predicates. When the listener encounters a goal that
matches a built-in predicate, it calls a predefined procedure.

Built-in predicates are usually written in the language used to implement the
listener. They can perform functions that have nothing to do with logical
theorem proving, such as writing to the console. For this reason they are
sometimes called extra-logical predicates.

However, since they appear as Prolog goals they must be able to respond to
either a call from the left or a redo from the right. Its response in the redo
case is referred to as its behavior on backtracking.

We will introduce specific built-in predicates as we need them. Here are the I/O
predicates that will let us control the output of our query.
\begin{description}
\item[write/1]\ \\This predicate always succeeds when called, and has the side
effect of writing its argument to the console.It always fails on backtracking.
Backtracking does not undo the side effect.
\item[nl/0]\ \\Succeeds, and starts a new line. Like write, it always succeeds
when called, and fails on backtracking.
\item[tab/1]\ \\It expects the argument to be an integer and tabs that number of
spaces. It succeeds when called and fails on backtracking.
\end{description}

Figure 4.4 is a stylized picture of a goal showing its internal control
structure. We will compare this with the internal flow of control of various
built-in predicates.

\fig{\\Figure 4.4. Internal flow of control through a normal
goal}{prolog/adventure/fig44.png}{width=0.4\textwidth}

In figure 4.4, the upper left diamond represents the decision point after a
call. Starting with the first clause of a predicate, unification is attempted
between the query pattern and each clause, until either unification succeeds or
there are no more clauses to try. If unification succeeded, branch to exit,
marking the clause that successfully unified, if it failed, branch to fail.

The lower right diamond represents the decision point after a redo. Starting
with the most recent clause found in the predicate, unification is again
attempted between the query pattern and remaining clauses. If it succeeds,
branch to exit, if not, branch to fail.

The I/O built-in predicates differ from normal goals in that they never change
the direction of the flow of control. If they get control from the left, they
pass control to the right. If they get control from the right, they pass control
to the left as shown in figure 4.5.

\fig{\\Figure 4.5. Internal flow of control through an
I/O predicate}{prolog/adventure/fig45.png}{width=0.4\textwidth}

The output I/O predicates do not affect the variable table; however, they may
output values from it. They simply leave their mark at the console each time
control passes through them from left to right.

There are built-in predicates that do affect backtracking, and we have need of
one of them for the first example. It is \verb|fail/0|, and, as its name
implies, it always fails.

If \verb|fail/0| gets control from the left, it immediately passes control back
to the redo port of the goal on the left. It will never get control from the
right, since it never allows control to pass to its right. Figure 4.6 shows its
internal control structure.

\fig{\\Figure 4.6. Internal flow of control through the \var{fail/0}
predicate}{prolog/adventure/fig46.png}{width=0.3\textwidth}

Previously we relied on the listener to display variable bindings for us, and
used the semicolon (;) response to generate all of the possible solutions. We
can now use the I/O built-in predicates to display the variable bindings, and
the \verb|fail/0| predicate to force backtracking so all solutions are
displayed.

Here then is the query that lists everything in the kitchen.
\begin{verbatim}
?- location(X, kitchen), write(X) ,nl, fail.
apple
broccoli
crackers
no
\end{verbatim}

The final \verb'no' means the query failed, as it was destined to, due to the
\verb|fail/0|.

Figure 4.7 shows the control flow through this query.

\fig{\\Figure 4.7. Flow of control through query with built-in 
predicates}{prolog/adventure/fig47.png}{width=0.3\textwidth}



     
\secup