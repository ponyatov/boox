\secrel{Simple Queries}\label{adv3}\secdown

Now that we have some facts in our Prolog program, we can
\termdef{consult}{\prolog!consult} the program in the listener and
\termdef{query}{\prolog!query}, or \termdef{call}{\prolog!call}, the facts. This
chapter, and the next \ref{adv4}, will assume the Prolog program contains only
facts. Queries against programs with rules will be covered in a later chapter
\ref{5}.

Prolog queries work by pattern matching. The query pattern is called a
\termdef{goal}{\prolog!goal}. If there is a fact that matches the \term{goal},
then the \term{query} \emph{succeeds} and the listener responds with
\verb'yes.'\note{or \var{true.}} If there is \emph{no matching} fact, then the
query \emph{fails} and the listener responds with \verb'no.'\note{or
\var{false.}}

Prolog's \term{pattern matching} is called
\termdef{unification}{\prolog!unification}. In the case where the logicbase
contains only facts, unification succeeds if the following three conditions hold
simultaneously.
\begin{itemize}[nosep]
  \item 
The predicate named in the goal and logicbase are the same.
  \item 
Both predicates have the same arity.
  \item 
All of the arguments are the same.
\end{itemize}

Before proceeding, review figure 3.1, which has a listing of the program so far.

\lst{Figure 3.1. The listing of \prog{Nani Search} entered at this
point}{prolog/adventure/fig31.pl}{prolog}

\secup