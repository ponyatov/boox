\secrel{Getting Started}\secdown

Prolog stands for PROgramming in LOGic. It was developed from a foundation of
logical theorem proving and originally used for research in natural language
processing. Although its popularity has sprung up mainly in the artificial
intelligence (AI) community, where it has been used for applications such as
expert systems, natural language, and intelligent databases, it is also useful
for more conventional types of applications. It allows for more rapid
development and prototyping than most languages because it is semantically close
to the logical specification of a program. As such, it approaches the ideal of
executable program specifications\note{and fast prototyping and RAD}.

Programming in Prolog is significantly different from conventional procedural
programming and requires a readjustment in the way one thinks about programming.
Logical relationships are asserted, and Prolog is used to determine whether or
not certain statements are true, and if true, what variable bindings make them
true. This leads to a very declarative style of programming.

In fact, the term program does not accurately describe a Prolog collection of
executable facts, rules and logical relationships, so you will often see term
\term{logicbase} used in this book as well.

While Prolog is a fascinating language from a purely theoretical viewpoint, this
book will stress Prolog as a practical tool for application development.

Much of the book will be built around the writing of a short adventure game. The
adventure game is a good example since it contains mundane programming
constructs, symbolic reasoning, natural language, data, and logic.

Through exercises you will also build a simple expert system, an intelligent
genealogical logicbase, and a mundane customer order entry application.

You should create a source file for the game, and enter the examples from the
book as you go. You should also create source files for the other three programs
covered in the exercises. Sample source code for each of the programs is
included in the appendix \ref{advapp}.

The adventure game is called Nani Search. Your persona as the adventurer is that
of a three year old girl. The lost treasure with magical powers is your nani
(security blanket). The terrifying obstacle between you and success is a dark
room. It is getting late and you're tired, but you can't go to sleep without
your nani. Your mission is to find the nani.

\fig{\ This is Nani}{prolog/adventure/nani.jpg}{width=0.3\textwidth}

Nani Search is composed of
\begin{itemize}[nosep]
  \item 
A read and execute command loop
  \item 
A natural language input parser
  \item 
Dynamic facts/data describing the current environment
  \item 
Commands that manipulate the environment
  \item 
Puzzles that must be solved
\end{itemize}

You control the game by using simple English commands (at the angle bracket (>)
prompt) expressing the action you wish to take. You can go to other rooms, look
at your surroundings, look in things, take things, drop things, eat things,
inventory the things you have, and turn things on and off.

\paragraph{Figure 1.1. A sample run of Nani Search}
\begin{verbatim}
You are in the kitchen.
You can see: apple, table, broccoli
You can go to: cellar, office, dining room

> go to the cellar

You can't go to the cellar because it's dark in the cellar,
and you're afraid of the dark.

> turn on the light

You can't reach the switch and there's nothing to stand on.

> go to the office

You are in the office.
You can see the following things: desk
You can go to the following rooms: hall, kitchen

> open desk

The desk contains:
  flashlight
  crackers

> take the flashlight

You now have the flashlight

> kitchen

You are in the kitchen

> turn on the light

flashlight turned on.
...
\end{verbatim}

Figure 1.1 shows a run of a completed version of Nani Search. As you develop
your own version you can of course change the game to reflect your own ideas of
adventure.

The game will be implemented from the bottom up, because that fits better with
the order in which the topics will be introduced. Prolog is equally adept at
supporting top-down or inside-out program development.

A Prolog logicbase exists in the listener's workspace as a collection of small
modular units, called \term{predicates}. They are similar to subroutines in
conventional languages, but on a smaller scale.

The predicates can be added and tested separately in a Prolog program, which
makes it possible to incrementally develop the applications described in the
book. Each chapter will call for the addition of more and more predicates to the
game. Similarly, the exercises will ask you to add predicates to each of the
other applications.

We will start with the Nani Search logicbase and quickly move into the commands
that examine that logicbase. Then we will implement the commands that manipulate
the logicbase.

Along the way there will be diversions where the same commands are rewritten
using a different approach for comparison. Occasionally a topic will be covered
that is critical to Prolog but has little application in Nani Search.

One of the final tasks will be putting together the top-level command processor.
We will finish with the natural language interface.

The goal of this book is to make you feel comfortable with
\begin{itemize}[nosep]
  \item 
The Prolog logicbase of facts and rules
  \item 
The built-in theorem prover that allows Prolog to answer questions about the
logicbase (backtracking search)
  \item 
How logical variables are used (They are different from the variables in most
languages.)
  \item 
Unification, the built in pattern matcher
  \item 
Extra-logical features (like read and write that make the language practical)
  \item 
How to control Prolog's execution behavior
\end{itemize}

\secrel{Jumping In}

As with any language, the best way to learn Prolog is to use it. This book is
designed to be used with a Prolog listener, and will guide you through the
building of four applications.
\begin{enumerate}[nosep]
  \item 
Adventure game
  \item 
Intelligent genealogical logicbase
  \item 
Expert system
  \item 
Customer order entry business application
\end{enumerate}
The adventure game will be covered in detail in the main body of the text, and
the others you will build yourself based on the exercises at the end of each
chapter.

There will be two types of example code throughout the book. One is code, meant
to be entered in a source file, and the other is interactions with the listener.
The listener interactions are distinguished by the presence of the question mark
and dash (?-) listener prompt.

Here is a two-line program, meant to help you learn the mechanics of the editor
and your listener.
\begin{verbatim}
mortal(X) :- person(X).
person(socrates).
\end{verbatim}

In the \prog{Amzi! Eclipse IDE}, first create a project for your source files.
Select \menu{File>New>Project} on the main menu, then click on \menu{Prolog} and
\menu{Project}, and enter the name of your project, \prog{adventure}.
Next, create a new source file. Select \menu{File>New>File}, and enter the name
of your file, \file{mortal.pro}. Enter the program in the edit window, paying
careful attention to upper and lowercase letters and punctuation. Then select
\menu{File>Save} from the menu.

Next, start the Prolog listener by selecting \menu{Run>Run As>Interpreted
Project}. Loading the source code in the \prog{Listener} is called
\term{consulting}. You should see a message indicating that your source file,
\file{mortal.pro}, was consulted. This message is followed by the typical
listener prompt.
\begin{verbatim}
?-
\end{verbatim}
\emph{Entering the source code} in the Listener is called \term{consulting}.
Select \menu{Listener>Consult} from the main menu, and select \file{mortal.pro}
from the file menu. You can also consult a Prolog source file directly from the
listener prompt like this.
\begin{verbatim}
?- consult(mortal).
yes
\end{verbatim}

See the documentation and/or online help for details on the Amzi! listener and
Eclipse IDE.

In all the listener examples in this book, you enter the text after the prompt
(?), the rest is provided by Prolog. When working with Prolog, it is important
to remember to include the final period \keys{.} and to press the \keys{return}
key. If you forget the period (and you probably will), you can enter it on the
next line with a \keys{.}\keys{return}.

Once you've loaded the program, try the following Prolog queries.
\begin{verbatim}
?- mortal(socrates).
yes
?- mortal(X).
X = socrates.
\end{verbatim}

Now let's change the program. First type \verb'quit.' to end the listener. Go
back to the edit window and add the line
\begin{verbatim}
person(plato).
\end{verbatim}
after the \verb|person(socrates)| line.

Select \menu{Run>Run As>Interpreted Project} to start the listener again with
your updated source file. And test it.

\begin{verbatim}
?- mortal(plato).
yes
\end{verbatim}

One more test. Enter this query in the listener.

\begin{verbatim}
?- write('Hello World').
Hello World
yes
\end{verbatim}

You are now ready to learn Prolog.

\secrel{Logic Programming}

Let's look at the simple example in more detail. In classical logic we might say
``All people are mortal'', or, rephrased for Prolog, ``For all X, X is mortal if
X is a person''.

\begin{verbatim}
mortal(X) :- person(X).
\end{verbatim}
Similarly, we can assert the simple fact that Socrates is a person.
\begin{verbatim}
person(socrates).
\end{verbatim}
From these two logical assertions, Prolog can now prove whether or not Socrates is mortal.
\begin{verbatim}
?- mortal(socrates).
\end{verbatim}
The listener responds
\begin{verbatim}
yes
\end{verbatim}
We could also ask "Who is mortal?" like this
\begin{verbatim}
?- mortal(X).
\end{verbatim}
and receive the response
\begin{verbatim}
X = socrates
\end{verbatim}

This declarative style of programming is one of Prolog's major strengths. It
leads to code that is easier to write and easier to maintain. For the most part,
the programmer is freed from having to worry about control structures and
transfer of control mechanisms. This is done automatically by Prolog.

By itself, however, a logical theorem prover is not a practical programming
tool. A programmer needs to do things that have nothing to do with logic, such
as read and write terms. A programmer also needs to manipulate the built-in
control structure of Prolog in order for the program to execute as desired.

The following example illustrates a Prolog program that prints a report of all
the known mortals. It is a mixture of pure logic from before, extra-logical I/O,
and forced control of the Prolog execution behavior. The example is illustrative
only, and the concepts involved will be explained in later chapters.

First add some more philosophers to the \verb'mortal' source in order to make
the report more interesting. Place them after \verb'person(plato).'
\begin{verbatim}
person(zeno).
person(aristotle).
\end{verbatim}

Next add the report-writing code, again being careful with punctuation and
upper- and lowercase. Note that the format of this program is the same as that
used for the logical assertions.
\begin{verbatim}
mortal_report:-  
  write('Known mortals are:'),nl,
  mortal(X),
  write(X),nl,
  fail.
\end{verbatim}

\lst{Figure 1.2. Sample program}{prolog/adventure/plato.pl}{prolog}

Listing 1.2 contains the full program, with some optional comments, indicated by
the percent sign (\%) at the beginning of a line. Load the program in the
listener and try it. Note that the syntax of calling the report code is the same
as the syntax used for posing the purely logical queries.

\begin{verbatim}
?- mortal_report.
Known mortals are:
socrates
plato
zeno
aristotle
false.
\end{verbatim}

The final \verb'no' or \verb'false' is from Prolog, and will be explained later.

You should now be able to create and edit source files for Prolog, and be able
to load and use them from a Prolog listener.

You have had your first glimpse of Prolog and should understand that it is
fundamentally different from most languages, but can be used to accomplish the
same goals and more.

\secrel{Jargon}

With any field of knowledge, the critical concepts of the field are embedded in
the definitions of its technical terms. Prolog is no exception. When you
understand terms such as \term{predicate}, \term{clause}, \term{backtracking},
and \term{unification} you will have a good grasp of Prolog. This section
defines the terms used to describe Prolog programs, such as predicate and
clause. Execution-related terms, such as backtracking and unification will be
introduced as needed throughout the rest of the text.

\emph{Prolog jargon is a mixture of programming terms}, database terms, and
logic terms. You have probably heard most of the terms before, \emph{but} in
Prolog \emph{they don't necessarily mean what you think they mean}.

In Prolog the normally clear distinction between data and procedure becomes
blurred. This is evident in the vocabulary of Prolog. Almost every concept in
Prolog can be referred to by synonymous terms. One of the terms has a procedural
flavor, and the other a data flavor.

We can illustrate this at the highest level. A Prolog \term{program} is a Prolog
\term{logicbase}. As we introduce the vocabulary of Prolog, synonyms (from
Prolog or other computer science areas) for a term will follow in parentheses.
For example, at the highest level we have a Prolog
\termdef{program}{\prolog!program} (\term{logicbase}).

The Prolog program is composed of \termdef{predicates}{predicate}
(\termdef{procedures}{\prolog!procedure}, \termdef{record
types}{\prolog!record}, \termdef{\textbf{relations}}{relation}). Each is defined
by its name and a number called \termdef{arity}{\prolog!arity}. The arity is the
fixed number of \term{arguments} (\term{attributes}, \term{fields}) the
predicate has. Two predicates with the same name and different arity are
considered to be \emph{different} predicates.

n our sample program we saw three examples of predicates. Each of these three
predicates has a distinctly different flavor.
\begin{description}
\item[person/1]\ \\
looks like multiple \emph{data record}s with one data field for each.
\item[mortal\_report/0]\ \\
looks like a \emph{procedure} with no arguments.
\item[mortal/1]\ \\
a logical assertion or \emph{rule} that is somewhere in between data and
procedure.
\end{description}

Each predicate in a program is defined by the existence of one or more
\termdef{clauses}{clause} in the logicbase. In the example program, the
predicate \verb'person/1' has four clauses. The other predicates have only one
clause.

A clause can be either a \termdef{fact}{\prolog!fact} or a
\termdef{rule}{\prolog!rule}. The three clauses of the \verb'person/1' predicate
are all \emph{facts}. The single clauses of \verb|mortal_report/0| and
\verb|mortal/1| are both \emph{rules}.

\secup
