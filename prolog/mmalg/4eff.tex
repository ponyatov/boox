\secrel{ EFFICIENT MULTIEQUATION SELECTION }\label{mmalg4}

In this section we show how to implement efficiently the operation of selecting a
multiequation "on top" of the partial ordering in step (1.1) of Algorithm 3. 

The idea is to associate with every multiequation a counter which contains the
number of other occurrences in U of the variables in its left-hand side. This
counter is initialized by scanning the whole U part at the beginning. Of course, a
multiequation whose counter is set to zero is on top of the partial ordering. 

For instance, let us again consider system (3):
\begin{verbatim}
U: {[0] {x} = (f(xl,g(x2, x3), x2, b), f(g(h(a, x~), xe), xl,
h(a, x4), x4)),
[2] {xl} = 6,
[3] {x2} = 6,
[1] (x3} = 6,
[2] {x4} = 6,
[1] {xa} -- 6};
T:(). 
\end{verbatim}

Here square brackets enclose the counters associated with each multiequation.
Since only the first multiequation has its counter set to zero, it is selected to be
transferred. Counters of the other multiequations are easily updated by 
decrementing them whenever an occurrence of the corresponding variable appears in
the left-hand side of a multiequation in the frontier computed in step (1.2.1).
When two or more multiequations in U are merged in the compactification phase,
the counter associated with the new multiequation is obviously set to a value
which is the sum of the contents of the old counters. 

The next steps are as follows:
\begin{verbatim}
U: {[0] (Xl} = (g(h(a, x~), x2), g(x2, x3)),
[2] {x2} = (h(a, x4)),
[1] (x~) = o,
[1] {x,} = (b),
[1] {x~} = ~};
T: ( (x} = (f(xl, x~, x2, x4))).
U: {[0] {x2, x3} = (h(a, x4), h(a, x~)),
[1] {x4} = (b),
[1] {x~} = o};
T: ({x} = (f(x,, x,, x2, x,)),
{x,} = (g(x2, x3))).
U: {[0] {x4, xs} = (b)};
T: ({x} = (f(x,, xl, x2, x4)),
{x,} = (g(x2, x3)),
{x2, x3} = (h(a, x4))).
U: ~;
T: ({x} = (f(xl, xl, x2, x4)),
{xl} = (g(x2, x3)),
{x2, x3} = (h(a, x4)),
{x4, x~} = (b)). 
\end{verbatim}
