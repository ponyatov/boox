\secrel{Solving Systems of Multiequations}

For convenience, in what follows, we want to give a structure to a set of
multiequations. Thus we introduce the concept of system of multiequations. A
system R is a pair (T, U), where T is a sequence and U is a set of multiequations
(either possibly empty), such that

(1) the sets of variables which constitute the left-hand sides of all multiequations
in both T and U contain all variables and are disjoint;

(2) the right-hand sides of all multiequations in T consist of no more than one
term; and

(3) all variables belonging to the left-hand side of some multiequation in T can
only occur in the right-hand side of any preceding multiequation in T. 

We now present an algorithm for solving a given system R of multiequations.
When the computation starts, the T part is empty, and every step of the following
Algorithm 2 consists of "transferring" a multiequation from the U part, that is,
the unsolved part, to the T part, that is, the triangular or solved part of R. When
the Upart of R is empty, the system is essentially solved. In fact, the solution can
be obtained by substituting the variables backward. Notice that, by keeping a
solved system in this triangular form, we can hope to find efficient algorithms for
unification even when the mgu has a size which is exponential with respect to the
size of the initial system. For instance, the mgu of the set of multiequations
\begin{verbatim} 
{{Xl} = ~,
{x~} = ~,
{x3} = 0,
{x4} = (h(x3, h(x2, x2)), h(h(h (xl, xl), x2), x3))} 
\end{verbatim}
is
\begin{verbatim}
{(h(xl, Xl), x2), (h(h(xl, Xl), h(Xl, Xl)), x3),
(h(h(h(Xl, Xl), h(xl, Xl)), h(h(xl, Xl), h(Xl, Xl))), X4)}. 
\end{verbatim} 
However, we can give an equivalent solved system with empty U part and whose
T part is
\begin{verbatim}
({x,} --- (h(x3, x3)),
{x3} = (h(x2, x2)),
{X2) = (h(Xl, xl)),
{xl} = o), 
\end{verbatim}
from which the mgu can be obtained by substituting backward. 

Given a system R = (T, U) with an empty T part, an equivalent system with
an empty U part can be computed with the following algorithm. 

\paragraph{Algorithm 2}\ \\
\begin{enumerate}
  \item 
(1) repeat
\begin{enumerate}
  \item 
(1.1) Select a multiequation S = M of U with M \# ~5.
  \item 
(1.2) Compute the common part C and the frontier F ofM. IfM has no common part,
stop with failure (clash).
  \item 
(1.3) If the left-hand sides of the frontier of M contain some variable of S, stop with
failure (cycle).
  \item 
(1.4) Transform U using multiequation reduction on the selected mnltiequation and
compactification.
  \item 
(1.5) Let S = \{xl ..... Xn). Apply the substitution ~ = {(C, xl) ..... (C, x,)}
to all terms in the right-hand side of the multiequations of U.
  \item 
(1.6) Transfer the multiequation S = (C) from U to the end of T.
until the U part of R contains only multiequations, if any, with empty right-hand
sides.
\end{enumerate}
  \item 
(2) Transfer all the mnltiequations of U (all with M = ~D) to the end of T, and stop with
success. 
\end{enumerate}
 
