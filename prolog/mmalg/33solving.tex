\secrel{Solving Systems of Multiequations}

For convenience, in what follows, we want to give a structure to a set of
multiequations. Thus we introduce the concept of system of multiequations. A
system R is a pair (T, U), where T is a sequence and U is a set of multiequations
(either possibly empty), such that

(1) the sets of variables which constitute the left-hand sides of all multiequations
in both T and U contain all variables and are disjoint;

(2) the right-hand sides of all multiequations in T consist of no more than one
term; and

(3) all variables belonging to the left-hand side of some multiequation in T can
only occur in the right-hand side of any preceding multiequation in T. 

We now present an algorithm for solving a given system R of multiequations.
When the computation starts, the T part is empty, and every step of the following
Algorithm 2 consists of "transferring" a multiequation from the U part, that is,
the unsolved part, to the T part, that is, the triangular or solved part of R. When
the Upart of R is empty, the system is essentially solved. In fact, the solution can
be obtained by substituting the variables backward. Notice that, by keeping a
solved system in this triangular form, we can hope to find efficient algorithms for
unification even when the mgu has a size which is exponential with respect to the
size of the initial system. For instance, the mgu of the set of multiequations
\begin{verbatim} 
{{Xl} = ~,
{x~} = ~,
{x3} = 0,
{x4} = (h(x3, h(x2, x2)), h(h(h (xl, xl), x2), x3))} 
\end{verbatim}
is
\begin{verbatim}
{(h(xl, Xl), x2), (h(h(xl, Xl), h(Xl, Xl)), x3),
(h(h(h(Xl, Xl), h(xl, Xl)), h(h(xl, Xl), h(Xl, Xl))), X4)}. 
\end{verbatim} 
However, we can give an equivalent solved system with empty U part and whose
T part is
\begin{verbatim}
({x,} --- (h(x3, x3)),
{x3} = (h(x2, x2)),
{X2) = (h(Xl, xl)),
{xl} = o), 
\end{verbatim}
from which the mgu can be obtained by substituting backward. 

Given a system R = (T, U) with an empty T part, an equivalent system with
an empty U part can be computed with the following algorithm. 

\paragraph{Algorithm 2}\ \\
\begin{enumerate}
  \item 
(1) repeat
\begin{enumerate}
  \item 
(1.1) Select a multiequation S = M of U with M \# ~5.
  \item 
(1.2) Compute the common part C and the frontier F ofM. IfM has no common part,
stop with failure (clash).
  \item 
(1.3) If the left-hand sides of the frontier of M contain some variable of S, stop with
failure (cycle).
  \item 
(1.4) Transform U using multiequation reduction on the selected mnltiequation and
compactification.
  \item 
(1.5) Let S = \{xl ..... Xn). Apply the substitution ~ = {(C, xl) ..... (C, x,)}
to all terms in the right-hand side of the multiequations of U.
  \item 
(1.6) Transfer the multiequation S = (C) from U to the end of T.
until the U part of R contains only multiequations, if any, with empty right-hand
sides.
\end{enumerate}
  \item 
(2) Transfer all the mnltiequations of U (all with M = ~D) to the end of T, and stop with
success. 
\end{enumerate}
 
Of course, if we want to use this algorithm for unifying two terms tl and t2, we
have to construct an initial system with empty T part and with the following U
part:
\begin{verbatim}
\[{{x) = (tl, t2), {xl} = 6, {x2} = O ..... {x,} = 6}\]
\end{verbatim}
where xl, x2 .... , Xn are all the variables in t~ and t2 and x is a new variable which
does not occur in ti and t2. For instance, let tl = f(x~, g(x2, xs), x2, b) and
t2 = f(g(h(a, xs), x2), x~, h(a, x4), x4). The initial system is as follows:
\begin{verbatim}
U: {{x} = (f(xl,g(x2, x3), x2, b), f(g(h(a, x~), x2), xl, h(a, x4), x4)),
{x~} = 6, (x2} = 6, {x3} = ;D, (x4} = 6, {xs} = 6}; (3)
T:().  
\end{verbatim}

After the first iteration of Algorithm 2 we get
\begin{verbatim}
U: {(x~} = (g(h(a, x~), x2), g(x2, x3)),
{x2} = (h(a, x4)),
(x~) = 0,
(x4} = (b),
(xs} = ~);
T: ( {x} = (f(xl, xl, x2, x4))). 
\end{verbatim}

We now eliminate variable x2, obtaining
\begin{verbatim}
U: ({Xl) = (g(h(a, xs), h(a, x4)), g(h(a, x4), x3)),
{x3} =6,
(x4} = (b),
{x5 ) = O};
T: ( (x} = (f(xl, Xl, x2, x4)),
{x2} = (h(a, x4))). 
\end{verbatim}

By eliminating variable xl, we get
\begin{verbatim}
U: {(x3} = (h(a, x4)),
{x,, xs} = (b));
T: ( (x} = (f(xl, xi, x2, x4)),
(x2} = (h(a, x4)),
(xl} = (g(h(a, x4), x3))). 
\end{verbatim}

Finally, by eliminating first the set {x4, xs} and then {x3}, we get the solved
system
\begin{verbatim}
U: O;
T: ((x} = (f(x~, Xl, X2, X4)),
(X2) = (h(a, x4)),
{Xl) = (g(h(a, x4), xz)),
(x4, xs} = (b),
{x3) = (h(a, b))). 
\end{verbatim}

We can now prove the correctness of Algorithm 2. 

THEOREM 3.2. Algorithm 2 always terminates. If it stops with failure, then
the given system has no unifier. If it stops with success, the resulting system is
equivalent to the given system and has an empty unsolved part.

PROOF. All transformations obtain systems equivalent to the given one. In fact,
in step (1.4) multiequation reduction obtains a set of equations which (according
to Theorem 3.1) is equivalent, and compactification transforms it again into a
system. Step (1.5) applies substitution only to the terms in U, and its feasibility
can be proved as in Theorem 2.2. Step (1.6) can be applied since the multiequation
S = (C), introduced during multiequation reduction, has not been modified by
compactification, due to the condition tested in step (1.3). For the same condition,
transferring multiequation S = (C) from U to T still leaves a system. Step (2) is
clearly feasible.

If the algorithm stops with failure, then, by Theorem 3.1, the system presently
denoted by R (equivalent to the given one) has no solution. Otherwise, the final
system clearly has an empty U part. Finally, the algorithm always terminates
since at every cycle some variable is eliminated from the U part. $\square$ 

It is easy to see that, for a given system, the size of the final system depends
heavily on the order of elimination of the multiequations. For instance, given the
same system as discussed earlier,
\begin{verbatim}
U: {{xl) = ~,
{x2} = (h(xl, Xl)),
{x3} = (h(x2, x2)),
{x,) = (h(x3, x3))};
T:(), 
\end{verbatim}

and eliminating the variables in the order x2, xz, x4, Xx, we get the final system
\begin{verbatim}
U: 0;
T: ({x2) -- (h(xm, Xl)),
{x3} = (h(h(Xl, xl), h(Xl, xl))),
{x4} = (h(h(h(Xl, Xl), h(xl, xl)), h(h(xl, xl), h(Xl, Xl)))),
{x~ } = 0). 
\end{verbatim}

If instead we eliminate the variables in the order x4, x3, x2, xl, we get
\begin{verbatim}
U: O;
T: ({x4} = (h(x3, x3)),
(x3} = (h(x2, x2)),
{x2} -- (h(Xl, Xx)),
{x, } = O). 
\end{verbatim}
