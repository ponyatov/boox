\secrel{Basic Definitions}

In this section we present an extension of the previous formalism to model our
algorithm more closely. We first introduce the concept of multiequation. A multiequation
is the generalization of an equation, and it allows us to group together
many terms which should be unified. To represent multiequations we use the
notation S -- M where the left-hand side S is a nonempty set of variables and the
right-hand side M is a multiset 1 of nonvariable terms. An example is
\[{xl, x2, x3} = (tl, t2).\]
\note{A multiset is a family of elements in which no ordering exists but in which many identical elements
may occur.}

The solution (unifier) of a multiequation is any substitution which makes all
terms in the left- and right-hand sides identical. 

A multiequation can be seen as a way of grouping many equations together.
For instance, the set of equations
\[Xl ---- X2;\]
\[X3 = Xl;\]
\[tl = Xl;\]
\[X2 ---- t2;\]
\[tl = t2\]
can be transformed into the above multiequation, since every unifier of this set
of equations makes the terms of all equations identical. To be more precise, given
a set of equations SE, let us define a relation RSE between pairs of terms as
follows: tl RSE t2 iff the equation tl = t2 belongs to SE. Let/tSE be the reflexive,
symmetric, and transitive closure of RSE. 

Now we can say that a set of equations SE corresponds to a multiequation
S = M iff all terms of SE belong to S U M and for every tr and ts E S U M we have
tr RSE t,. 

It is easy to see that many different sets of equations may correspond to a
given multiequation and that all these sets are equivalent. Thus the set of
solutions (unifiers) of a multiequation coincides with the set of solutions of any
corresponding set of equations. 

Similar definitions can be given for a set of multiequations Z by introducing a
relation Rz between pairs of terms which belong to the same multiequation. A set
of equations SE corresponds to a set of multiequations Z iff
\[ti/~SE tj ** ti Rz tj\]
for all terms t~, tj of SE or Z.  
 