\secrel{IMPLEMENTATION}\label{mmalg6}

In order to describe the last details of our algorithm, we present here a PASCAL
implementation. In Figure 1 we have the definitions of data types. All data
structures used by the algorithm are dynamically created data structures connected
through pointers. The UPart of a system has two lists of multiequations:
Equations, which contains all initial multiequations, and ZeroCounterMultEq,
which contains all multiequations with zero counter. Furthermore, the field
MultEqNumber contains the number of multiequations in the UPart. A multiequation,
besides having the fields Counter, S, and M, has a field VarNumber,
which contains the number of variables in S and is used during compactification.
The pairs Pi = (S i, Mi), which are the arguments of a multiterm, have type
TempMultEq. Finally, all occurrences of a variable point to the same Variable
object, which points to the multiequation containing it in its left-hand side. 

\paragraph{Figure 2}
\begin{verbatim}
procedure Unify(R: System);
var Mult: MultiEquation ;
Frontier: ListOf TempMultEq ;
begin
repeat
SelectMultiEquation(R ~. U, Mult);
if not(Mult~.M=Nil) then
begin
Frontier := Nil;
Reduce(Multi.M, Frontier);
Compact(Frontier, R ~. U)
end;
R ~.T := NewListOfMultEq(Mult, R ~.T)
until R ~. U.MultEqNumber = 0
end (*Unify*);
\end{verbatim}

\paragraph{Figure 3}
\begin{verbatim}
procedure SelectMultiEquation(var U: UPart; var Mult: MultiEquation);
begin
if U.ZeroCounterMultEq = Nil then fail('cycle');
Mult := U~eroCounterMultEq~. Value;
U~eroCounterMultEq := U.ZeroCounterMultEqT.Next;
U.MultEqNumber := U.MultEqNumber - 1
end ( * SelectMultiEquation *);
\end{verbatim}

ject, which points to the multiequation containing it in its left-hand side.
The types "ListOf... ," not given in Figure 1, are all implemented as records
with two fields: Value and Next. Finally, QueueOfVariables is an abstract type
with operations CreateListOfVars, IsEmpty, HeadOf, RemoveHead, and Append,
which have a constant execution time. 

In Figure 2 we rephrase Algorithm UNIFY as a PASCAL procedure. Procedure
SelectMultiEquation selects from the UPart of the system a multiequation which
is "on top" of the partial ordering, by taking it from the ZeroCounterMultEq list.
Its implementation is given in Figure 3. 

Procedure Reduce, given in Figure 4, computes the common part and the
frontier of the selected multiequation. This procedure modifies the right-hand
side of this multiequation so that it contains directly the common part. Note that
the frontier is represented as a list of TempMultEq instead of as a list of
multiequations. 

Finally, in Figure 5 we give procedure Compact, which performs compactification
by repeatedly merging multiequations. When two multiequations are
merged, one of them is erased, and thus all pointers to it from its variables must
be moved to the other multiequation. To minimize the computing cost, we always
erase the multiequation with the smallest number of variables in its left-hand
side. Procedure MergeMultiTerms is given in Figure 6. 

A detailed complexity analysis of a similar implementation is given in [13].
There it is proved that an upper bound to execution time is the sum of two terms,
one linear with the total number of symbols in the initial system and another one
n log n with the number of distinct variables. 

\paragraph{Figure 4}
\begin{verbatim}
procedure Reduce(M: MultiTerm; var Frontier: ListOfTernpMultEq);
var Arg" ListOfTempMultEq;
begin
Arg := MT.Args;
while not(Arg = Nil) do
begin
if IsEmpty(Arg T. Value T.S) then Reduce(ArgO. Value T.M, Frontier)
else
begin
Frontier := NewListOfTempMultEq(Arg T. Value, Frontier);
ArgT. Value := NewTempMultEq( CreateQueueOfVars(HeadOf(Arg~. ValueT.S) ) , Nil)
end;
Arg := ArgT.Next
end
end (*Reduce*); 
\end{verbatim}

Here we want only to point out that the nonlinear behavior stems from the
operation described above of moving all pointers directed from variables to
multiequations, whenever two multiequations are merged. To see how this can
happen, let us consider the problem of unifying the two terms
\[f(xl, x3, xs, xT, xl, xs, xl)\]
and
\[f(x2, x4, x6, x8, x3, x7, x5).\]

During the first iteration of Unify we get a frontier whose multiequations are
the pairs (xl, x2), (x3, x4), (x~, x6), (xT, xs), (xl, x3), (xs, xT), and (xl, xs). By
executing Compact with this frontier, we see that it moves one pointer for each
of the first four elements of the frontier, two pointers for each of the next two
elements, and four pointers for the last element. Thus, it has an n log n complexity. 

As a final remark, we point out that we might modify the worst-case behavior
of our algorithm with a different implementation of the operation of multiequation
merging. In fact, we might represent sets of variables as trees instead of as lists,
and we might use the well-known UNION-FIND algorithms [1] to add elements
and to access them. In this case the complexity would be of the order of m. G(m),
where G is an extremely slowly growing function (the inverse of the Ackermann
function). However, m would be, in this case, the number of variable occurrences.  