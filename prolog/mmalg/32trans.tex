\secrel{Transformations of Sets of Multiequations}

We now introduce a few transformations of sets of multiequations, which are
generalizations of the transformations presented in Section 2. 

We first define the common part and the frontier of a multiset of terms
(variables or not). The common part of a multiset of terms M is a term which,
intuitively, is obtained by superimposing all terms of M and by taking the part
which is common to all of them starting from the root. For instance, given the
multiset of terms
\[(f(xl, g(a, f(xs, b))), f(h(c), g(x2, f(b, xs))), f(h(x4), g(x6, x3))),\]
the common part is
\[f(xl, g(x2, x3)).\]

The frontier is a set of multiequations, where every multiequation is associated
with a leaf of the common part and consists of all subterms (one for each term of 
M) corresponding to that leaf. The frontier of the above multiset of terms is
\begin{verbatim}
{{x~} = (h(c), h(x4)),
{x2, x6} = (a),
{x3} = (f(xs, b), f(b, xD)).
\end{verbatim}

Note that if there is a clash of function symbols among some terms of a multiset
of terms M, then M has no common part and frontier. In this case the terms of M
are not unifiable. 

The common part and the frontier can be defined more precisely by means of
a function DEC which takes a multiset of terms M as argument and returns either
"failure," in which case M has neither common part nor frontier, or a pair (C(M),
F(M) ) where C(M) is the common part of M and F(M) is the frontier of M. 

In the definition of DEC we use the following notation: 

\begin{tabular}{l l}
head(t) & is the root function symbol of term t, for t ~ V. \\
Pi & is the ith projection, defined by\\&
\verb|\[Pi(f(tl .... ,tn))=ti for f~An and l\_<i\_n;\]|\\
make&is a function which transforms a multiset of terms M into a multiequa\\
multeq & tion whose left-hand side is the set of all variables in M and whose\\
&right-hand side is the multiset of all terms in M which \\&are not variables;
and \\
$U$ & is the union for multisets. \\
\end{tabular}
\begin{lstlisting}
DEC(M) = ff 3t ~ M, t E V
	then (t, {makemulteq(M)} )
	else if 3n, 3 f E A,, Yt E M, head(t) = f
		then if n ffi 0
		then ( f, O)
		else if Vi (1 __ i _ n), DEC(Mi) ~ failure
		where Mi -- OteM Pi(t)
		then (f(C(M1) ..... C(M,)), UTffil F(Mi))
		else failure
		else failure.   
\end{lstlisting}

We can now define the following transformation: 

Multiequation Reduction. Let Z be a set of multiequations containing a
multiequation S -- M such that M is nonempty and has a common part C and a
frontier F. The new set Z' of multiequations is obtained by replacing S = M with
the union of the multiequation S = (C) and of all the multiequations of F:
\[Z'ffi(Z- {SffiM})U{S=(C)} UF.\]

THEOREM 3.1. Let S = M (M nonempty) be a multiequation of a set Z of
multiequations. If M has no common part, or if some variable in S belongs to
the left-hand side of some multiequation in the frontier F of M, then Z has no 
unifier. Otherwise, by applying multiequation reduction to the multiequation
S = M we get an equivalent set Z' ofmultiequations. 

PROOF. If the common part of M does not exist, then the multiequation S -- M
has no unifier, since two terms should be made equal having a different function
symbol in the corresponding subterms. Moreover, if some variable x of S occurs
in some left-hand side of the frontier, then it also occurs in some term t of M, and
thus the equation x = t, with x occurring in t, belongs to a set of equations
equivalent to Z. But, according to Theorem 2.2, this set has no unifier. 

To prove that Z and Z' are equivalent, we show first that a unifier of Z is also
a unifier of Z'. In fact, if a substitution ~ makes all terms of M equal, it also
makes equal all the corresponding subterms, in particular, all terms and variables
which belong to left- and right-hand sides of the same multiequation in the
frontier. The multiequation S = (C) is also satisfied by construction. Conversely,
if ~ satisfies Z', then the multiequation S -- M is also satisfied. In fact, all terms
in S and M are made equal--in their upper part (the common part) due to the
multiequation S -- (C) and in their lower part (the subterms not included in the
common part) due to the set of multiequations F. $\square$

We say that a set Z of multiequations is compact iff
\begin{verbatim}
\[Y(S =M), (S' =M'} ~Z: SA S' = ~.\]
\end{verbatim}
We can now introduce a second transformation, which derives a compact set of
multiequations. 

Compactification. Let Z be a noncompact set of multiequations. Let R be a
relation between pairs of multiequations of Z such that iS = M) R iS' = M') iff
S n S' \# O, and let/t be the transitive closure of R. The relation/~ partitions
the set Z into equivalence classes. To obtain the final compact set Z', all multiequations
belonging to the same equivalence class are merged; that is, they are
transformed into single multiequations by taking the union of their left- and
right-hand sides. 

Clearly, Z and Z' are equivalent, because the relation /~z between pairs of
terms, defined in Section 3.1, does not change by passing from Z to Z'. 
  