\secrel{An Efficient Unification Martelli/Montanary Algorithm}
\label{mmalg}\secdown

\cp{\url{http://www.nsl.com/misc/papers/martelli-montanari.pdf}}

\noindent\copyright \
ALBERTO MARTELLI
Consiglio Nazionale delle Ricerche
\\and\\
UGO MONTANARI
Universita di Pisa
\note{
Authors' present addresses: A. Martelli, Istituto di Scienze della Informazione,
Universit~ di Torino, Corso M. d'Azeglio 42, 1-10125 Torino, Italy; U.
Montanari, Istituto di Scienze della Informazione, Universit\& di Pisa, Corso
Italia 40, 1-56100 Pisa, Italy.

Permission to copy without fee all or part of this material is granted provided
that the copies are not made or distributed for direct commercial advantage, the
ACM copyright notice and the title of the publication and its date appear, and
notice is given that copying is by permission of the Association for Computing
Machinery. To copy otherwise, or to republish, requires a fee and/or specific
permission.

\copyright\ 1982 ACM 0164-0925/82/0400-0258 \$00.75

ACM Transactions on Programming Languages and Systems, Vol. 4, No. 2, April
 1982, Pages 258-282.
}\bigskip

\subsecly{Abstract}

The unification problem in first-order predicate calculus is described in
general terms as the solution of a system of equations, and a nondeterministic
algorithm is given. A new unification algorithm, characterized by having the
acyclicity test efficiently embedded into it, is derived from the
nondeterministic one, and a PASCAL implementation is given. A comparison with
other well-known unification algorithms shows that the algorithm described here
performs well in all cases.

Categories and Subject Descriptors: F.2.2 [Analysis of Algorithms and Problem
Complexity]: Nonnumerical Algorithms and Problems--complexity of proof
procedures; F.4.1 [Mathematical Logic and Formal Languages]: Mathematical
Logic--mechanical theorem proving; 1.2.3 [Artificial Intelligence]: Deduction
and Theorem Proving--resolution

General Terms: Algorithms, Languages, Performance, Theory 

\secrel{INTRODUCTION}\label{mmalg1}

In its essence, the unification problem in first-order logic can be expressed as
follows: Given two terms containing some variables, find, if it exists, the
simplest substitution (i.e., an assignment of some term to every variable) which
makes the two terms equal. The resulting substitution is called the \term{most
general unifier} and is unique up to variable renaming.

Unification was first introduced by Robinson [17, 18] as the central step of the
inference rule called resolution. This single, powerful rule can replace all the
axioms and inference rules of the first-order predicate calculus and thus was
immediately recognized as especially suited to mechanical theorem provers. In
fact, a number of systems based on resolution were built and tried on a variety
of different applications [5]. Even though further research made it apparent
that resolution systems are difficult to direct during proof search and thus are
often prone to combinatorial explosion [6], new impetus to the research in this
area was given by Kowalski's idea of interpreting predicate logic as a
programming language [10]. Here predicate logic clauses are seen as procedure
declarations, and procedure invocation represents a resolution step. From this
viewpoint, theorem provers can be regarded as interpreters for programs written
in predicate logic, and this analogy suggests efficient implementations [3, 25].

Resolution, however, is not the only application of the unification algorithm.
In fact, its pattern matching nature can be exploited in many cases where
symbolic expressions are dealt with, such as, for instance, in interpreters for
equation languages [4, 11], in systems using a database organized in terms of
productions [19], in type checkers for programming languages with a complex type
structure [14], and in the computation of critical pairs for term rewriting
systems [9].

The unification algorithm constitutes the heart of all the applications listed
above, and thus its performance affects in a crucial way the global efficiency
of each. The unification algorithm as originally proposed can be extremely
inefficient; therefore, many attempts have been made to find more efficient
algorithms [2, 7, 13, 15, 16, 22]. Unification algorithms have also been
extended to the case of higher order logic [8] and to deal directly with
associativity and commutativity [20]. The problem was also tackled from a
computational complexity point of view, and linear algorithms were proposed
independently by Martelli and Montanari [13] and Paterson and Wegman [15].

In the next section we give some basic definitions by representing the
unification problem as the solution of a system of equations. A nondeterministic
algorithm, which comprehends as special cases most known algorithms, is then
defined and proved correct. In Section 3 we present a new version of this
algorithm obtained by grouping together all equations with some member in
common, and we derive from it a first version of our unification algorithm.

In Sections 4 and 5 we present the main ideas which make the algorithm
efficient, and the last details are described in Section 6 by means of a PASCAL
implementation.

Finally, in Section 7, the performance of this algorithm is compared with that
of two well-known algorithms, Huet's [7] and Paterson and Wegman's [15]. This
analysis shows that our algorithm has uniformly good performance for all classes
of data considered.


\secrel{UNIFICATION AS THE SOLUTION OF A SET OF EQUATIONS: A NONDETERMINISTIC
ALGORITHM}

In this section we introduce the basic definitions and give a few theorems which
are useful in proving the correctness of the algorithms. Our ay of stating the
unification problem is slightly more general than the classical one due to
Robinson [18] and directly suggests a number of possible solution methods.

Let
\[A= \bigcup_{i=0,1,..} A_i \quad (A_i \bigcap A_j = \varnothing, i \neq j)\] be
a ranked alphabet, where $A_i$ contains the $i$-adic function symbols
(the elements of $A_0$ are constant symbols). Furthermore, let $V$ be the
alphabet of the variables.
The \term{terms} are defined recursively as follows:

(1) constant symbols and variables are terms;

(2) if $t_1,..,t_n \ (n \geq 1)$ are terms and $f \in A_n$,
then $f(t_1,..,t_n)$ is a term.

A \term{substitution} $\vartheta$ is a mapping from variables to terms,
with $\vartheta(x)=x$ almost everywhere. A substitution can be represented
by a finite set of ordered pairs
$\vartheta={(t_1,x_1),(t_2,x_2),..,(t_m,x_m)}$
where $t_i$ are terms and $x_i$ are distinct variables,
$i = 1,..,m$. To apply a substitution $\vartheta$ to a term $t$, we
simultaneously substitute all occurrences in $t$ of every variable $x_i$ in a
pair $(t_i, x_i)$ of $\vartheta$ with the corresponding term $t_i$. We call the
resulting term $t_\vartheta$.

For instance, given a term $t = f(x_1, g(x_2, a)$ and a substitution 
$\vartheta = {(h(x_2),x_1),(b,x_2)}$, 
we have $t_\vartheta = f(f(x_2),g(b),a)$ and $t_{\vartheta\vartheta} =
f(h(b),g(b),a))$.

The standard unification problem can be written as an equation \[t'=t''\]

A solution of the equation, called a \term{unifier}, is any substitution
$\vartheta$, if it exists, which makes the two terms identical. For instance,
two unifiers of the equation $f(x_1,h(x_1),x_2)=f(g(x_3),x_4,x_3))$ are
$\vartheta_1={(g(x_3),x_1),(x_3,x_2),(h(g(x_3)),x_4)}$ and
$\vartheta_2={(g(a),x_1),(a,x_2),(a,x_3),(h(g(a)),x_4)}$.

In what follows it is convenient also to consider sets of equations
\[t'_j=t''_j, \quad j=1,..,k\]

Again, a \term{unifier} is any substitution which makes all pairs of terms
$t'_j,t''_j$ identical simultaneously.

Now we are interested in finding transformations which produce \emph{equivalent}
sets of equations, namely, transformations which preserve the sets of all
unifiers.
Let us introduce the following two transformations:

\paragraph{(1) Term Reduction.} Let

\begin{equation}\label{mm1}
f(t'_1,t'_2,..,t'_n)=f(t''_1,t''_2,..,t''_n), \quad f \in A_n
\end{equation} 
be an equation where both terms are not variables and where the two root
function symbols are equal. The new set of equations is obtained by replacing
this equation with the following ones: 
\begin{align}\label{mm2}
t'_1 &= t''_1\\
t'_2 &= t''_2\\
&.\\
&.\\
&.\\
t'_n &= t''_n
\end{align}
If $n = 0$, then $f$ is a constant symbol, and the equation is simply erased.

\paragraph{(2) Variable Elimination.} Let $x = t$ be an equation where $x$ is a
variable and $t$ is any term (variable or not). The new set of equations is
obtained by applying the substitution $\vartheta={(t,x)}$ to both terms of all
other equations in the set (without erasing $x = t$).

We can prove the following theorems: 

\paragraph{THEOREM 2.1.} \textit{Let $S$ be a set of equations and let
$f'(t'_1,..t'_n)=f''(t''_1,..,t''_n)$ be an equation of $S$. If $f' \neq f''$,
then $S$ has no unifier. Otherwise, the new set of equations $S'$, obtained by
applying term reduction to the given equation, is equivalent to $S$.}

\paragraph{PROOF.} If $f' \neq f''$, then no substitution can make the
two terms identical.
If $f' = f"$, any substitution which satisfies \ref{mm2} also satisfies
\ref{mm1}, and conversely for the recursive definition of term. $\square$

\paragraph{THEOREM 2.2.} \textit{Let $S$ be a set of equations, and let us apply
variable elimination to some equation $x = t$, getting a new set of equations
$S'$. If variable $x$ occurs in $t$ (but $t$ is not $x$), then $S$ has no
unifier; otherwise, $S$ and $S'$ are equivalent.}

\paragraph{PROOF.} Equation $x = t$ belongs both to $S$ and to $S'$, and thus
any unifier $\vartheta$ (if it exists) of $S$ or of $S'$ must unify $x$ and $t$;
that is, $x_\vartheta$ and $t_\vartheta$ are identical. Now let $t_1 = t_2$ be
any other equation of $S$, and let $t'_1 = t'_2$ be the corresponding equation
in $S'$. Since $t'_1$ and $t'_2$ have been obtained by substituting $t$ for
every occurrence of $x$ in $t_1$ and $t_2$, respectively, we have
$t_{1_\vartheta}=t'_{1_\vartheta}$ and $t_{2_\vartheta}=t'_{2_\vartheta}$. Thus,
any unifier of $S$ is also a unifier of $S'$ and vice versa. Furthermore, if
variable $x$ occurs in $t$ (but $t$ is not $x$), then no substitution
$\vartheta$ can make $x$ and $t$ identical, since $x_\vartheta$ becomes a
subterm of $t_\vartheta$, and thus $S$ has no unifier. $\square$

A set of equations is said to be \term{in solved form} iff it satisfies the
following conditions: 

(1) the equations are $x_j = t_j, j = 1,..,k$;

(2) every variable which is the left member of some equation occurs only there.

A set of equations in solved form has the obvious unifier
\[\vartheta = {(t_1,x_1),(t_2,x_2),..,(t_k,x_k)}\]

If there is any other unifier, it can be obtained as
\[0 = {(t,, x~), (t2, x2) .... , (tk, xk)} U a\]
where a is any substitution which does not rewrite variables xl .... , xk. Thus t~
is called a most general unifier (mgu ).  

The following nondeterministic algorithm shows how a set of equations can be
transformed into an equivalent set of equations in solved form. 

\paragraph{Algorithm 1 }\ \\

Given a set of equations, repeatedly perform any of the following transformations. If no
transformation applies, stop with success. 

(a) Select any equation of the form
\[t=x\]
where t is not a variable and x is a variable, and rewrite it as 
\[x=t.\]

(b) Select any equation of the form
\[X=X\] 
where x is variable, and erase it.  

(c) Select any equation of the form
\[t' = t"\]
where t' and t" are not variables. If the two root function symbols are different, stop with
failure; otherwise, apply term reduction. 

(d) Select any equation of the form
\[x=t\]
where x is a variable which occurs somewhere else in the set of equations and
where t \# x. If x occurs in t, then stop with failure; otherwise, apply variable elimination. 

As an example, let us consider the following set of equations:
\[g(x2) = xl;\]
\[f(xl, h(xl), x2) = f(g(x3), x4, x3).\]

By applying transformation (c) of Algorithm 1 to the second equation we get
\[g(x2) = xl;\]
\[xl = g(x3);\]
\[h(x~) = x4;\]
\[X2 =X3.\]

By applying transformation (d) to the second equation we get
\[g(x2) = g(xs);\]
\[xl = g(x3);\]
\[h(g(x3)) = x4;\]
\[X2 ~- X3.\]

We now apply transformation (c) to the first equation and transformation (a) to
the third equation:
\[X2 ~ X3\]
\[xl = g(x3);\]
\[Xa = h(g(x3));\]
\[X2 ----X3.\]

Finally, by applying transformation (d) to the first equation and transformation
(b) to the last equation, we get the set of equations in solved form:
\[X2 ~- X3 ;\]
\[xl = g(x3);\]
\[x4 = h(g(x3)).\]

Therefore, an mgu of the given system is
\[= {(g(x~), x~), (x3, x2), (h(g(x3)), x4)}.\] 

The following theorem proves the correctness of Algorithm 1. 

\paragraph{THEOREM 2.3}. Given a set of equations S,

(i) Algorithm 1 always terminates, no matter which choices are made.

(ii) If Algorithm 1 terminates with failure, S has no unifier. If Algorithm 1
terminates with success, the set S has been transformed into an equivalent
set in solved form. 
 
\paragraph{PROOF}.\\

(i) Let us define a function F mapping any set of equations S into a triple of
natural numbers (nl, n2, n3). The first number, n~, is the number of variables in
S which do not occur only once as the left-hand side of some equation. The
second number, n2, is the total number of occurrences of function symbols in S.
The third number, n3, is the sum of the numbers of equations in S of type x = x
and of type t = x, where x is a variable and t is not. Let us define a total ordering
on such triples as follows:
\begin{align*}
(n~, n~, n~) > (n~', n2, " n~')\ if\ &n~ > n~'\\
&or\ n~ = n~\ and\ n2 > n2\\
&or\ n~=n" 1 a n- 1 n2 ' ---- n2 "\ and\ n3 ' > n3.\\ 
\end{align*}   

With the above ordering, N 3 becomes a well-founded set, that is, a set where no
infinite decreasing sequence exists. Thus, if we prove that any transformation of
Algorithm 1 transforms a set S in a set S' such that F(S') < F(S), we have
proved the termination. In fact, transformations (a) and (b) always decrease n3
and, possibly, n~. Transformation (c) can possibly increase n3 and decrease nl,
but it surely decreases n2 (by two). Transformation (d) can possibly change n3
and increase n2, but it surely decreases n~. 

(ii) If Algorithm 1 terminates with failure, the thesis immediately follows from
Theorems 2.1 and 2.2. If Algorithm 1 terminates with success, the resulting set of
equations S' is equivalent to the given set S. In fact, transformations (a) and (b)
clearly do not change the set of unifiers, while for transformations (c) and (d) this
fact is stated in Theorems 2.1 and 2.2. Finally, S' is in solved form. In fact, if (a),
(b), and (c) cannot be applied, it means that the equations are all in the form
x = t, with t \# x. If (d) cannot be applied, that means that every v.arialSle
which is the left-hand side of some equation occurs only there.
$\square$\bigskip

The above nondeterministic algorithm provides a widely general version from
which most unification algorithms [2, 3, 7, 13, 15, 16, 18, 22-24] can be derived by
specifying the order in which the equations are selected and by defining suitable
concrete data structures. For instance, Robinson's algorithm [18] might be
obtained by considering the set of equations as a stack. 


\secrel{AN ALGORITHM WHICH EXPLOITS A PARTIAL ORDERING AMONG SETS
OF VARIABLES}\secdown
\secrel{Basic Definitions}

In this section we present an extension of the previous formalism to model our
algorithm more closely. We first introduce the concept of multiequation. A multiequation
is the generalization of an equation, and it allows us to group together
many terms which should be unified. To represent multiequations we use the
notation S -- M where the left-hand side S is a nonempty set of variables and the
right-hand side M is a multiset 1 of nonvariable terms. An example is
\[{xl, x2, x3} = (tl, t2).\]
\note{A multiset is a family of elements in which no ordering exists but in which many identical elements
may occur.}

The solution (unifier) of a multiequation is any substitution which makes all
terms in the left- and right-hand sides identical. 

A multiequation can be seen as a way of grouping many equations together.
For instance, the set of equations
\[Xl ---- X2;\]
\[X3 = Xl;\]
\[tl = Xl;\]
\[X2 ---- t2;\]
\[tl = t2\]
can be transformed into the above multiequation, since every unifier of this set
of equations makes the terms of all equations identical. To be more precise, given
a set of equations SE, let us define a relation RSE between pairs of terms as
follows: tl RSE t2 iff the equation tl = t2 belongs to SE. Let/tSE be the reflexive,
symmetric, and transitive closure of RSE. 

Now we can say that a set of equations SE corresponds to a multiequation
S = M iff all terms of SE belong to S U M and for every tr and ts E S U M we have
tr RSE t,. 

It is easy to see that many different sets of equations may correspond to a
given multiequation and that all these sets are equivalent. Thus the set of
solutions (unifiers) of a multiequation coincides with the set of solutions of any
corresponding set of equations. 

Similar definitions can be given for a set of multiequations Z by introducing a
relation Rz between pairs of terms which belong to the same multiequation. A set
of equations SE corresponds to a set of multiequations Z iff
\[ti/~SE tj ** ti Rz tj\]
for all terms t~, tj of SE or Z.  
 
\secrel{Transformations of Sets of Multiequations}

We now introduce a few transformations of sets of multiequations, which are
generalizations of the transformations presented in Section 2. 

We first define the common part and the frontier of a multiset of terms
(variables or not). The common part of a multiset of terms M is a term which,
intuitively, is obtained by superimposing all terms of M and by taking the part
which is common to all of them starting from the root. For instance, given the
multiset of terms
\[(f(xl, g(a, f(xs, b))), f(h(c), g(x2, f(b, xs))), f(h(x4), g(x6, x3))),\]
the common part is
\[f(xl, g(x2, x3)).\]

The frontier is a set of multiequations, where every multiequation is associated
with a leaf of the common part and consists of all subterms (one for each term of 
M) corresponding to that leaf. The frontier of the above multiset of terms is
\begin{verbatim}
{{x~} = (h(c), h(x4)),
{x2, x6} = (a),
{x3} = (f(xs, b), f(b, xD)).
\end{verbatim}

Note that if there is a clash of function symbols among some terms of a multiset
of terms M, then M has no common part and frontier. In this case the terms of M
are not unifiable. 

The common part and the frontier can be defined more precisely by means of
a function DEC which takes a multiset of terms M as argument and returns either
"failure," in which case M has neither common part nor frontier, or a pair (C(M),
F(M) ) where C(M) is the common part of M and F(M) is the frontier of M. 

In the definition of DEC we use the following notation: 

\begin{tabular}{l l}
head(t) & is the root function symbol of term t, for t ~ V. \\
Pi & is the ith projection, defined by\\&
\verb|\[Pi(f(tl .... ,tn))=ti for f~An and l\_<i\_n;\]|\\
make&is a function which transforms a multiset of terms M into a multiequa\\
multeq & tion whose left-hand side is the set of all variables in M and whose\\
&right-hand side is the multiset of all terms in M which \\&are not variables;
and \\
$U$ & is the union for multisets. \\
\end{tabular}
\begin{lstlisting}
DEC(M) = ff 3t ~ M, t E V
	then (t, {makemulteq(M)} )
	else if 3n, 3 f E A,, Yt E M, head(t) = f
		then if n ffi 0
		then ( f, O)
		else if Vi (1 __ i _ n), DEC(Mi) ~ failure
		where Mi -- OteM Pi(t)
		then (f(C(M1) ..... C(M,)), UTffil F(Mi))
		else failure
		else failure.   
\end{lstlisting}

We can now define the following transformation: 

Multiequation Reduction. Let Z be a set of multiequations containing a
multiequation S -- M such that M is nonempty and has a common part C and a
frontier F. The new set Z' of multiequations is obtained by replacing S = M with
the union of the multiequation S = (C) and of all the multiequations of F:
\[Z'ffi(Z- {SffiM})U{S=(C)} UF.\]

THEOREM 3.1. Let S = M (M nonempty) be a multiequation of a set Z of
multiequations. If M has no common part, or if some variable in S belongs to
the left-hand side of some multiequation in the frontier F of M, then Z has no 
unifier. Otherwise, by applying multiequation reduction to the multiequation
S = M we get an equivalent set Z' ofmultiequations. 

PROOF. If the common part of M does not exist, then the multiequation S -- M
has no unifier, since two terms should be made equal having a different function
symbol in the corresponding subterms. Moreover, if some variable x of S occurs
in some left-hand side of the frontier, then it also occurs in some term t of M, and
thus the equation x = t, with x occurring in t, belongs to a set of equations
equivalent to Z. But, according to Theorem 2.2, this set has no unifier. 

To prove that Z and Z' are equivalent, we show first that a unifier of Z is also
a unifier of Z'. In fact, if a substitution ~ makes all terms of M equal, it also
makes equal all the corresponding subterms, in particular, all terms and variables
which belong to left- and right-hand sides of the same multiequation in the
frontier. The multiequation S = (C) is also satisfied by construction. Conversely,
if ~ satisfies Z', then the multiequation S -- M is also satisfied. In fact, all terms
in S and M are made equal--in their upper part (the common part) due to the
multiequation S -- (C) and in their lower part (the subterms not included in the
common part) due to the set of multiequations F. $\square$

We say that a set Z of multiequations is compact iff
\begin{verbatim}
\[Y(S =M), (S' =M'} ~Z: SA S' = ~.\]
\end{verbatim}
We can now introduce a second transformation, which derives a compact set of
multiequations. 

Compactification. Let Z be a noncompact set of multiequations. Let R be a
relation between pairs of multiequations of Z such that iS = M) R iS' = M') iff
S n S' \# O, and let/t be the transitive closure of R. The relation/~ partitions
the set Z into equivalence classes. To obtain the final compact set Z', all multiequations
belonging to the same equivalence class are merged; that is, they are
transformed into single multiequations by taking the union of their left- and
right-hand sides. 

Clearly, Z and Z' are equivalent, because the relation /~z between pairs of
terms, defined in Section 3.1, does not change by passing from Z to Z'. 
  
\secrel{Solving Systems of Multiequations}

For convenience, in what follows, we want to give a structure to a set of
multiequations. Thus we introduce the concept of system of multiequations. A
system R is a pair (T, U), where T is a sequence and U is a set of multiequations
(either possibly empty), such that

(1) the sets of variables which constitute the left-hand sides of all multiequations
in both T and U contain all variables and are disjoint;

(2) the right-hand sides of all multiequations in T consist of no more than one
term; and

(3) all variables belonging to the left-hand side of some multiequation in T can
only occur in the right-hand side of any preceding multiequation in T. 

We now present an algorithm for solving a given system R of multiequations.
When the computation starts, the T part is empty, and every step of the following
Algorithm 2 consists of "transferring" a multiequation from the U part, that is,
the unsolved part, to the T part, that is, the triangular or solved part of R. When
the Upart of R is empty, the system is essentially solved. In fact, the solution can
be obtained by substituting the variables backward. Notice that, by keeping a
solved system in this triangular form, we can hope to find efficient algorithms for
unification even when the mgu has a size which is exponential with respect to the
size of the initial system. For instance, the mgu of the set of multiequations
\begin{verbatim} 
{{Xl} = ~,
{x~} = ~,
{x3} = 0,
{x4} = (h(x3, h(x2, x2)), h(h(h (xl, xl), x2), x3))} 
\end{verbatim}
is
\begin{verbatim}
{(h(xl, Xl), x2), (h(h(xl, Xl), h(Xl, Xl)), x3),
(h(h(h(Xl, Xl), h(xl, Xl)), h(h(xl, Xl), h(Xl, Xl))), X4)}. 
\end{verbatim} 
However, we can give an equivalent solved system with empty U part and whose
T part is
\begin{verbatim}
({x,} --- (h(x3, x3)),
{x3} = (h(x2, x2)),
{X2) = (h(Xl, xl)),
{xl} = o), 
\end{verbatim}
from which the mgu can be obtained by substituting backward. 

Given a system R = (T, U) with an empty T part, an equivalent system with
an empty U part can be computed with the following algorithm. 

\paragraph{Algorithm 2}\ \\
\begin{enumerate}
  \item 
(1) repeat
\begin{enumerate}
  \item 
(1.1) Select a multiequation S = M of U with M \# ~5.
  \item 
(1.2) Compute the common part C and the frontier F ofM. IfM has no common part,
stop with failure (clash).
  \item 
(1.3) If the left-hand sides of the frontier of M contain some variable of S, stop with
failure (cycle).
  \item 
(1.4) Transform U using multiequation reduction on the selected mnltiequation and
compactification.
  \item 
(1.5) Let S = \{xl ..... Xn). Apply the substitution ~ = {(C, xl) ..... (C, x,)}
to all terms in the right-hand side of the multiequations of U.
  \item 
(1.6) Transfer the multiequation S = (C) from U to the end of T.
until the U part of R contains only multiequations, if any, with empty right-hand
sides.
\end{enumerate}
  \item 
(2) Transfer all the mnltiequations of U (all with M = ~D) to the end of T, and stop with
success. 
\end{enumerate}
 

\secup

\secup