\secrel{An Efficient Unification Martelli/Montanary Algorithm}
\label{mmalg}\secdown

\cp{\url{http://www.nsl.com/misc/papers/martelli-montanari.pdf}}

\noindent\copyright \
ALBERTO MARTELLI
Consiglio Nazionale delle Ricerche
\\and\\
UGO MONTANARI
Universita di Pisa
\note{
Authors' present addresses: A. Martelli, Istituto di Scienze della Informazione,
Universit~ di Torino, Corso M. d'Azeglio 42, 1-10125 Torino, Italy; U.
Montanari, Istituto di Scienze della Informazione, Universit\& di Pisa, Corso
Italia 40, 1-56100 Pisa, Italy.

Permission to copy without fee all or part of this material is granted provided
that the copies are not made or distributed for direct commercial advantage, the
ACM copyright notice and the title of the publication and its date appear, and
notice is given that copying is by permission of the Association for Computing
Machinery. To copy otherwise, or to republish, requires a fee and/or specific
permission.

\copyright\ 1982 ACM 0164-0925/82/0400-0258 \$00.75

ACM Transactions on Programming Languages and Systems, Vol. 4, No. 2, April
 1982, Pages 258-282.
}\bigskip

\subsecly{Abstract}

The unification problem in first-order predicate calculus is described in
general terms as the solution of a system of equations, and a nondeterministic
algorithm is given. A new unification algorithm, characterized by having the
acyclicity test efficiently embedded into it, is derived from the
nondeterministic one, and a PASCAL implementation is given. A comparison with
other well-known unification algorithms shows that the algorithm described here
performs well in all cases.

Categories and Subject Descriptors: F.2.2 [Analysis of Algorithms and Problem
Complexity]: Nonnumerical Algorithms and Problems--complexity of proof
procedures; F.4.1 [Mathematical Logic and Formal Languages]: Mathematical
Logic--mechanical theorem proving; 1.2.3 [Artificial Intelligence]: Deduction
and Theorem Proving--resolution

General Terms: Algorithms, Languages, Performance, Theory 

\secrel{INTRODUCTION}\label{mmalg1}

In its essence, the unification problem in first-order logic can be expressed as
follows: Given two terms containing some variables, find, if it exists, the
simplest substitution (i.e., an assignment of some term to every variable) which
makes the two terms equal. The resulting substitution is called the \term{most
general unifier} and is unique up to variable renaming.

Unification was first introduced by Robinson [17, 18] as the central step of the
inference rule called resolution. This single, powerful rule can replace all the
axioms and inference rules of the first-order predicate calculus and thus was
immediately recognized as especially suited to mechanical theorem provers. In
fact, a number of systems based on resolution were built and tried on a variety
of different applications [5]. Even though further research made it apparent
that resolution systems are difficult to direct during proof search and thus are
often prone to combinatorial explosion [6], new impetus to the research in this
area was given by Kowalski's idea of interpreting predicate logic as a
programming language [10]. Here predicate logic clauses are seen as procedure
declarations, and procedure invocation represents a resolution step. From this
viewpoint, theorem provers can be regarded as interpreters for programs written
in predicate logic, and this analogy suggests efficient implementations [3, 25].

Resolution, however, is not the only application of the unification algorithm.
In fact, its pattern matching nature can be exploited in many cases where
symbolic expressions are dealt with, such as, for instance, in interpreters for
equation languages [4, 11], in systems using a database organized in terms of
productions [19], in type checkers for programming languages with a complex type
structure [14], and in the computation of critical pairs for term rewriting
systems [9].

The unification algorithm constitutes the heart of all the applications listed
above, and thus its performance affects in a crucial way the global efficiency
of each. The unification algorithm as originally proposed can be extremely
inefficient; therefore, many attempts have been made to find more efficient
algorithms [2, 7, 13, 15, 16, 22]. Unification algorithms have also been
extended to the case of higher order logic [8] and to deal directly with
associativity and commutativity [20]. The problem was also tackled from a
computational complexity point of view, and linear algorithms were proposed
independently by Martelli and Montanari [13] and Paterson and Wegman [15].

In the next section we give some basic definitions by representing the
unification problem as the solution of a system of equations. A nondeterministic
algorithm, which comprehends as special cases most known algorithms, is then
defined and proved correct. In Section 3 we present a new version of this
algorithm obtained by grouping together all equations with some member in
common, and we derive from it a first version of our unification algorithm.

In Sections 4 and 5 we present the main ideas which make the algorithm
efficient, and the last details are described in Section 6 by means of a PASCAL
implementation.

Finally, in Section 7, the performance of this algorithm is compared with that
of two well-known algorithms, Huet's [7] and Paterson and Wegman's [15]. This
analysis shows that our algorithm has uniformly good performance for all classes
of data considered.


\secrel{UNIFICATION AS THE SOLUTION OF A SET OF EQUATIONS: A NONDETERMINISTIC
ALGORITHM}

In this section we introduce the basic definitions and give a few theorems which
are useful in proving the correctness of the algorithms. Our ay of stating the
unification problem is slightly more general than the classical one due to
Robinson [18] and directly suggests a number of possible solution methods.

Let
\[A= \bigcup_{i=0,1,..} A_i \quad (A_i \bigcap A_j = \varnothing, i \neq j)\] be
a ranked alphabet, where $A_i$ contains the $i$-adic function symbols
(the elements of $A_0$ are constant symbols). Furthermore, let $V$ be the
alphabet of the variables.
The \term{terms} are defined recursively as follows:

(1) constant symbols and variables are terms;

(2) if $t_1,..,t_n \ (n \geq 1)$ are terms and $f \in A_n$,
then $f(t_1,..,t_n)$ is a term.

A \term{substitution} $\vartheta$ is a mapping from variables to terms,
with $\vartheta(x)=x$ almost everywhere. A substitution can be represented
by a finite set of ordered pairs
$\vartheta={(t_1,x_1),(t_2,x_2),..,(t_m,x_m)}$
where $t_i$ are terms and $x_i$ are distinct variables,
$i = 1,..,m$. To apply a substitution $\vartheta$ to a term $t$, we
simultaneously substitute all occurrences in $t$ of every variable $x_i$ in a
pair $(t_i, x_i)$ of $\vartheta$ with the corresponding term $t_i$. We call the
resulting term $t_\vartheta$.

For instance, given a term $t = f(x_1, g(x_2, a)$ and a substitution 
$\vartheta = {(h(x_2),x_1),(b,x_2)}$, 
we have $t_\vartheta = f(f(x_2),g(b),a)$ and $t_{\vartheta\vartheta} =
f(h(b),g(b),a))$.

The standard unification problem can be written as an equation \[t'=t''\]

A solution of the equation, called a \term{unifier}, is any substitution
$\vartheta$, if it exists, which makes the two terms identical. For instance,
two unifiers of the equation $f(x_1,h(x_1),x_2)=f(g(x_3),x_4,x_3))$ are
$\vartheta_1={(g(x_3),x_1),(x_3,x_2),(h(g(x_3)),x_4)}$ and
$\vartheta_2={(g(a),x_1),(a,x_2),(a,x_3),(h(g(a)),x_4)}$.

In what follows it is convenient also to consider sets of equations
\[t'_j=t''_j, \quad j=1,..,k\]

Again, a \term{unifier} is any substitution which makes all pairs of terms
$t'_j,t''_j$ identical simultaneously.

Now we are interested in finding transformations which produce \emph{equivalent}
sets of equations, namely, transformations which preserve the sets of all
unifiers.
Let us introduce the following two transformations:

\paragraph{(1) Term Reduction.} Let

\begin{equation}\label{mm1}
f(t'_1,t'_2,..,t'_n)=f(t''_1,t''_2,..,t''_n), \quad f \in A_n
\end{equation} 
be an equation where both terms are not variables and where the two root
function symbols are equal. The new set of equations is obtained by replacing
this equation with the following ones: 
\begin{align}\label{mm2}
t'_1 &= t''_1\\
t'_2 &= t''_2\\
&.\\
&.\\
&.\\
t'_n &= t''_n
\end{align}
If $n = 0$, then $f$ is a constant symbol, and the equation is simply erased.

\paragraph{(2) Variable Elimination.} Let $x = t$ be an equation where $x$ is a
variable and $t$ is any term (variable or not). The new set of equations is
obtained by applying the substitution $\vartheta={(t,x)}$ to both terms of all
other equations in the set (without erasing $x = t$).

We can prove the following theorems: 

\paragraph{THEOREM 2.1.} \textit{Let $S$ be a set of equations and let
$f'(t'_1,..t'_n)=f''(t''_1,..,t''_n)$ be an equation of $S$. If $f' \neq f''$,
then $S$ has no unifier. Otherwise, the new set of equations $S'$, obtained by
applying term reduction to the given equation, is equivalent to $S$.}

\paragraph{PROOF.} If $f' \neq f''$, then no substitution can make the
two terms identical.
If $f' = f"$, any substitution which satisfies \ref{mm2} also satisfies
\ref{mm1}, and conversely for the recursive definition of term. $\square$

\paragraph{THEOREM 2.2.} \textit{Let $S$ be a set of equations, and let us apply
variable elimination to some equation $x = t$, getting a new set of equations
$S'$. If variable $x$ occurs in $t$ (but $t$ is not $x$), then $S$ has no
unifier; otherwise, $S$ and $S'$ are equivalent.}

\paragraph{PROOF.} Equation $x = t$ belongs both to $S$ and to $S'$, and thus
any unifier $\vartheta$ (if it exists) of $S$ or of $S'$ must unify $x$ and $t$;
that is, $x_\vartheta$ and $t_\vartheta$ are identical. Now let $t_1 = t_2$ be
any other equation of $S$, and let $t'_1 = t'_2$ be the corresponding equation
in $S'$. Since $t'_1$ and $t'_2$ have been obtained by substituting $t$ for
every occurrence of $x$ in $t_1$ and $t_2$, respectively, we have
$t_{1_\vartheta}=t'_{1_\vartheta}$ and $t_{2_\vartheta}=t'_{2_\vartheta}$. Thus,
any unifier of $S$ is also a unifier of $S'$ and vice versa. Furthermore, if
variable $x$ occurs in $t$ (but $t$ is not $x$), then no substitution
$\vartheta$ can make $x$ and $t$ identical, since $x_\vartheta$ becomes a
subterm of $t_\vartheta$, and thus $S$ has no unifier. $\square$

A set of equations is said to be \term{in solved form} iff it satisfies the
following conditions: 

(1) the equations are $x_j = t_j, j = 1,..,k$;

(2) every variable which is the left member of some equation occurs only there.

A set of equations in solved form has the obvious unifier
\[\vartheta = {(t_1,x_1),(t_2,x_2),..,(t_k,x_k)}\]

If there is any other unifier, it can be obtained as
\[0 = {(t,, x~), (t2, x2) .... , (tk, xk)} U a\]
where a is any substitution which does not rewrite variables xl .... , xk. Thus t~
is called a most general unifier (mgu ).  

The following nondeterministic algorithm shows how a set of equations can be
transformed into an equivalent set of equations in solved form. 

\paragraph{Algorithm 1 }\ \\

Given a set of equations, repeatedly perform any of the following transformations. If no
transformation applies, stop with success. 

(a) Select any equation of the form
\[t=x\]
where t is not a variable and x is a variable, and rewrite it as 
\[x=t.\]

(b) Select any equation of the form
\[X=X\] 
where x is variable, and erase it.  

(c) Select any equation of the form
\[t' = t"\]
where t' and t" are not variables. If the two root function symbols are different, stop with
failure; otherwise, apply term reduction. 

(d) Select any equation of the form
\[x=t\]
where x is a variable which occurs somewhere else in the set of equations and
where t \# x. If x occurs in t, then stop with failure; otherwise, apply variable elimination. 

As an example, let us consider the following set of equations:
\[g(x2) = xl;\]
\[f(xl, h(xl), x2) = f(g(x3), x4, x3).\]

By applying transformation (c) of Algorithm 1 to the second equation we get
\[g(x2) = xl;\]
\[xl = g(x3);\]
\[h(x~) = x4;\]
\[X2 =X3.\]

By applying transformation (d) to the second equation we get
\[g(x2) = g(xs);\]
\[xl = g(x3);\]
\[h(g(x3)) = x4;\]
\[X2 ~- X3.\]

We now apply transformation (c) to the first equation and transformation (a) to
the third equation:
\[X2 ~ X3\]
\[xl = g(x3);\]
\[Xa = h(g(x3));\]
\[X2 ----X3.\]

Finally, by applying transformation (d) to the first equation and transformation
(b) to the last equation, we get the set of equations in solved form:
\[X2 ~- X3 ;\]
\[xl = g(x3);\]
\[x4 = h(g(x3)).\]

Therefore, an mgu of the given system is
\[= {(g(x~), x~), (x3, x2), (h(g(x3)), x4)}.\] 

The following theorem proves the correctness of Algorithm 1. 

\paragraph{THEOREM 2.3}. Given a set of equations S,

(i) Algorithm 1 always terminates, no matter which choices are made.

(ii) If Algorithm 1 terminates with failure, S has no unifier. If Algorithm 1
terminates with success, the set S has been transformed into an equivalent
set in solved form. 
 
\paragraph{PROOF}.\\

(i) Let us define a function F mapping any set of equations S into a triple of
natural numbers (nl, n2, n3). The first number, n~, is the number of variables in
S which do not occur only once as the left-hand side of some equation. The
second number, n2, is the total number of occurrences of function symbols in S.
The third number, n3, is the sum of the numbers of equations in S of type x = x
and of type t = x, where x is a variable and t is not. Let us define a total ordering
on such triples as follows:
\begin{align*}
(n~, n~, n~) > (n~', n2, " n~')\ if\ &n~ > n~'\\
&or\ n~ = n~\ and\ n2 > n2\\
&or\ n~=n" 1 a n- 1 n2 ' ---- n2 "\ and\ n3 ' > n3.\\ 
\end{align*}   

With the above ordering, N 3 becomes a well-founded set, that is, a set where no
infinite decreasing sequence exists. Thus, if we prove that any transformation of
Algorithm 1 transforms a set S in a set S' such that F(S') < F(S), we have
proved the termination. In fact, transformations (a) and (b) always decrease n3
and, possibly, n~. Transformation (c) can possibly increase n3 and decrease nl,
but it surely decreases n2 (by two). Transformation (d) can possibly change n3
and increase n2, but it surely decreases n~. 

(ii) If Algorithm 1 terminates with failure, the thesis immediately follows from
Theorems 2.1 and 2.2. If Algorithm 1 terminates with success, the resulting set of
equations S' is equivalent to the given set S. In fact, transformations (a) and (b)
clearly do not change the set of unifiers, while for transformations (c) and (d) this
fact is stated in Theorems 2.1 and 2.2. Finally, S' is in solved form. In fact, if (a),
(b), and (c) cannot be applied, it means that the equations are all in the form
x = t, with t \# x. If (d) cannot be applied, that means that every v.arialSle
which is the left-hand side of some equation occurs only there.
$\square$\bigskip

The above nondeterministic algorithm provides a widely general version from
which most unification algorithms [2, 3, 7, 13, 15, 16, 18, 22-24] can be derived by
specifying the order in which the equations are selected and by defining suitable
concrete data structures. For instance, Robinson's algorithm [18] might be
obtained by considering the set of equations as a stack. 


\secrel{AN ALGORITHM WHICH EXPLOITS A PARTIAL ORDERING AMONG SETS
OF VARIABLES}\secdown
\secrel{Basic Definitions}

In this section we present an extension of the previous formalism to model our
algorithm more closely. We first introduce the concept of multiequation. A multiequation
is the generalization of an equation, and it allows us to group together
many terms which should be unified. To represent multiequations we use the
notation S -- M where the left-hand side S is a nonempty set of variables and the
right-hand side M is a multiset 1 of nonvariable terms. An example is
\[{xl, x2, x3} = (tl, t2).\]
\note{A multiset is a family of elements in which no ordering exists but in which many identical elements
may occur.}

The solution (unifier) of a multiequation is any substitution which makes all
terms in the left- and right-hand sides identical. 

A multiequation can be seen as a way of grouping many equations together.
For instance, the set of equations
\[Xl ---- X2;\]
\[X3 = Xl;\]
\[tl = Xl;\]
\[X2 ---- t2;\]
\[tl = t2\]
can be transformed into the above multiequation, since every unifier of this set
of equations makes the terms of all equations identical. To be more precise, given
a set of equations SE, let us define a relation RSE between pairs of terms as
follows: tl RSE t2 iff the equation tl = t2 belongs to SE. Let/tSE be the reflexive,
symmetric, and transitive closure of RSE. 

Now we can say that a set of equations SE corresponds to a multiequation
S = M iff all terms of SE belong to S U M and for every tr and ts E S U M we have
tr RSE t,. 

It is easy to see that many different sets of equations may correspond to a
given multiequation and that all these sets are equivalent. Thus the set of
solutions (unifiers) of a multiequation coincides with the set of solutions of any
corresponding set of equations. 

Similar definitions can be given for a set of multiequations Z by introducing a
relation Rz between pairs of terms which belong to the same multiequation. A set
of equations SE corresponds to a set of multiequations Z iff
\[ti/~SE tj ** ti Rz tj\]
for all terms t~, tj of SE or Z.  
 
\secrel{Transformations of Sets of Multiequations}

We now introduce a few transformations of sets of multiequations, which are
generalizations of the transformations presented in Section 2. 

We first define the common part and the frontier of a multiset of terms
(variables or not). The common part of a multiset of terms M is a term which,
intuitively, is obtained by superimposing all terms of M and by taking the part
which is common to all of them starting from the root. For instance, given the
multiset of terms
\[(f(xl, g(a, f(xs, b))), f(h(c), g(x2, f(b, xs))), f(h(x4), g(x6, x3))),\]
the common part is
\[f(xl, g(x2, x3)).\]

The frontier is a set of multiequations, where every multiequation is associated
with a leaf of the common part and consists of all subterms (one for each term of 
M) corresponding to that leaf. The frontier of the above multiset of terms is
\begin{verbatim}
{{x~} = (h(c), h(x4)),
{x2, x6} = (a),
{x3} = (f(xs, b), f(b, xD)).
\end{verbatim}

Note that if there is a clash of function symbols among some terms of a multiset
of terms M, then M has no common part and frontier. In this case the terms of M
are not unifiable. 

The common part and the frontier can be defined more precisely by means of
a function DEC which takes a multiset of terms M as argument and returns either
"failure," in which case M has neither common part nor frontier, or a pair (C(M),
F(M) ) where C(M) is the common part of M and F(M) is the frontier of M. 

In the definition of DEC we use the following notation: 

\begin{tabular}{l l}
head(t) & is the root function symbol of term t, for t ~ V. \\
Pi & is the ith projection, defined by\\&
\verb|\[Pi(f(tl .... ,tn))=ti for f~An and l\_<i\_n;\]|\\
make&is a function which transforms a multiset of terms M into a multiequa\\
multeq & tion whose left-hand side is the set of all variables in M and whose\\
&right-hand side is the multiset of all terms in M which \\&are not variables;
and \\
$U$ & is the union for multisets. \\
\end{tabular}
\begin{lstlisting}
DEC(M) = ff 3t ~ M, t E V
	then (t, {makemulteq(M)} )
	else if 3n, 3 f E A,, Yt E M, head(t) = f
		then if n ffi 0
		then ( f, O)
		else if Vi (1 __ i _ n), DEC(Mi) ~ failure
		where Mi -- OteM Pi(t)
		then (f(C(M1) ..... C(M,)), UTffil F(Mi))
		else failure
		else failure.   
\end{lstlisting}

We can now define the following transformation: 

Multiequation Reduction. Let Z be a set of multiequations containing a
multiequation S -- M such that M is nonempty and has a common part C and a
frontier F. The new set Z' of multiequations is obtained by replacing S = M with
the union of the multiequation S = (C) and of all the multiequations of F:
\[Z'ffi(Z- {SffiM})U{S=(C)} UF.\]

THEOREM 3.1. Let S = M (M nonempty) be a multiequation of a set Z of
multiequations. If M has no common part, or if some variable in S belongs to
the left-hand side of some multiequation in the frontier F of M, then Z has no 
unifier. Otherwise, by applying multiequation reduction to the multiequation
S = M we get an equivalent set Z' ofmultiequations. 

PROOF. If the common part of M does not exist, then the multiequation S -- M
has no unifier, since two terms should be made equal having a different function
symbol in the corresponding subterms. Moreover, if some variable x of S occurs
in some left-hand side of the frontier, then it also occurs in some term t of M, and
thus the equation x = t, with x occurring in t, belongs to a set of equations
equivalent to Z. But, according to Theorem 2.2, this set has no unifier. 

To prove that Z and Z' are equivalent, we show first that a unifier of Z is also
a unifier of Z'. In fact, if a substitution ~ makes all terms of M equal, it also
makes equal all the corresponding subterms, in particular, all terms and variables
which belong to left- and right-hand sides of the same multiequation in the
frontier. The multiequation S = (C) is also satisfied by construction. Conversely,
if ~ satisfies Z', then the multiequation S -- M is also satisfied. In fact, all terms
in S and M are made equal--in their upper part (the common part) due to the
multiequation S -- (C) and in their lower part (the subterms not included in the
common part) due to the set of multiequations F. $\square$

We say that a set Z of multiequations is compact iff
\begin{verbatim}
\[Y(S =M), (S' =M'} ~Z: SA S' = ~.\]
\end{verbatim}
We can now introduce a second transformation, which derives a compact set of
multiequations. 

Compactification. Let Z be a noncompact set of multiequations. Let R be a
relation between pairs of multiequations of Z such that iS = M) R iS' = M') iff
S n S' \# O, and let/t be the transitive closure of R. The relation/~ partitions
the set Z into equivalence classes. To obtain the final compact set Z', all multiequations
belonging to the same equivalence class are merged; that is, they are
transformed into single multiequations by taking the union of their left- and
right-hand sides. 

Clearly, Z and Z' are equivalent, because the relation /~z between pairs of
terms, defined in Section 3.1, does not change by passing from Z to Z'. 
  
\secrel{Solving Systems of Multiequations}

For convenience, in what follows, we want to give a structure to a set of
multiequations. Thus we introduce the concept of system of multiequations. A
system R is a pair (T, U), where T is a sequence and U is a set of multiequations
(either possibly empty), such that

(1) the sets of variables which constitute the left-hand sides of all multiequations
in both T and U contain all variables and are disjoint;

(2) the right-hand sides of all multiequations in T consist of no more than one
term; and

(3) all variables belonging to the left-hand side of some multiequation in T can
only occur in the right-hand side of any preceding multiequation in T. 

We now present an algorithm for solving a given system R of multiequations.
When the computation starts, the T part is empty, and every step of the following
Algorithm 2 consists of "transferring" a multiequation from the U part, that is,
the unsolved part, to the T part, that is, the triangular or solved part of R. When
the Upart of R is empty, the system is essentially solved. In fact, the solution can
be obtained by substituting the variables backward. Notice that, by keeping a
solved system in this triangular form, we can hope to find efficient algorithms for
unification even when the mgu has a size which is exponential with respect to the
size of the initial system. For instance, the mgu of the set of multiequations
\begin{verbatim} 
{{Xl} = ~,
{x~} = ~,
{x3} = 0,
{x4} = (h(x3, h(x2, x2)), h(h(h (xl, xl), x2), x3))} 
\end{verbatim}
is
\begin{verbatim}
{(h(xl, Xl), x2), (h(h(xl, Xl), h(Xl, Xl)), x3),
(h(h(h(Xl, Xl), h(xl, Xl)), h(h(xl, Xl), h(Xl, Xl))), X4)}. 
\end{verbatim} 
However, we can give an equivalent solved system with empty U part and whose
T part is
\begin{verbatim}
({x,} --- (h(x3, x3)),
{x3} = (h(x2, x2)),
{X2) = (h(Xl, xl)),
{xl} = o), 
\end{verbatim}
from which the mgu can be obtained by substituting backward. 

Given a system R = (T, U) with an empty T part, an equivalent system with
an empty U part can be computed with the following algorithm. 

\paragraph{Algorithm 2}\ \\
\begin{enumerate}
  \item 
(1) repeat
\begin{enumerate}
  \item 
(1.1) Select a multiequation S = M of U with M \# ~5.
  \item 
(1.2) Compute the common part C and the frontier F ofM. IfM has no common part,
stop with failure (clash).
  \item 
(1.3) If the left-hand sides of the frontier of M contain some variable of S, stop with
failure (cycle).
  \item 
(1.4) Transform U using multiequation reduction on the selected mnltiequation and
compactification.
  \item 
(1.5) Let S = \{xl ..... Xn). Apply the substitution ~ = {(C, xl) ..... (C, x,)}
to all terms in the right-hand side of the multiequations of U.
  \item 
(1.6) Transfer the multiequation S = (C) from U to the end of T.
until the U part of R contains only multiequations, if any, with empty right-hand
sides.
\end{enumerate}
  \item 
(2) Transfer all the mnltiequations of U (all with M = ~D) to the end of T, and stop with
success. 
\end{enumerate}
 

\secrel{The Unification Algorithm}

Looking at Algorithm 2, it is clear that the main source of complexity is step
(1.5), since it may make many copies of large terms. In the following--and this is
the heart of our algorithm--we show that, if the system has unifiers, then there
always exists a multiequation in U (if not empty) such that by selecting it we do
not need step (1.5) of the algorithm, since the variables in its left-hand side do
not occur elsewhere in U. We need the following definition. 

Given a system R, let us consider the subset Vu of variables obtained by making
the union of all left-hand sides Si of the multiequations in the U part of R. Since
the sets Si are disjoint, they determine a partition of Vu. We now define a relation
on the classes Si of this partition: we say that Si < Sj iff there exists a variable
of Si occurring in some term of Mj, where Mj is the right-hand side of
the multiequation whose left-hand side is Sj. We write <* for the transitive closure
of <. 

Now we can prove the following theorem and corollary. 

THEOREM 3.3. If a system R has a unifier, then the relation <* is a partial
ordering.
PROOF. If Si < \$i, then, in all unifiers of the system, the term substituted
for every variable in Si must be a strict subterm of the term substituted for every
variable in Sj. Thus, if the system has a unifier, the graph of the relation < cannot
have cycles. Therefore, its transitive closure must be a partial ordering.
$\square$

COROLLARY. If the system R has a unifier and its U part is nonempty, there
exists a multiequation S ffi M such that the variables in S do not occur elsewhere
inU.

PROOF. Let S = M be a multiequation such that S is "on top" of the partial
ordering < * (i.e., ~3Si, S < Si). The variables in S occur neither in the other lefthand
sides of U (since they are disjoint) nor in any right member Mi of U, since
otherwise S < Si. $\square$

We can now refine the nondeterministic Algorithm 2 giving the general version
of our unification algorithm for a system of multiequations R = (T, U). 

Algorithm 3: UNIFY, the Unification Algorithm
\begin{enumerate}
  \item 
(1) repeat
\begin{enumerate}
  \item 
(1.1) Select a multiequation S = M of U such that the variables in S do not occur
elsewhere in U. If a multiequation with this property does not exist, stop with
failure (cycle).
  \item 
(1.2) ifMis empty
then transfer this multiequation from U to the end of T.
else begin
\begin{enumerate}
  \item 
(1.2.1) Compute the common part C and the frontier F of M. If M has
no common part, stop with failure (clash).
  \item 
(1.2.2) Transform U using multiequation reduction on the selected
multiequation and compactification.
  \item 
(1.2.3) Transfer the multiequation S = (C) from U to the end of T.
end
until the U part of R
\end{enumerate}
\end{enumerate}
(2) stop with success.  
\end{enumerate}
 
A few comments are needed. Besides step (1.5) of Algorithm 2, we have also
erased step (1.3) for the same reason. Furthermore, in Algorithm 2 we were forced
to wait to transfer multiequations with empty right-hand sides since substitution
in that case would have required a special treatment. 

By applying Algorithm UNIFY to the system which was previously solved with
Algorithm 2, we see that we must first eliminate variable x, then variable x,, then
variables x2 and x3 together, and, finally, variables x4 and x5 together, getting the
following final system:
\begin{verbatim}
U:~
T: ({x} = (f(xl, xl, x2, x,)),
{Xl} = (g(x2, x3)),
{x2, x3} = (h(a, x4)),
{x,, xs} = (b)).  
\end{verbatim}

Note that the solution obtained using Algorithm UNIFY is more concise than
the solution previously obtained using Algorithm 2, for two reasons. First,
variables x2 and x3 have been recognized as equivalent; second, the right member
of x~ is more factorized. This improvement is not casual but is intrinsic in the
ordering behavior of Algorithm UNIFY.

To summarize, Algorithm UNIFY is based mainly on the two ideas of keeping
the solution in a factorized form and of selecting at each step a multiequation in
such a way that no substitution ever has to be applied. Because of these two
facts, the size of the final system cannot be larger than that of the initial one.
Furthermore, the operation of selecting a multiequation fails if there are cycles
among variables, and thus the so-called occur-check is built into the algorithm,
instead of being performed at the last step as in other algorithms [2, 7]. 
\secup
\secrel{ EFFICIENT MULTIEQUATION SELECTION }\label{mmalg4}

In this section we show how to implement efficiently the operation of selecting a
multiequation "on top" of the partial ordering in step (1.1) of Algorithm 3. 

The idea is to associate with every multiequation a counter which contains the
number of other occurrences in U of the variables in its left-hand side. This
counter is initialized by scanning the whole U part at the beginning. Of course, a
multiequation whose counter is set to zero is on top of the partial ordering. 

For instance, let us again consider system (3):
\begin{verbatim}
U: {[0] {x} = (f(xl,g(x2, x3), x2, b), f(g(h(a, x~), xe), xl,
h(a, x4), x4)),
[2] {xl} = 6,
[3] {x2} = 6,
[1] (x3} = 6,
[2] {x4} = 6,
[1] {xa} -- 6};
T:(). 
\end{verbatim}

Here square brackets enclose the counters associated with each multiequation.
Since only the first multiequation has its counter set to zero, it is selected to be
transferred. Counters of the other multiequations are easily updated by 
decrementing them whenever an occurrence of the corresponding variable appears in
the left-hand side of a multiequation in the frontier computed in step (1.2.1).
When two or more multiequations in U are merged in the compactification phase,
the counter associated with the new multiequation is obviously set to a value
which is the sum of the contents of the old counters. 

The next steps are as follows:
\begin{verbatim}
U: {[0] (Xl} = (g(h(a, x~), x2), g(x2, x3)),
[2] {x2} = (h(a, x4)),
[1] (x~) = o,
[1] {x,} = (b),
[1] {x~} = ~};
T: ( (x} = (f(xl, x~, x2, x4))).
U: {[0] {x2, x3} = (h(a, x4), h(a, x~)),
[1] {x4} = (b),
[1] {x~} = o};
T: ({x} = (f(x,, x,, x2, x,)),
{x,} = (g(x2, x3))).
U: {[0] {x4, xs} = (b)};
T: ({x} = (f(x,, xl, x2, x4)),
{x,} = (g(x2, x3)),
{x2, x3} = (h(a, x4))).
U: ~;
T: ({x} = (f(xl, xl, x2, x4)),
{xl} = (g(x2, x3)),
{x2, x3} = (h(a, x4)),
{x4, x~} = (b)). 
\end{verbatim}

\secrel{IMPROVING THE UNIFICATION ALGORITHM FOR NONUNIFYING DATA}\label{mmalg5}

In the case of nonunifying data, Algorithm 3 can stop with failure in two ways:
either in step (1.1) if a cycle has been detected, or in step (1.2.1) if a clash occurs.
In this section we show how to anticipate the latter kind of failure without
altering the structure of the algorithm. 

Let us first give the following definition. Two terms are consistent iff either at
least one of them is a variable or they are both nonvariable terms with the same
root function symbol and pairwise consistent arguments. This definition can be
extended to the case of more than two terms by saying that they are consistent
iff all pairs of terms are consistent. For instance, the three terms f(x, g(a, y)),
f(b, x), and f(x, y) are consistent although they are not unifiable. 

We now modify Algorithm UNIFY by requiring all terms in the right-hand
side of a multiequation to be consistent, for every multiequation. Thus, we stop
with clash failure as soon as this requirement is not Satisfied. This new version of
the algorithm is still correct since, if there are two inconsistent terms in the same
multiequation, they will never unify. 

In this way, clashes are detected earlier. In fact, in the Algorithm 3 version of
UNIFY a clash can be detected while computing the common part and the
frontier of the right-hand side of the selected multiequation, whereas in the new
version of UNIFY the same error is detected in the compactification phase of a
previous iteration. 

An efficient implementation of the consistency check when two multiequations
are merged requires a suitable representation for right-hand sides of multiequations.
Thus, instead of choosing the obvious solution of representing every righthand
side as a list of terms, we represent it as a multiterm, defined as follows. 

A multiterm can be either empty or of the form f(P1 ..... Pn) where f E A, and
Pi (i = 1 ..... n) is a pair (Si, Mi) consisting of a set of variables Si and a
multiterm ]Vii. Furthermore, Si and Mi cannot both be empty. 

For instance, the multiset of consistent terms
\[(f(x, g(a, y)), f(b, x), f(x, y))\]
can be represented with the multiterm
\[f(((x), b), ({x,y},g((O, a), ((y), ~)))).\]

 By representing right-hand sides in this way we have no loss of information,
since the only operations which we have to perform on them are the operation of
merging two right-hand sides and the operation of computing the common part
and the frontier, which can be described as follows:
\begin{verbatim}
MERGE (M', M " ) =
case M' of
O: M";
f'((Si M~), , tS' M' ~"
case M" of
O: M';
f"((S~',M~') ..... (Sn", M~")):
iff' -- f" and MERGE(M~, M[') # failure (i -- 1 ..... n)
then f'((Si O S~', MERGE(MI, M;')) .....
(S~ ~J S t'n, MERGE(M~', M,,))"~
else failure
endcase
endcase
COMMONPART(f((S1, M1) ..... (S,, Mn))) = f(P1, --., Pn)
where Pi = if Si = ~ then COMMONPART(Mi)
else ANYOF(SD (i = 1 .... , n) 
\end{verbatim}
\begin{verbatim}
where function ANYOF(S~) returns an element of set Si. 
\end{verbatim}
\begin{verbatim}
UPart = record
MultEqNumber: Integer;,
ZeroCounterMultEq, Equations: ListOfMultEq
end;
System = TPSystem;
PSystem = record
T: ListOfMultEq;
U: UPart
end;
MultiTerm = ~PMultiTerm;
PMultiTerm = record
Fsymb: FunName;
Args: ListOfTempMultEq
end;
MultiEquation = ~PMultiEquation;
PMultiEquation = record
Counter, VarNumber: Integer;
S: ListOfVariables;
M: Multi Term
end;
TempMultEq = ~PTempMultEq;
PTempMultEq = record
S: QueueOfVariables;
M: MultiTerrn
end;
Variable = TPVariable;
P Variable = record
Name: VarName;
M: MultiEquation
end; 
\end{verbatim}
\begin{verbatim}
FRONTIER(f((S,, M1) ..... (Sn, Mn) )) = F1 [..J ... (.J Fn
where Fi = if Si = O then FRONTIER(M/)
else {Si = Mi} (i = 1, ..., n). 
\end{verbatim}

Note that the common part and the frontier are defined only for nonempty
multiterms and that they always exist. 

\secrel{IMPLEMENTATION}\label{mmalg6}

In order to describe the last details of our algorithm, we present here a PASCAL
implementation. In Figure 1 we have the definitions of data types. All data
structures used by the algorithm are dynamically created data structures connected
through pointers. The UPart of a system has two lists of multiequations:
Equations, which contains all initial multiequations, and ZeroCounterMultEq,
which contains all multiequations with zero counter. Furthermore, the field
MultEqNumber contains the number of multiequations in the UPart. A multiequation,
besides having the fields Counter, S, and M, has a field VarNumber,
which contains the number of variables in S and is used during compactification.
The pairs Pi = (S i, Mi), which are the arguments of a multiterm, have type
TempMultEq. Finally, all occurrences of a variable point to the same Variable
object, which points to the multiequation containing it in its left-hand side. 

\paragraph{Figure 2}
\begin{verbatim}
procedure Unify(R: System);
var Mult: MultiEquation ;
Frontier: ListOf TempMultEq ;
begin
repeat
SelectMultiEquation(R ~. U, Mult);
if not(Mult~.M=Nil) then
begin
Frontier := Nil;
Reduce(Multi.M, Frontier);
Compact(Frontier, R ~. U)
end;
R ~.T := NewListOfMultEq(Mult, R ~.T)
until R ~. U.MultEqNumber = 0
end (*Unify*);
\end{verbatim}

\paragraph{Figure 3}
\begin{verbatim}
procedure SelectMultiEquation(var U: UPart; var Mult: MultiEquation);
begin
if U.ZeroCounterMultEq = Nil then fail('cycle');
Mult := U~eroCounterMultEq~. Value;
U~eroCounterMultEq := U.ZeroCounterMultEqT.Next;
U.MultEqNumber := U.MultEqNumber - 1
end ( * SelectMultiEquation *);
\end{verbatim}

ject, which points to the multiequation containing it in its left-hand side.
The types "ListOf... ," not given in Figure 1, are all implemented as records
with two fields: Value and Next. Finally, QueueOfVariables is an abstract type
with operations CreateListOfVars, IsEmpty, HeadOf, RemoveHead, and Append,
which have a constant execution time. 

In Figure 2 we rephrase Algorithm UNIFY as a PASCAL procedure. Procedure
SelectMultiEquation selects from the UPart of the system a multiequation which
is "on top" of the partial ordering, by taking it from the ZeroCounterMultEq list.
Its implementation is given in Figure 3. 

Procedure Reduce, given in Figure 4, computes the common part and the
frontier of the selected multiequation. This procedure modifies the right-hand
side of this multiequation so that it contains directly the common part. Note that
the frontier is represented as a list of TempMultEq instead of as a list of
multiequations. 

Finally, in Figure 5 we give procedure Compact, which performs compactification
by repeatedly merging multiequations. When two multiequations are
merged, one of them is erased, and thus all pointers to it from its variables must
be moved to the other multiequation. To minimize the computing cost, we always
erase the multiequation with the smallest number of variables in its left-hand
side. Procedure MergeMultiTerms is given in Figure 6. 

A detailed complexity analysis of a similar implementation is given in [13].
There it is proved that an upper bound to execution time is the sum of two terms,
one linear with the total number of symbols in the initial system and another one
n log n with the number of distinct variables. 

\paragraph{Figure 4}
\begin{verbatim}
procedure Reduce(M: MultiTerm; var Frontier: ListOfTernpMultEq);
var Arg" ListOfTempMultEq;
begin
Arg := MT.Args;
while not(Arg = Nil) do
begin
if IsEmpty(Arg T. Value T.S) then Reduce(ArgO. Value T.M, Frontier)
else
begin
Frontier := NewListOfTempMultEq(Arg T. Value, Frontier);
ArgT. Value := NewTempMultEq( CreateQueueOfVars(HeadOf(Arg~. ValueT.S) ) , Nil)
end;
Arg := ArgT.Next
end
end (*Reduce*); 
\end{verbatim}

Here we want only to point out that the nonlinear behavior stems from the
operation described above of moving all pointers directed from variables to
multiequations, whenever two multiequations are merged. To see how this can
happen, let us consider the problem of unifying the two terms
\[f(xl, x3, xs, xT, xl, xs, xl)\]
and
\[f(x2, x4, x6, x8, x3, x7, x5).\]

During the first iteration of Unify we get a frontier whose multiequations are
the pairs (xl, x2), (x3, x4), (x~, x6), (xT, xs), (xl, x3), (xs, xT), and (xl, xs). By
executing Compact with this frontier, we see that it moves one pointer for each
of the first four elements of the frontier, two pointers for each of the next two
elements, and four pointers for the last element. Thus, it has an n log n complexity. 

As a final remark, we point out that we might modify the worst-case behavior
of our algorithm with a different implementation of the operation of multiequation
merging. In fact, we might represent sets of variables as trees instead of as lists,
and we might use the well-known UNION-FIND algorithms [1] to add elements
and to access them. In this case the complexity would be of the order of m. G(m),
where G is an extremely slowly growing function (the inverse of the Ackermann
function). However, m would be, in this case, the number of variable occurrences.  
\secrel{COMPARISONS WITH OTHER ALGORITHMS}\label{mmalg7}

In this section we compare the performance of our algorithm with that of two
well-known algorithms: Huet's algorithm [7], which has an almost linear time
complexity, and Paterson and Wegman's algorithm [15], which is theoretically
the best having a linear complexity.

\paragraph{Figure 5}
\begin{verbatim}
procedure Compact(Frontier: ListOfTempMultEq; var U: UPart);
var Vars: QueueOfVariables;
V: Variable;
Mult, Mult 1: MultiEquation ;
procedure MergeMultEq(var Mult: MultiEquation ; Mult l: MultiEquation );
vat Multt: MultiEquation;
V: Variable;
Vars : L istOfVariab les ;
begin
if not(Mult = Mult 1) then
begin
if Mult T. VarNumber < Mult 1T. VarNumber then
begin
Multt := Mult;
Mult := Multl;
Mult 1 := Multt
end;
MultT.Counter := MultT.Counter + Multl~.Counter;
Mult T. VarNumber := Mult ~. VarNumber + Mult 1 T. VarNumber;
Vars := Mult l T.S;
repeat
V := Vars'~.Value;
Vars := VarsT.Next;
V ~.M := Mult;
Mult T.S := NewListOfVariables( V, Mult T.S)
until Vars = Nil;
MergeMultiTerms(MultT.M, Mult l T.M);
U.MultEqNumber := U.MultEqNumber- 1
end
end (*MergeMultEq*);
begin
while not(Frontier = Nil) do
begin
Vars := Frontier T. ValueT.S;
V := HeadOf(Vars);
RemoveHead( Vars);
Mult := VT.M;
MultT.Counter := MultT.Counter - 1;
while not IsEmpty(Vars) do
begin
V := HeadOf(Vars);
RemoveHead( Vars);
Multl :-- VT.M;
Mult l T.Counter := Mult l ~.Counter - 1;
MergeMultEq(Mult, Mult 1)
end;
MergeMulti Terms(Mult T.M, Frontier T. Value~.M ) ;
ifMultT.Counter = 0 then
U.ZeroCounterMultEq := NewListOfMultEq(Mult, U.ZeroCounterMultEq);
Frontier := FrontierT.Next
end
end (*Compact*); 
\end{verbatim} 

\paragraph{Figure 6}
\begin{verbatim}
procedure MergeMultiTerms(var M: MultiTerm ; MI: MultiTerm);
var Arg, Argl: ListOfTempMultEq;
begin
ifM = Nil then M := M1
else if not(M1 = Nil) then
begin
if not (M "f .Fsymb = M l ~.Fsymb) then fail(' clash' )
else
begin
Arg := M~.Args;
Argl := MIT.Args;
while not(Arg = Nil) do
begin
Append(Arg~. Value~.S, Argl ~. Value~.S);
MergeMultiTerms(Arg~. Value~.M, Argl ~. Value~.M );
Arg := ArgT.Next;
Argl := ArglT.Next
end
end
end
end (*MergeMultiTerms*); 
\end{verbatim}

As an example of the assertion made at the end of Section 2, let us give a
sketchy description of the two algorithms using the terminology of this paper.
Both algorithms deal with sets of multiequations whose left-hand sides are
disjoint and whose right-hand sides consist of only one term of depth one, that is, 
of the form f(x,, ..., x,) where x, ..... Xn are variables. For instance, 
\begin{verbatim}
{x,} = f(x~, x3, x,); 
{x2} --- a;
(xa} = g(x2);
(x4} = a;
(x5} ffi f(x6, xT, xs);
{x6} = a;
{x7} = g(xa);
(4)
{xs} = O. 
\end{verbatim}

Furthermore, we have a set S of equations whose left- and right-hand sides are 
variables; for instance, 
\begin{verbatim}
S: {x, = xs}. 
\end{verbatim}

A step of both algorithms consists of choosing an equation from S, merging the
two corresponding multiequations, and adding to S the new equations obtained
as the outcome of the merging. For instance, after the first step we have
\begin{verbatim}
{x,, xs} = f(x2, x~, x4);
{X2} = a;
{x3) = g(x2);
{x4} = a;
{x~} = a;
{xv} = g(xa);
{x8 } = O;
S: {x2 = x6, x3 ffi x7, x4 = xs). 
\end{verbatim}

The two algorithms differ in the way they select the equation from S. In Huet's
algorithm S is a list; at every step, the first element of it is selected, and the new
equations are added at the end of the list. The algorithm stops when S is empty,
and up to this point it has not yet checked the absence of cycles. Thus, there is
a last step which checks whether the final multiequations are partially ordered. 

The source of the nonlinear behavior of this algorithm is the same as for our
algorithm, that is, the access to multiequations after they have been merged. To
avoid this, Paterson and Wegman choose to merge two multiequations only when
their variables are no longer accessible. For instance, from (5) their algorithm
selects x3 = x7 because x2 and xs are still accessible from the third and sixth
multiequation, respectively, getting
\begin{verbatim}
{xl, xs} = f(x2, x3, x,);
{x2} = a;
{x3, xT} = g(x2);
{x4} = a;
{x6} =- a;
{xs} = O;
S: {x2 = xs, x2 = x6, x4 = xs}. 
\end{verbatim}

To select the multiequations to be merged, this algorithm "climbs" the partial
ordering among multiequations until it finds a multiequation which is "on top";
thus the detection of cycles is intrinsic in this algorithm. 

Let us now see how our algorithm works with the above example. The initial
system of multiequations is
\begin{verbatim}
U: {[0] {Xl, X5} = f(( {x2, x6}, O), ({x3, xT}, gD), ({x4, xs}, ~)),
[2] {x2} -= a,
[1] {x3} = g(({x2), O)),
[1] (x4} = a,
[1] {x6} = a,
[1] {xT} = g({{xs}, O)),
[2] {xs ) = ~};
T: (). 
\end{verbatim}

The next step is
\begin{verbatim}
U: {[1] {x2, x6} = a,
[0] {x3, xT) = g(({x2, xs}, ~)),
[1] {x4, xs} = a};
T: ((x~, xs} = f(x2, x~, x4)),
\end{verbatim}
and so on. 

In this algorithm the equations containing the pairs of variables to be unified
are kept in the multiterms, and the mergings are delayed until the corresponding
multiequation is eliminated. 

An important difference between our algorithm and the others is that our
algorithm may use terms of any depth. This fact entails a gain in efficiency,
because it is certainly simpler to compute the common part and the frontier of
deep terms than to merge multiequations step by step. Note, however, that this
feature might also be added to the other algorithms. For instance, by adding the
capability of dealing with deep terms to Paterson and Wegman's algorithm, we
essentially obtain a linear algorithm which was independently discovered by the
authors [13]. 

In order to compare the essential features of the three algorithms, we notice
that they can stop either with success or with failure for the detection of a cycle
or with failure for the detection of a clash. Let Pm, Pc, and Pt be the probabilities
of stopping with one of these three events, respectively. We consider three
extreme cases: 

(1) Pm >> Pc, Pt (very high probability of stopping with success). Paterson
and Wegman's algorithm is asymptotically the best, because it has a linear
complexity whereas the other two algorithms have a comparable nonlinear
complexity. 

However, in a typical application, such as, for example, a theorem prover, the
unification algorithm is not used for unifying very large terms, but instead it is
used a great number of times for unifying rather small terms each time. In this
case we cannot exploit the asymptotically growing difference between linear and
nonlinear algorithms, and the computing times of the three algorithms will be
comparable, depending on the efficiency of the implementation. 

An experimental comparison of these algorithms, together with others, was
carried out by Trum and Winterstein [21]. The algorithms were implemented in
the same language, PASCAL, with similar data structures, and tried on five
different classes of unifying test data. Our algorithm had the lowest running time
for all test data. In fact, our algorithm is more efficient than Huet's because it
does not need a final acyclicity test, and it is more efficient than Paterson and
Wegman's because it needs simpler data structures. 

(2) Pc >> Pt >> Pm (very high probability of detecting a cycle). Paterson and
Wegman's algorithm is the best because it starts merging two multiequations
only when it is sure that there are no cycles above them. Our algorithm is also
good because cycle detection is embedded in it. In contrast, Huet's algorithm
must complete all mergings before being able to detect a cycle, and thus it has a
very poor performance. 

(3) Pt >> Pc >> Pm (very high probability of detecting a clash). Huet's algorithm
is the best because, if it stops with a clash, it has not paid any overhead for
cycle detection. Our algorithm is better than Paterson and Wegman's because
clashes are detected during multiequation merging and because our algorithm
may merge some multiequations earlier, like {x2, x6} and {x4, Xs} in the above
example. On the other hand, mergings which are delayed by our algorithm, by
putting them in multiterms, cannot be done earlier by the other algorithm
because they refer to multiequations which are still accessible. The difference in
the performance of the two algorithms may become quite large if terms of any
depth are allowed.
\secrel{CONCLUSION}\label{mmalg8}

A new unification algorithm has been presented. Its performance has been
compared with that of other well-known algorithms in three extreme cases: high
probability of stopping with success, high probability of detecting a cycle, and
high probability of detecting a clash. Our algorithm was shown to have a good
performance in all the cases, and thus presumably in all the intermediate cases,
whereas the other algorithms had a poor performance in some cases. 

Most applications of the unification algorithm, such as, for instance, a resolution
theorem prover or the interpreter of an equation language, require repeated use
of the unification algorithm. The algorithm described in this paper can be very
efficient even in this case, as the authors have shown in [12]. There they have
proposed to merge this unification algorithm with Boyer and Moore's technique
for storing shared structures in resolution-based theorem provers [3] and have
shown that, by using the unification algorithm of this paper instead of the
standard one, an exponential saving of computing time can be achieved. Furthermore,
the time spent for initializations, which might be heavy for a single
execution of the unification algorithm, is there reduced through a close integration
of the unification algorithm into the whole theorem prover. 
\secly{REFERENCES}
\begin{enumerate}
  \item
1. AHO, A.V., HOPCROFT, J.E., AND ULLMAN, J.D. The Design and Analysis of Computer Algorithms.
Addison-Wesley, Reading, Mass., 1974.
  \item
2. BAXTER, L.D. A practically linear unification algorithm. Res. Rep. CS-76-13, Dep. of Applied
Analysis and Computer Science, Univ. of Waterloo, Waterloo, Ontario, Canada.
  \item
3. BOYER, R.S., AND MOORE, J.S. The sharing of structure in theorem-proving programs. In
Machine Intelligence, vol. 7, B. Meltzer and D. Michie (Eds.). Edinburgh Univ. Press, Edinburgh,
Scotland, 1972, pp. 101-116.
  \item
4. BURSTALL, R.M., AND DARLINGTON, J. A transformation system for developing recursive programs.
J. ACM 24, 1 (Jan. 1977), 44-67.
  \item
5. CHANG, C.L., AND LEE, R.C. Symbolic Logic and Mechanical Theorem Proving. Academic
Press, New York, 1973.
  \item
6. HEWITT, C. Description and Theoretical Analysis (Using Schemata) of PLANNER: A Language
for Proving Theorems and Manipulating Models in a Robot. Ph.D. dissertation, Dep. of Mathematics,
Massachusetts Institute of Technology, Cambridge, Mass., 1972.
  \item
7. HUET, G. R6solution d'6quations dans les langages d'ordre 1, 2 ..... 0:. Th~se d'6tat, Sp6cialit6
Math~matiques, Universit~ Paris VII, 1976.
  \item
8. HUET, G.P. A unification algorithm for typed ?,-calculus. Theor. Comput. Sci. 1, 1 (June 1975),
27-57.
  \item
9. KNUTH, D.E., AND BENDIX, P.B. Simple word problems in universal algebras. In Computational
Problems in Abstract Algebra, J. Leech (Ed.). Pergamon Press, Eimsford, N.Y., 1970, pp. 263-297.
  \item
10. KOWALSKI, R. Predicate logic as a programming language. In Information Processing 74,
Elsevier North-Holland, New York, 1974, pp. 569-574.
  \item
11. LEVI, G., AND SIROVICH, F. Proving program properties, symbolic evaluation and logical procedural
semantics. In Lecture Notes in Computer Science, vol. 32: Mathematical Foundations of
Computer Science 1975. Springer-Verlag, New York, 1975, pp. 294-301.
  \item
12. MARTELLI, A., AND MONTANARI, U. Theorem proving with structure sharing and efficient
unification. Internal Rep. S-77-7, Ist. di Scienze della Informazione, University of Pisa, Pisa, Italy;
also in Proceedings of the 5th International Joint Conference on Artificial Intelligence, Boston,
1977, p. 543.
  \item
13. MARTELLI, A., AND MONTANARI, V. Unification in linear time and space: A structured 
presentation. Internal Rep. B76-16, Ist. di Elaborazione delle Informazione, Consiglio Nazionale delle
Ricerche, Pisa, Italy, July 1976.
  \item
14. MILNER, R. A theory of type polymorphism in programming. J. Comput. Syst. Sci. 17, 3 (Dec.
1978), 348-375.
  \item
15. PATERSON, M.S., AND WEGMAN, M.N. Linear unification. J. Comput. Syst. Sci. 16, 2 (April
1978), 158-167.
  \item
16. ROBINSON, J.A. Fast unification. In Theorem Proving Workshop, Oberwolfach, W. Germany,
Jan. 1976.
  \item
17. ROBINSON, J.A. Computational logic: The unification computation. In Machine Intelligence,
vol. 6, B. Meltzer and D. Michie (Eds.). Edinburgh Univ. Press, Edinburgh, Scotland, 1971, pp.
63-72.
  \item
18. ROBINSON, J.A. A machine-oriented logicbased on the resolution principle. J. ACM 12, 1 (Jan.
1965), 23-41.
  \item
19. SHORTLIFFE, E.H. Computer-Based Medical Consultation: MYCIN. Elsevier North-Holland,
New York, 1976.
  \item
20. STICKEL, M.E. A complete unification algorithm for associative-commutative functions. In
Proceedings of the 4th International Joint Conference on Artificial Intelligence, Tbilisi, U.S.S.R.,
1975, pp. 71-76.
  \item
21. TRUM, P., AND WINTERSTEIN, G. Description, implementation, and practical comparison of
unification algorithms. Internal Rep. 6/78, Fachbereich Informatik, Univ. of Kaiserlautern, W.
Germany.
  \item
22. VENTURINI ZILLI, M. Complexity of the unification algorithm for first-order expressions. Calcolo
12, 4 (Oct.-Dec. 1975), 361-372.
  \item
23. VON HENKE, F.W., AND LUCKHAM, D.C. Automatic program verificationIII: A methodology for
verifying programs. Stanford Artificial Intelligence Laboratory Memo AIM-256, Stanford Univ.,
Stanford, Calif., Dec. 1974.
  \item
24. WALDINGER, R.J., AND LEVITT, K.N. Reasoning about programs. Artif. Intell. 5, 3 (Fall 1974),
235-316.
  \item
25. WARREN, D.H.D., PEREIRA, L.M., AND PEREIRA, F. PROLOG--The language and its implementation
compared with LISP. In Proceedings of Symposium on Artificial Intelligence and
Programming Languages, Univ. of Rochester, Rochester, N.Y., Aug. 15-17, 1977. Appeared as
joint issue: SIGPLAN Notices (ACM) 12, 8 (Aug. 1977), and SIGART Newsl. 64 (Aug. 1977),
109-115.
  
\end{enumerate}

Received September 1979; revised July 1980 and September 1981; accepted October 1981 


\secup