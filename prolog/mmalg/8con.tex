\secrel{CONCLUSION}\label{mmalg8}

A new unification algorithm has been presented. Its performance has been
compared with that of other well-known algorithms in three extreme cases: high
probability of stopping with success, high probability of detecting a cycle, and
high probability of detecting a clash. Our algorithm was shown to have a good
performance in all the cases, and thus presumably in all the intermediate cases,
whereas the other algorithms had a poor performance in some cases. 

Most applications of the unification algorithm, such as, for instance, a resolution
theorem prover or the interpreter of an equation language, require repeated use
of the unification algorithm. The algorithm described in this paper can be very
efficient even in this case, as the authors have shown in [12]. There they have
proposed to merge this unification algorithm with Boyer and Moore's technique
for storing shared structures in resolution-based theorem provers [3] and have
shown that, by using the unification algorithm of this paper instead of the
standard one, an exponential saving of computing time can be achieved. Furthermore,
the time spent for initializations, which might be heavy for a single
execution of the unification algorithm, is there reduced through a close integration
of the unification algorithm into the whole theorem prover. 