\secly{REFERENCES}
\begin{enumerate}
  \item
1. AHO, A.V., HOPCROFT, J.E., AND ULLMAN, J.D. The Design and Analysis of Computer Algorithms.
Addison-Wesley, Reading, Mass., 1974.
  \item
2. BAXTER, L.D. A practically linear unification algorithm. Res. Rep. CS-76-13, Dep. of Applied
Analysis and Computer Science, Univ. of Waterloo, Waterloo, Ontario, Canada.
  \item
3. BOYER, R.S., AND MOORE, J.S. The sharing of structure in theorem-proving programs. In
Machine Intelligence, vol. 7, B. Meltzer and D. Michie (Eds.). Edinburgh Univ. Press, Edinburgh,
Scotland, 1972, pp. 101-116.
  \item
4. BURSTALL, R.M., AND DARLINGTON, J. A transformation system for developing recursive programs.
J. ACM 24, 1 (Jan. 1977), 44-67.
  \item
5. CHANG, C.L., AND LEE, R.C. Symbolic Logic and Mechanical Theorem Proving. Academic
Press, New York, 1973.
  \item
6. HEWITT, C. Description and Theoretical Analysis (Using Schemata) of PLANNER: A Language
for Proving Theorems and Manipulating Models in a Robot. Ph.D. dissertation, Dep. of Mathematics,
Massachusetts Institute of Technology, Cambridge, Mass., 1972.
  \item
7. HUET, G. R6solution d'6quations dans les langages d'ordre 1, 2 ..... 0:. Th~se d'6tat, Sp6cialit6
Math~matiques, Universit~ Paris VII, 1976.
  \item
8. HUET, G.P. A unification algorithm for typed ?,-calculus. Theor. Comput. Sci. 1, 1 (June 1975),
27-57.
  \item
9. KNUTH, D.E., AND BENDIX, P.B. Simple word problems in universal algebras. In Computational
Problems in Abstract Algebra, J. Leech (Ed.). Pergamon Press, Eimsford, N.Y., 1970, pp. 263-297.
  \item
10. KOWALSKI, R. Predicate logic as a programming language. In Information Processing 74,
Elsevier North-Holland, New York, 1974, pp. 569-574.
  \item
11. LEVI, G., AND SIROVICH, F. Proving program properties, symbolic evaluation and logical procedural
semantics. In Lecture Notes in Computer Science, vol. 32: Mathematical Foundations of
Computer Science 1975. Springer-Verlag, New York, 1975, pp. 294-301.
  \item
12. MARTELLI, A., AND MONTANARI, U. Theorem proving with structure sharing and efficient
unification. Internal Rep. S-77-7, Ist. di Scienze della Informazione, University of Pisa, Pisa, Italy;
also in Proceedings of the 5th International Joint Conference on Artificial Intelligence, Boston,
1977, p. 543.
  \item
13. MARTELLI, A., AND MONTANARI, V. Unification in linear time and space: A structured 
presentation. Internal Rep. B76-16, Ist. di Elaborazione delle Informazione, Consiglio Nazionale delle
Ricerche, Pisa, Italy, July 1976.
  \item
14. MILNER, R. A theory of type polymorphism in programming. J. Comput. Syst. Sci. 17, 3 (Dec.
1978), 348-375.
  \item
15. PATERSON, M.S., AND WEGMAN, M.N. Linear unification. J. Comput. Syst. Sci. 16, 2 (April
1978), 158-167.
  \item
16. ROBINSON, J.A. Fast unification. In Theorem Proving Workshop, Oberwolfach, W. Germany,
Jan. 1976.
  \item
17. ROBINSON, J.A. Computational logic: The unification computation. In Machine Intelligence,
vol. 6, B. Meltzer and D. Michie (Eds.). Edinburgh Univ. Press, Edinburgh, Scotland, 1971, pp.
63-72.
  \item
18. ROBINSON, J.A. A machine-oriented logicbased on the resolution principle. J. ACM 12, 1 (Jan.
1965), 23-41.
  \item
19. SHORTLIFFE, E.H. Computer-Based Medical Consultation: MYCIN. Elsevier North-Holland,
New York, 1976.
  \item
20. STICKEL, M.E. A complete unification algorithm for associative-commutative functions. In
Proceedings of the 4th International Joint Conference on Artificial Intelligence, Tbilisi, U.S.S.R.,
1975, pp. 71-76.
  \item
21. TRUM, P., AND WINTERSTEIN, G. Description, implementation, and practical comparison of
unification algorithms. Internal Rep. 6/78, Fachbereich Informatik, Univ. of Kaiserlautern, W.
Germany.
  \item
22. VENTURINI ZILLI, M. Complexity of the unification algorithm for first-order expressions. Calcolo
12, 4 (Oct.-Dec. 1975), 361-372.
  \item
23. VON HENKE, F.W., AND LUCKHAM, D.C. Automatic program verificationIII: A methodology for
verifying programs. Stanford Artificial Intelligence Laboratory Memo AIM-256, Stanford Univ.,
Stanford, Calif., Dec. 1974.
  \item
24. WALDINGER, R.J., AND LEVITT, K.N. Reasoning about programs. Artif. Intell. 5, 3 (Fall 1974),
235-316.
  \item
25. WARREN, D.H.D., PEREIRA, L.M., AND PEREIRA, F. PROLOG--The language and its implementation
compared with LISP. In Proceedings of Symposium on Artificial Intelligence and
Programming Languages, Univ. of Rochester, Rochester, N.Y., Aug. 15-17, 1977. Appeared as
joint issue: SIGPLAN Notices (ACM) 12, 8 (Aug. 1977), and SIGART Newsl. 64 (Aug. 1977),
109-115.
  
\end{enumerate}

Received September 1979; revised July 1980 and September 1981; accepted October 1981 
