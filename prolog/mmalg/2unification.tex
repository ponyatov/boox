\secrel{UNIFICATION AS THE SOLUTION OF A SET OF EQUATIONS: A NONDETERMINISTIC
ALGORITHM}

In this section we introduce the basic definitions and give a few theorems which
are useful in proving the correctness of the algorithms. Our ay of stating the
unification problem is slightly more general than the classical one due to
Robinson [18] and directly suggests a number of possible solution methods.

Let
\[A= \bigcup_{i=0,1,..} A_i \quad (A_i \bigcap A_j = \varnothing, i \neq j)\] be
a ranked alphabet, where $A_i$ contains the $i$-adic function symbols
(the elements of $A_0$ are constant symbols). Furthermore, let $V$ be the
alphabet of the variables.
The \term{terms} are defined recursively as follows:

(1) constant symbols and variables are terms;

(2) if $t_1,..,t_n \ (n \geq 1)$ are terms and $f \in A_n$,
then $f(t_1,..,t_n)$ is a term.

A \term{substitution} $\vartheta$ is a mapping from variables to terms,
with $\vartheta(x)=x$ almost everywhere. A substitution can be represented
by a finite set of ordered pairs
$\vartheta={(t_1,x_1),(t_2,x_2),..,(t_m,x_m)}$
where $t_i$ are terms and $x_i$ are distinct variables,
$i = 1,..,m$. To apply a substitution $\vartheta$ to a term $t$, we
simultaneously substitute all occurrences in $t$ of every variable $x_i$ in a
pair $(t_i, x_i)$ of $\vartheta$ with the corresponding term $t_i$. We call the
resulting term $t_\vartheta$.

For instance, given a term $t = f(x_1, g(x_2, a)$ and a substitution 
$\vartheta = {(h(x_2),x_1),(b,x_2)}$, 
we have $t_\vartheta = f(f(x_2),g(b),a)$ and $t_{\vartheta\vartheta} =
f(h(b),g(b),a))$.

The standard unification problem can be written as an equation \[t'=t''\]

A solution of the equation, called a \term{unifier}, is any substitution
$\vartheta$, if it exists, which makes the two terms identical. For instance,
two unifiers of the equation $f(x_1,h(x_1),x_2)=f(g(x_3),x_4,x_3))$ are
$\vartheta_1={(g(x_3),x_1),(x_3,x_2),(h(g(x_3)),x_4)}$ and
$\vartheta_2={(g(a),x_1),(a,x_2),(a,x_3),(h(g(a)),x_4)}$.

In what follows it is convenient also to consider sets of equations
\[t'_j=t''_j, \quad j=1,..,k\]

Again, a \term{unifier} is any substitution which makes all pairs of terms
$t'_j,t''_j$ identical simultaneously.

Now we are interested in finding transformations which produce \emph{equivalent}
sets of equations, namely, transformations which preserve the sets of all
unifiers.
Let us introduce the following two transformations:

\paragraph{(1) Term Reduction.} Let

\begin{equation}\label{mm1}
f(t'_1,t'_2,..,t'_n)=f(t''_1,t''_2,..,t''_n), \quad f \in A_n
\end{equation} 
be an equation where both terms are not variables and where the two root
function symbols are equal. The new set of equations is obtained by replacing
this equation with the following ones: 
\begin{align}\label{mm2}
t'_1 &= t''_1\\
t'_2 &= t''_2\\
&.\\
&.\\
&.\\
t'_n &= t''_n
\end{align}
If $n = 0$, then $f$ is a constant symbol, and the equation is simply erased.

\paragraph{(2) Variable Elimination.} Let $x = t$ be an equation where $x$ is a
variable and $t$ is any term (variable or not). The new set of equations is
obtained by applying the substitution $\vartheta={(t,x)}$ to both terms of all
other equations in the set (without erasing $x = t$).

We can prove the following theorems: 

\paragraph{THEOREM 2.1.} \textit{Let $S$ be a set of equations and let
$f'(t'_1,..t'_n)=f''(t''_1,..,t''_n)$ be an equation of $S$. If $f' \neq f''$,
then $S$ has no unifier. Otherwise, the new set of equations $S'$, obtained by
applying term reduction to the given equation, is equivalent to $S$.}

\paragraph{PROOF.} If $f' \neq f''$, then no substitution can make the
two terms identical.
If $f' = f"$, any substitution which satisfies \ref{mm2} also satisfies
\ref{mm1}, and conversely for the recursive definition of term. $\square$

\paragraph{THEOREM 2.2.} \textit{Let $S$ be a set of equations, and let us apply
variable elimination to some equation $x = t$, getting a new set of equations
$S'$. If variable $x$ occurs in $t$ (but $t$ is not $x$), then $S$ has no
unifier; otherwise, $S$ and $S'$ are equivalent.}

\paragraph{PROOF.} Equation $x = t$ belongs both to $S$ and to $S'$, and thus
any unifier $\vartheta$ (if it exists) of $S$ or of $S'$ must unify $x$ and $t$;
that is, $x_\vartheta$ and $t_\vartheta$ are identical. Now let $t_1 = t_2$ be
any other equation of $S$, and let $t'_1 = t'_2$ be the corresponding equation
in $S'$. Since $t'_1$ and $t'_2$ have been obtained by substituting $t$ for
every occurrence of $x$ in $t_1$ and $t_2$, respectively, we have
$t_{1_\vartheta}=t'_{1_\vartheta}$ and $t_{2_\vartheta}=t'_{2_\vartheta}$. Thus,
any unifier of $S$ is also a unifier of $S'$ and vice versa. Furthermore, if
variable $x$ occurs in $t$ (but $t$ is not $x$), then no substitution
$\vartheta$ can make $x$ and $t$ identical, since $x_\vartheta$ becomes a
subterm of $t_\vartheta$, and thus $S$ has no unifier. $\square$

A set of equations is said to be \term{in solved form} iff it satisfies the
following conditions: 

(1) the equations are $x_j = t_j, j = 1,..,k$;

(2) every variable which is the left member of some equation occurs only there.

A set of equations in solved form has the obvious unifier
\[\vartheta = {(t_1,x_1),(t_2,x_2),..,(t_k,x_k)}\]

 
  