\secrel{The Unification Algorithm}

Looking at Algorithm 2, it is clear that the main source of complexity is step
(1.5), since it may make many copies of large terms. In the following--and this is
the heart of our algorithm--we show that, if the system has unifiers, then there
always exists a multiequation in U (if not empty) such that by selecting it we do
not need step (1.5) of the algorithm, since the variables in its left-hand side do
not occur elsewhere in U. We need the following definition. 

Given a system R, let us consider the subset Vu of variables obtained by making
the union of all left-hand sides Si of the multiequations in the U part of R. Since
the sets Si are disjoint, they determine a partition of Vu. We now define a relation
on the classes Si of this partition: we say that Si < Sj iff there exists a variable
of Si occurring in some term of Mj, where Mj is the right-hand side of
the multiequation whose left-hand side is Sj. We write <* for the transitive closure
of <. 

Now we can prove the following theorem and corollary. 

THEOREM 3.3. If a system R has a unifier, then the relation <* is a partial
ordering.
PROOF. If Si < \$i, then, in all unifiers of the system, the term substituted
for every variable in Si must be a strict subterm of the term substituted for every
variable in Sj. Thus, if the system has a unifier, the graph of the relation < cannot
have cycles. Therefore, its transitive closure must be a partial ordering.
$\square$

COROLLARY. If the system R has a unifier and its U part is nonempty, there
exists a multiequation S ffi M such that the variables in S do not occur elsewhere
inU.

PROOF. Let S = M be a multiequation such that S is "on top" of the partial
ordering < * (i.e., ~3Si, S < Si). The variables in S occur neither in the other lefthand
sides of U (since they are disjoint) nor in any right member Mi of U, since
otherwise S < Si. $\square$

We can now refine the nondeterministic Algorithm 2 giving the general version
of our unification algorithm for a system of multiequations R = (T, U). 

Algorithm 3: UNIFY, the Unification Algorithm
\begin{enumerate}
  \item 
(1) repeat
\begin{enumerate}
  \item 
(1.1) Select a multiequation S = M of U such that the variables in S do not occur
elsewhere in U. If a multiequation with this property does not exist, stop with
failure (cycle).
  \item 
(1.2) ifMis empty
then transfer this multiequation from U to the end of T.
else begin
\begin{enumerate}
  \item 
(1.2.1) Compute the common part C and the frontier F of M. If M has
no common part, stop with failure (clash).
  \item 
(1.2.2) Transform U using multiequation reduction on the selected
multiequation and compactification.
  \item 
(1.2.3) Transfer the multiequation S = (C) from U to the end of T.
end
until the U part of R
\end{enumerate}
\end{enumerate}
(2) stop with success.  
\end{enumerate}
 
A few comments are needed. Besides step (1.5) of Algorithm 2, we have also
erased step (1.3) for the same reason. Furthermore, in Algorithm 2 we were forced
to wait to transfer multiequations with empty right-hand sides since substitution
in that case would have required a special treatment. 

By applying Algorithm UNIFY to the system which was previously solved with
Algorithm 2, we see that we must first eliminate variable x, then variable x,, then
variables x2 and x3 together, and, finally, variables x4 and x5 together, getting the
following final system:
\begin{verbatim}
U:~
T: ({x} = (f(xl, xl, x2, x,)),
{Xl} = (g(x2, x3)),
{x2, x3} = (h(a, x4)),
{x,, xs} = (b)).  
\end{verbatim}

Note that the solution obtained using Algorithm UNIFY is more concise than
the solution previously obtained using Algorithm 2, for two reasons. First,
variables x2 and x3 have been recognized as equivalent; second, the right member
of x~ is more factorized. This improvement is not casual but is intrinsic in the
ordering behavior of Algorithm UNIFY.

To summarize, Algorithm UNIFY is based mainly on the two ideas of keeping
the solution in a factorized form and of selecting at each step a multiequation in
such a way that no substitution ever has to be applied. Because of these two
facts, the size of the final system cannot be larger than that of the initial one.
Furthermore, the operation of selecting a multiequation fails if there are cycles
among variables, and thus the so-called occur-check is built into the algorithm,
instead of being performed at the last step as in other algorithms [2, 7]. 