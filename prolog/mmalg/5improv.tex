\secrel{IMPROVING THE UNIFICATION ALGORITHM FOR NONUNIFYING DATA}\label{mmalg5}

In the case of nonunifying data, Algorithm 3 can stop with failure in two ways:
either in step (1.1) if a cycle has been detected, or in step (1.2.1) if a clash occurs.
In this section we show how to anticipate the latter kind of failure without
altering the structure of the algorithm. 

Let us first give the following definition. Two terms are consistent iff either at
least one of them is a variable or they are both nonvariable terms with the same
root function symbol and pairwise consistent arguments. This definition can be
extended to the case of more than two terms by saying that they are consistent
iff all pairs of terms are consistent. For instance, the three terms f(x, g(a, y)),
f(b, x), and f(x, y) are consistent although they are not unifiable. 

We now modify Algorithm UNIFY by requiring all terms in the right-hand
side of a multiequation to be consistent, for every multiequation. Thus, we stop
with clash failure as soon as this requirement is not Satisfied. This new version of
the algorithm is still correct since, if there are two inconsistent terms in the same
multiequation, they will never unify. 

In this way, clashes are detected earlier. In fact, in the Algorithm 3 version of
UNIFY a clash can be detected while computing the common part and the
frontier of the right-hand side of the selected multiequation, whereas in the new
version of UNIFY the same error is detected in the compactification phase of a
previous iteration. 

An efficient implementation of the consistency check when two multiequations
are merged requires a suitable representation for right-hand sides of multiequations.
Thus, instead of choosing the obvious solution of representing every righthand
side as a list of terms, we represent it as a multiterm, defined as follows. 

A multiterm can be either empty or of the form f(P1 ..... Pn) where f E A, and
Pi (i = 1 ..... n) is a pair (Si, Mi) consisting of a set of variables Si and a
multiterm ]Vii. Furthermore, Si and Mi cannot both be empty. 

For instance, the multiset of consistent terms
\[(f(x, g(a, y)), f(b, x), f(x, y))\]
can be represented with the multiterm
\[f(((x), b), ({x,y},g((O, a), ((y), ~)))).\]

 By representing right-hand sides in this way we have no loss of information,
since the only operations which we have to perform on them are the operation of
merging two right-hand sides and the operation of computing the common part
and the frontier, which can be described as follows:
\begin{verbatim}
MERGE (M', M " ) =
case M' of
O: M";
f'((Si M~), , tS' M' ~"
case M" of
O: M';
f"((S~',M~') ..... (Sn", M~")):
iff' -- f" and MERGE(M~, M[') # failure (i -- 1 ..... n)
then f'((Si O S~', MERGE(MI, M;')) .....
(S~ ~J S t'n, MERGE(M~', M,,))"~
else failure
endcase
endcase
COMMONPART(f((S1, M1) ..... (S,, Mn))) = f(P1, --., Pn)
where Pi = if Si = ~ then COMMONPART(Mi)
else ANYOF(SD (i = 1 .... , n) 
\end{verbatim}
\begin{verbatim}
where function ANYOF(S~) returns an element of set Si. 
\end{verbatim}
\begin{verbatim}
UPart = record
MultEqNumber: Integer;,
ZeroCounterMultEq, Equations: ListOfMultEq
end;
System = TPSystem;
PSystem = record
T: ListOfMultEq;
U: UPart
end;
MultiTerm = ~PMultiTerm;
PMultiTerm = record
Fsymb: FunName;
Args: ListOfTempMultEq
end;
MultiEquation = ~PMultiEquation;
PMultiEquation = record
Counter, VarNumber: Integer;
S: ListOfVariables;
M: Multi Term
end;
TempMultEq = ~PTempMultEq;
PTempMultEq = record
S: QueueOfVariables;
M: MultiTerrn
end;
Variable = TPVariable;
P Variable = record
Name: VarName;
M: MultiEquation
end; 
\end{verbatim}
\begin{verbatim}
FRONTIER(f((S,, M1) ..... (Sn, Mn) )) = F1 [..J ... (.J Fn
where Fi = if Si = O then FRONTIER(M/)
else {Si = Mi} (i = 1, ..., n). 
\end{verbatim}

Note that the common part and the frontier are defined only for nonempty
multiterms and that they always exist. 
