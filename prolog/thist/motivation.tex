\secrel{Motivation}

If you are used to imperative programming languages and then learn a functional
or logic programming language, you may first work successfully through a few
easy exercises and then suddenly find yourself unable to solve apparently simple
tasks in the declarative language. You may for example find yourself asking:
``How do I even increase the value of a variable\,?'', or ``How do I even remove
an element from a list\,?'', or more generally ``How do I even apply a simple
transformation to this data structure\,?'' etc.

Invariably, the solution to such problems is to think in terms of
\term{relations} between different entities. Examples of such entities are
variables, lists, trees, \term{states} etc.

For instance, when you find yourself asking ``How do I even remove an element
\var{E} from a list\,?'', think declaratively and describe a \term{relation}
between two lists: One list may contain the element \var{E}, and the second list
contains all elements of the first list which are not equal to \var{E}.
Actually, as you already see, we are describing a relation between \emph{three}
entities in this case: Two lists, and an element. You can express this relation
in \prolog\ by stating the conditions that make the relation \term{hold}:
\begin{lstlisting}[language=prolog]
        list1_element_list2([], _, []).
        list1_element_list2([E|Ls1], E, Ls2) :-
                list1_element_list2(Ls1, E, Ls2).
        list1_element_list2([L|Ls1], E, [L|Ls2]) :-
                dif(L, E),
                list1_element_list2(Ls1, E, Ls2).
\end{lstlisting}
                
This relation has one argument for each of these entities, and is usable in all
directions: It can answer much more general questions than just ``What does a
list look like if we remove all occurrences of the element \var{E}\,?''. You can
also use it to answer for example: ``Which element, if any, \emph{has been}
removed in a given example\,?'', or to answer the most general query: ``For
which 3 entities does this relation even hold\,?''. This generality is the
reason why an imperative name like \verb"remove/3" would not be a good fit in
this case.

As another example, when you find yourself asking: ``How do I even apply a
transformation to a tree\,?'', again think declaratively and describe a
\term{relation} between two trees: one tree \emph{before} the transformation and
one tree \emph{after} the transformation.

Notice that \term{functions} are a special kind of relations, so most of the
things below hold for functional as well as logic programming languages. Logic
programming languages typically allow for more general solutions with less
effort, since predicates can often be used in several directions.

In this text, I give several examples of relations between states: states in
puzzles, states in programs, states in compilers etc., so that you see how
various tasks can be expressed in declarative languages. The core idea is the
same in all these examples: We think in terms of relations between states. This
is in contrast to imperative languages, where we often think in terms of
destructive modifications to a state.
