\secrel{Generating Code from Trees }

A more general approach to code generation is based on “walks” in the syntaxtree
of a program. We start by describing Warren’s approach [31] for generating
code for a fictitious machine. This computer performs arithmetic operations
using a single accumulator. The corresponding instructions are ADD, MULT,
SUB, and DIV. Operations of the type ADDI, MULTI, and so on, are also
available, and consider the value immediately following them as the second
operand in the computation. LOAD and ST0 commands are of course present,
as well as the unconditional transfer (JUMP) or conditional ones such as J xx,
where xx is EQ, NE, GT, and so on. The input/output commands are simply
READ and WRITE. The generator consists of the clause encode-statement which
identifies the node of the syntax-tree and constructs the corresponding code. The
generated code is a list of instructions and labels, (possibly containing embedded
sublists), for instance,
\begin{verbatim}
[. . . label(LI), [instr(LOAD, X), instr(ADDI, 3)], . . -1 
\end{verbatim}

In Warren’s paper the arguments of instructions are stored in a dictionary,
but remain unbound to actual memory addresses until the very final phase of the
compiler. At that time an assembler determines the addresses of labels, and an
allocator binds the addresses of the variables and reserves the number of memory
locations needed to run the compiled program. We now present some fragments
of Prolog programs that perform the generation. An assignment of an expression
Expr to a variable X is translated into the list whose head is the generated code
for the expression followed by the instruction ST0 X. The procedure encodestatement
has three arguments: the syntax-tree, the dictionary Diet, and the
resulting code. We have
\begin{verbatim}
encode-statement(assign(name(X), Expr),
Diet,
[Exprcode, instr(sto, Addr)]) :-
lookup(X, Addr, Diet),
encode-expr(Expr, Diet, Exprcode). 
\end{verbatim}

The procedure lookup stores the new variable X if it is not yet entered in Diet
and retrieves the unbound variable representing its address (see Section 4). 

The procedure encode-expr can handle two shapes of arithmetic expression
syntax-trees. In the first the right operand is a leaf (i.e., a variable or a constant).
In the second the right operand is a subtree. The syntax-tree for arithmetic
expressions has internal nodes labeled by the operator Op. The more complex
case where the right operand is a subtree is presented below. Its action is to
translate expr(Op, Exprl, Expr2) (in which Expr2 is of the form expr(Op, Anyl,
Any2) into the sequence containing

(1) the code for Expr2,

(2) the instruction ST0 temp,

(3) the code for Exprl, and finally,

(4) the code for the instruction specified by Op. 

An added argument N is needed to specify the pool of temporary locations. Its
initial value is zero. In Prolog we have
\begin{verbatim}
encode-subexpr(expr(Op, Exprl, Expr2), N, Diet,
[Exprkode, instr(sto, Addr), Exprlcode, instr(Opcode, Addr)]) :-
complex (Expr2),
lookup(N, Addr, Diet),
encode-subexpr(Expr2, N, Diet, ExprZcode),
NlisN+l,
encode-szbexpr(Exprl, Nl, Diet, Exprlcode),
memoryop (Op, Opcode).
complex(expr(Op, Anyl, AnyP)).
memoryop(+, add).
memoryop(*, m&t).
\end{verbatim}

The code generated for the previous expression A* (A* B + C - C* D) now
becomes
\begin{verbatim}
LOAD C
MULT D
ST0 TO
LOAD
MULT ii
ADD C
SUB To
ST0 T,,
LOAD A
MULT T,, 
\end{verbatim}
Note that since a right subtree is evaluated before a left one, the code for C* D
is the first to be generated. 

The use of labels is illustrated by the generation of code for while statements.
The translation consists of transforming the syntax-tree while (Test, Do) into the
code
\begin{verbatim}
hbel(L1): (encode Test)
(encode Do )
jump Ll
lubel(L2): 
\end{verbatim}

Note that a new argument (L2) is needed in the procedure that encodes tests
to generate the jump to the exit label. The Prolog program to achieve the
translation parallels the above description.
\begin{verbatim}
encode-statement (while (Test, Do), Diet,
[hbel(Ll), Testcode, Docode, in&( jump, Ll), bbel(L2)]) :-
encode-test (Test, Diet, L2, Testcode),
encode-statement(Do, Diet, Docode). 
\end{verbatim}

