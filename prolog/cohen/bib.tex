\secly{REFERENCES}

\begin{enumerate}
  \item 
1. AHO, A. V., AND ULLMAN, J. D. Principles of Compiler Design. Addison-Wesley, Reading, Mass.,
1979.
  \item 
2. BACKHOUSE, R. C. Syntax of Programming Langooges. Prentice-Hail, Englewood Cliffs, N.J.,
1979.
  \item 
3. CAMPBELL, J. A. (ED.) Implementations of Prolog. Wiley, New York, 1984.
  \item 
4. CARLSSON, M. A microcoded unifier for Lisp machine Prolog. In IEEE Proceedings 1985
Symposium on Logic Programming (Boston, July 1985), IEEE, New York, 1985,162-171.
  \item 
5. CATTELL, R. G. G. Automatic derivation of code generators from machine descriptions. ACM
Trans. Programm. Long. Syst. 2,2 (Apr. 1980), 173-190.
  \item 
6. CLOCKSIN, W. F., AND MELLISH, C. S. Programming in Prolog. Springer-Verlag, New York,
1981(2nd ed., 1984).
  \item 
7. COHEN, J., SITVER, R., AND AUTY, D. Evaluating and improving recursive descent parsers.
IEEE Trans. Softw. Eng. SE-5,5 (Sept. 1979), 472-480.
  \item 
8. COHEN, J. Describing Prolog by ita interpretation and compilation. Commun. ACM 28, 12
(Dec. 1985), 1311-1324.
  \item 
9. COLMERAUER, A. Les Grammaires de Metamorphose. Groupe d’Intelligence Artificielle, Univ.
of Marseilles-Luminy, 1975. (Appears as Metamorphosis grammars in Natural Language Communication
with Computers. L. Bale, Ed., Springer-Verlag, New York, 1978, 133-189.)
  \item 
10. COLMERAUER, A., KANOUI, H., AND VAN CANEGHEM, M. Prolog II. Groupe d’Intelligence
Artificielle, Univ. of Marseilles-Luminy, 1982.
  \item 
11. COLMERAUER, A. Prolog and infinite trees. In Logic Programming, Clark and Tarnlund (Eds.),
Academic Press, New York, 1982,231-251.
  \item 
12. COUPET, S., AND DUPLESSIS, F. Prolog programs for transforming regular expressions into the
corresponding minimal finite state recognizers. Memoire de D.E.A., GIA, Univ. of Marseilles,
June 1984 (in French).
  \item 
13. EARLEY, J. An efficient context-free parsing algorithm. Commun. ACM 13, 2 (Feb. 1970),
94-102.
  \item 
14. GIANNESINI, F., AND COHEN, J. Parser generation and grammar manipulations using Prolog’s
infinite trees. J. Logic Programm. (Oct. 1984), 253-265.
  \item 
15. GLANVILLE, R. A machine independent algorithm for code generation and its use in retargetable
compilers. Ph.D. dissertation, Univ. of California, Berkeley, 1977.
  \item 
16. GRAHAM, S. L., HENRY, R. R., AND SHULMAN, R. A. An experiment in table driven generation.
In Proceedings SIGPLAN Symposium on Compiler Construction 17,6 (June 1982), 32-43.
  \item 
17. GRIES, D. Compiler Construction for Digital Computers. Wiley, New York, 1971. 
  \item 
18. GRIFFITHS, T. V., AND PETRICK, S. R. On the relative efficiencies of context-free grammar
recognizers. Commun. ACM 8,5 (May 1965), 289-300.
  \item 
19. ICHBIAH, J. D., AND MORSE, S. P. A technique for generating almost optimal Floyd-Evans
productions for precedence grammars. Commun. ACM 13,8 (Aug. 1970), 501-508.
  \item 
20. KOWALSKI, R. Logic for Problem Soluing. North-Holland, Amsterdam, 1979.
  \item 
21. MELLISH, ‘C. S. Some global optimizations for a Prolog compiler. J. Logic Programm. 2, 1
(Apr. 1985), 43-66.
  \item 
22. MICKUNAS, M. D., AND MODRY, J. A. Automatic error recovery for LR parsers. Commun. ACM
21,6 (June 1978), 459-465.
  \item 
23. NAISH, L. Automating control for logic programs. J. Logic Programm. 3 (1985), 167-183.
  \item 
24. PEREIRA, F. C. N., AND WARREN, D. H. D. Parsing as deduction. In Proceedings of the 21st
Annual Meeting of the Association for Computational Linguistics (Cambridge, Mass., June 1983),
Association for Computational Linguistics, 1983, 137-144.
  \item 
25. PEREIRA, F. C. N., AND WARREN, D. H. D. Definite clause grammars for language analysis.
Artif. Intell. 13 (1980), 231-278.
  \item 
26. PIQUE, J. F. Drawing trees and their equations in Prolog. In Proceedings of the 2nd Znternutionul
Logic Programming Conference (Uppsala, 1984), Ord and Furm, Uppsala, 1984,23-33.
  \item 
27. TANENBAUM, A. S., VAN STAVEREN, H., AND STEVENSON, J. W. Using peephole optimization
on intermediate code. ACM Trans. Programm. Lung. Syst. 4,1 (Jan. 1982), 21-36.
  \item 
28. UEHARA, K., OCHITANI, R., AND KAKUSHO, 0. A bottom-up parser based on predicate logic. In
IEEE Proceedings 1984, International Symposium on Logic Programming (Atlantic City, N.J.,
Feb. 1984), IEEE, New York, 1984,220-227.
  \item 
29. WAITE, W. M., AND GOOS, G. Compiler Construction. Springer-Verlag, New York, 1984.
  \item 
30. WARREN, D. H. D. Applied logic-its use and implementation as a programming tool. Ph.D.
dissertation, Univ. of Edinburgh, 1977 (also appeared as Tech. Note 290, SRI International,
1983).
  \item 
31. WARREN, D. H. D. Logic programming and compiler writing. Softw. Pratt. Exper. 10 (Feb.
1980), 97-125. 
\end{enumerate}
 