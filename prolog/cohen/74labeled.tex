\secrel{Code Generation from a Labelled Tree}\label{cohen74}

We conclude this section by presenting the Prolog programs implementing the
optimal code generation applicable to labelled trees as described in [ 1] 

The labelling phase consists of a postorder walk on a syntax-tree in which left
leaves are labelled with a 1 and right leaves with a 0. Interior nodes are labelled
by max(left, right) if the left label is different from the right one; otherwise the
interior node is labelled with right + 1. The label of the root specifies the total
(optimal) number of registers needed to code the syntax-tree without using
temporary locations. In Prolog the labelling is accomplished by the clause Zubel,
having four parameters: (1) the original syntax-tree, (2) a mark denoting a left
or right branch, (3) the generated labelled tree, and (4) the node label itself. 
\begin{verbatim}
label(uar(X), left, uar(X, l), 1).
lubel(uar(X), right, uar(X, 0), 0).
label(expr(Op, Left, Right), 2, expr(Op, El, E2, Label), N) :-
hbel(L.eft, left, El, L.ubell),
lubel(Right, right, E2, LabelZ),
nmx(Label1, LabelP, Label).
mux(N, N, Nl) :- Nl is N + 1.
m&N, Nl, N) :- N > Nl.
max(N1, N, N) :- N-c Nl. 
\end{verbatim}

The actual code generation algorithm is practically the same as that presented
on p. 544 of [l]. The parameters of gencode are (1) the labelled syntax-tree,
(2) the register stack, (3) the maximum number of registers, and (4) the next
available temporary location. The procedure will output code for the expression
in such a way that the result of the expression is stored in the register at the top
of the register stack. (We remind the reader that any program that uses write to
produce its results can easily be modified to store the results in a list, as described
in Section 4). 

The first two clauses consider the simplest cases dealing with leaves. The third
clause is applicable only when the right subexpression is a tree. It finds the labels
of the two subexpressions and calls gencodel which generates code using the
minimal number of registers and temporaries. We purposely avoided the use of
cuts by ensuring that each of the clauses deal with mutually exclusive cases, and
so no backtracking is possible. 

\begin{lstlisting}[language=prolog]
% case 0: left expression is a leaf.
gencode(var(X, l), [Reg ] RestR], Max, Temp) :- print(moue, X, Reg).
% case 1: right expression is a leaf.
gencode(expr(Op, X, uar( Y, 0), Label), [Reg ] RestR], Max, Temp) :-
	gencode(X, [Reg ] RestR], Max, Temp),
	write(Op, Y, Reg).
% cases 2a, 2b, and 2c: left and right expressions are trees.
gencode(expr(Op, L, R, Any-Label), Regs, Max, Temp) :-
	labelvalue(L, NL),
	labelvalue(R, NR), NR > 0
	gencodel(expr(Op, L, R, Any-Label), Regs, Max, NL, NR, Temp).
% case 2a: Left expression can be computed without temporaries.
gencodel(expr(Op, L, R, Any-Label), [Regl, Reg2 ] RestR], Max, Nl, N2, Temp) :-
	Nl < N2, Nl < Max,
	gencode(R, [RegZ, Regl ] RestR], Max, Temp),
	gencode(L, [Regl ] RestR], Max, Temp),
	write (Op, Reg2, Regl).
% case 2b: Right expression can be computed without temporaries.
gencodel(expr(Op, L, R, Any-Label), [Regl, Reg2 ] RestR], Max, Nl, N2, Temp) :-
	N2 =< Nl, N2 <Max,
	gencode(L, [Regl, Reg2 ] RestR], Max, Temp),
	gencode(R, [Reg:! ] RestR], Max, Temp),
	write(Op, Reg2, Regl).
% case 2~: temporaries are required.
gencodel(expr(Op, L, R, Any-Label), [Reg ] RestR], Max, Nl, N2, Temp) :-
	Nl 2 Max, N2 > Max,
	gencode(R, [Reg ] RestR], Max, Temp),
	write(move, Reg, t (Temp)),
	NextTemp is Temp + 1,
	gencode(L, [Reg ] RestR], Max, NextTemp),
	write(Op, t(Temp), Reg).
% miscellaneous.
labelvalue(expr(Any1, Any2, Any3, N), N).
labelvalue(var(Any1, N), N). 
\end{lstlisting}

Assuming that two registers are available, the code generated for the expression
((A - B)/(C - D))l((E - J’MG - f0) is
\begin{verbatim}
MOVE E, R0
SUB F, Ro
MOVE G, R,
SUB H, RI
DIV RI, Ro
MOVE R,,, T,,
MOVE A, R,,
SUB B, R,,
MOVE C, R,
SUB D, R,
DIV RI, Ro
DIV To, Ro
\end{verbatim}

We conclude this section by pointing out that Cattell’s method of code generation
is a prime candidate for prototyping using Prolog. In his dissertation,
Cattell [5] proposes a method for formalizing and automatically deriving code
generators from machine descriptions. The method consists of constructing a
syntax-tree-like description for each instruction in the machine’s repertoire.
A special tree-matching program then generates code sequences by combining
the available instruction syntax-trees so that they match the syntax-tree representing
a source program. Cattell’s approach combines AI techniques with those
in current use in compiler construction. 

Although the code generators described herein are specialized to the case of
Algol-like programs, Prolog has already proved its usefulness in writing Prolog
compilers [3]. At Brandeis we have developed a Prolog compiler that compiles
Prolog programs into equivalent C programs [8]. A remarkable feature of these
compilers is their conciseness and the ease with which they can be changed to
generate code in various target languages. 
