\secrel{Peephole Optimization}\label{cohen82}

Table I summarizes some of the typical peephole optimizations that can be
performed after code generation. The table also indicates the source program
segments, resulting in code that can be optimized in this manner. 

We first note that if Warren’s approach (Section 7.2) is used for code generation,
an additional pass is needed to “flatten” the list that makes up the generated
code. This list contains sublists resulting from the order in which the clauses
endcode statements are activated. Assuming that the code consists of a list of
elements separated by (right-associative) semicolons, it is a simple matter to
express in Prolog the optimizations in Table I:
\begin{lstlisting}[language=prolog]
% if pattern is found perform the optimization.
peep([instr(sto, X), instr(load, X) 1 L], [instr(sto, X) 1 M]) :-peep&, M).
peep([instr(subi, 0) I L], M) :-peep@, M).
peep([lubel(A), instr(jump, A) 1 L], M) :- peep([instr(jump, A), L], M).
% keep trying with tail of list.
ped[X I Ll, IX I Ml) :-peep& MI.
pw([ I, 1 I).
\end{lstlisting}

\paragraph{Table I. Peephole Optimization}\ \\
\begin{tabular}{l l l}
\hline
Source code & Compiled code & Optimization code \\
\hline
a := \ldots & STO a & STO a \\
b := a \ldots & LOAD a & \\
\hline
if a<0 then \ldots & SUBI 0 & $\varepsilon$ \\
\hline
while \ldots do & JUMP a & JUMP b \\
if \ldots then \ldots else; & \ldots & \ldots \\
& a: JUMP b & a: JUMP b \\
\hline 
\end{tabular}\bigskip

Note that the above program can handle the nondeterministic situations arising
when the code to be inspected using the first parameter renders more than one
peep clause applicable. The resulting nondeterministic searches could result in 
longer processing times. It is up to the designer to decide whether this overhead
brings significant gains in the execution of the optimized code. A careful ordering
of the clauses and a judicious introduction of cuts can reduce some of the
overhead. It is also possible to control the amount of backtracking by introducing
and tallying costs which, when exceeded, trigger the choice of alternate paths
(see Section \ref{cohen9}). 

Finally, it should be mentioned that David Hildum and the first author were
able to implement, using Prolog, all of the (over one hundred) peephole optimizations
applicable to a P-code-like intermediate language [27]. This was achieved
by prototyping a language for specifying the transformations. An interesting
aspect of this implementation is that DCG rules are used to generate another set
of DCG rules needed to match and replace patterns. 
