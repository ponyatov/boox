
\secrel{A Machine-Independent Algorithm for Code Generation }

An alternate approach to code generation is that proposed by Glanville and
Graham [15, 16]. It is assumed that by syntax-directed translation a source
program is translated into its prefix Polish counterpart. A second syntax-directed
translation of the prefix code then produces actual machine code. The interesting
feature of this approach is that the grammar used to recognize the prefix takes
into consideration the description of the machine for which code is generated.
Consider a register machine whose operations are of the type
\[LOAD M, R\]
\[ADD R1, R2 or ADD M, R\]
\[ST0 R, M\]
\[ADDI C, R\]
where M is a memory address, C is a constant, and R is a register, and the first
argument is the source, the second the destination. To simplify the presentation,
we assume an unlimited pool of registers. The problem of dealing with a limited
number of registers is discussed in the next section.

The grammar rule
\[R + op R var 1 var\]
describes a prefix string in which the last operand is always a variable, (e.g., as
in + + a b c). The code to be generated in this case can be triggered by semantic
actions corresponding to the rules 

\[R + var\]
Action: Load variable into register r

\[R + op R var\]
Action: 1. recognize (recursively) the left operand R assuming that it will
use register r

2. generate the code: op var r

Similar grammar rules are applicable for generating code when the last operand
is a constant. The more general case corresponds to the grammar rule
\[R + op R R 1 var 1 const\]

In this case a new register is needed before recursing to the second R. Also, a
register becomes available after recognizing the second R. A natural way of
implementing the Glanville-Graham approach is through the use of DCGs. The
following simplified grammar rules express assignments:
\[A + := var R\]
\[R+opRvar~opRconst\]
\[R-+opRR\]
\[R + var I const\]

Note that this is an ambiguous grammar, and therefore the use of cuts at the end
of each clause is recommended to avoid generating multiple solutions. The
recursive descent compiler generated from the DCGs opts, whenever possible, to
the first rule defining R, instead of the more general second rule. 

The procedures listed below specify the syntax-directed translation of prefix
Polish into assembly language according to the above grammar rules. The
procedure reg corresponds to the nonterminal R and has three parameters:

(1) generated assembly language sequence,

(2) register containing the final result,

(3) dictionary for storing variables.   

Although the presented program assumes an unlimited number of registers, it is
fairly straightforward to modify it to consider a finite number only. This can be
done by adding extra parameters to the procedure reg. 

The first two clauses of reg treat the special cases where the second parameter
is a variable or a constant:
\begin{verbatim}
% Rule:R+OpRvar.
reg([Sl, imtr(Op, Addr, Rl)], Rl, D) --+
arithop(Op, Optwe),
regC% RL D),
[uar(Var)l,
(lookup( Var, D, Addr), !).
\end{verbatim}
\begin{verbatim}
% Rule: R --, Op R const.
reg([Sl, instr(Constop, C, Rl)], Rl, D) --+
arithop(Op, Optwe),
redsl, RI, D),
bdC)l,
{constop(op, Con-stop), !). 
\end{verbatim}

where arithop and constop are defined as
\begin{align*}
arithop(sub, noncommute) &\rightarrow\ [-I. &arithopbdd, commute) --+ [+I.\\
arithop(diu, noncommute) &\rightarrow\ [/I. &arithp(nult, commute) --+ [*I.\\
constop(sub, subi). &&constop(diu, diui).\\
constop(add, addi). &&constop(mult, multi). \\
\end{align*}

It is possible to perform some optimization in the case of commutative operations.
For that purpose two additional DCG clauses are included to process the rules:
\[R + op var R and R + op const R\]

The DCG clause for the first of these rules is given below.
\begin{verbatim}
% Rule: R -+ op uar R(op is commutatiue).
reg([Sl, instr(Op, Addr, Rl)], Rl, D) --+
arithop(Op, commute),
[udVar)l,
reg(S1, RI, D),
(lookup( Var, Addr, D), !I.  
\end{verbatim}

The more complex DCG given below corresponds to the rule R + op R R.
\begin{verbatim}
% Rule:R-*opRR.
reg([Sl, S2, instr(Op, R2, Rl)], Rl, D) --+
arithop(Op),
redS1, RI, D),
(R2 is Rl + 1),
reg(S2, R2, D), (!). 
\end{verbatim}

Two recursive calls are made to reg to determine the subsequences Sl and S2
representing the code for calculating the two operands. The simple DCG clauses
for the rules R + uar and R + const generate the necessary instructions that
load a register with a variable or with a constant.
\begin{verbatim}
% Rule: R + var.
reg(instr(load, Addr, Rl), Rl, D) --+
[MVar)l,
{lookup( Var, D, Addr), !).
% Rule: R + const.
reg(instr(hzdc, C, Rl), Rl, D) --+
[co=dC)l, 1% 
\end{verbatim}

Finally, we present the DCG clause for generating an assignment expressed in
prefix by the rule
\[A + := var R.\]
\begin{verbatim}
% Code generator for assignments.
instruction([Sl, instr(store, 1, Addr)], D) --+
[assign, uar (Vur)],
red% 0, D),
(lookup( Var, Addr, D)]. 
\end{verbatim}

Notice that the chosen grammar relies extensively on backtracking for recognizing
the appropriate rule. For example, consider the two rules
\[R --+ Op R const\]
\[R --+ Op R var\]
and the input string (+ + 5 c d). Although the first rule will not apply, it will
nonetheless be tried, and the code for the expression (+ 5 c) will be generated
before backtracking. The same code will then have to be regenerated when the
second rule is applied. This can be avoided by considering the following transformed
equivalent grammar:
\[R --+ Op R R2\]
\[R2 --+ Var 1 Const\]

This transformation can be easily generalized to the case at hand, and the
resulting parser will not rely on backtracking so there will be no need to insert
cuts into the program. 

An example of the code generated by this technique for the expression
,*(A* B + C - C*(D - E)) is
\begin{verbatim}
LOAD B, RO
MULT A, RO
ADD C, RO
LOAD D, Rl
SUB E, Rl
MULT C, RI
SUB Rl, RO
MULT A. RO  
\end{verbatim}
