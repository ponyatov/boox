\secrel{Compile-Time Evaluation }\label{cohen81}

Compile-time evaluation of numerical expressions and algebraic simplification
are easily performed by transforming the syntax-trees of arithmetic expressions
into equivalent trees containing fewer nodes. Both of the procedures evaluate
and simplify have as arguments the initial and final trees. They also have a
similar structure: recursive calls are made to process the left and right branches
until the leaves are reached. Then the auxiliary procedure simp is called to
perform the actual simplifications. This allows successive simplifications to be
performed. 
\begin{lstlisting}[language=prolog]
% leaves are left unchanged
evaluute(const(X), const(X)).
evahte(var(X), var(X)). 
% internal nodes are optimized after each of its subtrees
% has been optimized
evaluate(expr(Op, Left, Right), Optexp) :-
	eualuate(left, Optleft),
	evaluate(Right, Optright),
	simp(expr(Op, Optkft, Optright), Optexp).
simp(expr(Op, const(X), const(Y)), const(Z)) :-
	Temp =. . [Op, X, Y], Z is Temp. 
\end{lstlisting}

(In the Edinburgh syntax [6], the operation Temp =. . [Op, X, Y] binds Temp to
the term Op(X, Y)). Note that, unfortunately, this procedure is unable to simplify
expressions such as a + 3 + 2 into a + 5. This may be achieved by writing a
simple procedure that transforms left-associative expressions into equivalent
right-associative expressions. The procedure that performs algebraic simplifications
is
\begin{lstlisting}[language=prolog]
simplify(expr(Op, X, Y), U) :-
	simplify(X, Left),
	simplify( Y, Right),
	simp(expr(Op, Left, Right), U).
simplify(X, X). 
\end{lstlisting}

As before, the auxiliary procedure simp performs the actual simplifications.
\begin{lstlisting}[language=prolog]
simp(expr(Op, X, const(O)), X) :- addop(Op).
simp(expr(Op, X, const(l)), X) :- multop(Op).
simp(expr(*, X, const(O)), const(0)).
simp(expr(*, const(O), X), const(0)).
simp(X, X).
addop(+). addop(-). muZtop(*). m&top(/). 
\end{lstlisting}
