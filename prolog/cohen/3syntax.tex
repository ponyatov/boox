\secrel{SYNTAX-DIRECTED TRANSLATION }\label{cohen3}

This type of translation consists of triggering semantic actions specified by the
programmer, once selected syntactic constructs are found by a parser. In the case
of the bottom-up parsers described in Section 2.1, it suffices to add a third
parameter to the reduce clauses specifying the rule number and to modify the
parser so that a semantic action (specified by the rule number) will take place
just after the reduction. For example, in order to translate arithmetic expressions
into postfix Polish notation, the corresponding reduce for the first rule of G1
becomes
\begin{verbatim}
reduce(W), t(+), n(e) I Xl, [n(e) I Xl, 0
\end{verbatim}
The modified parser contains two additional parameters: (1) a stack, Sem, which
will be manipulated by the action procedure and (2) a parameter, Result, which
will be bound to the final result of the semantic actions:
\begin{verbatim}
% accept and bind Result to the semantic parameter.
wp-trunslate([$], [n(e)], Result, Result).
% try to perform a reduction and a semantic action.
wp-transkzte([ W 1 Input], [S I Stack], Sem, Result) :-
&y-reduce@, W),
reduce([S I Stack], NewStack, RuleNumber),
action(RuleNumber, [S I Stuck], Sem, NewSem),
wp-translute([ W I Input], NewStack, NewSem, Result). 
\end{verbatim}
\begin{verbatim}
% try a shift.
wp-transZate([ W 1 Input], [S 1 Stack], Sem, Result) :-
try_shift(S, W),
wp-translute(Znput, [W, S 1 Stack], Sem, Result).  
\end{verbatim}
The parser can then he equipped with actions by adding rules which specify how
the temporary semantic parameter is to be modified for each rule. The following
action procedure constructs parse trees for the arithmetic expressions defined by
grammar G1:
\begin{verbatim}
syntax-tree(Znput, Tree) :- wp-trandute(Znput, [ 1, [ 1, Tree).
action(1, Stack, [Xl, X2 1 T], [plus(X2, Xl) IT]).
action(3, Stack, [Xl, X2 I 2’1, [times(X2, Xl) I 2’1).
action(6, [t(Z,etter) I Stack], Temp, [Letter I Temp]).
action(X, Stack, Temp, Temp) :- X # 1, X # 3, X # 6. 
\end{verbatim}
The body of the last clause guarantees that no spurious actions are performed
should backtracking ever occur. Notice that the action procedure must have
access to the parsing stack (as is the case for rule 6) so that specific terminals
may be incorporated into the actions. A similar strategy is applicable in adding
actions to predictive parsers. 

All of the above descriptions of semantic actions utilize inherited attributes
and are admittedly standard. The main purpose of presenting them here is to
point out how succinct the descriptions become when Prolog is used. The truly
novel way of performing syntax-directed translation is that pioneered by
Colmerauer and widely utilized by Warren. That approach does not strictly
separate syntax from semantics as was done in this section. They have added
new parameters to the recursive descent parser described in Section 2.3, so that
the translation takes full advantage of the unification and goal-seeking features
of Prolog. Colmerauer’s approach is the subject of the next section. 

