\secrel{Parsing and Compiling Using Prolog}\label{cohen}\secdown
\href{https://drive.google.com/file/d/0B0u4WeMjO894eHpLcTE2bWU0SjQ/view?usp=sharing}{pdf}
\cp{\url{https://pdfs.semanticscholar.org/dd8d/c0deb336d90912a21ba8ec6f6c6fef4b4024.pdf}}

\copyright\ JACQUES COHEN and TIMOTHY J. HICKEY\\
Brandeis University
\bigskip

\subsecly{Abstract}

This paper presents the material needed for exposing the reader to the
advantages of using Prolog as a language for describing succinctly most of the
algorithms needed in prototyping and implementing compilers or producing tools
that facilitate this task. The available published material on the subject
describes one particular approach in implementing compilers using Prolog. It
consists of coupling actions to recursive descent parsers to produce
syntax-trees which are subsequently utilized in guiding the generation of
assembly language code. Although this remains a worthwhile approach, there is a
host of possibilities for Prolog usage in compiler construction. The primary aim
of this paper is to demonstrate the use of Prolog in parsing and compiling. A
second, but equally important, goal of this paper is to show that Prolog is a
labor-saving tool in prototyping and implementing many nonnumerical algorithms
which arise in compiling, and whose description using Prolog is not available in
the literature. The paper discusses the use of unification and nondeterminism in
compiler writing as well as means to bypass these (costly) features when they
are deemed unnecessary. Topics covered include bottom-up and top-down parsers,
syntax-directed translation, grammar properties, parser generation, code
generation, and optimixations. Newly proposed features that are useful in
compiler construction are also discussed. A knowledge of Prolog is assumed.

Categories and Subject Descriptors: D.l.O [Programming Techniques]: General;
D.2.m [Software Engineering]: Miscellaneous--rapid prototyping; D.3.4
[Programming Languages]: Processors; F.4.1. [Mathematical Logic and Formal
Languages]: Mathematical Logic--logic programming 1.2.3 [Artificial
Intelligence]: Deduction and Theorem Proving--logic programming

General Terms: Algorithms, Languages, Theory, Verification

Additional Key Words and Phrases: Code generation, grammar properties,
optimization, parsing

\bigskip
This work was supported by the NSF under grant DCR 8590881.

Authors’ address: Computer Science Department, Ford Hall, Brandeis University, Waltham, MA
02254.

Permission to copy without fee all or part of this material is granted provided that the copies are not
made or distributed for direct commercial advantage, the ACM copyright notice and the title of the
publication and its date appear, and notice is given that copying is by permission of the Association
for Computing Machinery. To copy otherwise, or to republish, requires a fee and/or specific
permission.

0 1987 ACM 0164-0925/87/0400-0125 \$00.75

ACM Transactions on Programming Languages and Systems, Vol. 9, No. 2, April 1967, Pages 125-163. 

\secrel{INTRODUCTION}

The seminal paper by Alain Colmerauer on Metamorphosis Grammars first
appeared in 1975 [9]. That paper spawned most of the developments in compiler
writing using Prolog, a great many of them due to David H. D. Warren. Warren’s
thesis [30], the paper summarizing it [31], and the related work on Definite
Clause Grammars [25] are practically the sole sources of reference on the
subject.\note{A recent book edited by Campbell [3] mostly covers the
implementation of Prolog itself. }

The available published material on the subject describes one particular approach
in implementing compilers using Prolog. It consists of coupling actions 
to recursive descent parsers to produce syntax-trees which are subsequently
utilized in guiding the generation of assembly language code. Although this
remains a worthwhile approach, there is a host of possibilities for Prolog usage
in compiler construction. The primary aim of this paper is to present the material
needed for exposing the reader to the advantages of using Prolog in parsing and
compiling. A second, but equally important, goal of this paper is to show that
Prolog is a labor-saving tool in prototyping and implementing many nonnumerical
algorithms which arise in compiling, and whose description using
Prolog is not available in the literature. Finally, a third goal is to present new
approaches to compiler design which use proposed extensions to Prolog. 

This paper is directed to compiler designers moderately familiar with Prolog,
who wish to explore the advantages and present drawbacks of using this language
for implementing language processors. The advantages of Prolog stem from two
important features of the language. 

(1) The use of unification as a general pattern-matching operation allowing
procedure parameters (logical variables) to be both input and output or to
remain unbound. Unification replaces the conditionals and assignments
which exist in most languages. 

(2) The ability to cope with nondeterministic situations, and therefore allow the
determination of multiple solutions to a given problem. 

From a subjective point of view, the main advantage of Prolog is that the
language has its foundations in logic, and it therefore encourages the user to
describe problems in a logical manner which facilitates the checking for correctness,
enhances program readability, and reduces the debugging effort. It will be
seen that unification and nondeterminism play an important role in compiler
design; however, using their full generality is often costly and unnecessary. These
issues are discussed throughout the paper whenever they become relevant.
Remarks are made in the last section about the efficiency of Prolog-written
compilers and the means to improve their performance. 

The Prolog proficiency assumed in this paper can be acquired by reading the
first few chapters of either Kowalski’s [20] or Clocksin and Mellish’s [6] books.
In particular, the reader should be at ease with elementary list processing and
with the predicate append. The concrete syntax used in this paper is that of
Edinburgh Prolog [6]. It is also assumed that the reader is familiar with compiler
design topics such as parsing, lexical analysis, code generation, optimizations,
and so on. These topics are covered in standard texts [l, 17,291. 



\secup