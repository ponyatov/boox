\secrel{Parsing and Compiling Using Prolog}\label{cohen}\secdown
\href{https://drive.google.com/file/d/0B0u4WeMjO894eHpLcTE2bWU0SjQ/view?usp=sharing}{pdf}
\cp{\url{https://pdfs.semanticscholar.org/dd8d/c0deb336d90912a21ba8ec6f6c6fef4b4024.pdf}}

\copyright\ JACQUES COHEN and TIMOTHY J. HICKEY\\
Brandeis University
\bigskip

\subsecly{Abstract}

This paper presents the material needed for exposing the reader to the
advantages of using Prolog as a language for describing succinctly most of the
algorithms needed in prototyping and implementing compilers or producing tools
that facilitate this task. The available published material on the subject
describes one particular approach in implementing compilers using Prolog. It
consists of coupling actions to recursive descent parsers to produce
syntax-trees which are subsequently utilized in guiding the generation of
assembly language code. Although this remains a worthwhile approach, there is a
host of possibilities for Prolog usage in compiler construction. The primary aim
of this paper is to demonstrate the use of Prolog in parsing and compiling. A
second, but equally important, goal of this paper is to show that Prolog is a
labor-saving tool in prototyping and implementing many nonnumerical algorithms
which arise in compiling, and whose description using Prolog is not available in
the literature. The paper discusses the use of unification and nondeterminism in
compiler writing as well as means to bypass these (costly) features when they
are deemed unnecessary. Topics covered include bottom-up and top-down parsers,
syntax-directed translation, grammar properties, parser generation, code
generation, and optimixations. Newly proposed features that are useful in
compiler construction are also discussed. A knowledge of Prolog is assumed.

Categories and Subject Descriptors: D.l.O [Programming Techniques]: General;
D.2.m [Software Engineering]: Miscellaneous--rapid prototyping; D.3.4
[Programming Languages]: Processors; F.4.1. [Mathematical Logic and Formal
Languages]: Mathematical Logic--logic programming 1.2.3 [Artificial
Intelligence]: Deduction and Theorem Proving--logic programming

General Terms: Algorithms, Languages, Theory, Verification

Additional Key Words and Phrases: Code generation, grammar properties,
optimization, parsing

\bigskip
This work was supported by the NSF under grant DCR 8590881.

Authors’ address: Computer Science Department, Ford Hall, Brandeis University, Waltham, MA
02254.

Permission to copy without fee all or part of this material is granted provided that the copies are not
made or distributed for direct commercial advantage, the ACM copyright notice and the title of the
publication and its date appear, and notice is given that copying is by permission of the Association
for Computing Machinery. To copy otherwise, or to republish, requires a fee and/or specific
permission.

0 1987 ACM 0164-0925/87/0400-0125 \$00.75

ACM Transactions on Programming Languages and Systems, Vol. 9, No. 2, April 1967, Pages 125-163. 

\secrel{INTRODUCTION}


\secup