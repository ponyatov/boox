\secrel{Parsing and Compiling Using Prolog}\label{cohen}\secdown
\href{https://drive.google.com/file/d/0B0u4WeMjO894eHpLcTE2bWU0SjQ/view?usp=sharing}{pdf}
\cp{\url{https://pdfs.semanticscholar.org/dd8d/c0deb336d90912a21ba8ec6f6c6fef4b4024.pdf}}

\copyright\ JACQUES COHEN and TIMOTHY J. HICKEY\\
Brandeis University
\bigskip

\subsecly{Abstract}

This paper presents the material needed for exposing the reader to the
advantages of using Prolog as a language for describing succinctly most of the
algorithms needed in prototyping and implementing compilers or producing tools
that facilitate this task. The available published material on the subject
describes one particular approach in implementing compilers using Prolog. It
consists of coupling actions to recursive descent parsers to produce
syntax-trees which are subsequently utilized in guiding the generation of
assembly language code. Although this remains a worthwhile approach, there is a
host of possibilities for Prolog usage in compiler construction. The primary aim
of this paper is to demonstrate the use of Prolog in parsing and compiling. A
second, but equally important, goal of this paper is to show that Prolog is a
labor-saving tool in prototyping and implementing many nonnumerical algorithms
which arise in compiling, and whose description using Prolog is not available in
the literature. The paper discusses the use of unification and nondeterminism in
compiler writing as well as means to bypass these (costly) features when they
are deemed unnecessary. Topics covered include bottom-up and top-down parsers,
syntax-directed translation, grammar properties, parser generation, code
generation, and optimixations. Newly proposed features that are useful in
compiler construction are also discussed. A knowledge of Prolog is assumed.

Categories and Subject Descriptors: D.l.O [Programming Techniques]: General;
D.2.m [Software Engineering]: Miscellaneous--rapid prototyping; D.3.4
[Programming Languages]: Processors; F.4.1. [Mathematical Logic and Formal
Languages]: Mathematical Logic--logic programming 1.2.3 [Artificial
Intelligence]: Deduction and Theorem Proving--logic programming

General Terms: Algorithms, Languages, Theory, Verification

Additional Key Words and Phrases: Code generation, grammar properties,
optimization, parsing

\bigskip
This work was supported by the NSF under grant DCR 8590881.

Authors’ address: Computer Science Department, Ford Hall, Brandeis University, Waltham, MA
02254.

Permission to copy without fee all or part of this material is granted provided that the copies are not
made or distributed for direct commercial advantage, the ACM copyright notice and the title of the
publication and its date appear, and notice is given that copying is by permission of the Association
for Computing Machinery. To copy otherwise, or to republish, requires a fee and/or specific
permission.

0 1987 ACM 0164-0925/87/0400-0125 \$00.75

ACM Transactions on Programming Languages and Systems, Vol. 9, No. 2, April 1967, Pages 125-163. 

\secrel{INTRODUCTION}

The seminal paper by Alain Colmerauer on Metamorphosis Grammars first
appeared in 1975 [9]. That paper spawned most of the developments in compiler
writing using Prolog, a great many of them due to David H. D. Warren. Warren’s
thesis [30], the paper summarizing it [31], and the related work on Definite
Clause Grammars [25] are practically the sole sources of reference on the
subject.\note{A recent book edited by Campbell [3] mostly covers the
implementation of Prolog itself. }

The available published material on the subject describes one particular approach
in implementing compilers using Prolog. It consists of coupling actions 
to recursive descent parsers to produce syntax-trees which are subsequently
utilized in guiding the generation of assembly language code. Although this
remains a worthwhile approach, there is a host of possibilities for Prolog usage
in compiler construction. The primary aim of this paper is to present the material
needed for exposing the reader to the advantages of using Prolog in parsing and
compiling. A second, but equally important, goal of this paper is to show that
Prolog is a labor-saving tool in prototyping and implementing many nonnumerical
algorithms which arise in compiling, and whose description using
Prolog is not available in the literature. Finally, a third goal is to present new
approaches to compiler design which use proposed extensions to Prolog. 

This paper is directed to compiler designers moderately familiar with Prolog,
who wish to explore the advantages and present drawbacks of using this language
for implementing language processors. The advantages of Prolog stem from two
important features of the language. 

(1) The use of unification as a general pattern-matching operation allowing
procedure parameters (logical variables) to be both input and output or to
remain unbound. Unification replaces the conditionals and assignments
which exist in most languages. 

(2) The ability to cope with nondeterministic situations, and therefore allow the
determination of multiple solutions to a given problem. 

From a subjective point of view, the main advantage of Prolog is that the
language has its foundations in logic, and it therefore encourages the user to
describe problems in a logical manner which facilitates the checking for correctness,
enhances program readability, and reduces the debugging effort. It will be
seen that unification and nondeterminism play an important role in compiler
design; however, using their full generality is often costly and unnecessary. These
issues are discussed throughout the paper whenever they become relevant.
Remarks are made in the last section about the efficiency of Prolog-written
compilers and the means to improve their performance. 

The Prolog proficiency assumed in this paper can be acquired by reading the
first few chapters of either Kowalski’s [20] or Clocksin and Mellish’s [6] books.
In particular, the reader should be at ease with elementary list processing and
with the predicate append. The concrete syntax used in this paper is that of
Edinburgh Prolog [6]. It is also assumed that the reader is familiar with compiler
design topics such as parsing, lexical analysis, code generation, optimizations,
and so on. These topics are covered in standard texts [l, 17,29]. 


\secrel{PARSING}\label{cohen2}

In this section we present parsers belonging to two main classes of parsing
algorithms, namely, bottom-up and top-down. Due to the backtracking capabilities
of Prolog, these parsers can in general handle nondeterministic and
ambiguous languages. An early paper by Griffiths and Petrick [18] describes
various parsing algorithms and their simulation by automata. There the amount
of nondeterminism is roughly specified by a selectivity matrix which guides the
parser in avoiding states leading to backtracking. A similar situation occurs in
the Prolog parsers described here. In compilers, interest is commonly restricted 
to deterministic languages. Backtracking may be prevented by a judicious use of
cuts(!) and/or by introducing assertions in the database that guide the parser in
avoiding dead-ends. 

A word about notation is in order. The grammar conventions are those in [l].
Edinburgh Prolog uses capital letters as variables, and therefore capitals cannot
be used to represent nonterminals unless they are quoted. In this paper, the
terms t( ) and n( ) denote, respectively, terminals and nonterminals. Quoting
may be necessary for specifying certain terminals (e.g., parentheses). For example,
the right-hand side (rhs) of the rule
\[F \rightarrow (E)\]
is described by the list 
\[[WV), de), W)‘)l. \]

Whenever stacks are used, they are also represented by lists whose leftmost
element is the top of the stack. 

This section does not pretend to make an exhaustive treatment of parsers.
We describe bottom-up and top-down parsers for both nondeterministic and
deterministic languages. A nondeterministic shift-reduce and a deterministic
weak-precedence parser are the bottom-up representatives. Their top-down counterparts
are, respectively, a predictive and an LL(1) parser. A recursive descent
version of the latter is also considered. Besides those described herein, we have
programmed and tested Earley’s algorithm [13] and a parser generator that
produces the necessary tables for parsing SLR( 1) grammars [ 11]. 

\secdown

\secrel{Bottom-Up}

A very simple (albeit inefficient) shift-reduce parser can be readily programmed
in Prolog. Its action consists of attempting to reduce whenever possible; otherwise
the window is shifted on to a stack and repeated reductions (followed by shifts)
take place until the main nonterminal appears by itself in the top of the stack.
Note that a reduction may be immediately followed by other reductions. A
reduction corresponds to the recognition of a grammar rule; for instance, the
reduction for the rule E + E + T occurs when E + T lies on the top of the stack.
It is then replaced by an E. This action is expressed by the unit clause
\[redme(Idt), H+), n(e) I Xl, Me) I Xl). \]

Let us consider the classical grammar describing arithmetic expressions: 
\[G1: E+E+T\]
\[E-T\]
\[T-+ T*F\]
\[T+F\]
\[F + (E)\]
\[F + (letter)\]
The appropriate sequence of reduce clauses follows immediately from the above
rules. To decrease the amount of backtracking it is convenient to order these
clauses so that rules with longer right-hand sides are tried before those with 
shorter rhs. We are now ready to present the parser. It has two parameters:
(1) a list representing the string being parsed, and (2) the list representing the
current stack. 
\begin{verbatim}
% try-reduce %
sr-parse(Input, Stack) :- reduce(Stack, NewStack),
% try-shift %
sr-parse(Input, NewStack).
sr-parse([ Window I Rest], Stack) :- sr-parse(Rest, [Window 1 Stack]). 
\end{verbatim} 
Assume that a marker (\$) is to be placed at the end of each input string. The
following acceptance clause accepts a string only when the marker is in the
window and the stack contains just an E.
\begin{verbatim}
% acceptance %
sr-paWL§l, Me)lh 
\end{verbatim}
Consider the input string a*b. We assume that a scanner is available to translate
it into the suitable list, understandable by the parser. Then the query
\begin{verbatim}
?- sr-PamWa), t(*), t(b), $1, [ I).
\end{verbatim}
will succeed. 

The above parser is very inefficient, since it relies heavily on backtracking to
eventually accept or refuse a string. Note that in parsing the string a*b, t(a) is
first shifted and successively reduced to an F, T, and (even) an E; the latter
being a faulty reduction. The parser is, however, capable of undoing these
reductions through backtracking. This inordinate amount of backtracking can
be controlled by a careful selection of the reductions and shifts that eliminate
possible blind-alleys. This is done in our next bottom-up parser, which is the
weak-precedence type [ 1,19].

The basic strategy is to consult a table made of unit clauses like
\[try-redue(Top-of-stack, Window). and try_shift(Top-of-stack, Window).\]
which command a reduction or a shift, depending on the elements lying on the
window and on the top of the stack. The problem of automatically generating
the above clauses from the grammar rules is addressed in Section 5. The weakprecedence
relations for the grammar Gi are represented by the clauses 
\[try_reduce(n(t), \$).\]
\[try_reduce(n(f), \$9.\]
\[. . .\]
\[try-reduce(t(‘)‘), t(+)).\]
\[\&-reduce(t(‘)‘), t(‘)‘)).\]
and
\[try-shift(t(+), t(‘(‘)).\]
\[tryshift(n(e), t(+)).\]
and so on.    

We now transform the previous sr-parser into a wp-parser which takes
advantage of the additional information to avoid backtracking. Using 
Griffiths and Petrick’s terminology [ 18], these unit clauses render the
algorithms selective. 
\begin{verbatim}
% acceptance %
w-~~~~4$1, [de)l).
% try-reduce %
wp-parse([ W 1 Input], [S 1 Stack]) :- try_reduce(S, W),
reduce([S 1 Stack], NewStack),
wp-purse([ W 1 Input], NewStack).
% try shift %
wp-parse([ WI Input], [S 1 Stack]) :- try-shift(S, W),
wp-purse(Zrzput, [W, S 1 Stack]). 
\end{verbatim}
Notice that if a grammar is truly a weak-precedence grammar (i.e., there are no
precedence conflicts and rules have distinct rhs), then backtracking will only
occur when try-reduce fails and try-shift has to be tried. Thus the query
\begin{verbatim}
?- wp-parse(Znput, [ I), print(accept).
\end{verbatim}
will print “accept”, and succeed if the Input string is in the language. If the string
is not in the language, the query will fail. The time complexity is proportional to
the length of Input. Error detection and recovery are discussed in Section 9.

A comment about the efficiency of this version of wp-parse is in order. Since
there will in general be a large number of try-reduce and try-shift rules, the
execution time of the wp-parser could be significantly reduced if a Prolog compiler
could branch directly to a clause having the appropriate constant as its first term
(for example, by constructing a hashing table at compile time). Recent and
planned Prolog optimizing compilers can indeed perform this branching [30].
The reader should also refer to [21] for a discussion of optimizations applicable
to deterministic Prolog programs, which render their efficiency closer to those of
conventional programs. 

Finally, note that it would be straightforward to extend this type of parser to
cover the syntactical analysis of bounded-context grammars, that is, those for
which a decision to reduce or shift is based on an inspection of m elements in
the top of the stack and a look-ahead of n elements in the input string. 

\secrel{Top-Down}

A Prolog implementation of predictive parsers [l] follows readily from the
programs described in the previous section. The grammar G2, below, generates
the same language as G1, but left-recursion has been replaced by right-recursion. 
\[Gz: E -+ TE’\]
\[E’ -+ + TE’\]
\[E’ 3 e\]
\[T +FT’\]
\[T’ --B * FT’\]
\[T’ + E\]
\[F *U-O\]
\[F + (letter)\]
The above rules are placed in the database using the unit clauses
\begin{verbatim}
rule(Non-terminal, Rhs).
\end{verbatim}
Examples are
\begin{verbatim}
ruMn(tprim), [t(*), n(f), nt@hne)l).
rule(n(tprime), [ I). 
\end{verbatim}

The parser predict(Input, Stack) has the same parameters as its predecessors,
namely: (1) the input string and (2) the current stack contents (initially n(e),
where E is the main nonterminal). The parser succeeds if the Input string is in
the language, and fails otherwise. 

The basic action of predict is to replace a nonterminal on the top of the parse
stack by the rhs of the rule defining that nonterminal. If a terminal element lies
on the top of the stack and if it matches the element W in the window, then
parsing proceeds by popping W and considering the next element of the input
string to be in the window. A string is accepted when the stack is empty and the
window contains the marker. In Prolog we have 
\begin{verbatim}
% acceptance.
predict(I$l, 1 I).
% try a possible rule.
predict(lnput, [n(N) 1 Stack]) :-
ruldn(N), Rh),
append(Rhs, Stack, NewStack),
predict&put, NewStack).
% match the terminals.
predict([t(W) 1 Input], [t(W) 1 Stack]) :- predict(lnput, Stack). 
\end{verbatim}
The above parser can handle nondeterministic or even ambiguous grammars, but
may become trapped in an infinite recursion loop if the grammar is left-recursive. 

To improve the efficiency when processing deterministic grammars, one could
again resort to placing additional information in the database. This is the case
for the next parser we consider, which is applicable to LL( 1) grammars, and does
not rely on backtracking. It will become apparent in Section 5 that it is straightforward
to generate tables for LL( 1) grammars [ 1]. These tables have as entries
the contents of the window t(W) and the nonterminal n(N) on the top of the
stack, and they specify the appropriate (unique) replacement by the rhs of the
rule defining N. Entries may be defined by unit clauses of the form 
\begin{verbatim}
entv(t(W), n(N), Rhs).
\end{verbatim}
for all pairs ( W, N) such that N J* W . . . . An LL(1) deterministic parser is
obtained by replacing the middle clause of predict by
\begin{verbatim}
predict[t( W) 1 Input], [n(N) 1 Stack]) :- entry(t(W), n(N), Rhs),
append(Rhs, Stack, NewStack),
predict([t(W) I Input], NewStack).   
\end{verbatim}

By properly selecting one among multiple entries, predict can deterministically
parse languages defined by ambiguous grammars, as is the case of the if then
else construct considered in [l, p. 191]. Moreover, the parser does not rely on
backtracking to accept a string. The complexity of the LL(1) parser is therefore
O(n) where n is the length of the input string.

\secrel{Recursive Descent }

All of the previously described parsers contain a general nucleus which drives
the parsing, the grammar rules being specified by unit clauses in the database.
Parser efficiency can be increased by establishing a direct mapping between
grammar rules and Prolog clauses. This is accomplished as in recursive descent
parsing: each procedure directly corresponds to a given grammar rule. As usual,
left-recursion is not allowed and has to be replaced by right-recursion to avoid
endless loops. 

There are three manners in which these parsers can be implemented in Prolog,
depending on the form of the input string. The first and the least efficient of
these is the one that uses the predicate append. The second uses links to define
the input string that appears as unit clauses in the database. Finally, the third,
which uses difference lists, is the most efficient, as will be seen by estimates of
the various complexities. The implementation of these versions is illustrated
using the grammar G3, generating a’kb”, n I 0. The notation t(T) and n(N) will
no longer be needed to differentiate between terminals and nonterminals, since
the nonterminals will be transformed into Prolog procedures which manipulate
terminal strings. 
\[G3: S+aSb\]
\[S4C\]
Every grammar rule is transformed into a clause whose argument is the list of
terminals derived from the defined nonterminal. Terminals are similarly handled
using unit clauses. We have 
\begin{verbatim}
s(ASB) :- uppend(A, SB, ASB),
~PP~W, B, SB),
a(A),
SW,
MB).
s(C) :- c(C).
ml).
WI).
4cl).   
\end{verbatim}
The appends are used to partition the list ASB as the concatenation of three
sublists A, S, B. Although the only partition for which the parser will succeed is
\[A = a, S = an-lcbnml, B = b,\]
this program will generate at least 2n incorrect partitions. Hence the number of
calls needed to append is at least n2. Note that the appends should precede the
calls of a(A), s(S), b(B). Otherwise, an infinite loop would occur. The above
program can be optimized by symbolic execution: the terms a(A), b(B), and c(C)
can be directly replaced by their unit clause counterparts, yielding 
\begin{verbatim}
s(ASB) :- uppend([a], SB, ASB),
wwMS, PI, SB),
SW.
ml).  
\end{verbatim}

The second approach for programming recursive descent parsers in Prolog is
the use of links. An input string such as [a, a, c, b, b] is represented by the unit
clauses link(i, t, i + l), stating that there is a terminal t located between positions
i and i + 1. In our case the input string aacbb becomes
hk(1, a, 2).
\begin{verbatim}
link(2, a, 3).
link(3, c, 4).
link(4, b, 5).
link(5, b, 6). 
\end{verbatim}
A clause recognizing a nonterminal will now have two parameters denoting the
leftmost and rightmost positions in the input string that will parse into the given
nonterminal. In our particular example we have
\begin{verbatim}
s(X1, X4) :- link(X1, a, X2),
s(X2, X3),
link(X3, b, X4).
s(X1, X2) :- link(X1, c, X2). 
\end{verbatim}
The as will be consumed by the n successive calls of the first two literals. Then,
only the second clause is applicable and the c is consumed. Finally, the unbound
variables X3, X4 are successively bound to the points separating the remaining
bs. The algorithm’s complexity is therefore linear. 

An efficient implementation of recursive descent parsers in Prolog makes use
of difference lists. If a nonterminal A generates a terminal string a! (i.e., A ==a*
a), that string can be represented by the difference of two lists U and V; V is a
sublist of U which has the same tail as U. For example, if U is [a, c, b, b, b] and
V is [b, b] the difference U - V defines the list [a, c, b], which for G3 parses into
an S. Warren [31] points out that the use of difference lists corresponds to having
the general link-like clause:
\begin{verbatim}
link([H 1 2’1, H, T) 
\end{verbatim}
which can be read as “the string position labelled by the list with head H and
tail T is connected by a symbol H to the string position labelled T.” A parser for
G3 using difference lists can be written as follows:
\begin{verbatim}
s(U, V) :- a(U, Vl), s(V1, V2), b(V2, V).
s( w, 2) :- c( w, 2). 
\end{verbatim}
For the terminals a, b, and c we have
\begin{verbatim}
am I VII, vu.
b([b I U21, U2).
c(k I u31, U3). 
\end{verbatim}
Symbolic execution allows us to find the values of U and Vl in the first clause:
\begin{verbatim}
U=[alUl], Vl=Ul 
\end{verbatim}
Similarly,
\begin{verbatim}
V2=[bIU2], V= U2
W=[cIu@], z=u3 
\end{verbatim}
Substituting the values of the above variables, we obtain the optimized program
\begin{verbatim}
s([a 1 Ul], U2) :- s(U1, [b I U2]).
4kl u31, U3). 
\end{verbatim}
(The above program could also have been derived using symbolic execution
by considering the first version of the parser with append and noticing that
if X - Y and Y - Z are difference lists, then append(X - Y, Y - 2, X - 2) is
a fact.) 

Let us follow the execution of the call
\[s(b, 0, c, b, bl, [ I).\]
Notice that Ul becomes [a, c, b, b] and U2 is [ ]. The next calls of S are
\[~([a, c, h bl, PII\]
\[s(k, h bl, P, bl)\] 
This last call matches only the second clause thus indicating a valid string. An
informal English description of the acceptance is as follows: successively remove
each a in the head of the first of the difference lists and add a b to the second
one. A string is accepted when no more as can be removed, the head of the first
list is a c, and the two lists contain the same number of bs. Therefore, for this
particular grammar, G3, the parsing is done in linear time with no backtracking.
The reader might have already surmised that the use of difference lists and of
symbolic executions illustrated in this example could be carried out automatically
from the given grammar rules. Clocksin and Mellish ([6, 1st ed., p. 237-2381]
present a short Prolog program that does the translation. 

\secup


\secrel{SYNTAX-DIRECTED TRANSLATION }\label{cohen3}

This type of translation consists of triggering semantic actions specified by the
programmer, once selected syntactic constructs are found by a parser. In the case
of the bottom-up parsers described in Section 2.1, it suffices to add a third
parameter to the reduce clauses specifying the rule number and to modify the
parser so that a semantic action (specified by the rule number) will take place
just after the reduction. For example, in order to translate arithmetic expressions
into postfix Polish notation, the corresponding reduce for the first rule of G1
becomes
\begin{verbatim}
reduce(W), t(+), n(e) I Xl, [n(e) I Xl, 0
\end{verbatim}
The modified parser contains two additional parameters: (1) a stack, Sem, which
will be manipulated by the action procedure and (2) a parameter, Result, which
will be bound to the final result of the semantic actions:
\begin{verbatim}
% accept and bind Result to the semantic parameter.
wp-trunslate([$], [n(e)], Result, Result).
% try to perform a reduction and a semantic action.
wp-transkzte([ W 1 Input], [S I Stack], Sem, Result) :-
&y-reduce@, W),
reduce([S I Stack], NewStack, RuleNumber),
action(RuleNumber, [S I Stuck], Sem, NewSem),
wp-translute([ W I Input], NewStack, NewSem, Result). 
\end{verbatim}
\begin{verbatim}
% try a shift.
wp-transZate([ W 1 Input], [S 1 Stack], Sem, Result) :-
try_shift(S, W),
wp-translute(Znput, [W, S 1 Stack], Sem, Result).  
\end{verbatim}
The parser can then he equipped with actions by adding rules which specify how
the temporary semantic parameter is to be modified for each rule. The following
action procedure constructs parse trees for the arithmetic expressions defined by
grammar G1:
\begin{verbatim}
syntax-tree(Znput, Tree) :- wp-trandute(Znput, [ 1, [ 1, Tree).
action(1, Stack, [Xl, X2 1 T], [plus(X2, Xl) IT]).
action(3, Stack, [Xl, X2 I 2’1, [times(X2, Xl) I 2’1).
action(6, [t(Z,etter) I Stack], Temp, [Letter I Temp]).
action(X, Stack, Temp, Temp) :- X # 1, X # 3, X # 6. 
\end{verbatim}
The body of the last clause guarantees that no spurious actions are performed
should backtracking ever occur. Notice that the action procedure must have
access to the parsing stack (as is the case for rule 6) so that specific terminals
may be incorporated into the actions. A similar strategy is applicable in adding
actions to predictive parsers. 

All of the above descriptions of semantic actions utilize inherited attributes
and are admittedly standard. The main purpose of presenting them here is to
point out how succinct the descriptions become when Prolog is used. The truly
novel way of performing syntax-directed translation is that pioneered by
Colmerauer and widely utilized by Warren. That approach does not strictly
separate syntax from semantics as was done in this section. They have added
new parameters to the recursive descent parser described in Section 2.3, so that
the translation takes full advantage of the unification and goal-seeking features
of Prolog. Colmerauer’s approach is the subject of the next section. 


\secrel{M-GRAMMARS AND DCGs }\label{cohen3}

A metamorphosis (or M-) grammar is a formalism which combines a Chomskytype
language definition with logic programming capabilities for manipulating
the semantic attributes needed to perform syntax-directed translations.
Colmerauer [9] maps general type-0 Chomsky rules into general logicprogramming
clauses, (i.e., those that may contain more than one predicate in
the left-hand side). A very useful subset of M-Grammars are Definite Clause
Grammars (DCGs), which are based on Chomsky’s context-free grammars. The
reader has undoubtedly noticed the similarity between Prolog clauses and context-free
grammar rules: they both have one term in the lhs and several (or none,
i.e., t) in the rhs. Prolog restricts itself to those special clauses called Horn or
Definite clauses, thus explaining the acronym. It will be seen shortly that
although DCGs are based on context-free grammars they are able to parse
context-sensitive ones as well. (In fact, any recursively enumerable language can
be recognized using DCGs with parameters.) 

DCGs are translated directly into Prolog clauses which include a recursive
descent parser using difference lists. For example, the DCG rules for recognizing
strings in G3 are
\[s --+ [cl.\]
\[s --+ [a], s, [b].\]
The syntax of DCGs is close to that of Prolog clauses. The ‘:-’ is replaced
by ‘--+‘, and terminals appear within square brackets. Most Prolog interpreters
automatically translate the above into the clauses:
\[s([c I m, LO).\]
\[s([a 1 LO], Ll) :- s(L0, [b I Ll]).\]
which have already been explained in Section 2.3. DCG terms usually contain
one or more arguments which are directly copied into their Prolog counterparts,
which also contain the difference list parameters. Our first example of usage of
DCGs is to determine the value of n for a given input string a”\&” (generated by
grammar GB ).
\[s(0) --+ [cl.\]
\[s(succW) --+ [al, SW), PI.\]
The added argument specifies that the recognition of a c implies a value of
N = 0. Each time an s surrounded by an a and a b is recognized, the value of N
increases by one (succ indicates the successor). The above DCGs are automatically
translated into
\[SK4 [c I w, LO).\]
\[s(succ(N), [a 1 LO], Ll) :- s(N, LO, [b I Ll]).\]
The call s(X, [a, a, c, b, b], [ ]) yields X = succ(succ(0)). The backtracking
capabilities of Prolog allow the call s(succ(succ(O)), X, [ ]) which yields X =
[a, a, c, b, b]. 

By employing a technique similar to the one illustrated by the previous
example, we can construct a parser s to recognize the language a”b”c”. It uses
the auxiliary procedure sequ.ence(X, N) (defined below) which parses a list of Xs
and binds N to the number of Xs found.
\[sequence(X, 0) --+ [ 1.\]
\[sequence(X, succ(N)) --+ [Xl, sequeme(X, IV)\]
\[s(N) --+ sequence(a, N), sequence(b, N), seqwme(c, IV)\]

Let us now consider the use of DCGs for translating arithmetic expressions
into their syntax-trees. We start with the simplified right-recursive grammar
rules:
\[E-T +E\]
\[E-T\]
\[T-a\]
Initially one would be tempted to use the DCG
\[eWw(X, Y)) --+ W3, [+I, e(y).\]
\[e(X) --+ t(X).\]
\[t(a) --9 [a].\]
These rules, however, translate a + a + a into plus (a, pZus(u, a)) which is rightassociative,
and therefore semantically incorrect. Some cunning is needed to 
circumvent this difficulty. Let us first rewrite the grammar rules as
\[E+TR\]
\[R++TR\]
\[R-,C\]
\[T+a\]
Our goal is to generate plus(plus(a, a), a) for the input string a + a + a. The
following DCG will do the proper translation:
\[eqv\$?3) --9 term(Tl), restexpr(T1, E).\]
\[restexpr(T1, E) --a [+I, term(T2), restezpr(plus(T1, T2), E).\]
\[restexpr(\&, E) --+ [ 1.\]
\[term(a) -3 [a].\]
When the clause expr recognizes the first term Tl in the expression, it passes
this term to the second clause, restexpr. If there is another term T2 following
Tl, then the composite term pb(T1, T2) is constructed and recursively passed
to restexpr. The first parameter of restexpr is used to build a left-recursive parse
tree, which is finally transmitted back to erpr by the third clause. 

Unfortunately, the above “contortions” are needed if one insists on using a
recursive descent parser to contruct a left-associative syntax tree. This particular
Prolog technique has become a standard idiom among DCG writers. A way out
of this predicament is to implement DCGs using bottom-up parsers. (This has
been proposed in [28].) At present these capabilities are not generally available
in existing Prolog interpreters and compilers. 

It is straightforward to generalize the above translation by introducing multilevel
grammar rules such as
\[Ei + TiRi\]
\[Ri + OpiTiRi\]
\[\& + \&+I\]
\[Ri + c\]
with 1 5 i 5 n and En+, + letter ] (El), where i denotes the precedence of the
operator opi. The corresponding DCG contains i as a parameter, and could allow
for redefining the priorities of the operators, therefore rendering the language
extensible. This approach is used in the Edinburgh version of Prolog.      

A very useful feature of DCGs is that parts of Prolog programs may appear in
their right-hand sides. This is done by surrounding the desired Prolog predicates
within curly brackets. Our next example illustrates the use of this feature to
perform the translation of arithmetic expressions into postfix notation directly
by a DCG that does not construct syntax trees. Our first example of this technique
will output the postfix notation.
\[e --+ t, r.\]
\[r --+ [+I, t, write(+)), r.\]
\[r --+ [ 1.\]
\[t --+ [a], (write(a)).\]
This procedure produces the postfix expression using side effects, and this
technique can be a drawback. One solution to this problem is to use difference 
lists to simulate the write procedure. To each DCG clause corresponding to a
nonterminal N we add difference list parameters representing the list of symbols
that are output during the recognition of N. (These difference lists are in addition
to those used in syntactic analysis). We have
\[e(F1, F3) --a t(F1, F2), r(F2, F3).\]
\[r(F1, F4) --+ [+I, t(F1, F2), writefik(+, F2, F3)), r(F3, F4).\]
\[r(F, F) --+ [ 1.\]
\[t(F1, F2) --+ [a], (writefik(a, Fl, F2)).\]
The output is simulated by the procedure writefile(Symbo1, Pos, NewPos) defined
by the unit clause
\[writefile(X, [X 1 B], B).\]

The call ?- e(F, [ 1, [a, +, a, +, a], [ ]]) produces F = [a, a, +, a, +]. This
example shows that difference lists can be used both to select parts of a list and to
construct a list. It is not hard to write a program that automatically performs
the translation from a DCG using write to a DCG using writefile and additional
difference lists. In the remainder of the paper we use the procedure write, and
leave to the interested reader the task of adding difference lists to avoid side
effects. 

The BNF of full-fledged programming languages can be readily transcribed
into DCGs that translate source programs into syntax-trees which can then be
either interpreted or used to generate code. We have tested the DCG needed to
process the entire Pascal language by translating input programs into syntaxtrees.
The following program fragments illustrate this construction for parts of a
mini-language. A while statement is defined by the DCG: 
\begin{verbatim}
statement(while(Test, Do)) --+
[while], test(Test), [do], statement(Do).
\end{verbatim}
A test may be defined by
\begin{verbatim}
test(test(Op, El, E2)) --+ expr(&l),
comP(oP),
expr(E2).
camp(=) --+ [=I.
cow(( )I --+ [( )I.
etc . . .  
\end{verbatim}
The translation of statements into P-code-like instructions is also easily achieved.
For example the statement while T do S can be directly translated into the
sequence
\begin{verbatim}
L: code for test T
jif(i.e, jump if false) to Exit
s
jump to L
Exit:       
\end{verbatim}
If labels are represented by terms of the form label(L) and the instructions by
instr(jif, L) or instr(jump, L), the translation is performed by the DCG:
\begin{verbatim}
statement([label(L), Test, S, instr( jump, L), lubel(Exit)]) --+
[while], test(Test, Exit), [do], statement(S). 
\end{verbatim}
where
\begin{verbatim}
test([Rl, R2, Op, in.str(jif, Exit)], Exit) --+
expr(E1, Rl),
comP(oP),
expr(E2, R2). 
\end{verbatim}
This example illustrates the elegant use of Prolog’s logical variables and unification
in compiling. Each of the variables L and Exit occur twice in the generated
code. When instantiated, each pair will be bound to the same actual value. This
instantiation may occur at a later stage when the final program is assembled and
storage is allocated. Even when using special compiler-writing tools such as
YACC, the implementation of similar constructs requires lengthier programs
since one has to keep track of locations that have to be updated when final
addresses become known. Prolog’s ability to postpone bindings is therefore of
great value in compiling. 

The advantage of using logical variables and delayed binding is also
apparent in managing symbol tables. Consider the procedure Zookup(ldentifier,
Property, Dictionary), in which Dictionary is a list containing the pairs
(identifier-property); lookup’s behavior is similar to that of the procedure
member(E, L) which tests if an element E is present or not in the list L. However,
lookup adds the pair to the Dictionary if it has not been previously added. We
have
\begin{verbatim}
lookup(l, P, [[I, P] 1 T] :- !.
lookup(1, P, [[II, Pl] 1 T] :- lookup(I, P, T).
\end{verbatim}
If lookup is initially called with an uninstantiated variable, the first clause will
create the new pair [I, P] as well as a new uninstantiated variable T. The cut is
needed to prevent backtracking once the desired pair is found or is created. 
Consider now the sequence of calls to lookup:
\begin{verbatim}
lookup(a, Xl, D), lookup(b, X2, D), lookup(a, X3, D). 
\end{verbatim}

The net effect of the above calls is to store the two pairs [a, Xl] and [b, X2] in
D and to bind X3 to Xl. Later on, when Xl and X2 are instantiated, X3 will
automatically be bound to the value of Xl. 

A similar approach is used in [31] to implement binary tables. In that paper,
table lookup is done in the code-generation phase after constructing the syntax
trees (see Section 7). If one wished to perform that operation while parsing, the
DCG rule defining a factor could be
\begin{verbatim}
WU, PI, D) --+ Went(Z (bokup(V, PI, D)J. 
\end{verbatim}
where ident(1) is constructed in a previous scanning pass and the property P is
determined while processing declarations. In this case lookup should be modified
to handle semantic errors such as undeclared identifiers. 


\secrel{GRAMMAR PROPERTIES }\label{cohen5}

This section makes extensive use of the built-in predicate setof which implicitly
relies on the nondeterministic capabilities of Prolog. In our view the use of this
and similar predicates in determining grammar properties is perfectly justifiable,
since, in this context, efficiency plays a secondary role: grammar properties are 
usually determined only a few of times when generating the parser and, although
it is important that the generated parser itself be efficient (and deterministic),
longer generation times are usually tolerable. 

We start by pointing out that it is easy to test whether a grammar is strictly
weak-precedence or LL(l), provided one knows the sets first(N), follow(N), and
last(N). The Prolog procedures for performing these tests follow the declarative
definitions closely and appear at the end of this section. We first show how
Prolog can be used to calculate these sets in the general case of context-free
grammars which may contain left-recursive nonterminals and e-rules. 

We assume that the rules for a grammar are stored in the database by assertions
like
\[rule(RuleNum, n(A), Rhs)\]
in which RuleNum is an integer number identifying a rule, Rhs is the list
representing the right-hand side of the rule defining the nonterminal A. Recall
that the elements of Rhs are identified by the terms of the form t(T) and n(N)
representing terminals and nonterminals. 

For each nonterminal N in the grammar, first(N) is the set of all (terminal or
nonterminal) symbols V such that N a* V . . . . To calculate the set first(A) for
a nonterminal A, we use the built-in procedure setof in conjunction with a
procedure first which finds a single element of this set. Thus, we make the toplevel
call:
\[allfirst(N, L) :- setof(X, first(N, [ 1, X), L).\]
The procedure \verb|first(Input, &a&, V)| has three parameters:

(1) an Input list representing a sequence (Y of terminals and/or nonterminals,

(2) a Stuck of rule numbers which keeps track of the already considered rules,

(3) a terminal or nonterminal element V such that (Y =J* V . . . .

There are three ways in which a symbol T can be the first element of a
sentential form derived from (Y: (1) it can be the first element of (Y, (2) it can be
the first element of a sentential form obtained by rewriting the first element in
cu (which must be a nonterminal in this case), or (3) it can be the first element
of a sentential form obtained by rewriting some of the initial nonterminals of (Y
into the empty string t. The following procedure contains a clause for each of
these three cases. The middle parameter \verb|&a&| is used to prevent looping by
prohibiting the consideration of previously used rules. The third clause uses the
procedure reduces-to-epsilon (defined below) to determine if a sequence of
nonterminals rewrites into 6 
\begin{verbatim}
first([Symbol 1 Rest], Stack, Symbol).
first([n(N) 1 Rest], Stack, Symbol) :-
rule(Number, n(N), Rks),
not(member(Number, Stack)),
first(Rks, [Number 1 Stuck], Symbol).
first(List, Stack, Symbol) :-
append(A, B, List),
Af[l,
reduces-to-epsilon(A),
first@, Stuck, Symbol). 
\end{verbatim}   
The predicate reduces-to-epsilon(A) will succeed if A represents a sequence cu of
nonterminals which rewrite into the empty string. If a sentential form reduces
to epsilon, then it must consist entirely of nonterminals that reduce to epsilon.
Moreover, if a nonterminal rewrites to epsilon, then there is a parse tree
representing this reduction such that no branch of the parse tree contains more
than one occurrence of any nonterminal. The translation of these two statements
into Prolog is straightforward. The procedure list-reduces-to-epsilon asserts that
a sequence of nonterminals List rewrites into epsilon if each of the nonterminals
does, and the procedure nt-reduces-to-epsilon asserts that a nonterminal N
reduces to epsilon if it rewrites into a sentential form that reduces to epsilon.
The stack parameter is used to guarantee that no branch of the parse tree
contains multiple occurrences of any nonterminal. 
\begin{verbatim}
reduces-to-epsilon(List) :-
list-reduces-to-epsilon(List, [ I).
list-reduces-to-epsilon([ 1, Stock).
list-reduces-to-epsilon([n(N) 1 Rest], Stack) :-
nt-reduces-to-epsih(n(N), [n(N) 1 Stack]),
list-reduces-to-epsilon(Rest, Stack).
nt-reduces-to-epsih(n(N), Stack) :-
rule(Number, n(N), Rhs),
not(intersect(Rhs, Stuck)),
list-reduces-to-epsilon(Rhs, Stack).
intersect(List1, L&2) :- member(X, L&l), member(X, L&2). 
\end{verbatim}
In weak-precedence, parsing reductions are called for when S > IV, where

(1) W is the terminal element in the window,

(2) S is the (terminal or nonterminal) element in the top of the stack,

(3) S > W if there exists a grammar rule

\[Y-a **- x,x,...,\]
where Xl ++ - - - S and WE first(Xz). 

Shifting occurs when S c W, that is, if there is a rule
\[Y-* *** sx, . . . . where W E first(X*).\]
To determine whether a language is of the weak-precedence type and to construct
the parsing tables, one needs to find for each nonterminal X the set last+(X),
consisting of all terminals and nonterminals V such that X J+ . . . V. This can
be done by finding the sets first(X) for the grammar that is obtained by reversing
the right-hand sides of the rules in the original grammar. It is easy to define a
procedure first-rev that finds the sets first(A) for the reversed grammar by
modifying the procedure first. The procedure to compute k\&+(A) is then
concisely expressed as follows:
\begin{verbatim}
lost-ph(n(X), 2) :-
rule(Number, n(X), Rhs),
reverse(Rhs, RRhs),
first-rev(RRhs, [ 1, Z).  
\end{verbatim}
As before, the set last+(A) can then be found using the setof predicate:
\[aUo.st-ph(n(A), L) :- setof(X, last-plus(n(A), [ 1, X), L).\]

The set follow(N) is also succinctly expressed in Prolog. There are two ways
in which a symbol V can be in the set follow(N): (1) there is a rule X --* \&V/3
such that V E first(@), or (2) there is a rule X + OlNp such that B rewrites into
epsilon, and V E follow(X). The Prolog procedure for follow consists of two
clauses closely paralleling these two cases. The middle parameter \verb|&a&| is
again used to prevent looping by prohibiting the multiple use of rules:
\begin{verbatim}
follow(n(N), Stack, Terminal) :-
rule(Number, n(X), Rhs),
not(member(Number, Stack)),
wwW& In(N) I Bl, Rh),
first(B, [ 1, Terminal).
follow(n(N), Stack, Terminal) :-
rule(Number, n(X), Rhs),
not(member(Number, Stack)),
~PP=U, [n(N) I Bl, Rh),
reduces-to-epsilon(B),
follow(n(X), [Number 1 Stack], Terminal).  
\end{verbatim}
The predicate all-follow below calculates the list of all follow symbols of a
nonterminal N:
\[all-follow(N, L) :- setof(X, follow(N, [ 1, X), L).\]
To assess the gains in program size and readability the reader may want to
compare the above programs with the English description of first and follow in
[l, p. 184] and with a Pascal version in [2]. As to efficiency, these programs
could be significantly improved by using the assert procedure to memorize previously
computed firsts and follows, thereby avoiding recomputation. (This technique,
called memoization, has been considered in [ 24].)  

The predicates first and follow and lust-plus can be used to test for the LL(1)
and weak-precedence grammar properties and to generate the parsing tables for
each of these types of grammars. For example, the clauses of the try\_reduce
procedure can be computed by the following procedure:
\begin{verbatim}
generate-reduces(L) :-
setof(try-reduce(X, Y), wp-greater(X, Y), L).
wp-greater(X, Y) :-
rule(RuleNum, n(N), Rhs),
appe~(Awl, [A, B I AnyPI, Rh),
last-plus(A, X),
first([B], Y). 
\end{verbatim}
and the try\_shift clauses can be generated in a similar manner. Once these
clauses have been computed and stored in the database, the grammar can be tested for
weak-precedence by the query:
\begin{verbatim}
not-weak-precedence :- try-reduce(S, W), try-shift(S, W).
not-weak-precedence :- rule(N, X, Rhs), rule(M, Y, Rhs), N # M 
\end{verbatim}
The first clause tests for reduce-shift conflicts, which could easily be reported to
the user for selecting the desired action. This choice enables the processing of
ambiguous grammars. The second clause tests if two grammar rules have identical
right-hand sides. 

The procedures to generate LL(1) tables and to test whether a grammar is
LL(1) can also be written concisely. The procedure that generates the tables
consists of a call to setof, combined with a procedure to find the firsts and follows
of the right-hand side of a rule:
\begin{verbatim}
generate-ill-table(L) :-
setof(entry(t(W), n(X), Rh), first-of-ruk(t(W), n(X), Rh), L).
first-of-r&( W, N, Rh) :-
ruk(Number, N, Rhs),
first(Rhs, W).
first-of-r&( W, N, Rhs) :-
rule(Number, N, Rh),
reduces-to-epsilon(Rhs),
follow(N, W). 
\end{verbatim}
To test whether a language is LL(1) we must show that the table constructed
above has no multiple entries. This can be done with a call to the procedure
\begin{verbatim}
n&-111, defined as follows:
not-ill :- entry(t( W), n(X), Rhsl), entry(t( W), n(X), Rhs2), Rhl # RhsP.
\end{verbatim}

The generation of the unit clauses for weak-precedence and LL(1) parsing
actually amounts to prototyping a parser generator. Additional discussion on this
topic is given in Section 6. The predicates first and last can also be used to
determine if a grammar contains a nonterminal that is left-recursive and also
right-recursive. This is a commonly used test for attempting to detect ambiguity
in context-free grammars. 

There is a host of grammar properties and transformations that could be
succinctly described in Prolog. A few that we have programmed are elimination
oft rules, general replacement of left-recursive rules by right-recursive ones, and
reduction to Chomsky and standard normal forms. Other properties that seem
likely candidates for description in Prolog are an attempt to determine if a
grammar is LL(k) or LR(k), and the reduction of an LR(K) grammar to LR(l). 


\secrel{LEXICAL SCANNERS AND PARSER GENERATION }\label{cohen6}

We first note that the syntax of regular expressions is quite similar to that of
arithmetic expressions. The union ( I) replaces the add operator and concatenation
(represented by a blank or period) replaces the multiplication operator. The
star operation may be represented by surrounding a starred sequence by curly
brackets. The translation of a regular expression into its syntax-tree is performed
either using DCGs (Section 4) or triggering the semantic actions described in
Section 3. For examle, the expression ((a 1 b).c).d is translated into the tree:
cone (star (cpnc (union (a, b), c)), d). We now present a recognizer accepting strings
defined by a regular expression given by its syntax-tree. The first argument of
the procedure ret is the syntax-tree, the other two are difference lists (as described
in Section 4).
\begin{verbatim}
rec(L, [L 1 U], U) :- letter(L).
rec(stur(X), U, V) :- rec(X, U, W), rec(stur(X), W, V).
rec(star(X), U, U).
rec(unzbn(X, Y), U, V) :- rec(X, U, V).
rec(union(X, Y), U, V) :- rec(Y, U, V).
rec(conc(X, Y), U, V) :- rec(X, U, W), rec(Y, W, V).  
\end{verbatim}
The above interpreter for regular expressions is admittedly inefficient, since it
relies heavily on backtracking. Nevertheless, it might be suitable for fast prototyping.
An efficient version of the recognizer may be obtained in three steps:

(1) translation of regular expressions into a nondeterministic automaton containing
t moves;

(2) reduction of the automaton in (1) to a deterministic one not containing e
moves (in the cases where the empty symbol is in the language, a complete
elimination of t moves is not possible);

(3) minimization of the automaton obtained in (2). 

The above steps are those performed by LEX, a scanner-generator package
developed at Bell Labs. A Brandeis student, Peter Appel, has prototyped a Prolog
version of LEX in less than one month. His program can handle practically all
features of LEX, but is admittedly slow compared with the original C-version of
that package. When compiled, his program can generate a scanner for a minilanguage
similar to that in the appendix of [l] in about four minutes. However,
it should be noted that Appel’s program is considerably (about five times) shorter
than the C-counterpart, and it took a fairly short time to develop. Since the
Prolog programs are deterministic, further gains in efficiency could be expected
by applying the optimizations suggested by Mellish [21]. In Section 9 we briefly
describe an alternate approach to scanner-generation using proposed extensions
of Prolog. 

In the remainder of this section we sketch two approaches for prototyping
parser generators. The first generates recursive descent parsers, whereas the
second produces SLR(1) parsers of the type used in YACC [ 1].

The recognizer of regular expressions presented earlier in this section can be
easily modified to recognize context-free languages specified by rules whose righthand
sides are themselves regular expressions. For example the rule
\[E --> T(T)\]
can be described by the unit clause
\[ruk(n(e), conc(n(t), star(conc(t(+), n(t))))).\]
The new clause for ret becomes
\[rec(n(A), U, V) :- rule(n(A), R), rec(R, U, V).\]
It is straightforward to prototype a parser generator by implementing the
following steps.

(a) Determine manually the syntax-trees for a grammar B specifying the syntax
of the grammar rules themselves. Each nonterminal N has its corresponding
syntax-tree TN asserted in the database by rule(N, 7~).

(b) Use the modified recognizer ret to parse strings of B, that is, a set of grammar
rules specifying a context-free grammar G.

(c) Attach actions to ret so that it produces the syntax-trees for the grammar G
being read. This step has been described in Section 3.

(d) Once the trees for G are generated, ret itself can be reused to parse the
strings generated by G.

A detailed description of the above steps appears in [14]. A further advantage of
this approach is the possibility it offers to generate efficient recursive descent
parsers [ 7]. One may “compile” assembly language code for a parser by “walking”
on the syntax-tree of a grammar G. 

Another option for parser generation is to use Prolog for producing the tables
for SLR(l) parsers, given a set of grammar rules. An item of a grammar G is a
production of G with a dot at some position of the right-hand side. Each item
can be computed as a triplet (N, D, L) in which N is the rule number, D the dot
position, also an integer, and L is the length of the right-hand side. (One could
also have used only N and D and recomputed L for each rule N whenever needed.)
States are implemented as lists of triplets. The main procedure generates all new
transition states stemming from a given state. Termination occurs when no new
states are generated. Ancillary procedures are needed to check if the element
preceding a dot in a triplet is a nonterminal, or to test if an item has the dot at
the end of a rule. This latter check is readily achieved by testing for (N, L, L).
Another auxiliary procedure determines all triplets that should be added to a
given state once it is found that that state contains items having a dot preceding
a nonterminal. 

The predicate follow (see Section 5) is called to determine the expected window
contents that trigger reductions. These correspond to states containing items
ending with a dot. As in YACC, the parser generator can produce tables with
multiple entries, allowing the user to select the appropriate entry which renders
the parsing deterministic. 

A Prolog version of YACC has been prototyped at Brandeis by Cindy Lurie.
Her program was developed in a couple of months. In addition to generating a
parser, it also produces the code embodying the error-detection and recovery
capabilities suggested by Mickunas and Modry [22]; the correction costs being
interactively supplied by the user (see Section 9). The performance of the Prolog
version of YACC is comparable to that of the LEX counterpart. The previous
remarks about the efficiency remain applicable. A word is in order about the
generated scanners and parsers. They are C-programs which, when optimized,
can approach the efficiency of those generated by LEX and YACC. 

 
\secrel{CODE GENERATION }\label{cohen7}\secdown

\secrel{Generating Code from Polish }

We start by describing a simple program that generates code for a single register
computer having the usual arithmetic operations, as well as the LOAD Vur and
STORE VW instructions, where Vur is the location of a variable. The DCG for
performing the translation is basically that used to generate the postfixed Polish
described in Section 4. The algorithm essentially operates as follows:

(1) When a variable is recognized it is placed on a stack.

(2) When an operation is recognized its two operands are on the top of the stack.
If these are variables the following instructions are generated:
\[LOAD 1st operand\]
\[Operation 2nd operand\]

Step (2) is continued by replacing the top elements of the stack with the mark
ccc to indicate that, at execution time, the result will be in the accumulator. To
take into account this mark we introduce the revised versions of (1) and (2),
which handle the cases where one of the operands is an occ mark.

(la) Before pushing a variable onto the stack it is necessary to check if the mark
ccc occupies the position just below the top of the stack. This indicates the
need of a temporary storage, since the accumulator was already utilized in
a previous operation and it contains a result that should not be destroyed.
Thus the mark act is replaced by Ti, the ith element of a pool of temporary
locations, and the following instruction is generated: ST0 Tie It is then
possible to push the recognized variable onto the stack.

(lb) If the penultimate element in the stack is not ccc, the variable is simply
pushed onto the stack.  

As for operators, two cases need to be considered: one for commutative
operations, the other for noncommutative ones. Let Sl be the top of the stack
and S2 the element just below it.

(2a) If neither Sl, nor S2 is an act, then code is generated as in step (2) above.

(2b, c) For the commutative operations (addition and multiplication) it suffices
to generate
Operation Sl if S2 is an ccc, or to generate
Operation S2 if Sl is an occ.

(2d) Noncommutative operations (subtraction and division) will check if Sl
is an act, in which case the instruction ST0 Ti has to be generated and
the stack updated with Ti instead of ccc, as is done in (la). The generation
proceeds as indicated in (2). The case where S2 is an ccc is processed as
in the case of commutative operations (2b).

The above description can be easily summarized in Prolog. For presentation
purposes, we assume that the arithmetic expression has been parsed into postfixed
Polish notation. We also assume that variables are represented by terms of the
form u (Name) and operators by terms op(Op). In an actual implementation the
semantic actions described below would be triggered directly from the DCG rules. 

The procedure gen-code(Polish, Stack, Temps) traverses the list Polish and
outputs the code as soon as it is generated. We remind the reader that a program
that produces output using writes can easily be modified so that it stores the
output in a list and thereby avoids relying on side effects to generate results (see
Section 4). The gen-code procedure is initiated with a call to the procedure
execute, defined by
\[execute(l) :- gen-code\& [ 1, [O]).\]
where L is the input in postfix. The operators and operands in the list L trigger
calls to the corresponding operator and operand clauses, which modify the Stack
as described above, and may either remove or return a location from the list 
Temps of available temporary locations:
\begin{verbatim}
gen-code([op(Op) 1 Rest], Stack, Temps) :-
operator(Op, Stack, NewStack, Temps, NewTemps),
gen-code(Rest, NewStack, NewTemps).
gen-code([u(X) 1 Rest], Stack, Temps) :-
operand(X, Stack, NewStack, Temps, NewTemps),
gen-code(Rest, NewStack, NewTemps).
gen-code([ 1, AnyStack, AnyTemps). 
\end{verbatim}

The operator and operand clauses have five parameters:

(1) the variable (or operator) being examined,

(2,3) the starting and resulting stack configurations,

(4,5) the starting and resulting lists of available temporary locations. 

The following remarks will help in understanding the semantic actions of the
procedures. The program assumes the availability of an unlimited number of
temporary locations which are reused whenever possible: a temporary is returned
to its stack after emitting an instruction of the type LOAD Ti or Op Tie The list
of available temporary locations is initialized to contain only the location T,,.
Whenever a new temporary is needed, it is taken from this list, and if the list
contains only one element a new temporary is generated (see the second clause
of get-temp below). The term t(X) is used to represent a temporary location.
\begin{verbatim}
% Case (la). ’
operand(X, [A, act 1 Stack], [u(X), A, t(I) 1 Stack], Temps, NewTemps) :-
get-temp(t(I), Temps, NewTemps),
write(st0, t(I)).
% Case (lb).
operand(X, Stack, [u(X) 1 Stack], Temps, Temps). 
\end{verbatim}
The first clause of operand guarantees that the accumulator is always the first
or second element of the stack, if it occurs at all. The other elements in the Stack
are either temporaries or variables:   
\begin{verbatim}
% Case (2b).
operator(Op, [A, act I Stack], [act 1 Stack], Temps, NewTemps) :-
codeop(Op, Instruction, AnyOpType),
gen-instr(Instruction, A, Temps, NewTemps).
% Case (2~).
operator(Op, [act, A I Stack], [act I Stack], Temps, NewTemps) :-
codeop(Op, Instruction, commute),
gen-instr(Instruction, A, Temps, NewTemps).
% Case (2d).
operator(Op, [act, A I Stack], [act I Stack], Temps, NewTemps) :-
codeop (Op, Instruction, rumcommute),
get-temp(t(Z), Temps, TempsO),
write(st0, t(Z)),
gen-instr(load, A, TempsO, Tempsl),
gen-instr(Instruction, t(I), Tempsl, NewTemps).
% Case (2a).
operator(Op, [A, B I Stack], [act I Stack], Temps, NewTemps) :- ’
A#acc,B#acc,
codeop(Op, Instruction, OpType),
gen-instr(load, B, Temps, Tempsl),
gen-instr(Instruction, A, Tempsl, NewTemps).
\end{verbatim}

Notice that at most one of the clauses for operator can succeed, since there can
be at most one act in the stack. Thus the ordering of the clauses is unimportant,
and there is no need for cuts. 

The remainder of the program consists of a few auxiliary procedures. The
procedure get-temp simulates the pop operation for a stack containing the
currently available temporary locations. Temporary locations are returned to the
stack by the first clause of gen-instr.
\begin{verbatim}
codeop(+, add, commute).
codeop (-, sub, noncommute).
codeop(*, mult, commute).
codeop(/, diu, noncommute).
gen-temp(tU), V, J I RI, [J I fW.
getj%y?y), 14, [Jl) :-
genhstr(Znstruction, t(Z), Temps, [I 1 Temps]) :-
write(Znstruction, t(Z)).
gen-instr(Znstruction, v(A), Temps, Temps) :- write(Znstruction, A). 
\end{verbatim}

The code generated for the expression A * (A * B + C - C* D) is
\begin{verbatim}
LOAD A
MULT B
ADD C
ST0 To
LOAD C
MULT D
ST0 Tl
LOAD T,,
SUB Tl
MULT A 
\end{verbatim}

An alternative approach to the method presented here is to generate new
Prolog variables to represent the temporaries as they are needed and to ensure,
in a subsequent pass, that their usage is optimized. 

\secrel{Generating Code from Trees }

A more general approach to code generation is based on “walks” in the syntaxtree
of a program. We start by describing Warren’s approach [31] for generating
code for a fictitious machine. This computer performs arithmetic operations
using a single accumulator. The corresponding instructions are ADD, MULT,
SUB, and DIV. Operations of the type ADDI, MULTI, and so on, are also
available, and consider the value immediately following them as the second
operand in the computation. LOAD and ST0 commands are of course present,
as well as the unconditional transfer (JUMP) or conditional ones such as J xx,
where xx is EQ, NE, GT, and so on. The input/output commands are simply
READ and WRITE. The generator consists of the clause encode-statement which
identifies the node of the syntax-tree and constructs the corresponding code. The
generated code is a list of instructions and labels, (possibly containing embedded
sublists), for instance,
\begin{verbatim}
[. . . label(LI), [instr(LOAD, X), instr(ADDI, 3)], . . -1 
\end{verbatim}

In Warren’s paper the arguments of instructions are stored in a dictionary,
but remain unbound to actual memory addresses until the very final phase of the
compiler. At that time an assembler determines the addresses of labels, and an
allocator binds the addresses of the variables and reserves the number of memory
locations needed to run the compiled program. We now present some fragments
of Prolog programs that perform the generation. An assignment of an expression
Expr to a variable X is translated into the list whose head is the generated code
for the expression followed by the instruction ST0 X. The procedure encodestatement
has three arguments: the syntax-tree, the dictionary Diet, and the
resulting code. We have
\begin{verbatim}
encode-statement(assign(name(X), Expr),
Diet,
[Exprcode, instr(sto, Addr)]) :-
lookup(X, Addr, Diet),
encode-expr(Expr, Diet, Exprcode). 
\end{verbatim}

The procedure lookup stores the new variable X if it is not yet entered in Diet
and retrieves the unbound variable representing its address (see Section 4). 

The procedure encode-expr can handle two shapes of arithmetic expression
syntax-trees. In the first the right operand is a leaf (i.e., a variable or a constant).
In the second the right operand is a subtree. The syntax-tree for arithmetic
expressions has internal nodes labeled by the operator Op. The more complex
case where the right operand is a subtree is presented below. Its action is to
translate expr(Op, Exprl, Expr2) (in which Expr2 is of the form expr(Op, Anyl,
Any2) into the sequence containing

(1) the code for Expr2,

(2) the instruction ST0 temp,

(3) the code for Exprl, and finally,

(4) the code for the instruction specified by Op. 

An added argument N is needed to specify the pool of temporary locations. Its
initial value is zero. In Prolog we have
\begin{verbatim}
encode-subexpr(expr(Op, Exprl, Expr2), N, Diet,
[Exprkode, instr(sto, Addr), Exprlcode, instr(Opcode, Addr)]) :-
complex (Expr2),
lookup(N, Addr, Diet),
encode-subexpr(Expr2, N, Diet, ExprZcode),
NlisN+l,
encode-szbexpr(Exprl, Nl, Diet, Exprlcode),
memoryop (Op, Opcode).
complex(expr(Op, Anyl, AnyP)).
memoryop(+, add).
memoryop(*, m&t).
\end{verbatim}

The code generated for the previous expression A* (A* B + C - C* D) now
becomes
\begin{verbatim}
LOAD C
MULT D
ST0 TO
LOAD
MULT ii
ADD C
SUB To
ST0 T,,
LOAD A
MULT T,, 
\end{verbatim}
Note that since a right subtree is evaluated before a left one, the code for C* D
is the first to be generated. 

The use of labels is illustrated by the generation of code for while statements.
The translation consists of transforming the syntax-tree while (Test, Do) into the
code
\begin{verbatim}
hbel(L1): (encode Test)
(encode Do )
jump Ll
lubel(L2): 
\end{verbatim}

Note that a new argument (L2) is needed in the procedure that encodes tests
to generate the jump to the exit label. The Prolog program to achieve the
translation parallels the above description.
\begin{verbatim}
encode-statement (while (Test, Do), Diet,
[hbel(Ll), Testcode, Docode, in&( jump, Ll), bbel(L2)]) :-
encode-test (Test, Diet, L2, Testcode),
encode-statement(Do, Diet, Docode). 
\end{verbatim}

\secrel{A Machine-Independent Algorithm for Code Generation }

An alternate approach to code generation is that proposed by Glanville and
Graham [15, 16]. It is assumed that by syntax-directed translation a source
program is translated into its prefix Polish counterpart. A second syntax-directed
translation of the prefix code then produces actual machine code. The interesting
feature of this approach is that the grammar used to recognize the prefix takes
into consideration the description of the machine for which code is generated.
Consider a register machine whose operations are of the type
\[LOAD M, R\]
\[ADD R1, R2 or ADD M, R\]
\[ST0 R, M\]
\[ADDI C, R\]
where M is a memory address, C is a constant, and R is a register, and the first
argument is the source, the second the destination. To simplify the presentation,
we assume an unlimited pool of registers. The problem of dealing with a limited
number of registers is discussed in the next section.

The grammar rule
\[R + op R var 1 var\]
describes a prefix string in which the last operand is always a variable, (e.g., as
in + + a b c). The code to be generated in this case can be triggered by semantic
actions corresponding to the rules 

\[R + var\]
Action: Load variable into register r

\[R + op R var\]
Action: 1. recognize (recursively) the left operand R assuming that it will
use register r

2. generate the code: op var r

Similar grammar rules are applicable for generating code when the last operand
is a constant. The more general case corresponds to the grammar rule
\[R + op R R 1 var 1 const\]

In this case a new register is needed before recursing to the second R. Also, a
register becomes available after recognizing the second R. A natural way of
implementing the Glanville-Graham approach is through the use of DCGs. The
following simplified grammar rules express assignments:
\[A + := var R\]
\[R+opRvar~opRconst\]
\[R-+opRR\]
\[R + var I const\]

Note that this is an ambiguous grammar, and therefore the use of cuts at the end
of each clause is recommended to avoid generating multiple solutions. The
recursive descent compiler generated from the DCGs opts, whenever possible, to
the first rule defining R, instead of the more general second rule. 

The procedures listed below specify the syntax-directed translation of prefix
Polish into assembly language according to the above grammar rules. The
procedure reg corresponds to the nonterminal R and has three parameters:

(1) generated assembly language sequence,

(2) register containing the final result,

(3) dictionary for storing variables.   

Although the presented program assumes an unlimited number of registers, it is
fairly straightforward to modify it to consider a finite number only. This can be
done by adding extra parameters to the procedure reg. 

The first two clauses of reg treat the special cases where the second parameter
is a variable or a constant:
\begin{verbatim}
% Rule:R+OpRvar.
reg([Sl, imtr(Op, Addr, Rl)], Rl, D) --+
arithop(Op, Optwe),
regC% RL D),
[uar(Var)l,
(lookup( Var, D, Addr), !).
\end{verbatim}
\begin{verbatim}
% Rule: R --, Op R const.
reg([Sl, instr(Constop, C, Rl)], Rl, D) --+
arithop(Op, Optwe),
redsl, RI, D),
bdC)l,
{constop(op, Con-stop), !). 
\end{verbatim}

where arithop and constop are defined as
\begin{align*}
arithop(sub, noncommute) &\rightarrow\ [-I. &arithopbdd, commute) --+ [+I.\\
arithop(diu, noncommute) &\rightarrow\ [/I. &arithp(nult, commute) --+ [*I.\\
constop(sub, subi). &&constop(diu, diui).\\
constop(add, addi). &&constop(mult, multi). \\
\end{align*}

It is possible to perform some optimization in the case of commutative operations.
For that purpose two additional DCG clauses are included to process the rules:
\[R + op var R and R + op const R\]

The DCG clause for the first of these rules is given below.
\begin{verbatim}
% Rule: R -+ op uar R(op is commutatiue).
reg([Sl, instr(Op, Addr, Rl)], Rl, D) --+
arithop(Op, commute),
[udVar)l,
reg(S1, RI, D),
(lookup( Var, Addr, D), !I.  
\end{verbatim}

The more complex DCG given below corresponds to the rule R + op R R.
\begin{verbatim}
% Rule:R-*opRR.
reg([Sl, S2, instr(Op, R2, Rl)], Rl, D) --+
arithop(Op),
redS1, RI, D),
(R2 is Rl + 1),
reg(S2, R2, D), (!). 
\end{verbatim}

Two recursive calls are made to reg to determine the subsequences Sl and S2
representing the code for calculating the two operands. The simple DCG clauses
for the rules R + uar and R + const generate the necessary instructions that
load a register with a variable or with a constant.
\begin{verbatim}
% Rule: R + var.
reg(instr(load, Addr, Rl), Rl, D) --+
[MVar)l,
{lookup( Var, D, Addr), !).
% Rule: R + const.
reg(instr(hzdc, C, Rl), Rl, D) --+
[co=dC)l, 1% 
\end{verbatim}

Finally, we present the DCG clause for generating an assignment expressed in
prefix by the rule
\[A + := var R.\]
\begin{verbatim}
% Code generator for assignments.
instruction([Sl, instr(store, 1, Addr)], D) --+
[assign, uar (Vur)],
red% 0, D),
(lookup( Var, Addr, D)]. 
\end{verbatim}

Notice that the chosen grammar relies extensively on backtracking for recognizing
the appropriate rule. For example, consider the two rules
\[R --+ Op R const\]
\[R --+ Op R var\]
and the input string (+ + 5 c d). Although the first rule will not apply, it will
nonetheless be tried, and the code for the expression (+ 5 c) will be generated
before backtracking. The same code will then have to be regenerated when the
second rule is applied. This can be avoided by considering the following transformed
equivalent grammar:
\[R --+ Op R R2\]
\[R2 --+ Var 1 Const\]

This transformation can be easily generalized to the case at hand, and the
resulting parser will not rely on backtracking so there will be no need to insert
cuts into the program. 

An example of the code generated by this technique for the expression
,*(A* B + C - C*(D - E)) is
\begin{verbatim}
LOAD B, RO
MULT A, RO
ADD C, RO
LOAD D, Rl
SUB E, Rl
MULT C, RI
SUB Rl, RO
MULT A. RO  
\end{verbatim}
   
\secup
\secrel{OPTIMIZATIONS}\label{cohen8}\secdown
\secrel{Compile-Time Evaluation }\label{cohen81}

Compile-time evaluation of numerical expressions and algebraic simplification
are easily performed by transforming the syntax-trees of arithmetic expressions
into equivalent trees containing fewer nodes. Both of the procedures evaluate
and simplify have as arguments the initial and final trees. They also have a
similar structure: recursive calls are made to process the left and right branches
until the leaves are reached. Then the auxiliary procedure simp is called to
perform the actual simplifications. This allows successive simplifications to be
performed. 
\begin{lstlisting}[language=prolog]
% leaves are left unchanged
evaluute(const(X), const(X)).
evahte(var(X), var(X)). 
% internal nodes are optimized after each of its subtrees
% has been optimized
evaluate(expr(Op, Left, Right), Optexp) :-
	eualuate(left, Optleft),
	evaluate(Right, Optright),
	simp(expr(Op, Optkft, Optright), Optexp).
simp(expr(Op, const(X), const(Y)), const(Z)) :-
	Temp =. . [Op, X, Y], Z is Temp. 
\end{lstlisting}

(In the Edinburgh syntax [6], the operation Temp =. . [Op, X, Y] binds Temp to
the term Op(X, Y)). Note that, unfortunately, this procedure is unable to simplify
expressions such as a + 3 + 2 into a + 5. This may be achieved by writing a
simple procedure that transforms left-associative expressions into equivalent
right-associative expressions. The procedure that performs algebraic simplifications
is
\begin{lstlisting}[language=prolog]
simplify(expr(Op, X, Y), U) :-
	simplify(X, Left),
	simplify( Y, Right),
	simp(expr(Op, Left, Right), U).
simplify(X, X). 
\end{lstlisting}

As before, the auxiliary procedure simp performs the actual simplifications.
\begin{lstlisting}[language=prolog]
simp(expr(Op, X, const(O)), X) :- addop(Op).
simp(expr(Op, X, const(l)), X) :- multop(Op).
simp(expr(*, X, const(O)), const(0)).
simp(expr(*, const(O), X), const(0)).
simp(X, X).
addop(+). addop(-). muZtop(*). m&top(/). 
\end{lstlisting}

\secup

\secrel{USING PROPOSED EXTENSIONS }\label{cohen9}\secdown

Several extensions have been proposed to enhance the capabilities of Prolog. The
reader is referred to [S] for a brief description of some of these extensions. ‘Iwo
of them are of special interest in compiler construction and are dealt with in this
section: the use of the built-in predicate freeze and unification involving infinite
trees. These features are available in Prolog II [lo], in the interpreter developed
by Carlsson [4] and in MU-Prolog [23]\,\note{One purpose of presenting them here
is to generate interest, so that they will become more generally available. }.

The predicate freeze (also referred to as lazy evaluation, or coroutining) has
the form
\[freeze ( Var, Procedure)\]
Its action is to immediately activate the given Procedure if the variable Var is
bound. Otherwise, the Procedure becomes a dormant goal until Var is bound. In
that event, Procedure becomes the next goal to be activated. It is straightforward
to write a metalevel interpreter that simulates the effect of freeze [8]. This
interpreter is admittedly inefficient. Nevertheless, when compiled, it is usable
for processing small examples. 

We illustrate the use of freeze in two contexts: (1) coroutining the scanning,
parsing, and code generation phases of a compiler, and (2) error detection and
recovery. 

The coroutining of the phases is particularly useful when parallel processing
is available: it allows the intermediate results of one phase to be transmitted to 
the subsequent phase and therefore speed-up the computation by triggering
simultaneous executions whenever possible. (In Warren’s compiler [31] the
phases are strictly sequential.) Consider the simple procedure
\begin{lstlisting}[language=prolog]
readlkt(L) :- read(X), readrest(X, L).
readrest(stop, [ I).
readrest(X, [X 1 L]) :- readrest(  
\end{lstlisting}
The built-in predicate read reads individual atoms, and readlist assembles them
into a list. The atom stop is used as a flag to terminate the reading. 

The availability of freeze allows one to write a writelist procedure which outputs
the elements of a list as soon as they are read:
\begin{verbatim}
writelist ([ I).
writeZist([H 1 T]) :- freeze(H, write(H)), freeze(T, writelist(T)).
\end{verbatim}
The query is: \verb|?- freeze(L, writelist(L readlist(|

 The same ideas can be used to alternately transfer control among the scanner,
the parser, and the code generator. The main procedure compile is
\begin{lstlisting}[language=prolog]
compile :- freeze (Tree, encode-statement (Tree, Diet, Code)),
	freeze(List, parse(Lkt, Tree)),
	scan(List).
\end{lstlisting}
The above states that purse can only be activated as soon as (a part of) a List is
available. Similarly, encodestatement is activated as soon as a (partially) instantiated
syntax-tree becomes available. It is of course necessary to “sprinkle”
additional freezes within parse and encode-statement. This is illustrated below
by examples. The translated DCG rule for parsing a while statement becomes
\begin{lstlisting}[language=prolog]
stutement(whik(Test, Do), [while 1 Dl], 04) :-
freeze(D1, test(Test, Dl, D2)),
freeze(D2, eq(D2, [do 1 D3])),
freeze(D3, statement(Do, D3,04)).  
\end{lstlisting}
The second and third parameters of statement and test are the difference lists for
parsing strings derived from the corresponding nonterminals. The procedure eq
is simply the unit clause eq(X, X) which unifies its arguments. The effect of
freezing on Dl, 02, and 03 is to allow test and the recursive call of statement to
be activated only when the pertinent information becomes available. Similarly,
the code generator for a while node of the syntax-tree (see Section 7.2) becomes
\begin{lstlisting}[language=prolog]
encode-statement(while(Test, Do), Diet, . . .) :-
freeze(Test, encode-test(Test, Diet, L2, Testcode)),
freeze(Do, encode-statement(Do, Diet, Docode)). 
\end{lstlisting}

Figure 1 shows the alternating flow of control among the scanner, parser, and
code generator while compiling a small program using the coroutining technique.
The reader might have already suspected that the introduction of freezes could
be done automatically. We have indeed developed programs that perform this
task, based on user-specified mode declarations (input or output) for each
parameter of a procedure. 

Another usage of freeze is in error detection and recovery. The following
example just illustrates the main ideas, which are based on the work of Mickunas 
and Modry [22]. At the top level the procedure recover has two parameters:
(1) the possibly erroneous input string and (2) the corrected string.
\begin{lstlisting}[language=prolog]
recover ( Tokens, Tree) :-
freeze(Filtered-tokens, parse(Filtered-tokens, Tree)),
correct( Tokens, Filtered-tokens, 0). 
\end{lstlisting}

\paragraph{Figure 1}\ \\\bigskip

The variable Filtered-tokens is initially unbound; purse will call the corresponding
procedures that use the difference lists. The third parameter of correct is the
initial cost of correction. The approach consists of attempting to insert or to
delete tokens in the input string so that an erroneous string becomes parsable. If
necessary, different costs for insertion and deletion, applicable to specific terminals,
can be specified by the designer. In the simplified version of correct listed
below, a unit cost is used for both operations. The database contains a unit clause
cost-ok(nax) in which mar is a number that controls the amount of backtracking. 
\begin{lstlisting}[language=prolog]
% final scan
correct([ 1, [ 1, Cost).
70 normal scan
correct([X 1 R], [X 1 Rl], Cost) :- correct(R, Rl, Cost).
% deletion
correct([X 1 R], Rl, Cost) :-
cost-ok(Cost),
Cost1 is cost + 1,
correct(R, Rl, Costl).
% insertion
correct(R, [I 1 Rl], Cost) :-
cost-ok(Cost),
Cost1 is cost + 1,
correct(R, Rl, Costl).
\end{lstlisting}
Note that the variable I in the insertion clause will be bound by the parser
according to the grammar rules. 

A more elaborate version of correct could reduce the amount of nondeterminism
by making insertions and deletions based on examining the (fragments of the) 
parse tree constructed prior to encountering an error. This is the approach
described in [ 22]. 

In addition to the two above uses of freeze, we have explored its application in
dataflow analysis. The iterative methods described in [l] can be implemented
using a variant of freeze in which the frozen variables simulate the incoming and
outgoing flow of information for each block. 

The other proposed extension of Prolog that is useful in compiler design deals
with the so-called infinite trees. It is Colmerauer’s contention that grammars,
flowcharts, and programs frequently specify loops or recursion [ll], which can
be conveniently described using directed graphs. Their use within Prolog requires
that the unification operation be extended to handle circular structures instead
of trees. 

An elegant and novel approach for implementing a scanner generator using
infinite trees has been developed by students of the University of Marseilles [12]
under the guidance of A. Colmerauer. It consists of using a special type of
unification to produce the minimal finite state automaton directly from a given
regular expression. Most Prolog interpreters perform unification only on trees.
A notable exception is the interpreter developed at Marseilles, which can unify
special kinds of graphs called infinite trees [lo]. For example, when the unit
clause eq (X, X) is matched with eq (A, stute(a(A ))), the resulting unification is
expressed by the infinite tree: 

%\fig{}{}{}

Terms representing states have an additional component specifying whether
the state is final or not. The procedure to translate a regular expression into the
corresponding minimal finite state automaton takes as input the expression given
by its syntax-tree and produces as result the infinite tree corresponding to the
minimal automaton. The highlights of this translation are given in what follows. 

If a node of the syntax-tree is a conc(L.eft, Right), one recursively determines
the automata corresponding to the Left and Right branches and “concatenates”
the two automata to obtain the result. Concatenation of two automata Al and
A2 is performed by checking whether the starting state of Al is final or not-find.
In the first case the resulting automata is the union of the automata A2 with the
concatenation of automata Al’ and A2, in which Al’ is a modified copy of Al in
which the starting state is considered to be not-find. In the second case the
concatenation of the two automata consists of specifying the proper transitions
between the final states of Al directly to the states that stem from the initial
state of A2. The union and star operations are processed similarly. 

A dictionary is “carried along” as a parameter to provide the information
needed to keep a single copy of each of the generated subautomata needed to
construct the desired one. Therefore, before proceeding to generate an automaton
corresponding to a subpart of a regular expression, the dictionary is used to check
if the translation has already been done. If so, the desired subautomaton is
retrieved from the dictionary. Otherwise, the automaton is determined and the
corresponding entry is placed in the dictionary. This per se does not guarantee
the construction of a minimal automaton. The program that “prints” the desired
infinite tree is actually the one responsible for the minimization [26]. Again with
the use of a dictionary, the printing program keeps unique copies of each subtree
of the given infinite tree and uses them every time identical subtrees are found.
This process has been proved to terminate [ll], and for the particular problem
at hand it yields the desired minimal automaton. 

The authors of this program [12] extended its capabilities to handle the
difference and intersection of regular expressions. The program hardly exceeds
three pages of code; it also uses another feature that is only available in Marseilles’
interpreters: the constraint \&\&ff(X, Y), meaning X \# Y is valid even when X
and Y are uninstantiated, therefore allowing the program’s execution to continue in
the forward mode. Backtracking is thus postponed until it is found that the
ensemble of constraints becomes unsatisfiable. 


 

\secup
\secrel{FINAL REMARKS }\label{cohen10}

In the previous sections we described in Prolog several algorithms that play an
important role in the design and construction of compilers. We hope it has
become apparent to the reader that the descriptions using Prolog are substantially
more concise than those which appear in current textbooks. For example, Aho
and Ullman often use a mixture of English, the language of sets, and control
primitives usually found in Pascal-like languages. The reader is urged to compare
some of their descriptions to those presented in this paper. 

The experience we gained with Prolog has convinced us of its effectiveness as
a language for rapid prototyping compilers and for developing ancillary tools.
Presently, the highest gains are achieved in the development of tools in which
performance is not of prime consideration. This is the case of automatically
producing code generators, parsers, and scanners. Even if the generation of these
components takes considerable computer time (say a few hours), the combined
man-machine effort may be inexpensive when compared to the human resources
needed to produce their hand-coded counterparts. Another area in which the
language has proved its usefulness is in the writing of compilers for Prolog itself.
It is fair to say that most Prolog compilers are written in Prolog. The gains are
substantial, especially because they have to process relatively short programs,
and compilation can be done incrementally as the procedures are developed. 

Yet another advantage of using Prolog programs is their ability to perform
computations both in the forward and reverse directions. It should therefore be
possible to decompile target code to obtain the corresponding source code.
Although this is in principle feasible, the use of “impure” Prolog features
such as the cut and the assignment (is) render the reverse execution impossible.
These problems may be circumvented by using the generalized diff, mentioned in
Section 9, and by ensuring that simple assignments such as those incrementing
the values of variables become backtrackable. 

Among the shortcomings of Prolog, it should be mentioned that the language
is still in evolution and that, presently, a suitable environment for developing
larger Prolog programs is not yet available. The language also suffers from the
nonexistence of a methodology for documentation, the lack of scoping for variables,
the ever-increasing number of parameters, and the resulting profusion of
identifier names. 

Benchmarks of the parsers and code generators described in this paper showed
that their interpretation is indeed slower than the compiled equivalent programs
written in C or in Pascal. Compiled Prolog programs running on a dedicated
workstation exhibit 5- to lo-fold speed-ups compared to their interpreted versions.
For example, the compiled version of Warren’s minicompiler enabled us
to generate code for sample programs containing a few hundred statements in a
couple of minutes. Such compilation speeds are still admittedly below those
attained by equivalent compilers written in C. However, we feel that there is a
great potential for improving considerably the performance of compilers written
in Prolog. The justifying arguments are as follows. 

The advantages of Prolog basically stem from the use of unification and
nondeterminism. The present price paid for the advantages are increasing demands
in memory and execution time. Since compilers are usually designed to
avoid nondeterministic situations, it is important to reduce Prolog’s interpreter
(or compiled code) overhead for dealing with these situations. Once it is known
that a Prolog program is deterministic, several optimizations can be carried out.
One of them is to eliminate the need of saving choice points for backtracking
purposes. The optimized program can then achieve the efficiency of the corresponding
programs written in a functional language (see [30]). 

In a recent paper, Mellish [21] provides weak conditions for determining
automatically if a set of Prolog procedures is deterministic. His method is based
on a dataflow analysis in which properties of programs are determined by
iteratively solving a system of equations. The efficiency of the compiled code can
also be increased by having the user supply, by a mode declaration, the nature
(input or output) of each parameter of a procedure. This allows the compiler not
only to discard certain nondeterministic situations, but also to replace costly
unifications by the simpler operations of assignments and conditionals. 

A possibility that should not be overlooked in the quest to speed-up Prolog
programs is the use of parallel processing. In contrast with most other languages,
Prolog offers an embarrassment of riches for exploiting parallelism. The experience
gained by empirical or theoretical analysis of parallel Prolog compilers may
therefore help to shed some light as to which particular approach yields better
speed-up gains. 

We feel that the initial investment spent in learning Prolog is largely compensated
for by the advantages accrued in having a shorter program-development
stage and achieving program descriptions that can easily be tried and tested in a
computer. It is also possible that other higher level languages such as SETL
could be used with the same purpose. What seems certain is that the availability
of these languages will make program description less verbose and more accurate.
In addition, they will spur the development of optimization techniques capable
of rendering efficient the descriptions that are not directly presented in an
efficient form. The history of the development of Fortran and other languages
indicates that this is not only a desirable goal but likely an unavoidable one. 



\secly{ACKNOWLEDGMENTS}

The first author’s initiation to Prolog developed from the close contacts that he
has had with the Groupe d’Intelligence Artificielle (GIA) at the University of
Marseilles, Luminy, in France. Alain Colmerauer, Michel van Caneghem, Henri 
Kanoui, Bob Pasero, and Francis Giannesini were all enthusiastic in sharing
with him their knowledge of the language they have developed and refined at
GIA. Three graduate students from Marseilles: Sylvie Duchenoy, Robert Kong
Win Chang, and Sophie Nabitz helped in testing the programs presented here.
In particular, Robert Kong, now at Brandeis, has dedicated countless hours in
helping us polish the paper. David Hildum implemented the Glanville-Graham
code-generation method described in Section 7. Peter Appel and Cindy Lurie did
their honor’s projects prototyping Prolog versions of LEX and YACC. We count
ourselves lucky to have had the opportunity to interact with the above-mentioned
persons. 

Finally, we wish to express our gratitude to a referee, David S. Warren, who,
following a meticulous reading of the original manuscript, helped identify the
major issues of Prolog usage in compiling and urged us to discuss them in the
revised paper. The thoughtful and detailed remarks made by this referee provided
an added incentive to improve the paper and reaffirmed our respect for the
refereeing process. 

\secly{REFERENCES}

\begin{enumerate}
  \item 
1. AHO, A. V., AND ULLMAN, J. D. Principles of Compiler Design. Addison-Wesley, Reading, Mass.,
1979.
  \item 
2. BACKHOUSE, R. C. Syntax of Programming Langooges. Prentice-Hail, Englewood Cliffs, N.J.,
1979.
  \item 
3. CAMPBELL, J. A. (ED.) Implementations of Prolog. Wiley, New York, 1984.
  \item 
4. CARLSSON, M. A microcoded unifier for Lisp machine Prolog. In IEEE Proceedings 1985
Symposium on Logic Programming (Boston, July 1985), IEEE, New York, 1985,162-171.
  \item 
5. CATTELL, R. G. G. Automatic derivation of code generators from machine descriptions. ACM
Trans. Programm. Long. Syst. 2,2 (Apr. 1980), 173-190.
  \item 
6. CLOCKSIN, W. F., AND MELLISH, C. S. Programming in Prolog. Springer-Verlag, New York,
1981(2nd ed., 1984).
  \item 
7. COHEN, J., SITVER, R., AND AUTY, D. Evaluating and improving recursive descent parsers.
IEEE Trans. Softw. Eng. SE-5,5 (Sept. 1979), 472-480.
  \item 
8. COHEN, J. Describing Prolog by ita interpretation and compilation. Commun. ACM 28, 12
(Dec. 1985), 1311-1324.
  \item 
9. COLMERAUER, A. Les Grammaires de Metamorphose. Groupe d’Intelligence Artificielle, Univ.
of Marseilles-Luminy, 1975. (Appears as Metamorphosis grammars in Natural Language Communication
with Computers. L. Bale, Ed., Springer-Verlag, New York, 1978, 133-189.)
  \item 
10. COLMERAUER, A., KANOUI, H., AND VAN CANEGHEM, M. Prolog II. Groupe d’Intelligence
Artificielle, Univ. of Marseilles-Luminy, 1982.
  \item 
11. COLMERAUER, A. Prolog and infinite trees. In Logic Programming, Clark and Tarnlund (Eds.),
Academic Press, New York, 1982,231-251.
  \item 
12. COUPET, S., AND DUPLESSIS, F. Prolog programs for transforming regular expressions into the
corresponding minimal finite state recognizers. Memoire de D.E.A., GIA, Univ. of Marseilles,
June 1984 (in French).
  \item 
13. EARLEY, J. An efficient context-free parsing algorithm. Commun. ACM 13, 2 (Feb. 1970),
94-102.
  \item 
14. GIANNESINI, F., AND COHEN, J. Parser generation and grammar manipulations using Prolog’s
infinite trees. J. Logic Programm. (Oct. 1984), 253-265.
  \item 
15. GLANVILLE, R. A machine independent algorithm for code generation and its use in retargetable
compilers. Ph.D. dissertation, Univ. of California, Berkeley, 1977.
  \item 
16. GRAHAM, S. L., HENRY, R. R., AND SHULMAN, R. A. An experiment in table driven generation.
In Proceedings SIGPLAN Symposium on Compiler Construction 17,6 (June 1982), 32-43.
  \item 
17. GRIES, D. Compiler Construction for Digital Computers. Wiley, New York, 1971. 
  \item 
18. GRIFFITHS, T. V., AND PETRICK, S. R. On the relative efficiencies of context-free grammar
recognizers. Commun. ACM 8,5 (May 1965), 289-300.
  \item 
19. ICHBIAH, J. D., AND MORSE, S. P. A technique for generating almost optimal Floyd-Evans
productions for precedence grammars. Commun. ACM 13,8 (Aug. 1970), 501-508.
  \item 
20. KOWALSKI, R. Logic for Problem Soluing. North-Holland, Amsterdam, 1979.
  \item 
21. MELLISH, ‘C. S. Some global optimizations for a Prolog compiler. J. Logic Programm. 2, 1
(Apr. 1985), 43-66.
  \item 
22. MICKUNAS, M. D., AND MODRY, J. A. Automatic error recovery for LR parsers. Commun. ACM
21,6 (June 1978), 459-465.
  \item 
23. NAISH, L. Automating control for logic programs. J. Logic Programm. 3 (1985), 167-183.
  \item 
24. PEREIRA, F. C. N., AND WARREN, D. H. D. Parsing as deduction. In Proceedings of the 21st
Annual Meeting of the Association for Computational Linguistics (Cambridge, Mass., June 1983),
Association for Computational Linguistics, 1983, 137-144.
  \item 
25. PEREIRA, F. C. N., AND WARREN, D. H. D. Definite clause grammars for language analysis.
Artif. Intell. 13 (1980), 231-278.
  \item 
26. PIQUE, J. F. Drawing trees and their equations in Prolog. In Proceedings of the 2nd Znternutionul
Logic Programming Conference (Uppsala, 1984), Ord and Furm, Uppsala, 1984,23-33.
  \item 
27. TANENBAUM, A. S., VAN STAVEREN, H., AND STEVENSON, J. W. Using peephole optimization
on intermediate code. ACM Trans. Programm. Lung. Syst. 4,1 (Jan. 1982), 21-36.
  \item 
28. UEHARA, K., OCHITANI, R., AND KAKUSHO, 0. A bottom-up parser based on predicate logic. In
IEEE Proceedings 1984, International Symposium on Logic Programming (Atlantic City, N.J.,
Feb. 1984), IEEE, New York, 1984,220-227.
  \item 
29. WAITE, W. M., AND GOOS, G. Compiler Construction. Springer-Verlag, New York, 1984.
  \item 
30. WARREN, D. H. D. Applied logic-its use and implementation as a programming tool. Ph.D.
dissertation, Univ. of Edinburgh, 1977 (also appeared as Tech. Note 290, SRI International,
1983).
  \item 
31. WARREN, D. H. D. Logic programming and compiler writing. Softw. Pratt. Exper. 10 (Feb.
1980), 97-125. 
\end{enumerate}
 

\secup