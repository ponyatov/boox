\secrel{Learn Datalog Today}\secdown

\cp{\url{http://www.learndatalogtoday.org/}}

Learn Datalog Today\ --- интерактивный учебник разработанный для вашего обучения
использованию Datomic-диалекта языка Datalog. \term{Datalog}\ --- язык запросов
декларативных баз данных, вышедший из языка \prolog\ и логического
программирования. Datalog имеет выразительность похожую на SQL.

\prog{Datomic}\ --- база данных интересного и инновационного нового типа, дающая
своим пользователям уникальный набор возможностей. Почитать о Datomic подронее
вы можете на официальном сайте \url{http://datomic.com}, архитектура системы
подробнее описана в
\href{http://www.infoq.com/articles/Architecture-Datomic}{InfoQ article}.

\bigskip
This tutorial was written on rainy days for the Lisp In Summer Projects 2013. If
you find bugs, or have suggestions on how to improve the tutorial, please visit
the project on github.

Many thanks to Robert Stuttaford for his careful proof reading/editing. I'd also
like to thank everyone who has contributed by fixing bugs and spelling mistakes.

\url{www.learndatalogtoday.org} \copyright 2013 Jonas Enlund

\href{https://github.com/jonase/learndatalogtoday}{github}

\url{lispinsummerprojects.org}

\input{prolog/datomic/notation}

\input{prolog/datomic/basic}

\secrel{Data Patterns}

\secrel{Parameterized Queries}

\secrel{More Queries}

\secrel{Predicates}

\secrel{Transformation Functions}

\secrel{Aggregates}

\secrel{Rules}

\secup % datomic