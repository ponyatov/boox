\secrel{Compiling $L_0$ programs}

Compiling a program term p is just a bit trickier, although not by much. Observe
that it assumes that a query ?-q will have built a term on the heap and set register
X
 to contain its address. Thus, unifying q to p can proceed by following the term
structure already present in X
 as long as it matches functor for functor the structure
of p. The only complication is that when an unbound REF cell is encountered
in the query term in the heap, then it is to be bound to a new term that is built on
the heap as an exemplar of the corresponding subterm in p. Therefore, an L
program
functions in two modes: a read mode in which data on the heap is matched
against, and a write mode in which a term is built on the heap exactly as is a
query term.

\paragraph{Figure 2.2: M machine instructions for query terms}
\begin{verbatim}
put structure f n
Xi
\end{verbatim}

\paragraph{Figure 2.3: Compiled code for L query ?-pZh}
\begin{verbatim}
put structure h
X % ?-X 
 h
set variable X 
set variable X
put structure f
 f
set value 
put structure p
 p
set value X % 
set value X % X
set value X
\end{verbatim}

As with queries, register allocation precedes translation of the textual form of a
program term into a machine instruction sequence. For example, the following
registers are allocated to program term pfhy
\begin{verbatim}
x1
..
x7
\end{verbatim}


Recall that compiling a query necessitates ordering its flattened form in such a
way as to build a term once its subterms have been built. Here, the situation is
reversed because query data from the heap are assumed available, even if only in
the form of unbound REF cells. Hence, a program term’s flattened form follows
a top-down order. For example, the program term pfhy
is put into the flattened sequence: X1..x7
 
As with query compiling, the flattened form of a program is tokenized for left-toright
processing and generates three kinds of machine instructions depending on
whether is met:

1. a register associated with a structure functor;

2. a first-seen register argument; or,

3. an already-seen register argument.

These instructions are,

1. get structure f n Xi

2. unify variable Xi

3. unify value Xi

respectively.

\paragraph{Figure 2.4: Compiled code for L program pfhY}
\begin{verbatim}

\end{verbatim}

Taking for example the program term pf
, the M
machine
instructions shown in Figure 2.4 are generated. Each of the two unify instructions
functions in two modes depending on whether a term is to be matched
from, or being built on, the heap. For building (write mode), the work to be
done is exactly that of the two set query instructions of Figure 2.2. For matching
(read mode), these instructions seek to recognize data from the heap as those
of the term at corresponding positions, proceeding if successful and failing otherwise.
In L,
failure aborts all further work. In read mode, these instructions set
a global register S to contain at all times the heap address of the next subterm to
be matched.

Variable binding creates the possibility that reference chains may be formed.
Therefore, dereferencing is performed by a function deref which, when applied
to a store address, follows a possible reference chain until it reaches either an unbound
REF cell or a non-REF cell, the address of which it returns. The effect of
dereferencing is none other than composing variable substitutions. Its definition
is given in Figure 2.5. We shall use the generic notation STORE[a] to denote the
contents of a term data cell at address a (whether heap, X register, or any other
global structure, yet to be introduced, containing term data cells). We shall use
specific area notation (e.g., HEAP[a]) whenever we want to emphasize that the
address a must necessarily lie within that area.

\paragraph{Figure 2.5: The deref operation}
\begin{verbatim}
function deref 
then return deref
else return a
end deref 
F
\end{verbatim}

Mode is set by get structure f n
Xi as follows: if the dereferenced value
of Xi is an unbound REF cell, then it is bound to a new STR cell pointing to f n
pushed onto the heap and mode is set to write; otherwise, if it is an STR cell
pointing to functor f n, then register S is set to the heap address following that
functor cell’s and mode is set to read. If it is not an STR cell or if the functor
is not f n, the program fails. Similarly, in read mode, unify variable Xi
sets register Xi to the contents of the heap at address S; in write mode, a new
unbound REF cell is pushed on the heap and copied into Xi. In both modes, S is
then incremented by one. As for unify value Xi, in read mode, the value of
Xi must be unified with the heap term at address S; in write mode, a new cell is
pushed onto the heap and set to the value of register Xi. Again, in either mode, S
is incremented. All three instructions are expressed explicitly in Figure 2.6.
In

In the definition of get structure f n
Xi, we write bindaddr
H to effectuate
the binding of the heap cell rather than HEAP[addr]  h REF
H i for reasons
that will become clear later. The bind operation is performed on two store
addresses, at least one of which is that of an unbound REF cell. Its effect, for
now, is to bind the unbound one to the other—i.e., change the data field of the
unbound REF cell to contain the address of the other cell. In the case where both
are unbound, the binding direction is chosen arbitrarily. Later, this will change
as a correctness-preserving measure in order to accommodate an optimization.
Also, we will see that bind is the logical place, when backtracking needs to be
considered, for recording effects to be undone upon failure (see Chapter 4, and
appendix Section B.2 on Page 113). If wished, bind may also be made to perform
the occurs-check test in order to prevent formation of cyclic terms (by failing at
that point). However, the occurs-check test is omitted in most actual Prolog implementations
in order not to impede performance.

\paragraph{Figure 2.6: M machine instructions for programs}
\begin{verbatim}

\end{verbatim}

We must also explicate the unify operation used in the matching phase (in read
mode). It is a unification algorithm based on the UNION/FIND method [AHU74],
where variable substitutions are built, applied, and composed through dereference
pointers. In M
(and in all later machines that will be considered here), this
unification operation is performed on a pair of store addresses. It uses a global
dynamic structure, an array of store addresses, as a unification stack (called PDL,
for Push-Down List). The unification operation is defined as shown in Figure 2.7,
where empty, push, and pop are the expected stack operations.

\paragraph{Exercise 2.2} Give heap representations for the terms fXgX and f b Y.
Let a and a be their respective heap addresses, and let aX and aY be the heap
addresses corresponding to variables X and Y , respectively. Trace the effects
of executing unifya a, verifying that it terminates with the eventual
dereferenced bindings from aX and aY corresponding to X
 b and Y g b a .
 
\paragraph{Exercise 2.3} Verify that the effect of executing the sequence of
instructions shown in Figure 2.4 right after that in Figure 2.3 produces the MGU
of the terms p Z h ZW  f  W and p f  X  h Yf  a   Y  . That is, the
(dereferenced) bindings corresponding to W
 f  a , X f  a , Y f  f  a  , Z f  f  a  .

\paragraph{Exercise 2.4} What are the respective sequences of M instructions for
L query term ?-p f  X  h Y f  a   Y   and program term p Z h ZW  f  W  ?

\paragraph{Exercise 2.5} After doing Exercise 2.4, verify that the effect of
executing the sequence you produced yields the same solution as that of Exercise
2.3.