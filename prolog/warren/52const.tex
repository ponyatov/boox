\secrel{Constants, lists, and anonymous variables}

To be fully consistent with the complete WAM unification instruction set and in
accordance with WAM Principle 3, we introduce special instructions for the specific
handling of -ary structures (i.e., constants), lists, and variables which appear
only once within a scope—so-called anonymous variables. These enhancements
will also be in the spirit of WAM Principles 1 and 2 as savings in heap space, code
size, and data movement will ensue.

Constants and lists are, of course, well handled by the structure oriented get,
put, and unify instructions. However, work and space are wasted in the process,
that need not really be. Consider the case of constants as, for instance, the
code in Figure 2.10, on Page 24. There, the sequence of instructions:
\[unify variable X\]
\[get structure a,X7\]
simply binds a register and proceeds to check the presence of, or build, the constant
a on the heap. Clearly, one register can be saved and data movement optimized
with one specialized instruction: unify constant a. The same 
situation in a query would simplify a sequence:
\[put structure c\]
\[Xi\]
into one specialized instruction set constant c. Similarly, put and get instructions
can thus be specialized from those of structures to deal specifically with
constants. Thus, we define a new sort of data cells tagged CON, indicating that the
cell’s datum is a constant. For example, a heap representation starting at address

 for the structure f b
ga
could be:
\begin{verbatim}
 g
	 CON a

f
 CON b
 STR 
E
\end{verbatim}

\paragraph{Exercise 5.1} Could the following (smaller) heap representation
starting at address

be an alternative for the structure f b ga? Why?

\begin{verbatim}
f
 CON b
 g
 CON a
He
\end{verbatim}

Heap space for constants can also be saved when loading a register with, or binding
a variable to, a constant. Rather than systematically occupying a heap cell
to reference, a constant can be simply assigned as a literal value. The following
instructions are thus added to I:

1. put constant c Xi

2. get constant c Xi

3. set constant c

4. unify constant c

and are explicitly defined in Figure 5.2.

\paragraph{Figure 5.2: Specialized instructions for constants}

\paragraph{Figure 5.3: Specialized instructions for lists}

Programming with linear lists being so privileged in Prolog, it makes sense to
tailor the design for this specific structure. In particular, non-empty list functors
need not be represented explicitly on the heap. Thus again, we define a fourth sort
for heap cells tagged LIS, indicating that the cell’s datum is the heap address of
the first element of a list pair. Clearly, to respect the subterm contiguity convention,
the second of the pair is always at the address following that of the first. The
following instructions (defined explicitly in Figure 5.3) are thus added to I:

1. put list Xi

2. get list Xi

For example, the code generated for query ?-pZ
ZW
f W,
using Prolog’s
notation for lists, is shown in Figure 5.4 and that for fact pf
X
Y
f a
Y ,
in Figure 5.5. Note the hidden presence of the atom  as list terminator.
Of c

Of course, having introduced specially tagged data cells for constants and nonempty
lists will require adapting accordingly the general-purpose unification algorithm
given in Figure 2.7. The reader will find the complete algorithm in appendix
Section B.2, on Page 117.

\paragraph{Exercise 5.2} In [War83], Warren also uses special instructionsput
nil Xi, get nil Xi, and to handle the list terminator constant . Define the effect of these instructions, and give explicit pseudo-code implementing them. Discuss their worth being
provided as opposed to using put constant  Xi, get constant  Xi,
set constant , and unify constant .

\paragraph{Figure 5.4: Specialized code for query ?-pZ
ZW
f W
p}

\paragraph{Figure 5.5: Specialized code for fact pf
X
Y
f a
Y }

Last in the rubric of specialized instructions is the case of single-occurrence variables
in non-argument positions (e.g., X in Figure 2.4 on Page 16, Figure 2.10 on
Page 24, and Figure 5.5 on Page 51). This is worth giving specialized treatment
insofar as no register need be allocated for these. In addition, if many occur in
a row as in f, say, they can be all be processed in one swoop, saving in
generated code size and time. We introduce two new instructions:

1. set void n

2. unify void n

whose effect is, respectively:

1. to push n new unbound REF cells on the heap;

2. in write mode, to behave as set void n and, in read mode, to skip the
next n heap cells starting at location S.

These are given explicitly in Figure 5.6.

Note finally, that an anonymous variable occurring as an argument of the head of
a clause can be simply ignored. Then indeed, the corresponding instruction:
\[get variable Xi\]
is clearly vacuous. Thus, such instructions are simply eliminated. The code for
fact p

gX
f

Y
, for example, shown in Figure 5.7, illustrates this point.

\paragraph{Exercise 5.3} What is the machine code generated for the fact p  
? What about the query ?-p   ?
