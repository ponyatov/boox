\secrel{Prolog}\label{warren4}\secdown

The language L (resp., the machine M) corresponds to pure Prolog, as it extends
the language L (resp., the machine M) to allow disjunctive definitions.
As in L, an L program is a set of procedure definitions. In L, a definition is
an ordered sequence of clauses (i.e., a sequence of facts or rules) consisting of
all and only those whose head atoms share the same predicate name. That name
is the name of the procedure specified by the definition. L queries are the same
as those of L. The semantics of L operates using top-down leftmost resolution,
an approximation of SLD resolution. Thus, in L, a failure of unification no
longer yields irrevocable abortion of execution but considers alternative choices
of clauses in the order in which they appear in definitions. This is done by chronological
backtracking; i.e., the latest choice at the moment of failure is reexamined
first.

It is necessary to alter M’s design so as to save the state of computation at each
procedure call offering alternatives to restore upon backtracking to this point of
choice. We call such a state a choice point. We thus need to analyze what information
must be saved as a choice point in order to create a record (a choice
point frame) wherefrom a correct state of computation can be restored to offer
another alternative, with all effects of the failed computation undone. Note that
choice point frames must be organized as a stack (just like environments) in order
to reflect the compounding of alternatives as each choice point spawns potentially
more alternatives to try in sequence.

To distinguish the two stacks, let us call the environment stack the AND-stack and
the choice point stack the OR-stack. As with the AND-stack, we organize the
OR-stack as a linked list. The head of this list always corresponds to the latest
choice point, and will be kept in a new global register B, such that upon failure,
computation is made to resume from the state recovered from the choice point
frame indicated by B. When the latest frame offers no more alternatives, it is
popped off the OR-stack by resetting B to its predecessor if one exists, otherwise
computation fails terminally

Clearly, if a definition contains only one clause, there is no need to create a choice
point frame, exactly as was the case in M. For definitions with more than one
alternative, a choice point frame is created by the first alternative; then, it is updated
(as far as which alternative to try next) by intermediate (but non ultimate)
alternatives; finally, it is discarded by the last alternative.

\secrel{Environment protection}

Before we go into the details of what exactly constitutes a choice frame, we must
ponder carefully the interaction between the AND-stack and the OR-stack. As
long as we considered (deterministic) L program definitions, it was clearly safe
to deallocate an environment frame allocated to a rule after successfully falling
off the end of the rule. Now, the situation is not quite so straightforward as later
failure may force reconsidering a choice from a computation state in the middle
of a rule whose environment has long been deallocated. This case is illustrated by
the following example program:
\begin{verbatim}
a :- bX0
cX00
bX0
:- eX00
c
00
eX0
:- f X00
eX0
:- gX00
f
00
g
00
\end{verbatim}

Executing ‘?-a0’ allocates an environment for a, then calls b. Next, an environment
for b is allocated, and e is called. This creates a choice point on the
OR-stack, and an environment for e is pushed onto the AND-stack. At this point
the two stacks look thus:\note{In these diagrams, the stacks grow downwards;
i.e., the stack top is the lower part.}
\[E  Environment for e\]
\[B  Choice point for e\]

The following call to f succeeds binding X to 	. The environment for e is deallocated,
then the environment for b is also deallocated. This leads to stacks looking
thus:
\[E  Environment for a\]
\[B  Choice point for e\]

Next, the continuation follows up with execution of a’s body, calling c, which
immediately hits failure. The choice point indicated by B shows an alternative
clause for e, but at this point b’s environment has been lost. Indeed, in a more
involved example where c proceeded deeper before failing, the old stack space for
b’s environment would have been overwritten by further calls in c’s body

Therefore, to avoid this kind of misfortune, a setup must be found to prevent
unrecoverable deallocation of environment frames whose creation chronologically
precedes that of any existing choice point. The idea is that every choice point
must “protect” from deallocation all environment frames already existing before
its creation. Now, since a stack reflects chronological order, it makes sense to
use the same stack for both environments and choice points. A choice point now
caps all older environments. In effect, as long as it is active, it forces allocation of
further environments on top of it, preventing the older environments’ stack space
to be overwritten even though they may explicitly be deallocated. This allows
their safe resurrection if needed by coming back to an alternative from this choice
point. Moreover, this “protection” lasts just as long as it is needed since as soon as
the choice point disappears, all explicitly deallocated environments can be safely
overwritten.

Hence, there is no need to distinguish between the AND-stack from the OR-stack,
calling the single one the stack. Choice point frames are stored in the stack along
with environments, and thus B’s value is an address in the stack.

Going back to our example above, the snapshot of the single stack at the same first
instant looks thus:
\[Environment for a\]
\[Environment for b\]
\[B  Choice point for e\]
\[E  Environment for e\]
and at the same second instant as before, the stack is such that having pushed on
it the choice point for e protects b’s deallocated environment (which may still be
needed by future alternatives given by e’s choice point), looking thus:
\[E  Environment for a\]
\[Deallocated environment for b\]
\[B  Choice point for e\]

Now, the computation can safely recover the state from the choice point for e
indicated by B, in which the saved environment to restore is the one current at
the time of this choice point’s creation—i.e., that (still existing) of b. Having no
more alternative for e after the second one, this choice point is discarded upon
backtracking, (safely) ending the protection. Execution of the last alternative for
e proceeds with a stack looking thus:
\[B \]
\[Environment for a\]
\[Environment for b\]
\[E  Environment for e\]

\secrel{What’s in a choice point}

When a chosen clause is attempted among those of a definition, it will create
side effects on the stack and the heap by binding variables residing there. These
effects must be undone when reconsidering the choice. A record must be kept of
those variables which need to be reset to ‘unbound’ upon backtracking. Hence,
we provide, along with the heap, the stack, the code area, and the PDL, a new
(and last!) data area called the trail (TRAIL). This trail is organized as an array
of addresses of those (stack or heap) variables which must be reset to ‘unbound’
upon backtracking. Note that it also works as a stack, and we need a new global
register TR always set to contain the top of the trail.

It is important to remark that not all bindings need to be remembered in the trail.
Only conditional bindings do. A conditional binding is one affecting a variable
existing before creation of the current choice point. To determine this, we will use
a new global register HB set to contain the value of H at the time of the latest choice
point.\note{Strictly speaking, register HB can in fact be dispensed with since, as we see next, its value is
that of H which will have been saved in the latest choice point frame.} Hence only bindings of heap (resp., stack) variables whose addresses
are less than HB (resp., B) need be recorded in the trail. We shall write traila
when
that this operation is performed on store address a. As mentioned before, it is
done as part of the bind operation.

Let us now think about what constitutes a computation state to be saved in a choice
point frame. Upon backtracking, the following information is needed:
\begin{description}
\item[The argument registers] A
, ..., An, where n is the arity of the procedure offering
alternative choices of definitions. This is clearly needed as the argument registers,
loaded by put instructions with the values of arguments necessary for goal being
attempted, are overwritten by executing the chosen clause.
\item[The current environment] (value of register E), to recover as a protected environment
as explained above.
\item[The continuation pointer] (value of register CP), as the current choice will overwrite
it.
\item[The latest choice point] (value of register B), where to backtrack in case all alternatives
offered by the current choice point fail. This acts as the link connecting
choice points as a list. It is reinstated as the value of the B register upon discarding
the choice point.
\item[The next clause], to try in this definition in case the currently chosen one fails.
This slot is updated at each backtracking to this choice point if more alternatives
exist.
\item[The current trail pointer] (value of register TR), which is needed as the boundary
where to unwind the trail upon backtracking. If computation comes back to this
choice point, this will be the address in the trail down to which all variables that
must be reset have been recorded.
\item[The current top of heap] (value of register H), which is needed to recover (garbage)
heap space of all the structures and variables constructed during the failed attempt
which will have resulted in coming back to this choice point.
\end{description}

In summary, a choice point frame is allocated on the stack looking thus:\note{In [War83], 
David Warren does not include the arity in a choice point, as we do here. He sets
up things slightly differently so that this number can always be quickly computed. He can do this
by making register B (and the pointers linking the choice point list) reference a choice point frame
at its end, rather than its start as is the case for environment frames. In other words, register B
contains the stack address immediately following the latest choice point frame, whereas register
E contains the address of the first slot in the environment. Thus, the arity of the latest choice
point predicate is always given by n
B  STACK[B  ]  . For didactic reasons, we chose
to handle E and B identically, judging that saving one stack slot is not really worth the entailed
complication of the code implementing the instructions.}
\begin{verbatim}
B n number
of arguments
B  
 A
 argument
register 

.
.
.
B  n An argument
register n
B  n  
 CE continuation
environment
B  n  	 CP continuation
pointer
B  n   B previous
choice point
B  n   BP next
clause
B  n   TR trail
pointer
B  n   H heap
pointer
\end{verbatim}

Note in passing that M’s explicit definition for allocate N must be altered in
order to work forM. This is because the top of stack is now computed differently
depending on whether an environment or choice point is the latest frame on the
stack. Namely, in M:
\begin{verbatim}
allocate N  if E  B
then newE  E  STACK[E  	]  
else newE  B  STACK[B]  
STACK[newE]  E
STACK[newE  
]  CP
STACK[newE  	]  N
E  newE
P  P  instruction sizeP
\end{verbatim}

To work with the foregoing choice point format, three new I instructions are
added to those already in I. They are to deal with the choice point manipulation
needed for multiple clause definitions. As expected, these instructions correspond,
respectively, to (1) a first, (2) an intermediate (but non ultimate), and (3) a last,
clause of a definition. They are:

1. try me else L

2. retry me else L

3. trust me

where L is an instruction label (i.e., an address in the code area). They have for
effect, respectively:

1. to allocate a new choice point frame on the stack setting its next clause field
to L and the other fields according to the current context, and set B to point
to it;

2. having backtracked to the current choice point (indicated by the current
value of the B register), to reset all the necessary information from it and
update its next clause field to L; and,

3. having backtracked to the current choice point, to reset all the necessary
information from it, then discard it by resetting B to its predecessor (the
value of the link slot).

With this setup, backtracking is effectively handled quite easily. All instructions
in which failure may occur (i.e., some unification instructions and all procedure
calls) must ultimately test whether failure has indeed occurred. If such is the case,
they must then set the instruction counter accordingly. That is, they perform the
following operation:
\[backtrack  P  STACK[B  STACK[B]  ]\]
as opposed to having P be unconditionally set to follow its normal (successful)
course. Naturally, if no more choice point exists on the stack, this is a terminal
failure and execution aborts. All the appropriate alterations of instructions regarding
this precaution are given in Appendix B.

The three choice point instructions are defined explicitly in Figures 4.1, 4.2, and 4.3,
respectively. In the definition of try me else L, we use a global variable
num of args giving the arity of the current procedure. This variable is set by
call that we must accordingly modify for M from its M form as follows:\note{As for num of args, it is legitimate to ask why this is not a global register like E, P, etc.,
in the design. In fact, the exact manner in which the number of arguments is retrieved at choice
point creation time is not at all explained in [War83, War88]. Moreover, upon private inquiry,
David H. D. Warren could not remember whether that was an incidental omission. So we chose to
introduce this global variable as opposed to a register as no such explicit register was specified for
the original WAM.}
\[call pn  CP  P  instruction sizeP\]
\[num of args  n\]
\[P  pn\]

As we just explained, we omit treating the case of failure (and therefore of backtracking)
where pn is not defined in this explicit definition of call pn. Its
obvious complete form is, as those of all instructions of the full WAM, given in
Appendix B.

Finally, the definitions of retry me else L and trust me, use an ancillary
operation, unwind trail, to reset all variables since the last choice point to an unbound
state. Its explicit definition can be found in Appendix B.

In conclusion, there are three patterns of code translations for a procedure definition
in L, depending on whether it has one, two, or more than two clauses. The
code generated in the first case is identical to what is generated for an L program
on M. In the second case, the pattern for a procedure pn is:
\[pn : try me else L\]
\[code for first clause\]
\[L : trust me\]
\[code for second clause\]
\begin{verbatim}
pn : try me else L
code for first clause
L : retry me else L
code for second clause .
.
. Lk
: retry me else Lk
code for penultimate clause
Lk : trust me
code for last clause
\end{verbatim}
where each clause is translated as it would be as a single L clause for M. For
example, M code for the definition:
\begin{verbatim}
pX
a
pb
X
pX
Y  :- pX
a
pb
Y 
\end{verbatim}
is given in Figure 4.4.

\paragraph{Figure 4.1: M choice point instruction try me else}

\paragraph{Figure 4.2: M choice point instruction retry me else}

\paragraph{Figure 4.3: M choice point instruction trust mentioned}

\paragraph{Figure 4.4: M code for a multiple clause definition}

\paragraph{Exercise 4.1} Trace the execution of L query ?-pc d with code in
Figure 4.4, giving all the successive states of the stack, the heap, and the trail.

\paragraph{Exercise 4.2} It is possible to maintain separate AND-stack and
OR-stack.
Discuss the alterations that would be needed to the foregoing setup to do so, ensuring a
correct management of environments and choice points.



\secup
