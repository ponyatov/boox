\secrel{Warren’s Abstract Machine\\Абстрактная машина
Варрена}\label{warren}\secdown

\cp{http://wambook.sourceforge.net/}

\copyright\ Hassan A\"it-Kaci \email{hak@cs.sfu.ca}

\copyright\ David H. D. Warren

\secly{Предисловие к репринтному изданию}

Этот докуент\ --- репринтное издание книги имеющей то же название, которая была
опубликована MIT Press, в 1991 году с кодом ISBN 0-262-51058-8 (мягкая обложка)
and ISBN 0-262-01123-9 (тканый переплет). Редакция книги MIT Press сейчас
не перездается, и права на издание были переданы автору.
Оригинальная версия\note{английская \url{http://wambook.sourceforge.net/}}\
была бесплатно доступна всем, кто хочет ее использовать в некоммерческих целях,
с веб-сайта автора:

\bigskip
\url{http://www.isg.sfu.ca/˜hak/documents/wam.html}
\bigskip

\textit{Сейчас ссылка недоступна, книга пеерехала на
\url{http://wambook.sourceforge.net/}}

\bigskip
Если вы используете ее, пожалуйста дайте мне знать кто вы и для каких целей
хотите ее использовать.

\bigskip
Thank you very much.

\bigskip
Hassan A\"it-Kaci

Burnaby, BC, Canada

May 1997

\secly{Предисловие}

Язык \prolog\ был задуман в начале 1970х Alain Colmerauer a и его коллегами из
Марсельского университета. Его реализация языка была первым практическим
воплощением концепции \term{логического программирования}, предложенной Robert
Kowalski. Ключевая идея логического программирования\ --- вычисления могут быть
выражены в виде конктролируемого вывода (дедукции) из набора декларативных
утверждений. Несмотря на то что эта область значительно развилась за последнее
время, \prolog\ остается наиболее фундаментальным и широко известным языком
логического программирования.

Первой реализацией \prolog а был интерпретатор, написанный на Фортране членами
группы Colmerauer а. Несмотря на очень ущербную в некотором смысле реализацию,
эта версия считается в некотором смысле первым камнем: она доказала
жизнеспособность \prolog а, помогла распространению языка, и заложила основные
принципы реализаций \prolog а. Следующим шагом возможно была \prolog-система для
PDPD-10, разработанная в Университете Эдинбурга мной и коллегами. Эта система
построена на базе техник Марсельской реализации, с добавлением понятия
компиляции \prolog а в низкоуровневый язык (в случае PDP-10 это машинный код), а
также различные техники экономии памяти. Позже я уточнил и абстрагировал
принципы реализации \prolog\ DEC-10 в то, что я называю \prog{WAM} (Warren
Abstract Machine).

\prog{WAM}\ --- абстрактная (виртуальная) машина с архитектурой памяти и набором
команд, заточенных под язык \prolog. Она может быть эффективно реализована на
широком наборе аппаратных архитектур, и служить целевой платформой для
переносимых компиляторов \prolog а. Сейчас она принимается как стандартный базис
при реализации \prolog а. Это конечно лично приятно, но неудобно в том, что WAM
слишком легко принимается как стандарт. Несмотря на то что WAM явилась
результатом длительной работы и большого опыта в реализации \prolog а, это
отнюдь не единственно возможный подход. Например, в то время как WAM применяет
\term{копирование структуры}\note{structure copying}\ для представления
\term{термов}\ \prolog а, метод \term{общих структур}\note{structure
sharing}, использованный в Марсельской и DEC-10 реализациях, все еще можно
рекомендовать к применению. Как бы то ни было, я считаю WAM хорошей отправной
точкой для изучения технологий реализации \prolog-машины.

К сожалению до сих пор не было хорошей книги для ознакомления с внутренним
устройством WAM. Мой оригинальный технический отчет слишком сложен, содержит
только скелетное описание \prolog-машины, и написан для опытного читателя.
Другие работы обсуждают WAM с различных точек зрения, но все же не могут быть
использованы в качестве хорошего вводного руководства.

Поэтому очень приятно видеть появление этого прекрасного учебника, написанного
Hassan A\"it-Kaci. Эту книгу приятно читать. Она объясняет WAM c большой
ясностью и элегантностью. Я думаю что читатели, интересующиеся информатикой,
найдут эту книгу очень стимулирующим введением в увлекательную тему\ ---
реализацию \prolog а. Я очень благодарен Хассану за донесение моей работы до
широкой аудитории.

\bigskip
\copyright\ David H. D. Warren

Бристоль, UK

Февраль 1991

\secly{Реализация машины вывода на \cpp}

В перевод книги Варрена мной\note{\email{dponyatov@gmail.com}} добавлен пример
реализации виртуальной машины вывода на \cpp. Исходные тексты находятся в
каталоге
\href{https://github.com/ponyatov/boox/tree/master/prolog/warren}{\dir{prolog/warren/}}.
Для вставки отдельных частей исходника по ходу книги полные файлы
\file{hpp.hpp}\ и \file{cpp.cpp} разделены на отдельные небольшие фрагменты в
каталогах \dir{hpp/} и \dir{cpp/}. Файл сборки \file{prolog/warren/Makefile}
содержит не только  
\lst{типовой блок компиляции \cpp\ программы}{prolog/warren/mk/cpp.mk}{make}

\secrel{1 Введение 3}\secdown

В 1983 году Дэвид Варрэн разработал абстрактную машину для реализации языка
\prolog, содержащую специальную архитектуру памяти и набор инструкций
\cite{War83}. Эта разработка стала известка как Warren Abstract Machine (WAM)
и стала стандартом де-факто для реализаций компиляторов \prolog а. В
\cite{War83} Варрэн описан WAM в минималистичном стиле, который слишком сложен
для понимания неподготовленным читателем, даже заранее знакомым в операциями
\prolog а. Слишком многое было несказанным, и very little is justified in clear
terms\note{David H. D. Warren поделился в частной беседе что он ``чувствовал
что WAM важна, но к деталям ее реализации вряд ли будет широкий интерес, поэтому
он использовал стиль личных заметок''}. Это привело к очень скудному количеству
поклонников WAM, которые могли был похвастаться пониманием деталей ее работы.
Обычно это были реализаторы \prolog а, которые решили уделить необходимое время
для обучения через делание и кропотливого достижения просветления.

\secrel{1.1 Существующая литература 3}

Свидетельством недостатка понимания может служить тот факт, что за первые шесть
лет было крайне мало публикаций о WAM, не говоря о том чтобы формально доказать
ее корректность. Кроме оригинального герметического доклада Варрэна
\cite{War83}, практически не было никаких официальных публикаций о WAM.
Несколько лет спустя группой Аргонской Национальной Лаборатории был выпущен
единственный черновой стандарт \cite{GLLO85}. Но следует отметить что этот
манускрипт был еще менее понятен, чем оригинальный отчет Варрэна. Его
недостатком была цель описать готовую WAM как есть, а не как пошагово
трансформируемый и оптимизируемый проект.

Стиль пошагового улучшения фактически был использован в публикации David Maier и
David S. Warren\note{Это другой человек, а не разработчик WAM, работа которого
вдохновила S.Warren на исследования. В свою очередь достаточно интересно что
David H. D. Warren позже работал над параллельной архитектурой реализации
\prolog а, поддерживая некоторые идеи, независимо предложенные David S.
Warren.}\ \cite{MW88}. В этой работе можно найти описание техник компиляции
\prolog а похожие на принципы WAM\note{chap.9}. Тем не менее мы считаем что
эта похвальная попытка все еще страдает от нескольких недостатков, если его
рассматривать как окончательный учебник.
Прежде всего эта работа описывает собственный достаточно близкий вариант WAM, но
строго говоря не ее саму. Так что описаны не все особенности WAM.
Более того, объяснения ограничены иллюстративными примерами, и редко четко и
исчерпывающие очерчивают контекст, в котором применяются некоторые оптимизации.
Во-вторых, часть посвященная компиляции \prolog а, идет очень поздно\ --- в
предпоследней главе, полагаясь в деталях реализации на свердетализированные
процедуры на Паскакле, и структуры данных, последовательно улчшаемые в течение
предыдущих разделов. Мы чувствуем что это уводит и запутывает читателя,
интересующегося абстрактной машиной. Наконец, несмотря на то что публикация
содержит серию последовательно улучшаемых вариантов реализации, этот учебник не
отделяет независимые части \prolog а в процессе. Все представленные версии\ ---
полные \prolog-машины. В результате, читатель интересующися выбором и сравнением
отдельных техник, которые он хочет применить, не может различить отдельные
техники в тексте. По всей справедливости, книга Майера и С.Варрэна имеет амбиции
быть первой книгой по логическому программирования. Так что они совершили
подвиг, охватывая так много материала, как теоретического так и практического, и
даже включили техники компиляции \prolog а. Более важно, что их книга была
первой доступной официальной публикацией, содержащей реальный учебник по
техникам WAM.

After the preliminary version of this book had been completed, another recent
publication containing a tutorial on the WAM was brought to this author’s
attention. It is a book due to Patrice Boizumault \cite{Boi88} whose Chapter 9
is devoted to explaining the WAM. There again, its author does not use a gradual
presentation of partial \prolog\ machines. Besides, it is written in French\ ---
a somewhat restrictive trait as far as its readership is concerned. Still,
Boizumault’s book is very well conceived, and contains a detailed discussion
describing an explicit implementation technique for the \var{freeze}
meta-predicate\note{chap.10}.

Even more recently, a formal verification of the correctness of a slight
simplification of the WAM was carried out by David Russinoff \cite{Rus89}. That
work deserves justified praise as it methodically certifies correctness of most
of the WAM with respect to \prolog’s SLD resolution semantics. However, it is
definitely not a tutorial, although Russinoff defines most of the notions he
uses in order to keep his work self-contained. In spite of this effort,
understanding the details is considerably impeded without working familiarity
with the WAM as a prerequisite. At any rate, Russinoff’s contribution is
nevertheless a \emph{premi\`ere} as he is the first to establish rigorously
something that had been taken for granted thus far. Needless to say, that report
is not for the fainthearted.
 
\secrel{1.2 Этот учебник 5}\secdown

\secrel{1.2.1 Disclaimer and motivation 5}

The length of this monography has been kept deliberately short. Indeed, this
author feels that the typical expected reader of a tutorial on the WAM would
wish to get to the heart of the matter quickly and obtain complete but short
answers to questions. Also, for reasons pertaining to the specificity of the
topic covered, it was purposefully decided not to structure it as a real
textbook, with abundant exercises and lengthy comments. Our point is to make the
WAM explicit as it was conceived by David H. D. Warren and to justify its
workings to the reader with convincing, albeit informal, explanations. The few
proposed exercises are meant more as an aid for understanding than as food for
further thoughts.

The reader may find, at points, that some design decisions, clearly correct as
they may be, appear arbitrarily chosen among potentially many other
alternatives, some of which he or she might favor over what is described. Also,
one may feel that this or that detail could be ``simplified'' in some local or
global way. Regarding this, we wish to underscore two points: (1) we chose to
follow Warren’s original design and terminology, describing what he did as
faithfully as possible; and, (2) we warn against the casual thinking up of
alterations that, although that may appear to be “smarter” from a local
standpoint, will generally bear subtle global consequences interfering with
other decisions or optimizations made elsewhere in the design.
This being said, we did depart in some marginal way from a few original WAM
details. However, where our deviations from the original conception are
proposed, an explicit mention will be made and a justification given.

Our motivation to be so conservative is simple: our goal is not to teach the
world how to implement Prolog optimally, nor is it to provide a guide to the
state of the art on the subject. Indeed, having contributed little to the craft
of Prolog implementation, this author claims glaring incompetence for carrying
out such a task. Rather, this work’s intention is to explain in simpler terms,
and justify with informal discussions, David H. D. Warren’s abstract machine
\emph{specifically} and \emph{exclusively}. Our source is what he describes in
\cite{War83, War88}. The expected achievement is merely the long overdue filling
of a gap so far existing for whoever may be curious to acquire \emph{basic}
knowledge of Prolog implementation techniques, as well as to serve as a spring
board for the expert eager to contribute further to this field for which the WAM
is, in fact, just the tip of an iceberg. As such, it is hoped that this
monograph would constitute an interesting and self-contained complement to basic
textbooks for general courses on logic programming, as well as to those on
compiler design for more conventional programming languages. As a stand-alone
work, it could be a quick reference for the computer professional in need of
direct access to WAM concepts.

\secrel{1.2.2 Organization of presentation 6}

Our style of teaching the WAM makes a special effort to consider carefully each
feature of the WAM design in isolation by introducing separately and
incrementally distinct aspects of Prolog. This allows us to explain as limpidly
as possible specific principles proper to each. We then stitch and merge the
different patches into larger pieces, introducing independent optimizations one
at a time, converging eventually to the complete WAM design as described in
\cite{War83} or as overviewed in \cite{War88}. Thus, in \ref{warren2}, we
consider unification alone. Then, we look at flat resolution (that is, Prolog
without backtracking) in \ref{warren3}. Following that, we turn to disjunctive
definitions and backtracking in \ref{warren4}. At that point, we will have a
complete, albeit na\"ive, design for pure Prolog. In \ref{warren5}, this
first-cut design will be subjected to a series of transformations aiming at
optimizing its performance, the end product of which is the full WAM. We have
also prepared an index for quick reference to most critical concepts used in the
WAM, something without which no (real) tutorial could possibly be complete.

It is expected that the reader already has a basic understanding of the
operational semantics of \prolog\ --- in particular, of unification and
backtracking. Nevertheless, to make this work also profitable to readers lacking
this background, we have provided a quick summary of the necessary \prolog\
notions in \ref{warrenA}. As for notation, we implicitly use the syntax of
so-called Edinburgh Prolog (see, for instance, \cite{CM84}), which we also
recall in that appendix. Finally, \ref{warrenB} contains a recapitulation of all
explicit definitions implementing the full WAM instruction set and its
architecture so as to serve as a complete and concise summary.

\secup
\secup
\secrel{Унификация\ --- ясно и просто}\label{warren2}\secdown

Напомним что терм (первого порядка)\ ---
\termdef{переменная}{\prolog!переменная} (задается большой буквой в начале
имени), \termdef{константа}{\prolog!константа} (задается маленькой буквой в
начале имени) или \termdef{терм}{\prolog!терм}\ --- структура вида
$f(t_1,\ldots,t_n)$, где $f$ символ называемый
\termdef{функтором}{\prolog|функтор} (записывается аналогично константе, с
маленькой буквы), а элементы $t_i$ тоже термы первого порядка\ ---
\termdef{субтермы}{\prolog!субтерм}. Число субтермов для данного функтора
предопределено, и называется \termdef{\'{а}рностью}{\prolog!арность} функтора.
Для обеспечения возможности использовать один и тот же символ с разной арностью,
мы должны использовать запись $f/n$, что обозначает функтор $f$ с арностью $n$.
Таким образом, два функтора равны только в том случае, если они имеют
\emph{одинаковые символ $f$ и арность $n$}. Разрешая случай $n=0$ можно
рассматривать константу как особый случай терма: константе $c$ соответствует
функтор $c/0$ с нулевой арностью.

Мы рассмотрим очень простой низкоуровневый\note{IL\--- Intermediate Language}\
язык $\mathcal L_0$. На этом языке мы можем описать два вида объектов:
\termdef{терм программы}{\prolog!терм программы} и \termdef{терм
запроса}{\prolog!терм запроса}. Оба этих вида запросов являются термами первого
порядка, но не переменными. Семантика $\mathcal L_0$ равносильна вычислению
самого общего унификатора программы или запроса. Что касается синтаксиса,
$\mathcal L_0$ будет описывать программу как \verb|t| и запрос как \verb$?-t$
где \verb$t$ является термом. Область видимости переменных ограничена термом
программы/запроса. Таким образом, \emph{значение программы не зависит от имен ее
переменных}. Интерпретатор для $\mathcal L_0$ будет использовать определенное
представление данных для термов и использовать алгоритм унификации для ее
операционной семантики. Затем мы опишем $\mathcal M_0 = (\mathcal D_0,\mathcal
I_0)$ , дизайн абстрактной машины для $\mathcal L_0$ содержащий представление
данных $\mathcal D_0$, над которыми выполняется множество $\mathcal I_0$
машинных инструкций.

Идея достаточно проста: имея определенных программный терм $p$, мы можем
выполнить лююбой запрос \verb|?-q|, и выполнение запроса завершится с ошибкой
если $p$ и $q$ не унифицируются, или будет успешным с привязкой переменных
в $q$ полученной при унификации запроса с $p$.

\secrel{Представление термов}\label{warren21}

Для начала давайте определим внутреннее представление термов в языке $\mathcal
L_0$. Мы будем использовать глобальный блок хранения данных в форме адресуемой
\termdef{кучи}{\prolog!куча} который мы назовем \var{HEAP}: массив ячеек данных.
Адресом ячейки в куче является индекс элемента массива \var{HEAP}.

Для представления произвольных термов в \var{HEAP} будет достаточно закодировать
переменные и ``структуры'' имеющие форму $f(@_1,..,@_n)$ где $f/n$ функтор и
$@_i$ ссылки на адреса кучи для $n$ субтермов. Таким образом существует два вида
данных, хранимых в куче: переменные и структуры термов. Явно заданные
\termdef{тэги}{\prolog|тэг}, появляющиеся как часть внутреннего формата ячеек
кучи, будут использоваться для различения между этими двумя типами
данных.\note{интересно рассмотреть расширение тэгирования для реализации
ООП и динамического контроля типов}

Переменная будет индентифицироваться как указатель, и представляться как одна
ячейка кучи, так что мы должны говорить о \termdef{ячейках
переменных}{\prolog!ячейка переменной}. Ячейка переменной отмечается тэгом
\class{REF}, и обозначается как $<REF,k>$ где $k$ адрес хранения, т.е. индекс в
\var{HEAP}. Этот механизм предназначен для облегчения связывания переменных
через установление ссылки на терм в переменной, которая связывается с этим
термом. Таким образом при связывании переменной адресная часть
\class{REF}-ячейки получает значение соответствующего адреса терма. Соглашение о
представлении \termdef{несвязанной переменной}{\prolog!несвязанная переменная}\
--- адресная часть \class{REF}-ячейки указывает на саму переменную. Таким
образом \emph{несвязанные переменные представляются \class{REF}-ячейкой со
ссылкой на саму себя}.

Структуры\ --- термы не являющиеся переменными. Формат кучи для представления
структуры $f(t_1,..,t_n)$ содержит $n+2$ ячеек кучи. Первые две ячейки не
обязательно смежные. По сути первая их этих двух ячеек выступает в роли
сортированного указателя на вторую ячейку, и в то же время сама выступает как
представление функтора $f/n$.\note{причина использования этой кажущеся странной
косвенной адресации\ --- реализация разделяемых структур (structure sharing)\
--- будет вскоре ясна} Остальные $n$ ячеек предназначены для упорядоченного хранения
ссылок на корни соответствующих $n$ субтермов. 

Детальнее, первая из $n+2$ ячеек представляющих терм $f(t_1,..,t_n)$
форматирована как тэгированная \termdef{структурная ячейка}{\prolog!структурная
ячейка}, которую можно записать как $<STR,k>$, содержит тэг \var{STR} и
указатель $k$ на \termdef{ячейку функтора}{\prolog!ячейка функтора}, храняющую
представление функтора $f/n$. Важно отметить что \emph{непосредственно за
\term{ячейкой функтора} в смежных адресах всегда следуют $n$ \term{структурных
ячеек}, представляющих каждый из $t_i$ субтермов}. Так что если $HEAP[k]=f/n$
то $HEAP[k+1]$ будет ссылаться на первый субтерм $t_1$, а $HEAP[k+n]$ будет
ссылаться на последний субтерм $t_n$.

\fig{\\Фиг. 2.1: Представление кучи для терма
$p(Z,h(Z,W),f(W))$}{prolog/warren/fig21.pdf}{height=0.5\textheight}
\label{warrenfig21}

\begin{tabular}{l l l}
0 & STR & 1 \\
1 & $h/2$ \\
2 & REF & 2\\
3 & REF & 3\\
4 & STR & 5\\
5 & $f/1$\\
6 & REF & 3\\
7 & STR & 8\\
8 & $p/3$\\
9 & REF & 2\\
10 & STR & 1\\
11 & STR & 5\\
\end{tabular}

\bigskip
Например, рассмотрим представление кучи для терма $p(Z,h(Z,W),f(W))$, начальная
ячейка которого находится по адресу 7 (иллюстрация \ref{warrenfig21}).
Отметим что \emph{для каждой} непривязанной переменной существует только одно
вхождение, представленное как \class{REF}-ячейка, в то время как другие ее
вхождения в исходный терм представляются как ссылки на первое вхождение
($Z=HEAP[2]$, $W=HEAP[3]$). Также обратите внимание что за структурными ячейками
по адресам 0, 4 и 7 \emph{сразу} следуют их ячейки функторов, но это не так для
адресов 10 и 11.


\secrel{Компиляция $\mathcal L_0$ запросов}

Согласно операционной семантике $\mathcal L_0$ обработка запроса состоит из
подготовке в решению уравнения с одной стороны. А именно, терм запроса $q$
транлируется в последовательность инструкций, целью которой является построение
экземпляра $q$ на куче из текстового представления $q$. Таким образом, из-за
древовидной структуры терма и множествественных вхождениях переменных,
необходимо, чтобы при обработке части терма где-то временно сохранялись части
терма, которые еще предстоит обработать, или переменные которые могут
встретиться еще раз далее по ходу работы. Для этой цели виртуальная машина
$\mathcal M_0$ наделена достаточным количеством (изменяемых)
\termdef{регистров}{\prolog!регистр} $X_1$, $X_2$,\ldots которые используются
для временного хранения данных кучи по мере создания промежуточных термов. Таким
образом, содержимое каждого регистра должно иметь формат ячейки кучи. Эти
изменяемые регистры выделяются для терма по мере доступности, так что (1)
регистр $X_1$ всегда распределяется для охватывающего терма, и (2) тот же
регистр распределяется для всех вхождений определенной переменной.
Например регистры для переменных терма $p(Z,h(Z,W),f(W))$ распределяются
\begin{equation*}
\begin{split}
X_1 &= p(X_2,X_3,X_4)\\
X_2 &= Z\\
X_3 &= h(X_2,X_5)\\
X_4 &= f(X_5)\\
X_5 &= W
\end{split}
\end{equation*}

Это равносильно тому что терм рассматривается как сплющенный конъюктивный набор
уравнений в форме $X_i=X$ или $X_i=f(X_{i_1},..,X_{i_n}), (n \geqslant 0)$ ,
где члены $X_i$ различные новые имена переменных. Есть два последствия
распределения регистров: (1) все внешние имена переменных (такие как $Z$ and $W$
в нашем примере) могут быть забыты; и (2) терм запроса может быть
трансформирован в его \termdef{сплющенную форму}{\prolog!сплющенная форма},
т.е. последовательность назначений регистров только в форме 
$X_i=f(X_{i_1},..,X_{i_n})$. Эта форма\ --- то, что контролирует построение
представления терма в куче. Таким образом, чтобы генерация кода слева направо
была хорошо обоснована, необходимо сформировать сплющенный терм запроса, так
чтобы гарантировать что \emph{имена регистров не могут использоваться в правых
частях присвоений (например как субтерм) до их инициализации}\note{if it has one
(viz., being the lefthand side)}. Например сплющенная форма терма запроса
$p(Z,h(Z,W),f(W))$ это последовательность
$X_3=h(X_2,X_5)$, $X_4=f(X_5)$, $X_1=p(X_2,X_3,X_4)$\note{исключена привязка
переменных на регистры $X_2$, $X_5$}.

Сканируя сплющенный терм запроса слева направо, каждый компонент в форме
$X_i=f(X_{i_1},..,X_{i_n})$ токенизируется в последовательность
$X_i=f/n,X_{i_1},..,X_{i_n}$ такую что после регистра ассоциированного с n-арным
функтором идет последовательность $n$ имен регистров. Так что в потоке таких
токенов полученных в результате токенизации полного сплющенного терма,
существует три вида элементов для обработки:
\begin{enumerate}
  \item 
регистр ассоциированный со структурным функтором;
  \item 
регистр-аргумент который не был нигде равнее встречен в потоке;
  \item 
регистр-аргумент который уже был упомянут в потоке.
\end{enumerate}

Из такого потока легко получить представление кучи используя метод управляемого
потоком токенов синтеза. Для реализации этого нужно выполнить сооответствующие
действия для каждого типа токенов:
\begin{enumerate}
  \item 
создать на куче новую ячейку STR (и примыкающий функтор) и скопировать эту
ячейку в указаный регистр;
  \item 
создать на куче новую ячейку REF содержащую собственный адрес, и скопировать
ее в указанный регистр;
  \item 
создать на куче новую ячейку и копировать в нее значение регистра.
\end{enumerate}

Each of these three actions specifies the effect of respective instructions of
the machine $M_0$ that we note:
\begin{enumerate}
  \item 
put structure f n Xi
  \item 
set variable Xi
  \item 
set value Xi
\end{enumerate}
respectively.

From the preceding considerations, it has become clear that the heap is implicitly
used as a stack for building terms. Namely, term parts being constructed are
incrementally piled on top of what already exists in the heap. Therefore, it is
necessary to keep the address of the next free cell in the heap somewhere, precisely
as for a stack.\note{As a matter of fact, in [War83], Warren refers to the heap
as the \emph{global stack}.} Adding to M a global register H containing at all
times the next available address on the heap, these three instructions are given explicitly
in Figure 2.2. For example, given that registers are allocated as above, the
sequence of instructions to build the query term $pZh$
is shown in Figure 2.3.

\paragraph{Exercise 2.1} Verify that the effect of executing the sequence of
instructions shown in Figure 2.3 (starting with H 
 )
does indeed yield a correct heap representation
for the term $pZh$ the one shown earlier as Figure 2.1, in
fact.

\secrel{2.3 Compiling L
programs . . . . . . . . . . . . . . . . . . . . . . . 13}

\secrel{2.4 Argument registers . . . . . . . . . . . . . . . . . . . . . . . . .
19}

\secup
\secrel{Flat Resolution}\label{warren3}\secdown

We now extend the language L0 into a language L where procedures are no longer
reduced only to facts but may also have bodies. A body defines a procedure as a
conjunctive sequence of atoms. Said otherwise, L is Prolog without backtracking.

An L program is a set of procedure definitions or (definite) clauses, at most
one per predicate name, of the form ‘a :- a0000an0’ where n 0  and the ai’s are
atoms. As before, when n  , the clause is called a fact and written without
the ‘:-’ implication symbol. When n  , the clause is called a rule, the atom a
is called its head, the sequence of atoms a0000an is called its body and atoms
composing this body are called goals. A rule with exactly one body goal is
called a chain (rule). Other rules are called deep rules. L queries are
sequences of goals, of the form ‘?-g0000gk0’ where k 0 . When k  , the query
is called the empty query. As in Prolog, the scope of variables is limited to
the clause or query in which they appear.

Executing a query ‘?-g0000gk0’ in the context of a program made up of a set of
procedure-defining clauses consists of repeated application of leftmost
resolution until the empty query, or failure, is obtained. Leftmost resolution
amounts to unifying the goal g0 with its definition’s head (or failing if none
exists) and, if this succeeds, executing the query resulting from replacing g0
by its definition body, variables in scope bearing the binding side-effects of
unification. Thus, executing a query in L either terminates with success (i.e.,
it simplifies into the empty query), or terminates with failure, or never
terminates. The “result” of an L query whose execution terminates with success
is the (dereferenced) binding of its original variables after termination.

Note that a clause with a non-empty body can be viewed in fact as a conditional
query. That is, it behaves as a query provided that its head successfully
unifies with a predicate definition. Facts merely verify this condition, adding
nothing new to the query but a contingent binding constraint. Thus, as a first
approximation, since an L query (resp., clause body) is a conjunctive sequence
of atoms interpreted as procedure calls with unification as argument passing,
instructions for it may simply be the concatenation of the compiled code of each
goal as an L0 query making it up. As for a clause head, since the semantics
requires that it retrieves arguments by unification as did facts in L0,
instructions for L0’s fact unification are clearly sufficient.

Therefore, M0 unification instructions can be used for L clauses, but with two
measures of caution: one concerning continuation of execution of a goal
sequence, and one meant to avoid conflicting use of argument registers.

\secrel{Facts}

Let us first only consider L facts. Note that L0 is all contained in L.
Therefore, it is natural to expect that the exact same compilation scheme for
facts carries over untouched from L0 to L. This is true up to a wee detail
regarding the proceed instruction. It must be made to continue execution, after
successfully returning from a call to a fact, back to the instruction in the
goal sequence following the call. To do this correctly, we will use another
global register CP, along with P, set to contain the address (in the code area)
of the next instruction to follow up with upon successful return from a call
(i.e., set to P instruction size P at procedure call time). Then, having exited
the called procedure’s code sequence, execution could thus be resumed as
indicated by CP. Thus, for L’s facts, we need to alter the effect of M0’s call
pn to:
cal
\[pn  CP  P  instruction sizeP0\]
\[P  pn0\]
and that of proceed to:
\[P  CP\]
As before, when the procedure pn is not defined, execution fails.
In summary, with the simple foregoing adjustment, L facts are translated exactly
as were L0 facts.
\secrel{Rules and queries}

We now must think about translating rules. A query is a particular case of a rule
in the sense that it is one with no head. It is translated exactly the same way,
but without the instructions for the missing head. The idea is to use L0’s instructions,
treating the head as a fact, and each goal in the body as an L0 query term
in sequence; that is, roughly translate a rule ‘p0000
:- p00000000pn00000’
following the pattern:
\begin{verbatim}
get arguments of p
put arguments of p0
call p0
.
.
.
put arguments of pn
call pn
\end{verbatim}

Here, in addition to ensuring correct continuation of execution, we must arrange
for correct use of argument registers. Indeed, since the same registers are used by
each goal in a query or body sequence to pass its arguments to the procedure it
invokes, variables that occur in many different goals in the scope of the sequence
need to be protected from the side effects of put instructions. For example, consider
the rule ‘pX
Y 0 :- qX
Z0
rZ
Y 00’ If the variables YZ
were allowed
to be accessible only from an argument register, no guarantee could be made that
they still would be after performing the unifications required in executing the body
of p.

Therefore, it is necessary that variables of this kind be saved in an environment
associated with each activation of the procedure they appear in. Variables which
occur in more than one body goal are dubbed permanent as they have to outlive
the procedure call where they first appear. All other variables in a scope that are
not permanent are called temporary. We shall denote a permanent variable as Yi,
and use Xi as before for temporary variables. To determine whether a variable is
permanent or temporary in a rule, the head atom is considered to be part of the
first body goal. This is because get and unify instructions do not load registers
for further processing. Thus, the variable X in the example above is temporary as
it does not occur in more than one goal in the body (i.e., it is not affected by more
than one goal’s put instructions).

Clearly, permanent variables behave like conventional local variables in a procedure.
The situation is therefore quite familiar. As is customary in programming
languages, we protect a procedure’s local variables by maintaining a run-time
stack of procedure activation frames in which to save information needed for the
correct execution of what remains to be done after returning from a procedure call.
We call such a frame an environment frame. We will keep the address of the latest
environment on top of the stack in a global register E.\note{In [War83], this stack is called the local stack to distinguish it from the global stack (see
Footnote 1 at the bottom of Page 13).}

As for continuation of execution, the situation for rules is not as simple as that
for facts. Indeed, since a rule serves to invoke further procedures in its body, the
value of the program continuation register CP, which was saved at the point of
its call, will be overwritten. Therefore, it is necessary to preserve continuation
information by saving the value of CP along with permanent variables.

Let us recapitulate: M is an augmentation of M0 with the addition of a new data
area, along with the heap (HEAP), the code area (CODE), and the push-down list
(PDL). It is called the stack (STACK) and will contain procedure activation frames.
Stack frames are called environments. An environment is pushed onto STACK
upon a (non-fact) procedure entry call, and popped from STACK upon return.
Thus, an allocate/deallocate pair of instructions must bracket the code
generated for a rule in order to create and discard, respectively, such environment
frames on the stack. In addition, deallocate being the ultimate instruction
of the rule, it must connect to the appropriate next instruction as indicated by
the continuation pointer that had been saved upon entry in the environment being
discarded.

Since the size of an environment varies with each procedure in function of its
number of permanent variables, the stack is organized as a linked list through a
continuation environment slot; i.e., a cell in each environment frame bearing the
stack index of the environment previously pushed onto the stack.

To sum up, two new I instructions for M are added to the ones we defined for
I0:

1. allocate

2. deallocate

with effect, respectively:

1. to allocate a new environment on the stack, setting its continuation environment
field to the current value of E, and its continuation point field to that
of CP; and,

2. to remove the environment frame at stack location E from the stack and
proceed, resetting P to the value of its CP field and E to the value of its CE
field.

To have proper effect, an allocate instruction needs to have access to the size
of the current environment in order to increment the value of E by the right stack
offset. The necessary piece of information is a function of the calling clause (i.e.,
the number of permanent variables occurring in the calling clause). Therefore, it
is easily statically available at the time the code for the calling clause is generated.
Now, the problem is to transmit this information to the called procedure that, if
defined as a rule (i.e., starting with an allocate), will need it dynamically,
depending on which clause calls it. A simple solution is to save this offset in the
calling clause’s environment frame from where it can be retrieved by a callee that
needs it. Hence, in M, an additional slot in an environment is set by allocate
to contain the number of permanent variables in the clause in question.

Summing up again, an M stack environment frame contains:

1. the address in the code area of the next instruction to execute upon (successful)
return from the invoked procedure;

2. the stack address of the previous environment to reinstate upon return (i.e.,
where to pop the stack to);

3. the offset of this frame on the stack (the number of permanent variables);
and,

4. as many cells as there are permanent variables in the body of the invoked
procedure (possibly none).

Such an M environment frame pushed on top of the stack looks thus:
\[E CE continuation environment\]

This necessitates giving allocate an explicit argument that is the number of
permanent variables of the rule at hand, such that, in M:
\[allocate N =\]

Similarly, the explicit definition of M’s deallocate is:
\[deallocate  =\]

With this being set up, the general translation scheme into M instructions for an
L rule ‘p0000
:- p00000000pn00000’
with N permanent variables will follow
the pattern:
\[p allocate N\]

For example, for L clause ‘pX
Y 0 :- qX
Z0
rZ
Y 00’, the corresponding
M code is shown in Figure 3.1.

\paragraph{Figure 3.1: M machine code for rule pX
Y 0 :- qX
Z0
rZ
Y 00}
\begin{verbatim}

\end{verbatim}

\paragraph{Exercise 3.1} GiveM code for L facts qa0 b and rb0 c and L query
?-pU0 V , then trace the code shown in Figure 3.1 and verify that the solution produced is
U 
 a0 V 
 c.
C  



\secup
\secrel{4 Prolog 33}\label{warren4}\secdown

\secrel{4.1 Environment protection . . . . . . . . . . . . . . . . . . . . . . .
34}

\secrel{4.2 What’s in a choice point . . . . . . . . . . . . . . . . . . . . . .
36}

\secup

\secrel{Optimizing the Design}\label{warren5}\secdown

Now that the reader is hopefully convinced that the design we have reached forms
an adequate target language and architecture for compiling pure Prolog, we can
begin transforming it in order to recover Warren’s machine as an ultimate design.
Therefore, since all optimizations considered here are part of the definitive design,
we shall now refer to the abstract machine gradually being elaborated as the WAM.
In the process, we shall abide by a few principles of design pervasively motivating
all the conception features of the WAM. We will repeatedly invoke these principles
in design decisions as we progress toward the full WAM engine, as more evidence
justifying them accrues.

\textsc{WAM PRINCIPLE 1} Heap space is to be used as sparingly as possible, as
terms built on the heap turn out to be relatively persistent.

\textsc{WAM PRINCIPLE 2} Registers must be allocated in such a way as to avoid
unnecessary data movement, and minimize code size as well.

\textsc{WAM PRINCIPLE 3} Particular situations that occur very often, even
though correctly handled by general-case instructions, are to be accommodated by special
ones if space and/or time may be saved thanks to their specificity.

In the light of WAM Principles 1, 2, and 3, we may now improve on M.

\paragraph{Figure 5.1: Better heap representation for term pZ hZW f W}
\begin{verbatim}

h
 REF 
 REF
 f 
 REF
 p
 REF 
 STR
 STR
\end{verbatim}

\secrel{Heap representation}

As many readers of [AK90] did, this reader may have wondered about the necessity
of the extra level of indirection systematically introduced in the heap by
an STR cell for each functor symbol. In particular, Fernando Pereira [Per90]
suggested that instead of that shown in Figure 2.1 on Page 11, a more economical
heap representation for pZ
hZW
f W
ought to be that of Figure 5.1, where
reference to the term from elsewhere must be from a store (or register) cell of the
form h STR
 i. In other words, there is actually no need to allot a systematic STR
cell before each functor cell.

As it turns out, only one tiny modification of one instruction is needed in order to
accommodate this more compact representation. Namely, the put structure
instruction is simplified to:
\begin{verbatim}
put structure f n
Xi  HEAP[H]  f n
Xi  h STR
H i
H  H  
\end{verbatim}

Clearly, this is not only in complete congruence with WAM Principle 1, but it also
eliminates unnecessary levels of indirection and hence speeds up dereferencing.

The main reason for our not having used this better heap representation in Section
2.1 was essentially didactic, wishing to avoid having to mention references
from outside the heap (e.g., from registers) before due time. In addition, we did
not bother bringing up this optimization in [AK90] as we are doing here, as we
had not realized that so little was in fact needed to incorporate it.\note{After dire reflection seeded by discussions with Fernando Pereira, we eventually realized that
this optimization was indeed cheap—a fact that had escaped our attention. We are grateful to him
for pointing this out. However, he himself warns [Per90]:

“Now, this representation (which, I believe, is the one used by Quintus, SICStus
Prolog, etc.) has indeed some disadvantages:

1. If there aren’t enough tags to distinguish functor cells from the other
cells, garbage collection becomes trickier, because a pointed-to value does not in
general identify its own type (only the pointer does).

2. If you want to use [the Huet-Fages] circular term unification algorithm,
redirecting pointers becomes messy, for the [same] reason...
In fact, what [the term representation in Section 2.1 is] doing is enforcing a convention
that makes every functor application tagged as such by the appearance of a
STR cell just before the functor word.”}


\secrel{5.2 Constants, lists, and anonymous variables . . . . . . . . . . . . .
47}

\secrel{5.3 A note on set instructions . . . . . . . . . . . . . . . . . . . . .
52}

\secrel{5.4 Register allocation . . . . . . . . . . . . . . . . . . . . . . . .
. 54}

\secrel{5.5 Last call optimization . . . . . . . . . . . . . . . . . . . . . . .
. 56}

\secrel{5.6 Chain rules . . . . . . . . . . . . . . . . . . . . . . . . . . . .
. 57}

\secrel{5.7 Environment trimming . . . . . . . . . . . . . . . . . . . . . . .
58}

\secrel{5.8 Stack variables . . . . . . . . . . . . . . . . . . . . . . . . . .
. 60}\secdown

\secrel{5.8.1 Variable binding and memory layout . . . . . . . . . . . . 62}

\secrel{5.8.2 Unsafe variables . . . . . . . . . . . . . . . . . . . . . . 64}

\secrel{5.8.3 Nested stack references . . . . . . . . . . . . . . . . . . . 67}

\secup

\secrel{5.9 Variable classification revisited . . . . . . . . . . . . . . . . .
. . 69}

\secrel{5.10 Indexing . . . . . . . . . . . . . . . . . . . . . . . . . . . . .
. . 75}

\secrel{5.11 Cut . . . . . . . . . . . . . . . . . . . . . . . . . . . . . . . .
. . 83}


\secup

\secrel{6 Conclusion 89}
\secrel{A Prolog in a Nutshell 91}\label{warrenA}
\secrel{B The WAM at a glance 97}\label{warrenB}\secdown

\secrel{B.1 WAM instructions . . . . . . . . . . . . . . . . . . . . . . . . . .
97}

\secrel{B.2 WAM ancillary operations . . . . . . . . . . . . . . . . . . . . .
112}

\secrel{B.3 WAM memory layout and registers . . . . . . . . . . . . . . . . .
117}

\secup


\secup
