\secrel{Argument registers}

Since we have in mind to use unification in Prolog for procedure invocation, we
can introduce a distinction between atoms (terms whose functor is a predicate)
and terms (arguments to a predicate). We thus extend L into a language L0
similar to L but where a program may be a set of first-order atoms each defining
at most one fact per predicate name. Thus, in the context of such a program,
execution of a query connects to the appropriate definition to use for solving a
given unification equation, or fails if none exists for the predicate invoked.

\paragraph{Figure 2.7: The \term{unify} operation}
\begin{verbatim}

\end{verbatim}

The set of instructions I contains all those in I.
In M, compiled code is stored in a code area (CODE), an addressable array of
data words, each containing a possibly labeled instruction over one or more
memory words consisting of an opcode followed by operands. For convenience, the
size of an instruction stored at address a (i.e., CODE[a]) will be assumed given
by the expression instruction sizea.
Labels are symbolic entry points into the code area that may be used as operands
of instructions for transferring control to the code labeled accordingly.
Therefore, there is no need to store a procedure name in the heap as it denotes
a key into a compiled instruction sequence. Thus, a new instruction call pn can
be used to pass control over to the instruction labeled with pn, or fail if none
such exists.

A global register P is always set to contain the address of the next instruction to
execute (an instruction counter). The standard execution order of instructions is
sequential. Unless failure occurs, most machine instructions (like all those seen
before) are implicitly assumed, to increment P by an appropriate offset in the
code area as an ultimate action. This offset is the size of the instruction at address
P. However, some instructions have for purpose to break the sequential order of
execution or to connect to some other instruction at the end of a sequence. These
instructions are called control instructions as they typically set P in a non-standard
way. This is the case of call pn, whose explicit effect, in the machine M0, is:
\[call pn  P  pnwhe\]
where the notation pn stands for the address in the code area of instruction
labeled pn. If the procedure pn is not defined (i.e., if that address is not
allocated in the code area), a unification failure occurs and overall execution
aborts. W

We also introduce another control instruction, proceed, which indicates the end
of a fact’s instruction sequence. These two new control instructions’ effects are
trivial for now, and they will be elaborated later. For our present purposes, it is
sufficient that proceed be construed as a no-op (i.e., just a code terminator),
and call pn as an unconditional “jump” to the start address of the instruction
sequence for program term with functor pn.

Having eliminated predicate symbols from the heap, the unification problem between
fact and query terms amounts to solving, not one, but many equations,
simultaneously. Namely, there are as many term roots in a given fact or query as
there are arguments to the corresponding predicate. Therefore, we must organize
registers quite specifically so as to reflect this situation. As we privileged X
 before
to denote the (single) term root, we generalize the convention to registers X

to Xn which will now always refer to the first to n-th arguments of a fact or
query atom. In other words, registers X
Xn are systematically allocated to term roots of an n-ary predicate’s arguments.
To emphasize this, we use a conspicuous notation, writing a register Ai rather
than Xi when it is being used as an argument of an atom. In that case, we refer
to that register as an argument register. Otherwise, where register Xi is not
used as an argument register, it is written Xi, as usual. Note that this is just
notation as the Ai’s are not new registers but the same old Xi’s used thus far.
For example, registers are now allocated for the variables of the atom pZ hZW f
W as follows:
\[A1\]
\[A2\]
\[A3\]
\[A4=W\]

Observe also that a new situation arises now as variables can be arguments and
thus must be handled as roots. Therefore, provision must be made for variables
to be loaded into, or extracted from, argument registers for queries and facts,
respectively. As before, the necessary instructions correspond to when a variable
argument is a first or later occurrence, either in a query or a fact. In a query,

1. the first occurrence of a variable in i-th argument position pushes a new
unbound REF cell onto the heap and copies it into that variable’s register as
well as argument register Ai; and,

2. a later occurrence copies its value into argument register Ai. Whereas, in a
fact,

3. the first occurrence of a variable in i-th argument position sets it to the value
of argument register Ai; and,

4. a later occurrence unifies it with the value of Ai.

The corresponding instructions are, respectively:

1. put variable Xn
Ai

2. put value Xn
Ai

3. get variable Xn
Ai

4. get value Xn
Ai

and are given explicitly in Figure 2.8. For example, Figure 2.9 shows code
generated for query ?- pZ hZW f W and Figure 2.10 that for fact pf X hY f a
Y .

\paragraph{Figure 2.8: $M_0$ instructions for variable arguments}
\begin{verbatim}

\end{verbatim}

\paragraph{Exercise 2.6} Verify that the effect of executing the sequence of M
instructions shown in Figure 2.9 produces the same heap representation as that produced by
the M code of Figure 2.3 (see Exercise 2.1).

\paragraph{Exercise 2.7} Verify that the effect of executing the sequence of M
instructions shown in Figure 2.10 right after that in Figure 2.9 produces the MGU of the
terms p Z h ZW  f  W   and p f  X  h Y f  a   Y  . That is, the binding W
 f  a , X f  a , Y f  f  a  , Z f  f  a  .
 
\paragraph{Exercise 2.8} What are the respective sequences of M0 instructions
for L query term ?-p f  X  h Y f  a   Y   and L program term p Z h ZW  f  W  ?

\paragraph{Exercise 2.9} After doing Exercise 2.8, verify that the effect of
executing the sequence you produced yields the same solution as that of Exercise 2.7.

\paragraph{Figure 2.9: Argument registers for L query ?-pZ hZW f W}
\begin{verbatim}

\end{verbatim}

\paragraph{Figure 2.10: Argument registers for L0 fact pf X hY f a Y }
\begin{verbatim}

\end{verbatim}