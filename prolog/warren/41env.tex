\secrel{Environment protection}

Before we go into the details of what exactly constitutes a choice frame, we must
ponder carefully the interaction between the AND-stack and the OR-stack. As
long as we considered (deterministic) L program definitions, it was clearly safe
to deallocate an environment frame allocated to a rule after successfully falling
off the end of the rule. Now, the situation is not quite so straightforward as later
failure may force reconsidering a choice from a computation state in the middle
of a rule whose environment has long been deallocated. This case is illustrated by
the following example program:
\begin{verbatim}
a :- bX0
cX00
bX0
:- eX00
c
00
eX0
:- f X00
eX0
:- gX00
f
00
g
00
\end{verbatim}

Executing ‘?-a0’ allocates an environment for a, then calls b. Next, an environment
for b is allocated, and e is called. This creates a choice point on the
OR-stack, and an environment for e is pushed onto the AND-stack. At this point
the two stacks look thus:\note{In these diagrams, the stacks grow downwards;
i.e., the stack top is the lower part.}
\[E  Environment for e\]
\[B  Choice point for e\]

The following call to f succeeds binding X to 	. The environment for e is deallocated,
then the environment for b is also deallocated. This leads to stacks looking
thus:
\[E  Environment for a\]
\[B  Choice point for e\]

Next, the continuation follows up with execution of a’s body, calling c, which
immediately hits failure. The choice point indicated by B shows an alternative
clause for e, but at this point b’s environment has been lost. Indeed, in a more
involved example where c proceeded deeper before failing, the old stack space for
b’s environment would have been overwritten by further calls in c’s body

Therefore, to avoid this kind of misfortune, a setup must be found to prevent
unrecoverable deallocation of environment frames whose creation chronologically
precedes that of any existing choice point. The idea is that every choice point
must “protect” from deallocation all environment frames already existing before
its creation. Now, since a stack reflects chronological order, it makes sense to
use the same stack for both environments and choice points. A choice point now
caps all older environments. In effect, as long as it is active, it forces allocation of
further environments on top of it, preventing the older environments’ stack space
to be overwritten even though they may explicitly be deallocated. This allows
their safe resurrection if needed by coming back to an alternative from this choice
point. Moreover, this “protection” lasts just as long as it is needed since as soon as
the choice point disappears, all explicitly deallocated environments can be safely
overwritten.

Hence, there is no need to distinguish between the AND-stack from the OR-stack,
calling the single one the stack. Choice point frames are stored in the stack along
with environments, and thus B’s value is an address in the stack.

Going back to our example above, the snapshot of the single stack at the same first
instant looks thus:
\[Environment for a\]
\[Environment for b\]
\[B  Choice point for e\]
\[E  Environment for e\]
and at the same second instant as before, the stack is such that having pushed on
it the choice point for e protects b’s deallocated environment (which may still be
needed by future alternatives given by e’s choice point), looking thus:
\[E  Environment for a\]
\[Deallocated environment for b\]
\[B  Choice point for e\]

Now, the computation can safely recover the state from the choice point for e
indicated by B, in which the saved environment to restore is the one current at
the time of this choice point’s creation—i.e., that (still existing) of b. Having no
more alternative for e after the second one, this choice point is discarded upon
backtracking, (safely) ending the protection. Execution of the last alternative for
e proceeds with a stack looking thus:
\[B \]
\[Environment for a\]
\[Environment for b\]
\[E  Environment for e\]
