\secrel{Компиляция $\mathcal L_0$ запросов}

Согласно операционной семантике $\mathcal L_0$ обработка запроса состоит из
подготовке в решению уравнения с одной стороны. А именно, терм запроса $q$
транлируется в последовательность инструкций, целью которой является построение
экземпляра $q$ на куче из текстового представления $q$. Таким образом, из-за
древовидной структуры терма и множествественных вхождениях переменных,
необходимо, чтобы при обработке части терма где-то временно сохранялись части
терма, которые еще предстоит обработать, или переменные которые могут
встретиться еще раз далее по ходу работы. Для этой цели виртуальная машина
$\mathcal M_0$ наделена достаточным количеством (изменяемых)
\termdef{регистров}{\prolog!регистр} $X_1$, $X_2$,\ldots которые используются
для временного хранения данных кучи по мере создания промежуточных термов. Таким
образом, содержимое каждого регистра должно иметь формат ячейки кучи. Эти
изменяемые регистры выделяются для терма по мере доступности, так что (1)
регистр $X_1$ всегда распределяется для охватывающего терма, и (2) тот же
регистр распределяется для всех вхождений определенной переменной.
Например регистры для переменных терма $p(Z,h(Z,W),f(W))$ распределяются
\begin{equation*}
\begin{split}
X_1 &= p(X_2,X_3,X_4)\\
X_2 &= Z\\
X_3 &= h(X_2,X_5)\\
X_4 &= f(X_5)\\
X_5 &= W
\end{split}
\end{equation*}

Это равносильно тому что терм рассматривается как сплющенный конъюктивный набор
уравнений в форме $X_i=X$ или $X_i=f(X_{i_1},..,X_{i_n}), (n \geqslant 0)$ ,
где члены $X_i$ различные новые имена переменных. Есть два последствия
распределения регистров: (1) все внешние имена переменных (такие как $Z$ and $W$
в нашем примере) могут быть забыты; и (2) терм запроса может быть
трансформирован в его \termdef{сплющенную форму}{\prolog!сплющенная форма},
т.е. последовательность назначений регистров только в форме 
$X_i=f(X_{i_1},..,X_{i_n})$. Эта форма\ --- то, что контролирует построение
представления терма в куче. Таким образом, чтобы генерация кода слева направо
была хорошо обоснована, необходимо сформировать сплющенный терм запроса, так
чтобы гарантировать что \emph{имена регистров не могут использоваться в правых
частях присвоений (например как субтерм) до их инициализации}\note{if it has one
(viz., being the lefthand side)}. Например сплющенная форма терма запроса
$p(Z,h(Z,W),f(W))$ это последовательность
$X_3=h(X_2,X_5)$, $X_4=f(X_5)$, $X_1=p(X_2,X_3,X_4)$\note{исключена привязка
переменных на регистры $X_2$, $X_5$}.

Сканируя сплющенный терм запроса слева направо, каждый компонент в форме
$X_i=f(X_{i_1},..,X_{i_n})$ токенизируется в последовательность
$X_i=f/n,X_{i_1},..,X_{i_n}$ такую что после регистра ассоциированного с n-арным
функтором идет последовательность $n$ имен регистров. Так что в потоке таких
токенов полученных в результате токенизации полного сплющенного терма,
существует три вида элементов для обработки:
\begin{enumerate}
  \item 
регистр ассоциированный со структурным функтором;
  \item 
регистр-аргумент который не был нигде равнее встречен в потоке;
  \item 
регистр-аргумент который уже был упомянут в потоке.
\end{enumerate}

Из такого потока легко получить представление кучи используя метод управляемого
потоком токенов синтеза. Для реализации этого нужно выполнить сооответствующие
действия для каждого типа токенов:
\begin{enumerate}
  \item 
создать на куче новую ячейку STR (и примыкающий функтор) и скопировать эту
ячейку в указаный регистр;
  \item 
создать на куче новую ячейку REF содержащую собственный адрес, и скопировать
ее в указанный регистр;
  \item 
создать на куче новую ячейку и копировать в нее значение регистра.
\end{enumerate}

Each of these three actions specifies the effect of respective instructions of
the machine $M_0$ that we note:
\begin{enumerate}
  \item 
put structure f n Xi
  \item 
set variable Xi
  \item 
set value Xi
\end{enumerate}
respectively.

From the preceding considerations, it has become clear that the heap is implicitly
used as a stack for building terms. Namely, term parts being constructed are
incrementally piled on top of what already exists in the heap. Therefore, it is
necessary to keep the address of the next free cell in the heap somewhere, precisely
as for a stack.\note{As a matter of fact, in [War83], Warren refers to the heap
as the \emph{global stack}.} Adding to M a global register H containing at all
times the next available address on the heap, these three instructions are given explicitly
in Figure 2.2. For example, given that registers are allocated as above, the
sequence of instructions to build the query term $pZh$
is shown in Figure 2.3.

\paragraph{Exercise 2.1} Verify that the effect of executing the sequence of
instructions shown in Figure 2.3 (starting with H 
 )
does indeed yield a correct heap representation
for the term $pZh$ the one shown earlier as Figure 2.1, in
fact.