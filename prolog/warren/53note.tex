\secrel{A note on \var{set} instructions}

Defining the simplistic language L
has allowed us to introduce, independently
of other Prolog considerations, all WAM instructions dealing with unification.
Strictly speaking, the set instructions we have defined are not part of the WAM
as described in [War83] or in [War88]. There, one will find that the corresponding
unify instructions are systematically used where we use set instructions.
The reason is, as the reader may have noticed, that indeed this is possible provided
that the put structure and put list instructions set mode to write.
Then, clearly, all set instructions are equivalent to unify instructions in write
mode. We chose to keep these separate as using set instructions after put instructions
is more efficient (it saves mode setting and testing) and makes the code
more perspicuous. Moreover, these instructions are more natural, easier to explain
and motivate as the data building phase of unification before matching work
comes into play.

\paragraph{Figure 5.6: Anonymous variable instructions}

\paragraph{Figure 5.7: Instructions for fact pgXY}

Incidentally, these instructions together with their unify homologues, make “onthe-fly”
copying part of unification, resulting in improved space and time consumption,
as opposed to the more naıve systematic copying of rules before using
them.