\secrel{Optimizing the Design}\label{warren5}\secdown

Now that the reader is hopefully convinced that the design we have reached forms
an adequate target language and architecture for compiling pure Prolog, we can
begin transforming it in order to recover Warren’s machine as an ultimate design.
Therefore, since all optimizations considered here are part of the definitive design,
we shall now refer to the abstract machine gradually being elaborated as the WAM.
In the process, we shall abide by a few principles of design pervasively motivating
all the conception features of the WAM. We will repeatedly invoke these principles
in design decisions as we progress toward the full WAM engine, as more evidence
justifying them accrues.

\textsc{WAM PRINCIPLE 1} Heap space is to be used as sparingly as possible, as
terms built on the heap turn out to be relatively persistent.

\textsc{WAM PRINCIPLE 2} Registers must be allocated in such a way as to avoid
unnecessary data movement, and minimize code size as well.

\textsc{WAM PRINCIPLE 3} Particular situations that occur very often, even
though correctly handled by general-case instructions, are to be accommodated by special
ones if space and/or time may be saved thanks to their specificity.

In the light of WAM Principles 1, 2, and 3, we may now improve on M.

\paragraph{Figure 5.1: Better heap representation for term pZ hZW f W}
\begin{verbatim}

h
 REF 
 REF
 f 
 REF
 p
 REF 
 STR
 STR
\end{verbatim}

\secrel{Heap representation}

As many readers of [AK90] did, this reader may have wondered about the necessity
of the extra level of indirection systematically introduced in the heap by
an STR cell for each functor symbol. In particular, Fernando Pereira [Per90]
suggested that instead of that shown in Figure 2.1 on Page 11, a more economical
heap representation for pZ
hZW
f W
ought to be that of Figure 5.1, where
reference to the term from elsewhere must be from a store (or register) cell of the
form h STR
 i. In other words, there is actually no need to allot a systematic STR
cell before each functor cell.

As it turns out, only one tiny modification of one instruction is needed in order to
accommodate this more compact representation. Namely, the put structure
instruction is simplified to:
\begin{verbatim}
put structure f n
Xi  HEAP[H]  f n
Xi  h STR
H i
H  H  
\end{verbatim}

Clearly, this is not only in complete congruence with WAM Principle 1, but it also
eliminates unnecessary levels of indirection and hence speeds up dereferencing.

The main reason for our not having used this better heap representation in Section
2.1 was essentially didactic, wishing to avoid having to mention references
from outside the heap (e.g., from registers) before due time. In addition, we did
not bother bringing up this optimization in [AK90] as we are doing here, as we
had not realized that so little was in fact needed to incorporate it.\note{After dire reflection seeded by discussions with Fernando Pereira, we eventually realized that
this optimization was indeed cheap—a fact that had escaped our attention. We are grateful to him
for pointing this out. However, he himself warns [Per90]:

“Now, this representation (which, I believe, is the one used by Quintus, SICStus
Prolog, etc.) has indeed some disadvantages:

1. If there aren’t enough tags to distinguish functor cells from the other
cells, garbage collection becomes trickier, because a pointed-to value does not in
general identify its own type (only the pointer does).

2. If you want to use [the Huet-Fages] circular term unification algorithm,
redirecting pointers becomes messy, for the [same] reason...
In fact, what [the term representation in Section 2.1 is] doing is enforcing a convention
that makes every functor application tagged as such by the appearance of a
STR cell just before the functor word.”}


\secrel{5.2 Constants, lists, and anonymous variables . . . . . . . . . . . . .
47}

\secrel{A note on \var{set} instructions}

Defining the simplistic language L
has allowed us to introduce, independently
of other Prolog considerations, all WAM instructions dealing with unification.
Strictly speaking, the set instructions we have defined are not part of the WAM
as described in [War83] or in [War88]. There, one will find that the corresponding
unify instructions are systematically used where we use set instructions.
The reason is, as the reader may have noticed, that indeed this is possible provided
that the put structure and put list instructions set mode to write.
Then, clearly, all set instructions are equivalent to unify instructions in write
mode. We chose to keep these separate as using set instructions after put instructions
is more efficient (it saves mode setting and testing) and makes the code
more perspicuous. Moreover, these instructions are more natural, easier to explain
and motivate as the data building phase of unification before matching work
comes into play.

\paragraph{Figure 5.6: Anonymous variable instructions}

\paragraph{Figure 5.7: Instructions for fact pgXY}

Incidentally, these instructions together with their unify homologues, make “onthe-fly”
copying part of unification, resulting in improved space and time consumption,
as opposed to the more naıve systematic copying of rules before using
them.
\secrel{5.4 Register allocation . . . . . . . . . . . . . . . . . . . . . . . .
. 54}

\secrel{5.5 Last call optimization . . . . . . . . . . . . . . . . . . . . . . .
. 56}

\secrel{5.6 Chain rules . . . . . . . . . . . . . . . . . . . . . . . . . . . .
. 57}

\secrel{5.7 Environment trimming . . . . . . . . . . . . . . . . . . . . . . .
58}

\secrel{5.8 Stack variables . . . . . . . . . . . . . . . . . . . . . . . . . .
. 60}\secdown

\secrel{5.8.1 Variable binding and memory layout . . . . . . . . . . . . 62}

\secrel{5.8.2 Unsafe variables . . . . . . . . . . . . . . . . . . . . . . 64}

\secrel{5.8.3 Nested stack references . . . . . . . . . . . . . . . . . . . 67}

\secup

\secrel{5.9 Variable classification revisited . . . . . . . . . . . . . . . . .
. . 69}

\secrel{5.10 Indexing . . . . . . . . . . . . . . . . . . . . . . . . . . . . .
. . 75}

\secrel{5.11 Cut . . . . . . . . . . . . . . . . . . . . . . . . . . . . . . . .
. . 83}


\secup
