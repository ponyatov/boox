\secrel{Представление термов}\label{warren21}

Для начала давайте определим внутреннее представление термов в языке $\mathcal
L_0$. Мы будем использовать глобальный блок хранения данных в форме адресуемой
\termdef{кучи}{\prolog!куча} который мы назовем \var{HEAP}: массив ячеек данных.
Адресом ячейки в куче является индекс элемента массива \var{HEAP}.

Для представления произвольных термов в \var{HEAP} будет достаточно закодировать
переменные и ``структуры'' имеющие форму $f(@_1,..,@_n)$ где $f/n$ функтор и
$@_i$ ссылки на адреса кучи для $n$ субтермов. Таким образом существует два вида
данных, хранимых в куче: переменные и структуры термов. Явно заданные
\termdef{тэги}{\prolog|тэг}, появляющиеся как часть внутреннего формата ячеек
кучи, будут использоваться для различения между этими двумя типами
данных.\note{интересно рассмотреть расширение тэгирования для реализации
ООП и динамического контроля типов}

Переменная будет индентифицироваться как указатель, и представляться как одна
ячейка кучи, так что мы должны говорить о \termdef{ячейках
переменных}{\prolog!ячейка переменной}. Ячейка переменной отмечается тэгом
\class{REF}, и обозначается как $<REF,k>$ где $k$ адрес хранения, т.е. индекс в
\var{HEAP}. Этот механизм предназначен для облегчения связывания переменных
через установление ссылки на терм в переменной, которая связывается с этим
термом. Таким образом при связывании переменной адресная часть
\class{REF}-ячейки получает значение соответствующего адреса терма. Соглашение о
представлении \termdef{несвязанной переменной}{\prolog!несвязанная переменная}\
--- адресная часть \class{REF}-ячейки указывает на саму переменную. Таким
образом \emph{несвязанные переменные представляются \class{REF}-ячейкой со
ссылкой на саму себя}.

Структуры\ --- термы не являющиеся переменными. Формат кучи для представления
структуры $f(t_1,..,t_n)$ содержит $n+2$ ячеек кучи. Первые две ячейки не
обязательно смежные. По сути первая их этих двух ячеек выступает в роли
сортированного указателя на вторую ячейку, и в то же время сама выступает как
представление функтора $f/n$.\note{причина использования этой кажущеся странной
косвенной адресации\ --- реализация разделяемых структур (structure sharing)\
--- будет вскоре ясна} Остальные $n$ ячеек предназначены для упорядоченного хранения
ссылок на корни соответствующих $n$ субтермов. 

Детальнее, первая из $n+2$ ячеек представляющих терм $f(t_1,..,t_n)$
форматирована как тэгированная \termdef{структурная ячейка}{\prolog!структурная
ячейка}, которую можно записать как $<STR,k>$, содержит тэг \var{STR} и
указатель $k$ на \termdef{ячейку функтора}{\prolog!ячейка функтора}, храняющую
представление функтора $f/n$. Важно отметить что \emph{непосредственно за
\term{ячейкой функтора} в смежных адресах всегда следуют $n$ \term{структурных
ячеек}, представляющих каждый из $t_i$ субтермов}. Так что если $HEAP[k]=f/n$
то $HEAP[k+1]$ будет ссылаться на первый субтерм $t_1$, а $HEAP[k+n]$ будет
ссылаться на последний субтерм $t_n$.

\fig{\\Фиг. 2.1: Представление кучи для терма
$p(Z,h(Z,W),f(W))$}{prolog/warren/fig21.pdf}{height=0.5\textheight}
\label{warrenfig21}

\begin{tabular}{l l l}
0 & STR & 1 \\
1 & $h/2$ \\
2 & REF & 2\\
3 & REF & 3\\
4 & STR & 5\\
5 & $f/1$\\
6 & REF & 3\\
7 & STR & 8\\
8 & $p/3$\\
9 & REF & 2\\
10 & STR & 1\\
11 & STR & 5\\
\end{tabular}

\bigskip
Например, рассмотрим представление кучи для терма $p(Z,h(Z,W),f(W))$, начальная
ячейка которого находится по адресу 7 (иллюстрация \ref{warrenfig21}).
Отметим что \emph{для каждой} непривязанной переменной существует только одно
вхождение, представленное как \class{REF}-ячейка, в то время как другие ее
вхождения в исходный терм представляются как ссылки на первое вхождение
($Z=HEAP[2]$, $W=HEAP[3]$). Также обратите внимание что за структурными ячейками
по адресам 0, 4 и 7 \emph{сразу} следуют их ячейки функторов, но это не так для
адресов 10 и 11.