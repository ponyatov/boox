\secrel{Heap representation}

As many readers of [AK90] did, this reader may have wondered about the necessity
of the extra level of indirection systematically introduced in the heap by
an STR cell for each functor symbol. In particular, Fernando Pereira [Per90]
suggested that instead of that shown in Figure 2.1 on Page 11, a more economical
heap representation for pZ
hZW
f W
ought to be that of Figure 5.1, where
reference to the term from elsewhere must be from a store (or register) cell of the
form h STR
 i. In other words, there is actually no need to allot a systematic STR
cell before each functor cell.

As it turns out, only one tiny modification of one instruction is needed in order to
accommodate this more compact representation. Namely, the put structure
instruction is simplified to:
\begin{verbatim}
put structure f n
Xi  HEAP[H]  f n
Xi  h STR
H i
H  H  
\end{verbatim}

Clearly, this is not only in complete congruence with WAM Principle 1, but it also
eliminates unnecessary levels of indirection and hence speeds up dereferencing.

The main reason for our not having used this better heap representation in Section
2.1 was essentially didactic, wishing to avoid having to mention references
from outside the heap (e.g., from registers) before due time. In addition, we did
not bother bringing up this optimization in [AK90] as we are doing here, as we
had not realized that so little was in fact needed to incorporate it.\note{After dire reflection seeded by discussions with Fernando Pereira, we eventually realized that
this optimization was indeed cheap—a fact that had escaped our attention. We are grateful to him
for pointing this out. However, he himself warns [Per90]:

“Now, this representation (which, I believe, is the one used by Quintus, SICStus
Prolog, etc.) has indeed some disadvantages:

1. If there aren’t enough tags to distinguish functor cells from the other
cells, garbage collection becomes trickier, because a pointed-to value does not in
general identify its own type (only the pointer does).

2. If you want to use [the Huet-Fages] circular term unification algorithm,
redirecting pointers becomes messy, for the [same] reason...
In fact, what [the term representation in Section 2.1 is] doing is enforcing a convention
that makes every functor application tagged as such by the appearance of a
STR cell just before the functor word.”}

