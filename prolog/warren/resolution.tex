\secrel{3 Flat Resolution 25}\label{warren3}\secdown

We now extend the language L0 into a language L where procedures are no longer
reduced only to facts but may also have bodies. A body defines a procedure as a
conjunctive sequence of atoms. Said otherwise, L is Prolog without backtracking.

An L program is a set of procedure definitions or (definite) clauses, at most
one per predicate name, of the form ‘a :- a0000an0’ where n 0  and the ai’s are
atoms. As before, when n  , the clause is called a fact and written without
the ‘:-’ implication symbol. When n  , the clause is called a rule, the atom a
is called its head, the sequence of atoms a0000an is called its body and atoms
composing this body are called goals. A rule with exactly one body goal is
called a chain (rule). Other rules are called deep rules. L queries are
sequences of goals, of the form ‘?-g0000gk0’ where k 0 . When k  , the query
is called the empty query. As in Prolog, the scope of variables is limited to
the clause or query in which they appear.

Executing a query ‘?-g0000gk0’ in the context of a program made up of a set of
procedure-defining clauses consists of repeated application of leftmost
resolution until the empty query, or failure, is obtained. Leftmost resolution
amounts to unifying the goal g0 with its definition’s head (or failing if none
exists) and, if this succeeds, executing the query resulting from replacing g0
by its definition body, variables in scope bearing the binding side-effects of
unification. Thus, executing a query in L either terminates with success (i.e.,
it simplifies into the empty query), or terminates with failure, or never
terminates. The “result” of an L query whose execution terminates with success
is the (dereferenced) binding of its original variables after termination.

Note that a clause with a non-empty body can be viewed in fact as a conditional
query. That is, it behaves as a query provided that its head successfully
unifies with a predicate definition. Facts merely verify this condition, adding
nothing new to the query but a contingent binding constraint. Thus, as a first
approximation, since an L query (resp., clause body) is a conjunctive sequence
of atoms interpreted as procedure calls with unification as argument passing,
instructions for it may simply be the concatenation of the compiled code of each
goal as an L0 query making it up. As for a clause head, since the semantics
requires that it retrieves arguments by unification as did facts in L0,
instructions for L0’s fact unification are clearly sufficient.

Therefore, M0 unification instructions can be used for L clauses, but with two
measures of caution: one concerning continuation of execution of a goal
sequence, and one meant to avoid conflicting use of argument registers.

\secrel{Facts}

Let us first only consider L facts. Note that L0 is all contained in L.
Therefore, it is natural to expect that the exact same compilation scheme for
facts carries over untouched from L0 to L. This is true up to a wee detail
regarding the proceed instruction. It must be made to continue execution, after
successfully returning from a call to a fact, back to the instruction in the
goal sequence following the call. To do this correctly, we will use another
global register CP, along with P, set to contain the address (in the code area)
of the next instruction to follow up with upon successful return from a call
(i.e., set to P instruction size P at procedure call time). Then, having exited
the called procedure’s code sequence, execution could thus be resumed as
indicated by CP. Thus, for L’s facts, we need to alter the effect of M0’s call
pn to:
cal
\[pn  CP  P  instruction sizeP0\]
\[P  pn0\]
and that of proceed to:
\[P  CP\]
As before, when the procedure pn is not defined, execution fails.
In summary, with the simple foregoing adjustment, L facts are translated exactly
as were L0 facts.
\secrel{3.2 Rules and queries . . . . . . . . . . . . . . . . . . . . . . . . .
. 27}


\secup