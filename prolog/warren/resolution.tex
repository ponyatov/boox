\secrel{3 Flat Resolution 25}\label{warren3}\secdown

We now extend the language L0 into a language L where procedures are no longer
reduced only to facts but may also have bodies. A body defines a procedure as a
conjunctive sequence of atoms. Said otherwise, L is Prolog without backtracking.

An L program is a set of procedure definitions or (definite) clauses, at most
one per predicate name, of the form ‘a :- a0000an0’ where n 0  and the ai’s are
atoms. As before, when n  , the clause is called a fact and written without
the ‘:-’ implication symbol. When n  , the clause is called a rule, the atom a
is called its head, the sequence of atoms a0000an is called its body and atoms
composing this body are called goals. A rule with exactly one body goal is
called a chain (rule). Other rules are called deep rules. L queries are
sequences of goals, of the form ‘?-g0000gk0’ where k 0 . When k  , the query
is called the empty query. As in Prolog, the scope of variables is limited to
the clause or query in which they appear.

Executing a query ‘?-g0000gk0’ in the context of a program made up of a set of
procedure-defining clauses consists of repeated application of leftmost
resolution until the empty query, or failure, is obtained. Leftmost resolution
amounts to unifying the goal g0 with its definition’s head (or failing if none
exists) and, if this succeeds, executing the query resulting from replacing g0
by its definition body, variables in scope bearing the binding side-effects of
unification. Thus, executing a query in L either terminates with success (i.e.,
it simplifies into the empty query), or terminates with failure, or never
terminates. The “result” of an L query whose execution terminates with success
is the (dereferenced) binding of its original variables after termination.

Note that a clause with a non-empty body can be viewed in fact as a conditional
query. That is, it behaves as a query provided that its head successfully
unifies with a predicate definition. Facts merely verify this condition, adding
nothing new to the query but a contingent binding constraint. Thus, as a first
approximation, since an L query (resp., clause body) is a conjunctive sequence
of atoms interpreted as procedure calls with unification as argument passing,
instructions for it may simply be the concatenation of the compiled code of each
goal as an L0 query making it up. As for a clause head, since the semantics
requires that it retrieves arguments by unification as did facts in L0,
instructions for L0’s fact unification are clearly sufficient.

Therefore, M0 unification instructions can be used for L clauses, but with two
measures of caution: one concerning continuation of execution of a goal
sequence, and one meant to avoid conflicting use of argument registers.

\secrel{Facts}

Let us first only consider L facts. Note that L0 is all contained in L.
Therefore, it is natural to expect that the exact same compilation scheme for
facts carries over untouched from L0 to L. This is true up to a wee detail
regarding the proceed instruction. It must be made to continue execution, after
successfully returning from a call to a fact, back to the instruction in the
goal sequence following the call. To do this correctly, we will use another
global register CP, along with P, set to contain the address (in the code area)
of the next instruction to follow up with upon successful return from a call
(i.e., set to P instruction size P at procedure call time). Then, having exited
the called procedure’s code sequence, execution could thus be resumed as
indicated by CP. Thus, for L’s facts, we need to alter the effect of M0’s call
pn to:
cal
\[pn  CP  P  instruction sizeP0\]
\[P  pn0\]
and that of proceed to:
\[P  CP\]
As before, when the procedure pn is not defined, execution fails.
In summary, with the simple foregoing adjustment, L facts are translated exactly
as were L0 facts.
\secrel{Rules and queries}

We now must think about translating rules. A query is a particular case of a rule
in the sense that it is one with no head. It is translated exactly the same way,
but without the instructions for the missing head. The idea is to use L0’s instructions,
treating the head as a fact, and each goal in the body as an L0 query term
in sequence; that is, roughly translate a rule ‘p0000
:- p00000000pn00000’
following the pattern:
\begin{verbatim}
get arguments of p
put arguments of p0
call p0
.
.
.
put arguments of pn
call pn
\end{verbatim}

Here, in addition to ensuring correct continuation of execution, we must arrange
for correct use of argument registers. Indeed, since the same registers are used by
each goal in a query or body sequence to pass its arguments to the procedure it
invokes, variables that occur in many different goals in the scope of the sequence
need to be protected from the side effects of put instructions. For example, consider
the rule ‘pX
Y 0 :- qX
Z0
rZ
Y 00’ If the variables YZ
were allowed
to be accessible only from an argument register, no guarantee could be made that
they still would be after performing the unifications required in executing the body
of p.

Therefore, it is necessary that variables of this kind be saved in an environment
associated with each activation of the procedure they appear in. Variables which
occur in more than one body goal are dubbed permanent as they have to outlive
the procedure call where they first appear. All other variables in a scope that are
not permanent are called temporary. We shall denote a permanent variable as Yi,
and use Xi as before for temporary variables. To determine whether a variable is
permanent or temporary in a rule, the head atom is considered to be part of the
first body goal. This is because get and unify instructions do not load registers
for further processing. Thus, the variable X in the example above is temporary as
it does not occur in more than one goal in the body (i.e., it is not affected by more
than one goal’s put instructions).

Clearly, permanent variables behave like conventional local variables in a procedure.
The situation is therefore quite familiar. As is customary in programming
languages, we protect a procedure’s local variables by maintaining a run-time
stack of procedure activation frames in which to save information needed for the
correct execution of what remains to be done after returning from a procedure call.
We call such a frame an environment frame. We will keep the address of the latest
environment on top of the stack in a global register E.\note{In [War83], this stack is called the local stack to distinguish it from the global stack (see
Footnote 1 at the bottom of Page 13).}

As for continuation of execution, the situation for rules is not as simple as that
for facts. Indeed, since a rule serves to invoke further procedures in its body, the
value of the program continuation register CP, which was saved at the point of
its call, will be overwritten. Therefore, it is necessary to preserve continuation
information by saving the value of CP along with permanent variables.

Let us recapitulate: M is an augmentation of M0 with the addition of a new data
area, along with the heap (HEAP), the code area (CODE), and the push-down list
(PDL). It is called the stack (STACK) and will contain procedure activation frames.
Stack frames are called environments. An environment is pushed onto STACK
upon a (non-fact) procedure entry call, and popped from STACK upon return.
Thus, an allocate/deallocate pair of instructions must bracket the code
generated for a rule in order to create and discard, respectively, such environment
frames on the stack. In addition, deallocate being the ultimate instruction
of the rule, it must connect to the appropriate next instruction as indicated by
the continuation pointer that had been saved upon entry in the environment being
discarded.

Since the size of an environment varies with each procedure in function of its
number of permanent variables, the stack is organized as a linked list through a
continuation environment slot; i.e., a cell in each environment frame bearing the
stack index of the environment previously pushed onto the stack.

To sum up, two new I instructions for M are added to the ones we defined for
I0:

1. allocate

2. deallocate

with effect, respectively:

1. to allocate a new environment on the stack, setting its continuation environment
field to the current value of E, and its continuation point field to that
of CP; and,

2. to remove the environment frame at stack location E from the stack and
proceed, resetting P to the value of its CP field and E to the value of its CE
field.

To have proper effect, an allocate instruction needs to have access to the size
of the current environment in order to increment the value of E by the right stack
offset. The necessary piece of information is a function of the calling clause (i.e.,
the number of permanent variables occurring in the calling clause). Therefore, it
is easily statically available at the time the code for the calling clause is generated.
Now, the problem is to transmit this information to the called procedure that, if
defined as a rule (i.e., starting with an allocate), will need it dynamically,
depending on which clause calls it. A simple solution is to save this offset in the
calling clause’s environment frame from where it can be retrieved by a callee that
needs it. Hence, in M, an additional slot in an environment is set by allocate
to contain the number of permanent variables in the clause in question.

Summing up again, an M stack environment frame contains:

1. the address in the code area of the next instruction to execute upon (successful)
return from the invoked procedure;

2. the stack address of the previous environment to reinstate upon return (i.e.,
where to pop the stack to);

3. the offset of this frame on the stack (the number of permanent variables);
and,

4. as many cells as there are permanent variables in the body of the invoked
procedure (possibly none).

Such an M environment frame pushed on top of the stack looks thus:
\[E CE continuation environment\]

This necessitates giving allocate an explicit argument that is the number of
permanent variables of the rule at hand, such that, in M:
\[allocate N =\]

Similarly, the explicit definition of M’s deallocate is:
\[deallocate  =\]

With this being set up, the general translation scheme into M instructions for an
L rule ‘p0000
:- p00000000pn00000’
with N permanent variables will follow
the pattern:
\[p allocate N\]

For example, for L clause ‘pX
Y 0 :- qX
Z0
rZ
Y 00’, the corresponding
M code is shown in Figure 3.1.

\paragraph{Figure 3.1: M machine code for rule pX
Y 0 :- qX
Z0
rZ
Y 00}
\begin{verbatim}

\end{verbatim}

\paragraph{Exercise 3.1} GiveM code for L facts qa0 b and rb0 c and L query
?-pU0 V , then trace the code shown in Figure 3.1 and verify that the solution produced is
U 
 a0 V 
 c.
C  



\secup