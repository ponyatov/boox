\secly{Реализация машины вывода на \cpp}

В перевод книги Варрена мной\note{\email{dponyatov@gmail.com}} добавлен пример
реализации виртуальной машины вывода на \cpp. Исходные тексты находятся в
каталоге
\href{https://github.com/ponyatov/boox/tree/master/prolog/warren}{\dir{prolog/warren/}}.
Для вставки отдельных частей исходника по ходу книги полные файлы
\file{hpp.hpp}\ и \file{cpp.cpp} разделены на отдельные небольшие фрагменты в
каталогах \dir{hpp/} и \dir{cpp/}. Файл сборки \file{prolog/warren/Makefile}
содержит не только  
\lst{\file{Makefile}: запуск программы и генерация
\file{log.log}}{prolog/warren/mk/exec.mk}{make}
\lst{\file{Makefile}: типовой блок компиляции \term{лексической}
программы}{prolog/warren/mk/lexprog.mk}{make}
\lst{\file{Makefile}: генерация кода
\term{синтаксического анализатора}}{prolog/warren/ypp/ypp.mk}{make}
\lst{\file{Makefile}: генерация кода
\term{лексера}}{prolog/warren/lpp/lpp.mk}{make} но и скрипты склейки файлов
исходников из частей в каталогах \dir{hpp/} \dir{cpp/} \dir{mk/}.

\lst{\file{hpp.hpp}: обертка одиночного
\#include}{prolog/warren/hpp/head.hpp}{C++}
\lst{\file{hpp.hpp}}{prolog/warren/hpp/foot.hpp}{C++}
\lst{\file{hpp.hpp}: типовые \#include}{prolog/warren/hpp/stdinc.hpp}{C++}
\lst{\file{hpp.hpp}: базовый класс для
структур WAM}{prolog/warren/hpp/wam.hpp}{C++}

\lst{\file{cpp.cpp}}{prolog/warren/cpp/main.cpp}{C++}

