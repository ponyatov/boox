\secrel{Факты, правила и запросы}\label{lpn1}\secdown

Это раздел имеет две главных цели:
\begin{enumerate}
  \item 
Дать несколько простых примеров программ на \prolog e. Это будет введением в три
базовых конструкции \prolog а: \term{факт} (fact), \term{правило} (rule) и
\term{запрос} (query). Также мы рассмотрим несколько других тем, таких как роль
логики в \prolog е, и важную идею \term{унификации} c выводом значений
переменных.
  \item 
Начало систематического изучения \prolog а через определение \term{термов},
\term{атомов}, \term{переменных} и других синтаксических конструкций.
\end{enumerate}

\secrel{ 1.1 Some Simple Examples}\secdown 

There are only three basic constructs in Prolog: facts, rules, and queries. A
collection of facts and rules is called a knowledge base (or a database) and
Prolog programming is all about writing knowledge bases. That is, Prolog
programs simply are knowledge bases, collections of facts and rules which
describe some collection of relationships that we find interesting.

So how do we use a Prolog program? By posing queries. That is, by asking
questions about the information stored in the knowledge base.

Now this probably sounds rather strange. It’s certainly not obvious that it has
much to do with programming at all. After all, isn’t programming all about
telling a computer what to do? But as we shall see, the Prolog way of
programming makes a lot of sense, at least for certain tasks; for example, it is
useful in computational linguistics and Artificial Intelligence (AI). But
instead of saying more about Prolog in general terms, let’s jump right in and
start writing some simple knowledge bases; this is not just the best way of
learning Prolog, it’s the only way.

\secrel{   Knowledge Base 1}

Knowledge Base 1 (KB1) is simply a collection of facts. Facts are used to state
 things that are unconditionally true of some situation of interest. For
 example, we can state that Mia, Jody, and Yolanda are women, that Jody plays
 air guitar, and that a party is taking place, using the following five facts:

\begin{verbatim} 
 woman(mia). 
   woman(jody). 
   woman(yolanda). 
   playsAirGuitar(jody). 
   party.
\end{verbatim}
   
This collection of facts is KB1. It is our first example of a Prolog program.
Note that the names mia , jody , and yolanda , the properties woman and
playsAirGuitar , and the proposition party have been written so that the first
letter is in lower-case. This is important; we will see why a little later on.

How can we use KB1? By posing queries. That is, by asking questions about the
information KB1 contains. Here are some examples. We can ask Prolog whether Mia
is a woman by posing the query:

\begin{verbatim} 
   ?-  woman(mia).
\end{verbatim}

Prolog will answer

\begin{verbatim} 
   yes
\end{verbatim}
   
for the obvious reason that this is one of the facts explicitly recorded in KB1.
Incidentally, we don’t type in the ?- . This symbol (or something like it,
depending on the implementation of Prolog you are using) is the prompt symbol
that the Prolog interpreter displays when it is waiting to evaluate a query. We
just type in the actual query (for example woman(mia) ) followed by . (a full
stop). The full stop is important. If you don’t type it, Prolog won’t start
working on the query.

Similarly, we can ask whether Jody plays air guitar by posing the following
query:

\begin{verbatim} 
  ?-  playsAirGuitar(jody).
\end{verbatim}

Prolog will again answer yes, because this is one of the facts in KB1. However,
suppose we ask whether Mia plays air guitar:

\begin{verbatim} 
   ?-  playsAirGuitar(mia).
\end{verbatim}

We will get the answer

\begin{verbatim} 
   no
\end{verbatim}

Why? Well, first of all, this is not a fact in KB1. Moreover, KB1 is extremely
simple, and contains no other information (such as the rules we will learn about
shortly) which might help Prolog try to infer (that is, deduce) whether Mia
plays air guitar. So Prolog correctly concludes that playsAirGuitar(mia) does
not follow from KB1.

Here are two important examples. First, suppose we pose the query:

\begin{verbatim} 
   ?-  playsAirGuitar(vincent).
\end{verbatim}

Again Prolog answers no. Why? Well, this query is about a person (Vincent) that
it has no information about, so it (correctly) concludes that
playsAirGuitar(vincent) cannot be deduced from the information in KB1.

Similarly, suppose we pose the query:

\begin{verbatim} 
   ?-  tatooed(jody).
\end{verbatim}
   
Again Prolog will answer no. Why? Well, this query is about a property (being
tatooed) that it has no information about, so once again it (correctly)
concludes that the query cannot be deduced from the information in KB1.
(Actually, some Prolog implementations will respond to this query with an error
message, telling you that the predicate or procedure tatooed is not defined; we
will soon introduce the notion of predicates.)

Needless to say, we can also make queries concerning propositions. For example,
if we pose the query

\begin{verbatim} 
   ?-  party.
\end{verbatim}

then Prolog will respond

\begin{verbatim} 
   yes
\end{verbatim}

and if we pose the query

\begin{verbatim} 
   ?-  rockConcert.
\end{verbatim}

then Prolog will respond

\begin{verbatim} 
   no
\end{verbatim}

exactly as we would expect.
   
\secrel{   Knowledge Base 2} 
\secrel{   Knowledge Base 3} 
\secrel{   Knowledge Base 4} 
\secrel{   Knowledge Base 5} 
\secup

\secrel{  1.2 Prolog Syntax}\secdown 
\secrel{   Atoms} 
\secrel{   Numbers} 
\secrel{   Variables} 
\secrel{   Complex terms} 
\secup

\secrel{  1.3 Exercises} 

\secrel{  1.4 Practical Session}

 
\secup
