\secrel{ 1.1 Some Simple Examples}\secdown 

There are only three basic constructs in Prolog: facts, rules, and queries. A
collection of facts and rules is called a knowledge base (or a database) and
Prolog programming is all about writing knowledge bases. That is, Prolog
programs simply are knowledge bases, collections of facts and rules which
describe some collection of relationships that we find interesting.

So how do we use a Prolog program? By posing queries. That is, by asking
questions about the information stored in the knowledge base.

Now this probably sounds rather strange. It’s certainly not obvious that it has
much to do with programming at all. After all, isn’t programming all about
telling a computer what to do? But as we shall see, the Prolog way of
programming makes a lot of sense, at least for certain tasks; for example, it is
useful in computational linguistics and Artificial Intelligence (AI). But
instead of saying more about Prolog in general terms, let’s jump right in and
start writing some simple knowledge bases; this is not just the best way of
learning Prolog, it’s the only way.

\secrel{   Knowledge Base 1}

Knowledge Base 1 (KB1) is simply a collection of facts. Facts are used to state
 things that are unconditionally true of some situation of interest. For
 example, we can state that Mia, Jody, and Yolanda are women, that Jody plays
 air guitar, and that a party is taking place, using the following five facts:

\begin{verbatim} 
 woman(mia). 
   woman(jody). 
   woman(yolanda). 
   playsAirGuitar(jody). 
   party.
\end{verbatim}
   
This collection of facts is KB1. It is our first example of a Prolog program.
Note that the names mia , jody , and yolanda , the properties woman and
playsAirGuitar , and the proposition party have been written so that the first
letter is in lower-case. This is important; we will see why a little later on.

How can we use KB1? By posing queries. That is, by asking questions about the
information KB1 contains. Here are some examples. We can ask Prolog whether Mia
is a woman by posing the query:

\begin{verbatim} 
   ?-  woman(mia).
\end{verbatim}

Prolog will answer

\begin{verbatim} 
   yes
\end{verbatim}
   
for the obvious reason that this is one of the facts explicitly recorded in KB1.
Incidentally, we don’t type in the ?- . This symbol (or something like it,
depending on the implementation of Prolog you are using) is the prompt symbol
that the Prolog interpreter displays when it is waiting to evaluate a query. We
just type in the actual query (for example woman(mia) ) followed by . (a full
stop). The full stop is important. If you don’t type it, Prolog won’t start
working on the query.

Similarly, we can ask whether Jody plays air guitar by posing the following
query:

\begin{verbatim} 
  ?-  playsAirGuitar(jody).
\end{verbatim}

Prolog will again answer yes, because this is one of the facts in KB1. However,
suppose we ask whether Mia plays air guitar:

\begin{verbatim} 
   ?-  playsAirGuitar(mia).
\end{verbatim}

We will get the answer

\begin{verbatim} 
   no
\end{verbatim}

Why? Well, first of all, this is not a fact in KB1. Moreover, KB1 is extremely
simple, and contains no other information (such as the rules we will learn about
shortly) which might help Prolog try to infer (that is, deduce) whether Mia
plays air guitar. So Prolog correctly concludes that playsAirGuitar(mia) does
not follow from KB1.

Here are two important examples. First, suppose we pose the query:

\begin{verbatim} 
   ?-  playsAirGuitar(vincent).
\end{verbatim}

Again Prolog answers no. Why? Well, this query is about a person (Vincent) that
it has no information about, so it (correctly) concludes that
playsAirGuitar(vincent) cannot be deduced from the information in KB1.

Similarly, suppose we pose the query:

\begin{verbatim} 
   ?-  tatooed(jody).
\end{verbatim}
   
Again Prolog will answer no. Why? Well, this query is about a property (being
tatooed) that it has no information about, so once again it (correctly)
concludes that the query cannot be deduced from the information in KB1.
(Actually, some Prolog implementations will respond to this query with an error
message, telling you that the predicate or procedure tatooed is not defined; we
will soon introduce the notion of predicates.)

Needless to say, we can also make queries concerning propositions. For example,
if we pose the query

\begin{verbatim} 
   ?-  party.
\end{verbatim}

then Prolog will respond

\begin{verbatim} 
   yes
\end{verbatim}

and if we pose the query

\begin{verbatim} 
   ?-  rockConcert.
\end{verbatim}

then Prolog will respond

\begin{verbatim} 
   no
\end{verbatim}

exactly as we would expect.
   
\secrel{   Knowledge Base 2}

Here is KB2, our second knowledge base:
\begin{verbatim} 
   happy(yolanda). 
   listens2Music(mia). 
   listens2Music(yolanda):-  happy(yolanda). 
   playsAirGuitar(mia):-  listens2Music(mia). 
   playsAirGuitar(yolanda):-  listens2Music(yolanda).
\end{verbatim}
There are two facts in KB2, listens2Music(mia) and happy(yolanda) . The last
three items it contains are rules.

Rules state information that is conditionally true of the situation of interest.
For example, the first rule says that Yolanda listens to music if she is happy,
and the last rule says that Yolanda plays air guitar if she listens to music.
More generally, the :- should be read as “if”, or “is implied by”. The part on
the left hand side of the :- is called the head of the rule, the part on the
right hand side is called the body. So in general rules say: if the body of the
rule is true, then the head of the rule is true too. And now for the key point:

If a knowledge base contains a rule head  :-  body, and Prolog knows that body
follows from the information in the knowledge base, then Prolog can infer head.

This fundamental deduction step is called \term{modus ponens}.

Let’s consider an example. Suppose we ask whether Mia plays air guitar:

\begin{verbatim} 
   ?-  playsAirGuitar(mia).
\end{verbatim}
Prolog will respond yes. Why? Well, although it can’t find playsAirGuitar(mia)
as a fact explicitly recorded in KB2, it can find the rule

\begin{verbatim} 
   playsAirGuitar(mia):-  listens2Music(mia).
\end{verbatim}
   
Moreover, KB2 also contains the fact listens2Music(mia) . Hence Prolog can use
the rule of modus ponens to deduce that playsAirGuitar(mia) .

Our next example shows that Prolog can chain together uses of modus ponens.
Suppose we ask:

\begin{verbatim} 
   ?-  playsAirGuitar(yolanda).
\end{verbatim}
Prolog would respond yes. Why? Well, first of all, by using the fact
happy(yolanda) and the rule

\begin{verbatim} 
   listens2Music(yolanda):-  happy(yolanda).
\end{verbatim}
Prolog can deduce the new fact listens2Music(yolanda) . This new fact is not
explicitly recorded in the knowledge base — it is only implicitly present (it is
inferred knowledge). Nonetheless, Prolog can then use it just like an explicitly
recorded fact. In particular, from this inferred fact and the rule

\begin{verbatim} 
   playsAirGuitar(yolanda):-  listens2Music(yolanda).
\end{verbatim}
   
it can deduce playsAirGuitar(yolanda) , which is what we asked it. Summing up:
   any fact produced by an application of modus ponens can be used as input to
   further rules. By chaining together applications of modus ponens in this way,
   Prolog is able to retrieve information that logically follows from the rules
   and facts recorded in the knowledge base.
   
The facts and rules contained in a knowledge base are called clauses. Thus KB2
contains five clauses, namely three rules and two facts. Another way of looking
at KB2 is to say that it consists of three predicates (or procedures). The three
predicates are:

\begin{verbatim} 
   listens2Music 
   happy 
   playsAirGuitar
\end{verbatim}
The happy predicate is defined using a single clause (a fact). The listens2Music
and playsAirGuitar predicates are each defined using two clauses (in one case,
two rules, and in the other case, one rule and one fact). It is a good idea to
think about Prolog programs in terms of the predicates they contain. In essence,
the predicates are the concepts we find important, and the various clauses we
write down concerning them are our attempts to pin down what they mean and how
they are inter-related.

One final remark. We can view a fact as a rule with an empty body. That is, we
can think of facts as conditionals that do not have any antecedent conditions,
or degenerate rules.   
 
\secrel{   Knowledge Base 3} 
\secrel{   Knowledge Base 4} 
\secrel{   Knowledge Base 5} 
\secup
