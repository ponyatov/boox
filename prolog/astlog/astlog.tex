\secrel{ASTLOG: Язык для анализа синтаксических деревьев}\secdown
\cp{http://www.cs.nyu.edu/~lharris/papers/crew.pdf}
\copyright\ Roger F. Crew \email{rfc@microsoft.com}\\
Microsoft Research
Microsoft Corporation
Redmond, WA 98052

\secly{Abstract}

We desired a facility for locating/analyzing syntactic artifacts in abstract
syntax trees of \ci/\cpp\ programs, similar to the facility \prog{grep} or
\prog{awk} provides for locating artifacts at the lexical level. \prolog, with
its implicit pattern-matching and backtracking capabilities, is a natural choice
for such an application. We have developed a \prolog\ variant that avoids the
overhead of translating the source syntactic structures into the form of a
\prolog\ database; this is crucial to obtaining acceptable performance on large
programs. An interpreter for this language has been implemented and used find
various kinds of syntactic bugs and other questionable constructs in real
programs like \prog{Microsoft SQL server} (450Klines) and \prog{Microsoft Word}
(2Mlines) in time comparable to the runtime of the actual compiler.

The model in which terms are matched against an implicit current object, rather
than simply proven against a database of facts, leads to a distinct ``inside-out
functional" programming style that is quite unlike typical \prolog, but one that
is, in fact, well-suited to the examination of trees. Also, various second-order
\prolog\ set-predicates may be implemented via manipulation of the current
object, thus retaining an important feature without entailing that the database
be dynamically extensible as the usual implementation does.


\secup