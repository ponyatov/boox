\secly{Introduction}
\subsecly{Who This Course Is For}

Anyone who:
\begin{itemize}[nosep]
  \item 
knows swi-Prolog reasonably well
  \item 
and wants to effectively generate or parse lists.
\end{itemize}
The second item goes far beyond the usual task of parsing text most programmers
associate with DCG's. We'll convert a tree to a list in this tutorial. A DCG
could convert a 2D array into a sparse array, or look for patterns in a data
stream.

But, we traditionally associate DCG's with parsing text. So we'll give you some
tools for parsing text as well.

\fig{}{prolog/dcg/dcguses.png}{height=0.3\textheight}

\subsecly{Getting The Most From This Course}

To get the most from this course, you'll need to
\begin{itemize}[nosep]
  \item 
Have a working swi-Prolog install
  \item 
Understand basic Prolog be able to use SWI-Prolog's environment
  \item 
Read the text
  \item 
Try each example program. Experiment!
  \item 
A collection of worked exercises and examples is on github
  \item 
Do the exercises
\end{itemize}

\noindent
Different people will have different backgrounds and learning styles.
Whatever works for you works.

\subsecly{Other resources}

\href{http://rowa.giso.de/languages/toki-pona/dcg/index.php}{Another DCG
tutorial}

\subsecly{Getting Stuck}

If you have questions and \emph{reasonable effort} doesn't answer them, drop me
email at aogborn (somechar) uh.edu. Please, if you're taking a beginning Prolog
course, ask your instructor. Questions about family trees will be ignored. But
if you're working on a real DCG related problem, feel free.

Asking on \#\#Prolog on freenode.net IRC is also a good way to get answers.
