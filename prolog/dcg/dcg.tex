\secrel{Using Definite Clause Grammars in SWI-Prolog}\secdown
\cp{\url{http://www.pathwayslms.com/swipltuts/dcg/}}
\copyright\ Anne Ogborn \email{aogborn@uh.edu}
\bigskip

Thanks to Markus Triska. Large sections of this tutorial are taken directly from
his tutorial, which is used by permission here.

\secly{Introduction}
\subsecly{Who This Course Is For}

Anyone who:
\begin{itemize}[nosep]
  \item 
knows swi-Prolog reasonably well
  \item 
and wants to effectively generate or parse lists.
\end{itemize}
The second item goes far beyond the usual task of parsing text most programmers
associate with DCG's. We'll convert a tree to a list in this tutorial. A DCG
could convert a 2D array into a sparse array, or look for patterns in a data
stream.

But, we traditionally associate DCG's with parsing text. So we'll give you some
tools for parsing text as well.

\fig{}{prolog/dcg/dcguses.png}{height=0.3\textheight}

\subsecly{Getting The Most From This Course}

To get the most from this course, you'll need to
\begin{itemize}[nosep]
  \item 
Have a working swi-Prolog install
  \item 
Understand basic Prolog be able to use SWI-Prolog's environment
  \item 
Read the text
  \item 
Try each example program. Experiment!
  \item 
A collection of worked exercises and examples is on github
  \item 
Do the exercises
\end{itemize}

\noindent
Different people will have different backgrounds and learning styles.
Whatever works for you works.

\subsecly{Other resources}

\href{http://rowa.giso.de/languages/toki-pona/dcg/index.php}{Another DCG
tutorial}

\subsecly{Getting Stuck}

If you have questions and \emph{reasonable effort} doesn't answer them, drop me
email at aogborn (somechar) uh.edu. Please, if you're taking a beginning Prolog
course, ask your instructor. Questions about family trees will be ignored. But
if you're working on a real DCG related problem, feel free.

Asking on \#\#Prolog on freenode.net IRC is also a good way to get answers.

\secrel{1 Definite Clause Grammars}\secdown

A Prolog \termdef{definite clause grammar}{\prolog!definite clause grammar}
(\termdef{DCG}{\prolog!DCG}) \emph{describes a Prolog list}.

Operationally, DCGs can be used to parse and generate lists.

\secrel{1 DCG rules}

A DCG is defined by DCG rules. A DCG rule has the form:
\begin{lstlisting}[language=prolog]
head --> body.
\end{lstlisting}
Analogous to normal Prolog rules with:

\begin{lstlisting}[language=prolog]
head :- body.
\end{lstlisting}
A rule's head is a (non variable) Prolog term.

A rule's body is a sequence of terminals and nonterminals, separated by commas.

A terminal is a Prolog list, which stands for the elements it contains.

\begin{lstlisting}[language=prolog]
some_terminals -->
     [this, is, a, teminal],
     [so, is, this],
     "code strings are also lists, so this too is a terminal".
\end{lstlisting}
A nonterminal refers to a DCG rule or other language construct, which stand for
the elements they themselves describe.

Declaratively, we can read the comma as "and then" in DCGs.

\fig{}{prolog/dcg/syntaxofdcg.png}{width=0.7\textwidth}

As an example, let us describe lists that only contain the atom 'a'. We shall
use the nonterminal as//0 to refer to such lists:

\begin{lstlisting}[language=prolog]
   as --> [].
   as --> [a], as.
\end{lstlisting}
The first rule says: The empty list is such a list. The second rule says: A list
containing the atom 'a' and then only atoms 'a' is also such a list.

To execute a grammar rule, we use Prolog's built-in phrase/2 predicate. The
first argument is a DCG body. phrase(Body, Ls) is true iff Body describes the
list Ls.

The most general query asks for all solutions:

\begin{lstlisting}[language=prolog]
  ?- phrase(as, Ls).
  Ls = [] ;
  Ls = [a] ;
  Ls = [a, a] ;
  Ls = [a, a, a] ;
  Ls = [a, a, a, a] ;
  etc.
\end{lstlisting}
Examples of more specific queries and the system's answers:

\begin{lstlisting}[language=prolog]
  ?- phrase(as, [a,a,a]).
  true.

  ?- phrase(as, [b,c,d]).
  false.

  ?- phrase(as, [a,X,a]).
  X = a.
\end{lstlisting}

Exercises:
1) run 1\_1. Add another DCG that creates an alternating series of a's and b's,
so your output should look like:
\begin{lstlisting}[language=prolog]
  Ls = [] ;
  Ls = [a] ;
  Ls = [a, b] ;
  Ls = [a, b, a] ;
  Ls = [a, b, a, b] ;
  etc.
\end{lstlisting}
2) Try the queries above in 'examples of more specific queries'

\secrel{2 More DCG Syntax}
\secrel{3 Capturing Input}
\secrel{4 Variables in Body}

\secup
\secrel{2 Relating Trees To Lists}\secdown

\secup
\secrel{3 Left Recursion}\secdown
\secup

\secrel{4 Right-hand Context Notation}
\secrel{5 Implicitly Passing States Around}
\secrel{6 Parsing From Files}

\secrel{7 Implementation}
\secrel{8 Error Handling}\secdown
\secrel{1 Resynching The Parser}
\secrel{2 Printing Line Numbers}
\secup
\secrel{9 A Few Practical Hints}\secdown
\secrel{1 \file{basics.pl}}
\secrel{2 Lexical Issues}
\secrel{Regular Expressions}
\secup

\secly{Conclusion}

In conclusion, I'd remind you - if you're working with lists, DCG's can definitely make your life easier. They're not just for parsing any more!

Thanks for taking this tutorial. If I can improve anything please email me at aogborn (hat) uh.edu.

If you make something beautiful, drop us a link.

Thanks,

This tutorial is based on a tutorial by Markus Triska, so a special nod to him.

Ulrich Neumerkel, Richard O'Keefe, Carlo Capelli, and Paulo Moura patiently explained many points on the swipl email list.

Michael Richter applied his thorough critical eye to the text.

Props to the Sanskrit grammarian Paañini, who first formalized grammar.

\fig{}{prolog/dcg/panini.jpg}{width=0.3\textwidth}

Annie

\secup