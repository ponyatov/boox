\secrel{Using Definite Clause Grammars in SWI-Prolog}\secdown
\cp{\url{http://www.pathwayslms.com/swipltuts/dcg/}}
\copyright\ Anne Ogborn \email{aogborn@uh.edu}
\bigskip

Thanks to Markus Triska. Large sections of this tutorial are taken directly from
his tutorial, which is used by permission here.

\secrel{1 Введение 3}\secdown

В 1983 году Дэвид Варрэн разработал абстрактную машину для реализации языка
\prolog, содержащую специальную архитектуру памяти и набор инструкций
\cite{War83}. Эта разработка стала известка как Warren Abstract Machine (WAM)
и стала стандартом де-факто для реализаций компиляторов \prolog а. В
\cite{War83} Варрэн описан WAM в минималистичном стиле, который слишком сложен
для понимания неподготовленным читателем, даже заранее знакомым в операциями
\prolog а. Слишком многое было несказанным, и very little is justified in clear
terms\note{David H. D. Warren поделился в частной беседе что он ``чувствовал
что WAM важна, но к деталям ее реализации вряд ли будет широкий интерес, поэтому
он использовал стиль личных заметок''}. Это привело к очень скудному количеству
поклонников WAM, которые могли был похвастаться пониманием деталей ее работы.
Обычно это были реализаторы \prolog а, которые решили уделить необходимое время
для обучения через делание и кропотливого достижения просветления.

\secrel{1.1 Существующая литература 3}

 

\secrel{1.2 Этот учебник 5 }\secdown

\secrel{1.2.1 Disclaimer and motivation . . . . . . . . . . . . . . . . . 5}

\secrel{1.2.2 Organization of presentation . . . . . . . . . . . . . . . . 6}

\secup
\secup
\secrel{1 Definite Clause Grammars}\secdown

A Prolog \termdef{definite clause grammar}{\prolog!definite clause grammar}
(\termdef{DCG}{\prolog!DCG}) \emph{describes a Prolog list}.

Operationally, DCGs can be used to parse and generate lists.

\secrel{1 DCG rules}

A DCG is defined by DCG rules. A DCG rule has the form:
\begin{lstlisting}[language=prolog]
head --> body.
\end{lstlisting}
Analogous to normal Prolog rules with:

\begin{lstlisting}[language=prolog]
head :- body.
\end{lstlisting}
A rule's head is a (non variable) Prolog term.

A rule's body is a sequence of terminals and nonterminals, separated by commas.

A terminal is a Prolog list, which stands for the elements it contains.

\begin{lstlisting}[language=prolog]
some_terminals -->
     [this, is, a, teminal],
     [so, is, this],
     "code strings are also lists, so this too is a terminal".
\end{lstlisting}
A nonterminal refers to a DCG rule or other language construct, which stand for
the elements they themselves describe.

Declaratively, we can read the comma as "and then" in DCGs.

\fig{}{prolog/dcg/syntaxofdcg.png}{width=0.7\textwidth}

As an example, let us describe lists that only contain the atom 'a'. We shall
use the nonterminal as//0 to refer to such lists:

\begin{lstlisting}[language=prolog]
   as --> [].
   as --> [a], as.
\end{lstlisting}
The first rule says: The empty list is such a list. The second rule says: A list
containing the atom 'a' and then only atoms 'a' is also such a list.

To execute a grammar rule, we use Prolog's built-in phrase/2 predicate. The
first argument is a DCG body. phrase(Body, Ls) is true iff Body describes the
list Ls.

The most general query asks for all solutions:

\begin{lstlisting}[language=prolog]
  ?- phrase(as, Ls).
  Ls = [] ;
  Ls = [a] ;
  Ls = [a, a] ;
  Ls = [a, a, a] ;
  Ls = [a, a, a, a] ;
  etc.
\end{lstlisting}
Examples of more specific queries and the system's answers:

\begin{lstlisting}[language=prolog]
  ?- phrase(as, [a,a,a]).
  true.

  ?- phrase(as, [b,c,d]).
  false.

  ?- phrase(as, [a,X,a]).
  X = a.
\end{lstlisting}

Exercises:
1) run 1\_1. Add another DCG that creates an alternating series of a's and b's,
so your output should look like:
\begin{lstlisting}[language=prolog]
  Ls = [] ;
  Ls = [a] ;
  Ls = [a, b] ;
  Ls = [a, b, a] ;
  Ls = [a, b, a, b] ;
  etc.
\end{lstlisting}
2) Try the queries above in 'examples of more specific queries'

\secrel{2 More DCG Syntax}
\secrel{3 Capturing Input}
\secrel{4 Variables in Body}

\secup
\secrel{2 Relating Trees To Lists}\secdown

\secup
\secrel{3 Left Recursion}\secdown
\secup

\secrel{4 Right-hand Context Notation}
\secrel{5 Implicitly Passing States Around}
\secrel{6 Parsing From Files}

\secrel{7 Implementation}
\secrel{8 Error Handling}\secdown
\secrel{1 Resynching The Parser}
\secrel{2 Printing Line Numbers}
\secup
\secrel{9 A Few Practical Hints}\secdown
\secrel{1 \file{basics.pl}}
\secrel{2 Lexical Issues}
\secrel{Regular Expressions}
\secup

\secly{Conclusion}

In conclusion, I'd remind you - if you're working with lists, DCG's can
definitely make your life easier. They're not just for parsing any more!

Thanks for taking this tutorial. If I can improve anything please email me at
\email{aogborn@uh.edu}.

If you make something beautiful, drop us a link.

Thanks,

This tutorial is based on a tutorial by \emph{Markus Triska}, so a special nod
to him.

\emph{Ulrich Neumerkel}, \emph{Richard O'Keefe}, \emph{Carlo Capelli}, and
\emph{Paulo Moura} patiently explained many points on the swipl email list.

\emph{Michael Richter} applied his thorough critical eye to the text.

Props to the Sanskrit grammarian \emph{Paañini}, who first formalized grammar.

\fig{}{prolog/dcg/panini.jpg}{width=0.3\textwidth}

Annie

\secup