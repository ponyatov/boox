\secrel{Настройка редактора/IDE \prog{\vim}}\label{vim}\secdown

При использовании редактора/IDE \prog{\vim}\ удобно настроить сочетания клавиш и
подсветку синтаксиса языков, которые вы использете так, как вам удобно.

\secrel{для вашего собственного скриптового языка}

Через какое-то время практики FSP у вас выработается один диалект скриптов для
всех программ, соответсвующий именно вашим вкусам в синтаксисе, и в этом случае
его нужно будет описать только в файлах ~/.vim/(ftdetect|syntax).vim, и
привязать их к расширениям через dot-файлы \vim\ в вашем домашнем
каталоге:\bigskip

\begin{tabular}{l l l}
filetype.vim & \vim & привязка расширений файлов (.src .log) к настройкам \vim
\\
syntax.vim & \vim & синтаксическая подсветка для скриптов \\
~/.vimrc & \linux & настройки для пользователя \\
~/vimrc & \win &\\
~/.vim/ftdetect/src.vim & \linux & привязка команд к расширению .src \\
~/vimfiles/ftdetect/src.vim & \win & \\
~/.vim/syntax/src.vim & \linux & синтаксис к расширению .src \\
~/vimfiles/syntax/src.vim & \win &\\
\end{tabular}

\secup