\secrel{\prog{skelex}: скелет программы в свободном
синтаксисе}\label{skelex}\secdown

В этом разделе описана общая структура любого проекта, использующего принципы
\term{программирования в свободном синтаксисе}, в виде примера определения
синтаксиса и семантики языка \bi.

Материал дублирует другие разделы, но может быть использован как вариант
\emph{минимизированного} языкового ядра FSP-проекта: нет комментариев, лишних
классов, подробного описания работы ядра и т.п., \emph{только краткие пояснения
и минимальный код}.

\secly{Структура проекта}

\lstx{Создание проекта}{skelex/mkproject.rc}

\begin{tabular}{l l l}
src.src & \bi & текст программы в свободном синтаксисе\\
log.log & \bi & лог интерпретатора \\
ypp.ypp & \prog{bison} & парсер синтаксиса \\
lpp.lpp & \prog{flex} & лексер \\
hpp.hpp & \cpp & хедеры \\
cpp.cpp & \cpp & ядро интерпретатора \\
Makefile & \prog{make} & скрипты сборки проекта \\
.gitignore & \prog{git} & маски файлов, не попадающие в git-проект\\
bat.bat & \win & helper запуска \vim \\
\end{tabular}

\lstx{\file{.gitignore}}{skelex/git.ignore}
\lstx{\file{bat.bat}}{skelex/bat.bat}

\subsecly{\file{Makefile}}
\lst{\file{Makefile}}{skelex/skelex.mk}{make}

\begin{description}
\item[\var{MODULE}] имя программного модуля, в примере получается
автоматически из имени каталога проекта; при компиляции интерпретатора
добавляется как глобальная константа, и может быть использована в скриптах. 
\item[\var{TAIL}] \verb$= -n7|-n17|<none>$ при успешном выполнении
интерпретатора выводятся последние \verb|$(TAIL)| строк лога, при отладке
скриптов удобно добавлять \emph{в конец программы} вывод отладочной информации.
Конкретное значение параметра команды \prog{tail} выбирается в зависимости от
настроек вашей IDE, для \prog{eclipse} на старом 15"\ мониторе мне удобен
\verb|TAIL=-n7|, для \vim\ и командной строки можно увеличить до
\verb|TAIL=-n17|.
\item[\var{CURDIR}] полный путь для текущего каталога
\item[\var{\$(notdir \ldots)}] функция выделяет из полного пути 
последний /элемент
\end{description}

\subsecly{\file{ypp.ypp}: синтаксический парсер}

Весь код между \verb|%{...%}| будет скопирован в выходной сгенерированный файл
\file{ypp.tab.cpp}

\lstx{Заголовочная часть с \cpp\ кодом}{skelex/ypp/head.ypp}

\lstx{используем универсальный тип для хранения дерева
разбора}{skelex/ypp/union.ypp}

\lstx{токены для скалярных типов}{skelex/ypp/tokscalars.ypp}
\lstx{правило для скалярных типов}{skelex/ypp/scalar.ypp}

\emph{символ, число и строка\ --- атомы информатики}

\lstx{токены для скобок}{skelex/ypp/brackets.ypp}

[L]eft/[R]ight [P]arens, [Q]uad, [C]url

\lstx{пачка операторов\ \ref{lexops}}{skelex/ypp/ops.ypp}

\lstx{типы выражений}{skelex/ypp/type.ypp}

\lstx{правила парсера помещаются между}{skelex/lpp/pp.lpp}

\lstx{REPL-цикл интерпретатора}{skelex/ypp/repl.ypp}

\lstx{скаляры}{skelex/ypp/scalar.ypp}

\lstx{выражения}{skelex/ypp/ex.ypp}

\subsecly{\file{lpp.lpp}: лексер}

Весь код между \verb|%{...%}| будет скопирован в выходной сгенерированный файл
\file{lex.yy.c}

\lstx{Заголовочная часть с \cpp\ кодом}{skelex/lpp/head.lpp}

определена дополнительная переменная \var{LexString}: буфер используемый при
разборе строк.

\lstx{опция}{skelex/lpp/yywrap.lpp}

подавляет вывод сообщений об отсутствии функции \var{yywrap} 

\lstx{опция включения счетчика нумерации строк}{skelex/lpp/lineno.lpp}

сохраняет в переменной \var{yylineno} номер текущей строки

\lstx{правила лексера помещаются между}{skelex/lpp/pp.lpp}

\lstx{строчные комментарии}{skelex/lpp/comment.lpp}

\lstx{разбор строк через специальное состояние лексера}{skelex/lpp/xstring.lpp}
\lstx{\ }{skelex/lpp/string.lpp}

\lstx{распознавание числел}{skelex/lpp/xnum.lpp}
\lstx{\ }{skelex/lpp/num.lpp}

\subsecly{\file{hpp.hpp}: хедеры}

\lst{\ }{skelex/hpp/head.hpp}{C++}
все остальное находится между препроцессорными ``скобками'',
блокирующими многократное включение кода
\lst{\ }{skelex/hpp/foot.hpp}{C++}

\lst{\var{\#include}}{skelex/hpp/include.hpp}{C++}

\lst{универсальный тип: Abstract Symbolic Type}{skelex/hpp/sym.hpp}{C++}

\lst{глобальная среда (таблица символов)}{skelex/hpp/env.hpp}{C++}

\lst{скаляры: строки}{skelex/hpp/string.hpp}{C++}
\lst{скаляры: числа}{skelex/hpp/num.hpp}{C++}

\lst{композиты}{skelex/hpp/list.hpp}{C++}

\lst{функционалы}{skelex/hpp/func.hpp}{C++}

\lst{интерфейс к лексеру/парсеру}{skelex/hpp/lex.hpp}{C++}

\subsecly{\file{cpp.cpp}: ядро интерпретатора}

\lst{\ }{skelex/cpp/hpp.cpp}{C++}
\lst{обработка ошибок синтаксического анализатора}{skelex/cpp/error.cpp}{C++}
\lst{функция \var{main()}}{skelex/cpp/main.cpp}{C++}
\lst{конструкторы AST}{skelex/cpp/sym.cpp}{C++}
\lst{дамп AST}{skelex/cpp/dump.cpp}{C++}
\lst{вычисление AST}{skelex/cpp/eval.cpp}{C++}

\lst{строки и \class{Sym::tagstr()}}{skelex/cpp/string.cpp}{C++}

\lst{числа}{skelex/cpp/num.cpp}{C++}

\lst{композиты}{skelex/cpp/list.cpp}{C++}

\lst{функционалы}{skelex/cpp/func.cpp}{C++}

\lst{глобальная таблица символов}{skelex/cpp/env.cpp}{C++}

\secly{Тестирование интерпретатора}\label{lextest}

\subsecly{Комментарии}

\lstx{\file{test/comment.src}}{skelex/src/comment.src}
\lstx{\file{test/comment.log}}{skelex/comment.log}

\subsecly{Скаляры и базовые композиты}

\lstx{\file{test/coretypes.src}}{skelex/src/coretypes.src}
\lstx{\file{test/coretypes.log}}{skelex/coretypes.log}

\secly{Операторы}\label{lexops}

\noindent\begin{tabular}{l l l}
\verb|A+B| & add & сложение \\
\verb|A-B| & sub & вычистание \\
\verb|A*B| & mul & умножение \\
\verb|A/B| & div & деление \\
\verb|A^B| & pow & возведение в степень \\
\verb|A>>B| & rsh & правый сдвиг \\
\verb|A<<B| & lsh & левый сдвиг \\
\hline
\end{tabular}

\noindent\begin{tabular}{l l l}
\verb|A>B| & great & больше \\
\verb|A=>B| & greateq & больше или равно \\
\verb|A<B| & less & меньше \\
\verb|A<=B| & lesseq & меньше или равно \\
\verb|A==B| & eq & равно \\
\verb|A!=B| & noteq & неравно\\
\verb|A&B| & and & и\\
\verb$A|B$ & or & или\\
\verb$A^B$ & xor & исключающее или\\
\verb$!A$ & not & не\\
\hline
\end{tabular}

\noindent\begin{tabular}{l l l}
\verb|A=B| & assign & назначение/присвоение переменной\\&&\emph{$A$
предварительно вычисляется}, результат является указателем на переменную\\
\verb|A@B| & apply & применение (функции) $A$ к (параметру) $B$\\
&&применимо не только к функциям: в общей случае $A$ может быть любым типом\\
&&в том числе классом: в роли конструктора объекта\\
\verb|~A| & quote & \emph{блокировка вычисления} выражения $A$ \\
\verb$A||B$ & map & применить распределенно $A$ \emph{к членам} $B$\\
&& функция \var{map}: $A$ функция, вычислить список $\rightarrow$ список\\
&& параллельное вычисление: $A$ constant-функция $f(x)=x$\\
&& \verb$A@B$ вычисляются параллельно при наличии поддержки в ядре
интерпретатора\\
\hline
\end{tabular}

\noindent\begin{tabular}{l l l}
\verb|A%B| & member & вложить $B$ как член $A$ \\
&&чаще всего используется в определении (добавлении) членов класса\\
\verb|A:B| & inherit & наследовать $B$ от $A$ \\
&&если $A$ составное, выполняется множественное наследование в порядке
итерации\\
&&если $A$ \emph{не класс}, выполняется наследование копированием\\
\verb|A.B| & index & доступ по индексу: $B$-ый член $A$\\
&&$B$ может быть именем или числовым индексом вложенного элемента из $A$\\
\hline
\end{tabular}

\noindent\begin{tabular}{l l l}
\verb|A<>B| & symm & симметричное правило замены $A\leftrightarrow B$\\
\verb|A>>B| & is & одностороннее правило замены $A\rightarrow B$\\
\verb|A<!>B| & notsym & симмектричный запрет замены $A\cancel\leftrightarrow
B$\\
\verb|A!>B| & notis & односторонний запрет замены $A\cancel\rightarrow B$\\
\end{tabular}

\secup

\secrel{Программирование в свободном синтаксисе: \prog{FSP}}\secdown

\secrel{Типичная структура проекта FSP: \textit{lexical skeleton}}\secdown

Скелет файловой структуры FSP-проекта = lexical skeleton = skelex 
\bigskip

\lstx{Создаем проект \prog{prog}\ из командной строки (\win):}{bi/fspskel.bat}
Создали каталог проекта, сгенерили набор пустых файлов (см. далее), и
запуститили батник-hepler который запустит \vim.

Для пользователей GitHub \verb|mkdir|\ надо заменить на 
\begin{verbatim}
git clone -o gh git@github.com:yourname/prog.git
cd prog
git gui &
...
\end{verbatim}

\bigskip

\begin{tabular}{l l l}
\file{src.src} & & исходный текст программы на вашем скриптовом языке \\
\file{log.log} & & лог работы ядра \bi\\
\file{ypp.ypp} & \prog{flex} & парсер \ref{biypp}\\
\file{lpp.lpp} & \prog{bison} & лексер \ref{bilpp}\\
\file{hpp.hpp} & \cpp & заголовочные файлы \ref{bihpp}\\
\file{cpp.cpp} & \cpp & код ядра \ref{bicpp}\\
\file{Makefile} & \prog{make} & зависимости между файлами и команды сборки (для
утилиты \prog{make})\\
\file{bat.bat} & \win & запускалка \prog{\vim}\ \ref{bibat}\\
\file{.gitignore} & \prog{git} & список масок временных и производных файлов
\ref{bigit}\\
\end{tabular}

\secrel{Настройки \vim}

При использовании редактора/IDE \prog{\vim}\ удобно настроить сочетания клавиш и
подсветку \emph{синтаксиса вашего скриптовго языка}\ так, как вам удобно. Для
этого нужно создать несколько файлов конфигурации .vim: по 2 файла\note{(1)
привязка расширения файла и (2) подсветка синтаксиса}\ для каждого диалекта
скрипт-языка\note{если вы пользуетесь сильно отличающимся синтаксисом, но
скорее всего через какое-то время практики FSP у вас выработается один диалект
для всех программ, соответсвующий именно вашим вкусам в синтаксисе, и в этом
случае его нужно будет описать только в файлах ~/.vim/(ftdetect|syntax).vim}, и
привязать их к расширениям через dot-файлы \vim\ в вашем домашнем каталоге.
Подробно конфигурирование \vim\ см. \ref{vim}. \bigskip 

\begin{tabular}{l l l}
filetype.vim & \vim & привязка расширений файлов (.src .log) к настройкам \vim
\\
syntax.vim & \vim & синтаксическая подсветка для скриптов \\
~/.vimrc & \linux & настройки для пользователя \\
~/vimrc & \win &\\
~/.vim/ftdetect/src.vim & \linux & привязка команд к расширению .src \\
~/vimfiles/ftdetect/src.vim & \win & \\
~/.vim/syntax/src.vim & \linux & синтаксис к расширению .src \\
~/vimfiles/syntax/src.vim & \win &\\
\end{tabular}

\secrel{Дополнительные файлы}

\begin{tabular}{l l l}
README.md & github & описание проекта для репоитория github \\
logo.png & github & логотип \\
logo.ico & \win & \\
rc.rc & \win & описание ресурсов: логотип, иконки приложения, меню,.. \\
\end{tabular}

% \file{rc.rc} & \ref{rc} & windres 
% 	& \\
% \file{logo.ico} && windres 
% 	& логотип в .ico формате \\
% \file{logo.png} &&
% 	& логотип в .png (для github README) \\
% \file{filetype.vim} & \ref{filetypevim} & (g)vim 
% 	& файлов cкриптов \\
% \file{syntax.vim} & \ref{syntaxvim} & (g)vim 
% 	&  \\
 
% \input{README.tex}

\secrel{Makefile}

Для сборки проекта используем команду \prog{make}\ или \prog{ming32-make}\ для
\win/\mingw. Прописываем в \file{Makefile}\ зависимости:

\lst{универсальный Makefile для fsp-проекта}{bi/Makefile}{make}

\begin{description}

\item{\file{./exe.exe}}\\ префикс ./ требуется для правильной работы
\prog{ming32-make}, поскольку в \linux\ исполняемый файл может иметь любое имя и
расширение, можем использовать .exe.

Для запуска транслятора используем простейший вариант\ --- перенаправление
потоков stdin/stdut на файлы, в этом случае не потребуется разбор параметров
командной строки, и получим подробную трассировку выполнения трансляции.

\item{переменные \var{C}\ и \var{H}} задают набор исходный файлов ядра
транслятора на \cpp:

\begin{description}
\item{\file{cpp.cpp}} реализация системы динамических типов данных,
наследованных от символьного типа AST \ref{ast}. Библиотека динамических классов
языка \bi\ \ref{bi}\ компактна, предоставляет достаточных набор типов
данных, и операций над ними. При необходимости вы можете легко написать свое
дерево классов, если вам достаточно только простого разбора.
\item{\file{hpp.hpp}} заголовочные файлы также используем из \bi\ \ref{bi}:
содержат декларации динамических типов и интерфейс лексического анализатора,
которые подходят для всех проектов
\item{ypp.tab.cpp ypp.tab.hpp} \cpp\ код синтаксического парсера, генерируемый
утилитой \prog{bison}\ \ref{parser}
\item{lex.yy.c} код лексического анализатора, генерируемый утилитой \prog{flex}\
\ref{flex}
\item{\var{CXXFLAGS}\verb| += gnu++11|} добавляем опцию диалекта \cpp,
необходимую для компиляции ядра \bi
\end{description}

\end{description}


% \secrel{bat.bat}\label{bat}
% 
% \lstx{bat.bat}{script/bat.bat}
% 
% \secrel{rc.rc}
% 
% \lstx{rc.rc}{script/rc.rc}
% 
% \bigskip\fig{}{../icons/hedgehog.png}{scale=2}

\secup
\secup
\secrel{Синтаксический анализ текстовых данных}\label{syntax}\secdown

\secrel{Универсальный \file{Makefile}}\label{lexmake}

Универсальный Makefile сделан на базе \ref{bimake}, с добавлением переменной
\var{APP}\ указывающий какой пример парсера следуует скомпилировать и выполнить.

Для хранения (и возможной обработки) отпарсенных данных используем ядро языка
\bi\ \ref{bicore}\ --- используем файлы \file{../bi/hpp.hpp}\ и
\file{../bi/cpp.cpp}. Ядро \emph{очень компактно}, но умеет работать со
скалярными, составными и функциональными данными, и содержит минимальную
реализацию \term{ядра динамического языка}.

\lstx{Универсальный \file{Makefile}}{parser/minimal.mk}

\secrel{\cpp\ интерфейс синтаксического анализатора}\label{lexinterface}

\begin{verbatim}
extern int yylex();             // получить код следующиго токена, и yylval.o 
extern int yylineno;            // номер текущей строки файла исходника
extern char* yytext;            // текст распознанного токена, asciiz
#define TOC(C,X) { yylval.o = new C(yytext); return X; }

extern int yyparse();           // отпарсить весь текущий входной поток токенов
extern void yyerror(string);    // callback вызывается при синтаксической ошибке
#include "ypp.tab.hpp"
\end{verbatim}

\secrel{Минимальный парсер}\label{miniparser}

Рассмотрим минимальный парсер, который может анализировать файлы текстовых
данных (например исходники программ), и вычленять из них последовательности
символов, которые можно отнести к \termdef{скалярам}: символ, строка и число.
\note{эти три типа можно назвать атомами computer science}

\lstx{Лексер \file{minimal.lpp}\
/\prog{flex}/}{parser/minimal.lpp}\label{minilexer}

\begin{description}
\item{(../bi/)\file{hpp.hpp}} содержит определения интерфейса лексера
\ref{lexinterface}, и ядра языка \bi\ \ref{bicore}\ для хранения результатов
разбора текстовых данных
\item{\var{noyywrap}} выключает использование функции \var{yywrap()}
\item{\var{yylineno}} включает отслеживание строки исходного файла, используется
при выводе сообщений об ошибках. В минимальном парсере не используется, но
требуется для сборки \bi-ядра.
\item{\verb|%%..%%|} набор правил группировки отдельных символов в
элементы данных\ --- \termdef{токены}{токен}, правила задаются с помощью \term{регулярных
выражений}
\item{\verb|TOC(Sym,SYM)|}\label{minisym} единственное правило, распознающее
любые группы сиволов как класс \class{bi::sym}: латинские буквы, цифры и символы
\_\ и .\ (точка)\note{точка добавлена, так часто используется в именах файлов}
\end{description}

\lstx{Парсер \file{minimal.ypp}\ /\prog{bison}/}{parser/minimal.ypp}

\begin{description}
\item{\file{hpp.hpp}} заголовок аналогичен лексеру \ref{minilexer}
\item{\verb|%defines %union|} указывает какие типы данных могут храниться в
узлах разобранного \termdef{синтаксического дерева}{синтаксическое дерево}.
Поскольку мы используем \bi-ядро, нам будет достаточно пользоваться
только классами языка \bi,
прежде всего универсальным символьным типом AST \ref{ast}\ и его прозводными
классами.
\item{\verb|%token|} описывает токены, которые может возвращать лексер
\ref{minlexer}, причем набор токенов должен быть согласованным между лексером и
парсером\note{определение токенов генерируется в файл \file{ypp.tab.hpp}}
\item{\verb|%type|} описывает типы синтаксических выражений, которые может
распознавать \termdef{грамматика}{грамматика} синтаксического анализатора, 
\item{\verb|REPL|} выражение, описывающее грамматику, аналогичную простейшему
варианту цикла REPL: Read Eval Print
Loop\note{чтение/вычисление/вывод/повторить}. В нашем случае часть вычисления
Eval не выполняется\note{разобранное выражение не вычисляется, хотя
используемое ядро \bi\ и поддерживает такой функционал}, а часть Print
выполняется через метод \verb|Sym.tagval()|, возвращающий котороткую строку
вида \verb|<класс:значение>|\ для найденного токена.
\item{\verb|ex|} (expression) универсальное символьное выражение языка \bi, в
нашем случае оно должно представлять только \verb|scalar|
\item{\verb|scalar|} выражение, представляющиее только распознаваемые скаляры:
\item{\verb|SYM|} символ, 
\item{\verb|STR|} строку \emph{или}
\item{\verb|NUM|} число\note{числа в грамматике языка \bi\ по типам не делятся,
токен соответствует как \class{int}, так и \class{num}}
\end{description}

В качестве тестового исходника возьмем \cpp\ код ядра языка \bi:
\file{../bi/cpp.cpp}:

\lstx{\file{minimal.src}: Тестовый исходник}{parser/minimal.src}

\lstx{\file{minimal.log}: Результат прогона}{parser/minimal.log}

Как видно по логу \file{minimal.log}, все группы сиволов, соответствующих
правилу лексера \verb|SYM|\ref{minisym}, распознались как объекты \bi, остальные
остались символами и попали в лог без изменений.


\secrel{Добавляем обработку комментариев}\label{minicomment}

В тестах программ и файлов конфигурации очень часто используются
\term{комментарии}. В языке \py, \bi\ и UNIX shell комментарием является все
от символа \#\ до конца строки.

Для обработки таких \termdef{строчных комментариев}{строчный комментарий}\
достаточно добавить одно правило лексера, \emph{обязательно первым правилом}:

\lstx{Лексер со строчными комментариями}{parser/comment.lpp}

Группа символов, начинающаяся с символа \#, затем идет ноль или более \verb|[]*|
любых символов не равных \verb|^|\ концу строки \verb|\n|.
Пустое тело правила: \cpp\ код в \verb|{}|\ скобках\ --- выполняется и ничего не
делает.

Тело правила SYM\ --- вызов макроса \verb|TOC(C,X)|, наоборот, при своем
выполнении создает токен, и возвращает код токена \verb|=SYM|.

\lstx{\file{comment.log}: Результат прогона}{parser/comment.log}

Как видно из лога, из вывода исчезли первые 2 строки, начинающиеся на \#, причем
концы этих строк остались (но не были как-либо распознаны).

\secrel{Разбор строк}\label{ministring}

Для разбора строк необходимо использовать лексер с применением
\termdef{состояний}{состояние лексера}. Строки имеют сильно отличающийся от
основного кода синтаксис, и для его обработки нужно \emph{переключать набор
правил лексера}.

\lstx{Лексер с состоянием для строк}{parser/string.lpp}\label{lexstring}

\begin{description}
\item{\verb|string LexString|} строковая буферная переменная, накапливающая
символы строки
\item{\verb|%x lexstring|} создание отдельного состояния лексера
\verb|lexstring|
\item{\verb|INITIAL|} основное состояние лексера
\item{\verb|<lexstring>\n|} правило конца строки позволяет использовать
многострочные строки\note{символ конца строки не распознается
метасимволом . (точка) в регулярном выражении, и требует явного указания}
\item{\verb|<lexstring>.|} любой символ в состоянии \verb|<lexstring>|
\end{description}

\lstx{Лог разбора со строками}{parser/string.log}

Обратите внимание, что ранее попадавшие в лог строки в двойных кавычках, типа
\verb|"]\n\n"|, стали распознаваться как строковые токены \verb|<str:']\n\n'>|.
\note{использованы 'одинарные кавычки' как в \py/\bi}

\secrel{Добавляем операторы}\label{miniops}

Для разбора языков программирования необходима поддержка операторов,
включим общепринятые одиночные операторы, операторы \cpp\ и \bi.
\emph{Скобки различного вида тоже будет рассматривать как операторы.}
Операторы реализованы в ядре \bi\ как отдельный класс \class{op}, зададим
пачку правил разбора операторов, создающих токены \verb|TOC(Op,XXX)|:

\lstx{Лексер с операторами}{parser/ops.lpp}

\lstx{Парсер с операторами}{parser/ops.ypp}

Лог уже стал нечитаем, переключаемся на древовидный вывод через метод
\verb|Sym.dump()|.

\lstx{Разбор с операторами}{parser/ops.log}

\secrel{Обработка вложенных структур (скобок)}

Обработка вложенных структур возможна только парсером, лексер оставляем
без изменений. Хранение вложенных структур в виде дерева\ --- главная фича
типа \bi\ AST\ref{ast}. Заменяем грамматическое выражение \verb|bracket|\ на
отдельные выражения для скобок:

\lstx{Парсер со скобками}{parser/brackets.ypp}

\lstx{Разбор со скобками}{parser/brackets.log}


\secup
\secrel{Синтаксический анализатор}\secdown

Синтаксис языка \bi\ был выбран алголо-подобным, более близким к современным
императивным языкам типа \cpp\ и \py. Использование типовых утилит-генераторов
позволяет легко описать синтаксис с инфиксными операторами и скобочной записью
для композитных типов\,\ref{bicompose}, и не заставлять пользователя
закапываться в море \lispовских скобок.

Инфиксный синтаксис для файлов конфигурации удобен неподготовленным
пользователям, а возможность определения пользовательских функций и объектная
система, встроенная в ядро \bi, дает богатейшие возможности по настройке
и кастомизации программ.

Единственной проблемой с точки зрения синтаксиса для начинающего пользователя
\bi\ может оказаться отказ от скобок при вызове функций, применение оператора
явной аппликации \verb|@|, и функциональные наклонности самого \bi,
претендующего на звание универсального \emph{объектного мета-языка}\ и
\emph{языка шаблонов}.

\secrel{\file{lpp.lpp}: лексер /flex/}\label{bilexer}
\secrel{\file{ypp.ypp}: парсер /bison/}\label{biparser}
\secup

