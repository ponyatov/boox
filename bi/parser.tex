\secrel{Синтаксический анализатор}\secdown

Синтаксис языка \bi\ был выбран алголо-подобным, более близким к современным
императивным языкам типа \cpp\ и \py. Использование типовых утилит-генераторов
позволяет легко описать синтаксис с инфиксными операторами и скобочной записью
для композитных типов\,\ref{bicompose}, и не заставлять пользователя
закапываться в море \lispовских скобок.

Инфиксный синтаксис для файлов конфигурации удобен неподготовленным
пользователям, а возможность определения пользовательских функций и объектная
система, встроенная в ядро \bi, дает богатейшие возможности по настройке
и кастомизации программ.

Единственной проблемой с точки зрения синтаксиса для начинающего пользователя
\bi\ может оказаться отказ от скобок при вызове функций, применение оператора
явной аппликации \verb|@|, и функциональные наклонности самого \bi,
претендующего на звание универсального \emph{объектного мета-языка}\ и
\emph{языка шаблонов}.

\secrel{\file{lpp.lpp}: лексер /flex/}\label{bilexer}
\secrel{\file{ypp.ypp}: парсер /bison/}\label{biparser}
\secup
