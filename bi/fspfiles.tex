\secrel{Программирование в свободном синтаксисе: \prog{FSP}}\secdown

\secrel{Типичная структура проекта FSP: \textit{lexical skeleton}}\secdown

Скелет файловой структуры FSP-проекта = lexical skeleton = skelex 
\bigskip

\lstx{Создаем проект \prog{prog}\ из командной строки (\win):}{bi/fspskel.bat}
Создали каталог проекта, сгенерили набор пустых файлов (см. далее), и
запуститили батник-hepler который запустит \vim.

Для пользователей GitHub \verb|mkdir|\ надо заменить на 
\begin{verbatim}
git clone -o gh git@github.com:yourname/prog.git
cd prog
git gui &
...
\end{verbatim}

\bigskip

\begin{tabular}{l l l}
\file{src.src} & & исходный текст программы на вашем скриптовом языке \\
\file{log.log} & & лог работы ядра \bi\\
\file{ypp.ypp} & \prog{flex} & парсер \ref{biypp}\\
\file{lpp.lpp} & \prog{bison} & лексер \ref{bilpp}\\
\file{hpp.hpp} & \cpp & заголовочные файлы \ref{bihpp}\\
\file{cpp.cpp} & \cpp & код ядра \ref{bicpp}\\
\file{Makefile} & \prog{make} & зависимости между файлами и команды сборки (для
утилиты \prog{make})\\
\file{bat.bat} & \win & запускалка \prog{\vim}\ \ref{bibat}\\
\file{.gitignore} & \prog{git} & список масок временных и производных файлов
\ref{bigit}\\
\end{tabular}

\secrel{Настройки \vim}

При использовании редактора/IDE \prog{\vim}\ удобно настроить сочетания клавиш и
подсветку \emph{синтаксиса вашего скриптовго языка}\ так, как вам удобно. Для
этого нужно создать несколько файлов конфигурации .vim: по 2 файла\note{(1)
привязка расширения файла и (2) подсветка синтаксиса}\ для каждого диалекта
скрипт-языка\note{если вы пользуетесь сильно отличающимся синтаксисом, но
скорее всего через какое-то время практики FSP у вас выработается один диалект
для всех программ, соответсвующий именно вашим вкусам в синтаксисе, и в этом
случае его нужно будет описать только в файлах ~/.vim/(ftdetect|syntax).vim}, и
привязать их к расширениям через dot-файлы \vim\ в вашем домашнем каталоге.
Подробно конфигурирование \vim\ см. \ref{vim}. \bigskip 

\begin{tabular}{l l l}
\file{filetype.vim} & \vim & привязка расширений файлов (.src .log) к настройкам
\vim
\\
\file{syntax.vim} & \vim & синтаксическая подсветка для скриптов \\
\file{~/.vimrc} & \linux & настройки для пользователя \\
\file{~/vimrc} & \win &\\
\file{~/.vim/ftdetect/src.vim} & \linux & привязка команд к расширению .src \\
\file{~/vimfiles/ftdetect/src.vim} & \win & \\
\file{~/.vim/syntax/src.vim} & \linux & синтаксис к расширению .src \\
\file{~/vimfiles/syntax/src.vim} & \win &\\
\end{tabular}

\secrel{Дополнительные файлы}

\begin{tabular}{l l l}
README.md & github & описание проекта для репоитория github \\
logo.png & github & логотип \\
logo.ico & \win & \\
rc.rc & \win & описание ресурсов: логотип, иконки приложения, меню,.. \\
\end{tabular}

\fig{\file{logo.png}: Логотип}{../icons/hedgehog.png}{height=0.35\textheight}

% \file{rc.rc} & \ref{rc} & windres 
% 	& \\
% \file{logo.ico} && windres 
% 	& логотип в .ico формате \\
% \file{logo.png} &&
% 	& логотип в .png (для github README) \\
% \file{filetype.vim} & \ref{filetypevim} & (g)vim 
% 	& файлов cкриптов \\
% \file{syntax.vim} & \ref{syntaxvim} & (g)vim 
% 	&  \\
 
% \input{README.tex}

\secrel{Makefile}

Для сборки проекта используем команду \prog{make}\ или \prog{ming32-make}\ для
\win/\mingw. Прописываем в \file{Makefile}\ зависимости:

\lst{универсальный Makefile для fsp-проекта}{bi/Makefile}{make}

\begin{description}

\item{\file{./exe.exe}}\\ префикс ./ требуется для правильной работы
\prog{ming32-make}, поскольку в \linux\ исполняемый файл может иметь любое имя и
расширение, можем использовать .exe.

Для запуска транслятора используем простейший вариант\ --- перенаправление
потоков stdin/stdut на файлы, в этом случае не потребуется разбор параметров
командной строки, и получим подробную трассировку выполнения трансляции.

\item{переменные \var{C}\ и \var{H}} задают набор исходный файлов ядра
транслятора на \cpp:

\begin{description}
\item{\file{cpp.cpp}} реализация системы динамических типов данных,
наследованных от символьного типа AST \ref{ast}. Библиотека динамических классов
языка \bi\ \ref{bi}\ компактна, предоставляет достаточных набор типов
данных, и операций над ними. При необходимости вы можете легко написать свое
дерево классов, если вам достаточно только простого разбора.
\item{\file{hpp.hpp}} заголовочные файлы также используем из \bi\ \ref{bi}:
содержат декларации динамических типов и интерфейс лексического анализатора,
которые подходят для всех проектов
\item{ypp.tab.cpp ypp.tab.hpp} \cpp\ код синтаксического парсера, генерируемый
утилитой \prog{bison}\ \ref{parser}
\item{lex.yy.c} код лексического анализатора, генерируемый утилитой \prog{flex}\
\ref{flex}
\item{\var{CXXFLAGS}\verb| += gnu++11|} добавляем опцию диалекта \cpp,
необходимую для компиляции ядра \bi
\end{description}

\end{description}


% \secrel{bat.bat}\label{bat}
% 
% \lstx{bat.bat}{script/bat.bat}
% 
% \secrel{rc.rc}
% 
% \lstx{rc.rc}{script/rc.rc}
% 
% \bigskip\fig{}{../icons/hedgehog.png}{scale=2}

\secup

\secup