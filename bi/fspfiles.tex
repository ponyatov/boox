\secrel{Программирование в свободном синтаксисе: \prog{FSP}}\secdown

\secrel{Типичная структура проекта FSP: \textit{lexical skeleton}}\secdown

Скелет файловой структуры FSP-проекта = lexical skeleton = skelex 
\bigskip

\lstx{Создаем проект \prog{prog}\ из командной строки (\win):}{bi/fspskel.bat}
Создали каталог проекта, сгенерили набор пустых файлов (см. далее), и
запуститили батник-hepler который запустит \gvim.

Для пользователей GitHub \verb|mkdir|\ надо заменить на 
\begin{verbatim}
git clone -o gh git@github.com:yourname/prog.git
cd prog
git gui &
...
\end{verbatim}

\bigskip

\begin{tabular}{l l l}
\file{src.src} & & исходный текст программы на вашем скриптовом языке \\
\file{log.log} & & лог работы ядра \bi\\
\file{ypp.ypp} & \prog{flex} & парсер \ref{biypp}\\
\file{lpp.lpp} & \prog{bison} & лексер \ref{bilpp}\\
\file{hpp.hpp} & \cpp & заголовочные файлы \ref{bihpp}\\
\file{cpp.cpp} & \cpp & код ядра \ref{bicpp}\\
\file{Makefile} & \prog{make} & зависимости между файлами и команды сборки (для
утилиты \prog{make})\\
\file{bat.bat} & \win & запускалка \prog{gvim}\ \ref{bibat}\\
\file{.gitignore} & \prog{git} & список масок временных и производных файлов
\ref{bigit}\\
\end{tabular}

% \file{rc.rc} & \ref{rc} & windres 
% 	& описание ресурсов: иконки приложения, меню,..\\
% \file{logo.ico} && windres 
% 	& логотип в .ico формате \\
% \file{logo.png} &&
% 	& логотип в .png (для github README) \\
% \file{filetype.vim} & \ref{filetypevim} & (g)vim 
% 	& привязка расширения файлов cкриптов \\
% \file{syntax.vim} & \ref{syntaxvim} & (g)vim 
% 	& синтаксическая подсветка для скриптов \\
 
% \input{README.tex}
% \input{makefile.tex}
% 
% \secrel{bat.bat}\label{bat}
% 
% \lstx{bat.bat}{script/bat.bat}
% 
% \secrel{rc.rc}
% 
% \lstx{rc.rc}{script/rc.rc}
% 
% \bigskip\fig{}{../icons/hedgehog.png}{scale=2}

\secup
