\secrel{DLR: Dynamic Language Runtime}\label{dlr}

\begin{framed}
DLR: Dynamic Language Runtime\ --- может использоваться как runtime-ядро
для реализации динамичеcких языков, или только в качестве библиотеки хранилища
данных
\end{framed}

\begin{description}
  \item[синтаксичеcкий парсер]
  для разбора текстовых данных, файлов конфигурации, скриптов и т.п.,
  необязателен. В результате разбора формируется синтаксическое дерево из
  динамических объектов DLR. По реализации может быть
  \begin{description}[nosep]
  \item[конфигурируемым в runtime]
  добавление/изменение/удаление правил правил грамматики в процессе работы
  программы
  \item[статическим] неизменный синтаксис, реализация в виде внешнего модуля,
  в самом простом случае достаточно использования \prog{flex}/\prog{bison}
  \end{description}
  \item[библиотека динамических типов данных] выполняет функции хранения
  данных, может быть реализована
  \begin{description}[nosep]
  \item[в \lisp-стиле]
  базовый набор скаляров \ref{biscalars}\ (символы, строки
  и числа) и тип \class{cons-ячейка}\ позволяющий конструировать составные 
  структуры данных
  \item[\bi-стиль]
  универсальный символьный тип \ref{ast}, позволяющий хранить как
  скаляры, так и вложенные элементы; в базовый тип \class{AST}\ заложено
  хранение типа данных \var{tag}, его значения \var{value}, и два способа
  вложенных хранилищ: плоский упорядоченный список \var{nest}\ и именованный
  неупорядоченный со строковыми ключами \var{pars}.
  
  От базового символьного типа наследуются
  \begin{description}[nosep]
  \item[композиты] структуры данных и
  \item[функционалы]
  \end{description}
  \end{description}
  \item[библиотека операций над данными]
  для преобразования данных и символьных вычислений на списках, деревьях,
  комбинаторах и т.п.
  \begin{description}[nosep]
  \item[\lisp] стандартная библиотека функций языка \lisp
  \item[\bi] каждый тип данных имеет набор унарных и
  бинарных \term{операторов}, реализованых в виде виртуальных методов классов
  \end{description}
  
  \item[менеджер памяти со сборщиком мусора]
  \item[подсистема облачных вычислений и кластеризации] распределение объектов и
  процессов вычисления между узлами кластера 
  \item[динамический компилятор] функциональных типов\ --- через библиотеку JIT
  LLVM
  \item[статический компилятор] \ \\
  \begin{description}[nosep]
  \item[в объектный код] через LLVM
  \item[кодогенератор \cpp]
  \end{description}
  \item[прикладные библиотеки] GUI, CAD/CAM/EDA, численные методы, цифровая
  обработка сигналов, сетевые сервера и протоколы,\ldots
\end{description}

\secrel{Система динамических типов}\label{bicore}\secdown
\secrel{\class{sym}: символ = Абстрактный Символьный Тип /AST/}\label{ast}

Использование класса \class{Sym}\ и виртуально наследованных от него
классов, позволяет реализовать на \cpp\ хранение и обработку \emph{любых}\
данных в виде деревьев\note{в этом АСТ близок к традиционной аббревиатуре AST: Abstract
Syntax Tree}. Прежде всего этот \termdef{символьный тип}{символьный тип}\
применяется при разборе текстовых форматов данных, и текстов программ.
\emph{Язык \bi\ построен как интерпретатор AST, примерно так же как язык \lisp\
использует списки}.

\bigskip

\begin{lstlisting}[language=C++]
// ============================================== ABSTRACT SYMBOLIC TYPE (AST)
struct Sym {
\end{lstlisting}

\begin{lstlisting}[language=C++,title=тип (класс) и значение элемента данных]
// ---------------------------------------------------------------------------
	string tag;							// data type / class
	string val;							// symbol value
\end{lstlisting}

\begin{lstlisting}[language=C++,title=конструкторы (токен используется в
лексере)]
// -------------------------------------------------------------- constructors
	Sym(string,string);					// <T:V>
	Sym(string);						// token
\end{lstlisting}


Хранение вложенных элементов реализовано через указатели на базовый тип
\class{Sym}. Благодаря виртуальному наследованию и использованию RTTI, этими
указателями можно пользоваться для работы с любыми другими наследованными типами
данных\note{числа, списки, высокоуровневые и скомпилированные функции,
элементы GUI,..}

\begin{lstlisting}[language=C++,
title=AST может хранить (и обрабатывать) вложенные элементы]
// --------------------------------------------------------- nest[]ed elements
	vector<Sym*> nest;
	void push(Sym*);
	void pop();
\end{lstlisting}

\begin{lstlisting}[language=C++,title=параметры (и поля класса)]
// -------------------------------------------------------------- par{}ameters
	map<string,Sym*> pars;
	void par(Sym*);						// add parameter
\end{lstlisting}

\begin{lstlisting}[language=C++,title=вывод дампа объекта в текстовом формате]
// ------------------------------------------------------------------- dumping
	virtual string dump(int depth=0);	// dump symbol object as text
	virtual string tagval();			// <T:V> header string
	string tagstr();					// <T:'V'> Str-like header string
	string pad(int);					// padding with tree decorators
\end{lstlisting}

\emph{Операции над \termdef{символами}{символ}\ выполняются через использование
набора \termdef{операторов}{оператор}:}

\begin{lstlisting}[language=C++,title=вычисление объекта]
// -------------------------------------------------------- compute (evaluate)
	virtual Sym* eval();
\end{lstlisting}

\begin{lstlisting}[language=C++,title=операторы]
// ----------------------------------------------------------------- operators
	virtual Sym* str();					// str(A)	string representation
	virtual Sym* eq(Sym*);				// A = B	assignment
	virtual Sym* inher(Sym*);			// A : B	inheritance
	virtual Sym* member(Sym*);			// A % B,C	named member (class slot)
	virtual Sym* at(Sym*);				// A @ B	apply
	virtual Sym* add(Sym*);				// A + B	add
	virtual Sym* div(Sym*);				// A / B	div
	virtual Sym* ins(Sym*);				// A += B	insert
};
\end{lstlisting}

\secrel{Скаляры}\label{biscalars}\secdown
\secrel{\class{str}: строка}
\secrel{\class{int}: целое число}
\secrel{\class{hex}: машинное hex}
\secrel{\class{bin}: бинарная строка}
\secrel{\class{num}: число с плавающей точкой}
\secup
\secrel{Композиты}\secdown
\secrel{\class{list}: плоский список}
\secrel{\class{cons}: cons-пара и списки в \lisp-стиле}
\secup
\secrel{Функционалы}\secdown
\secrel{\class{op}: оператор}
\secrel{\class{fn}: встроенная/скомпилированная функция}
\secrel{\class{lambda}: лямбда}
\secup
\secup

