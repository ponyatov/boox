\secly{\class{AST}: Абстрактный Символьный Тип /\class{Sym}/}

Использование класса \class{Sym}\ и виртуально наследованных от него
классов, позволяет реализовать на \cpp\ хранение и обработку \emph{любых}\
данных в виде деревьев\note{в этом АСТ близок к традиционной аббревиатуре AST: Abstract
Syntax Tree}. Прежде всего этот \termdef{символьный тип}{символьный тип}\
применяется при разборе текстовых форматов данных, и текстов программ.
\emph{Язык \bi\ построен как интерпретатор AST, примерно так же как язык \lisp\
использует списки}.

\bigskip

\begin{lstlisting}[language=C++]
// ============================================== ABSTRACT SYMBOLIC TYPE (AST)
struct Sym {
\end{lstlisting}

\begin{lstlisting}[language=C++,title=тип (класс) и значение элемента данных]
// ---------------------------------------------------------------------------
	string tag;							// data type / class
	string val;							// symbol value
\end{lstlisting}

\begin{lstlisting}[language=C++,title=конструкторы (токен используется в
лексере)]
// -------------------------------------------------------------- constructors
	Sym(string,string);					// <T:V>
	Sym(string);						// token
\end{lstlisting}


Хранение вложенных элементов реализовано через указатели на базовый тип
\class{Sym}. Благодаря виртуальному наследованию и использованию RTTI, этими
указателями можно пользоваться для работы с любыми другими наследованными типами
данных\note{числа, списки, высокоуровневые и скомпилированные функции,
элементы GUI,..}

\begin{lstlisting}[language=C++,
title=AST может хранить (и обрабатывать) вложенные элементы]
// --------------------------------------------------------- nest[]ed elements
	vector<Sym*> nest;
	void push(Sym*);
	void pop();
\end{lstlisting}

\begin{lstlisting}[language=C++,title=параметры (и поля класса)]
// -------------------------------------------------------------- par{}ameters
	map<string,Sym*> pars;
	void par(Sym*);						// add parameter
\end{lstlisting}

\begin{lstlisting}[language=C++,title=вывод дампа объекта в текстовом формате]
// ------------------------------------------------------------------- dumping
	virtual string dump(int depth=0);	// dump symbol object as text
	virtual string tagval();			// <T:V> header string
	string tagstr();					// <T:'V'> Str-like header string
	string pad(int);					// padding with tree decorators
\end{lstlisting}

\emph{Операции над \termdef{символами}{символ}\ выполняются через использование
набора \termdef{операторов}{оператор}:}

\begin{lstlisting}[language=C++,title=вычисление объекта]
// -------------------------------------------------------- compute (evaluate)
	virtual Sym* eval();
\end{lstlisting}

\begin{lstlisting}[language=C++,title=операторы]
// ----------------------------------------------------------------- operators
	virtual Sym* str();					// str(A)	string representation
	virtual Sym* eq(Sym*);				// A = B	assignment
	virtual Sym* inher(Sym*);			// A : B	inheritance
	virtual Sym* member(Sym*);			// A % B,C	named member (class slot)
	virtual Sym* at(Sym*);				// A @ B	apply
	virtual Sym* add(Sym*);				// A + B	add
	virtual Sym* div(Sym*);				// A / B	div
	virtual Sym* ins(Sym*);				// A += B	insert
};
\end{lstlisting}
