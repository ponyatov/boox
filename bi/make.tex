\secrel{Makefile}

Для сборки проекта используем команду \prog{make}\ или \prog{ming32-make}\ для
\win/\mingw. Прописываем в \file{Makefile}\ зависимости:

\lst{универсальный Makefile для fsp-проекта}{bi/Makefile}{make}

\begin{description}

\item{\file{./exe.exe}}\\ префикс ./ требуется для правильной работы
\prog{ming32-make}, поскольку в \linux\ исполняемый файл может иметь любое имя и
расширение, можем использовать .exe.

Для запуска транслятора используем простейший вариант\ --- перенаправление
потоков stdin/stdut на файлы, в этом случае не потребуется разбор параметров
командной строки, и получим подробную трассировку выполнения трансляции.

\item{переменные \var{C}\ и \var{H}} задают набор исходный файлов ядра
транслятора на \cpp:

\begin{description}
\item{\file{cpp.cpp}} реализация системы динамических типов данных,
наследованных от символьного типа AST \ref{ast}. Библиотека динамических классов
языка \bi\ \ref{bi}\ компактна, предоставляет достаточных набор типов
данных, и операций над ними. При необходимости вы можете легко написать свое
дерево классов, если вам достаточно только простого разбора.
\item{\file{hpp.hpp}} 
\end{description}

\end{description}
