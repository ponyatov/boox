\secly{Ф.И.Атауллаханов об учебниках США и России}

\copyright\ Доктор биологических наук Фазли Иноятович Атауллаханов.\\
МГУ им. М. В. Ломоносова, Университет Пенсильвании, США

\bigskip 
\url{http://www.nkj.ru/archive/articles/19054/}
\bigskip 

\ldots

У необходимости рекламировать науку есть важная обратная сторона: каждый
американский учёный непрерывно, с первых шагов и всегда, учится излагать свои
мысли внятно и популярно. В России традиции быть понятными у учёных нет. Как
пример я люблю приводить двух великих физиков: русского Ландау и американца
Фейнмана. Каждый написал многотомный учебник по физике. Первый\ --- знаменитый
``Ландау-Лифшиц'', второй\ --- ``Лекции по физике''. Так вот, ``Ландау-Лифшиц''
прекрасный справочник, но представляет собой полное издевательство над
читателем. Это типичный памятник автору, который был, мягко говоря, малоприятным
человеком. Он излагает то, что излагает, абсолютно пренебрегая своим читателем и
даже издеваясь над ним. А у нас целые поколения выросли на этой книге, и
считается, что всё нормально, кто справился, тот молодец. Когда я столкнулся с
``Лекциями по физике''\ Фейнмана, я просто обалдел: оказывается, можно
по-человечески разговаривать со своими коллегами, со студентами, с аспирантами.
Учебник Ландау\ --- пример того, как устроена у нас вся наука. Берёшь текст
русской статьи, читаешь с самого начала и ничего не можешь понять, а иногда
сомневаешься, понимает ли автор сам себя. Конечно, крупицы осмысленного и
разумного и оттуда можно вынуть. Но автор явно считает, что это твоя работа\ ---
их оттуда извлечь. Не потому, что он не хочет быть понятым, а потому, что его не
научили правильно писать. Не учат у нас человека ни писать, ни говорить понятно,
это считается неважным.

\ldots